\chapter{Podstata ekonometrie a ekonomických dat}

\section{Empirická analýza}

Empirická analýza využívá data k testování ekonomických teorií nebo k odhadu vztahů mezi ekonomickými 
veličinami. Samotné empirické analýze dat zpravidla předchází konstrukce formálního ekonomického 
modelu. Po té, co specifikujeme ekonomický model, je třeba ho transformovat v tzv. ekonometrický model, který má 
podobu rovnice
\begin{equation}
y_i = \beta_0 + \beta_1 x_{1i} + \beta_2 x_{2i} + ... + \beta_j x_{ji} + u_i.
\end{equation}
Parametry ekonometrického modelu jsou pomocí ekonometrických metod odhadnuty na základě podkladových dat. 
Následně je možné s pomocí ekonometrického modelu formulovat a testovat nejrůznější hypotézy.

\section{Struktura ekonomických data}

\subsection{Průřezová data (cross-sectional data)}
Průřezová data jsou data týkající se vzorku náhodně vybraného z tzv. podkladové populace, která se vztahují k 
jednomu konkrétnímu časovému okamžiku. Příkladem může být příjem jednoho tisíce 
zaměstnanců na základě daňového přiznání podaného za rok 2017.

\subsection{Časové řady}
Časové řady lze charakterizovat jako soubor pozorování týkajících se jedné nebo skupiny proměnných, které 
pokrývají určitou časovou periodu. Klasickým příkladem je vývoj reálného HDP České republiky v 
průběhu let 1993 až 2017.

\subsection{Sdružená průřezová data (pooled cross sections data)}
Sdružená průřezová data jsou syntézí průřezových dat a časových řad. Příkladem tohoto typu dat může 
být příjem dvou náhodně vybraných skupin zaměstnanců za rok 2016 a 2017, kdy jsme tyto dva na sobě 
nezávislé výběry sloučili s cílem zvětšit velikost pozorovaného vzorku popř. s cílem analyzovat vývoj průměrné mzdy v čase.

\subsection{Panelová data}
Panelová data se skládají z časových řad pro každý prvek z určitého datového setu. Příkladem panelových dat je 
vývoj příjmu konkrétních domáctností v letech 2007 až 2017. Na rozdíl od sdružených průřezových 
dat se tedy struktura náhodného vzorku v čase nemění.
