\chapter{Časové řady pro pokročilé}

\section{Nekonečně rozdělená zpoždění (IDL - infinite distributed lag)}

Model nekonečně rozdělených zpoždění je definován jako
\begin{equation}
y_t = \alpha + \delta_0 z_t + \delta_1 z_{t-1} + \delta_2 z_{t-2} + ... + u_t.
\end{equation}
Aby model (18.1) dával smysl, musí koeficient zpoždění $\delta_j$ konvergovat k nule s tím, jak $j 
\rightarrow \infty$. To znamená, že dočasné zvýšení veličiny 
$z$ o jednu jednotku nemá z dlouhodobého 
hlediska dopad na očekávanou hodnotu $y$. Nechť $z_i = 0$ s vyjímkou 
$t = 0$, kdy $z_0 = 1$. S tím, jak postupuje čas a $z_0$ 
se stává $z_1$, $z_2$ až $z_h$, se očekávaná hodnota $y_0$, 
$y_1$, $y_2$ až $y_h$ mění 
podle schématu
\begin{equation}
E[y_0] = \alpha + \delta_0,
\end{equation}
\begin{equation}
E[y_1] = \alpha + \delta_1,
\end{equation}
\begin{equation}
E[y_2] = \alpha + \delta_2,
\end{equation}
až
\begin{equation}
E[y_h] = \alpha + \delta_h.
\end{equation}
Parametr $\delta_h$ tak měří změnu v očekávané hodnotě $y$ po $h$ časových 
krocích. Protože pro $h \rightarrow \infty$ předpokládáme $\delta_h 
\rightarrow 0$, je dlouhodobá střední hodnota $y$ rovna $\alpha$. Dlouhodobá propenzita (LRP - long-term propensity) modelu (18.1) je tak
\begin{equation}
LRP = \delta_0 + \delta_1 + \delta_2 + \delta_3 + ...
\end{equation}
V praxi je $LPR$ často aproximováno konečným součtem ve tvaru $\delta_0 + 
\delta_1 + ... + \delta_p$ pro dostatečně vysoké $p$. Tímto způsobem lze např. kvatifikovat dopad zvýšení 
měnového agregátu na růst HDP.

V modelu (18.1) také předpokládáme, že očekávaná hodnota $u_i$ 
je nezávislá $z$, tj. předpokládáme tzv. striktní exogenitu
\begin{equation}
E[u_t|...,z_{t-2}, z_{t-1}, z_t, z_{t+1}, z_{t+2}, ...] = 0.
\end{equation}
Je důležité si uvědomit, že (18.7) nedovoluje zpětnou vazbu mezi $y_t$ a budoucími hodnotami $z$, protože $z_{t+h}$ musí být nekorelované s 
$u_t$ pro $h > 0$. Slabší verze (18.7) má pak tvar
\begin{equation}
E[u_t | z_t, z_{t-1}, ...] = 0,
\end{equation}
kde $u_t$ je nekorelované se současnou a minulými hodnotami $z$, avšak může být korelováné s 
budoucími hodnotami $z$. Nicméně (18.6) ani (18.7) nám neříkají nic o sériové korelaci $\{u_i\}$.

\subsection{Geometricky rozdělená zpoždění (GDL - geometric distributed lag) aka Koyckův model}

V případě geometricky rozdělených zpoždění je parametr $\delta_j$ definován jako
\begin{equation}
\delta_j = \gamma \rho^j, ~~~ |\rho| < 1, ~~~ j = 0, 1, 2, ...
\end{equation}
Tento vztah garantuje $\delta_j \rightarrow 0$ pro $j \rightarrow \infty$. Dlouhodobá propenzita modelu je
\begin{equation}
LRP = \frac{\gamma}{1 - \rho},
\end{equation}
a protože $|\rho| < 1$, je jeho směr zcela determinován znaménkem parametru $\gamma$.

Uvažume IDL model a čase $t$
\begin{equation}
y_t = \alpha + \gamma z_t + \gamma \rho z_{t-1} + \gamma \rho^2 z_{t-2} + ... + u_t
\end{equation}
a IDL model v čase $t$ vynásobený parametrem $\rho$
\begin{equation}
\rho y_{t-1} = \alpha + \gamma \rho z_{t-1} + \gamma \rho^2 z_{t-2} + \gamma \rho^3 z_{t-3} + ... + \rho u_{t-1}.
\end{equation}
Pokud od sebe odečteme (18.11) a (18.12), získáme
\begin{equation}
y_t - \rho y_{t - 1} = (1 - \rho)\alpha + \gamma z_t + u_t - \rho u_{t - 1},
\end{equation}
což lze dále upravit na
\begin{equation}
y_t = \alpha_0 + \gamma z_t + \rho y_{t-1} + u_t - \rho u_{t - 1},
\end{equation}
kde $\alpha_0 = (1 - \rho) \alpha$. (18.14) na první pohled vypadá 
jako standardní regresní model, pro který lze parametry 
$\alpha$, $\gamma$ a $\rho$ snadno odhadnout. Zdánlivá trivialita (18.14) je však zavádějící, protože chyba $\upsilon_t = u_t - \rho u_{t - 1}$ a $y_{t-1}$ jsou 
korelované. To má za následek zkrelené odhady parametrů $\gamma$ a $\rho$.

Jedním z případů, kdy $\upsilon_t$ musí být korelováno s $y_{t-1}$, je situace, kdy $u_t$ je nezávislé na $z_t$ a 
všech přechozích hodnotách $z$ a $y$. Pak je model (18.10) 
dynamicky kompletní a $u_t$ je tak nekorelované s 
$y_{t-1}$. Při splnění těchto předpokladů lze odvodit, že 
kovariance mezi $\upsilon_t$ a $y_{t-1}$ je $-\rho var[u_{t-1}] = -\rho 
\sigma_{u}^2$, což je rovno nule pouze pro $\rho = 0$. Lze si tak 
snadno všimnout, že $y_t$ musí být sériově korelované - vzhledem k tomu, že $\{u_i\}$ je 
sériově nekorelované, platí
\begin{multline}
E[\upsilon_t, \upsilon_{t-1}] =\\
E[u_t u_{t-1}] - \rho E[u_{t-1}^2] - \rho E[u_t u_{t-2}] + \rho^2 E[u_t, u_{t-2}] =\\
-\rho \sigma_u^2.
\end{multline}
Pro $j > 1$ je $E[\upsilon_t \upsilon_{t-j}] = 0$. Proto $\{y_t\}$ 
sleduje proces klouzavého průměru prvního řádu. Tento závěr 
a (18.14) vede k modelu, který byl odvozen z původního modelu, a 
který je charakteristický zpožděnou závislou 
veličinou $y_{t-1}$ a specifickým typem sériové korelace.

Jestliže předpokládáme, že $\{u_t\}$ sleduje AR(1) model, tj.
\begin{equation}
u_t = \rho u_{t - 1} + e_t,
\end{equation}
\begin{equation}
E[e_t|z_t, y_{t-1}, z_{t-1}, ...] = 0,
\end{equation}
lze parametry modelu (18.14) relativně snadno odhadnout. Pokud jsou totiž splněny podmínky (18.16) a (18.17), lze (18.14) přepsat do tvaru
\begin{equation}
y_t = \alpha_0 + \gamma z_t + \rho y_{t-1} + e_t,
\end{equation}
což je dynamicky kompletní model, pro který lze získat konzistentní a asymptoticky normální odhady 
parametrů pomocí OLS. Pokud $e_t$ navíc splňuje předpoklad homoskedasticity, jsou tyto odhady také efektivní. Po té, 
co odhadneme $\gamma$ a $\rho$, lze výpočíst dlouhodobou propenzitu modelu jako
\begin{equation}
\hat{LRP} = \frac{\hat{\gamma}}{1 - \hat{\rho}}.
\end{equation}

\subsection{Racionálně rozdělená zpoždění (RDL - rational distributed lag)}

Model s racioálně rozdělenými zpožděními má tvar
\begin{equation}
y_t = \alpha_0 + \gamma_0 z_t + \rho y_{t-1} + \gamma_1 z_{t-1} + \upsilon_t,
\end{equation}
kde $\upsilon_t = u_t - \rho u_{t-1}$. Lze dokázat, že (18.20) je ekvivalentní modelu s nekonečným časovým 
posunem, protože
\begin{multline}
y_t = \alpha + \gamma_0(z_t + \rho z_{t-1} + \rho^2 z_{t-2} + ...)\\
+ \gamma_1(z_{t - 1} + \rho z_{t-2} + \rho^2 z_{t-3} + ....) + u_t\\
= \alpha + \gamma_0 z_t + (\rho \gamma_0 + \gamma_1)z_{t-1} + \rho(\rho \gamma_0 + \gamma_1) z_{t-2}\\
+ \rho^2(\rho \gamma_0 + \gamma_1)z_{t-3} + ... + u_t,
\end{multline}
kde opět předpokládáme $|\rho| < 1$. Pro $h \ge 1$ je koeficient vysvětlující proměnné $z_{t-h}$ roven 
$\rho^{h-1}(\rho \gamma_0 + \gamma_1)$. Dlouhodobá propenzita modelu je rovna
\begin{equation}
LRP = \frac{\gamma_0 + \gamma_1}{1 - \rho}
\end{equation}
a lze ji odvodit dosazením $y_i = y^*$ a $z_i = z^*$ do (18.20).

\section{Testování jednotkového kořene}

\subsection{Dickey-Fullerův test}

Uvažujme AR(1) model

\begin{equation}
y_t = \alpha + \rho y_{t-1} + e_t,  ~~~ t = 1, 2, ...
\end{equation}
a předpokládejme
\begin{equation}
E[e_t|y_{t-1}, y_{t-2}, ...] = 0.
\end{equation}
Jestliže $\{y_t\}$ sleduje (18.23), pak má $\{y_t\}$ jednotkový 
kořen právě tehdy a jen tehdy, jestliže $\rho = 1$. Nulová 
hypotéza je tedy definována jako
\begin{equation}
H_0: \rho = 1
\end{equation}
a alternativní hypotéza je zpravidla definována jako
\begin{equation}
H_1: \rho < 1,
\end{equation}
kde v praxi implicitně předpokládáme $0 < \rho < 1$, protože $\rho < 0$ je v časových řadách, které 
podezíráme z jednotkového kořene, velmi vyjímečné.

Jesliže $|\rho| <1$, je $\{y_i\}$ stabilní AR(1) proces, což znamená, že $\{u_i\}$ je slabě závislé nebo 
asymptoticky nekorelované.

V kapitole 11 jsme dokázali, že $cor[y_t, y_{t+h}] = \rho^h \rightarrow 0$ pro 
$|\rho| < 1$. Proto je testování nulové hypotézy proti 
alternativní hypotéze ve skutečnosti testem, zda-li $\{y_t\}$ 
je $I(1)$ proces proti testu, že $\{y_t\}$ je $I(0)$ proces.

Jednoduchý test na jednotkový kořen spočívá v odečtení $y_{t-1}$ od obou stran rovnice (18.23) a definování 
$\theta$ jako $\theta = \rho - 1$. Výsledkem je model
\begin{equation}
\Delta y_t = \alpha + \theta y_{t-1} + e_t.
\end{equation}

Při splnění předpokladu (18.24) je (18.27) dynamicky kompletní, a proto by se mohlo zdát, že stačí pouze otestovat $H_0: 
\theta = 0$ proti $H_1: \theta < 0$. V případě platnosti hypotézi $H_0$ je 
však $y_{t-1}$ I(1) procesem a centrální limitní theorém nutný pro asymptoticky normální 
rozdělení $t$ statistiky odhadovaného parametru $\theta$ tak neplatí ani pro výběr velkého rozsahu. Asymptotické 
rozdělení $t$ statistiky při platné hypotéze $H_0$ je známé jako Dickey-Fullerovo rozdělení a 
odpovídající test je tak známý jako Dickey-Fullerův (DF) test. V 
rámci testu porovnáváme standardní $t$ statistiku parametru $\theta$ 
v modelu (18.27) s kritickými hodnotami uvedenými v tabulce (18.1). 
Pokud je $t$ statistika menší než odpovídající kritická hodnota, 
nulovou hypotézu o existenci nulového kořene na dané hladině významnosti zamítneme.

\begin{table}
\begin{center}
\begin{tabular}{| l | c | c | c | c |}
\hline
\bf{hladina významnosti} & 1.0\% & 2.5\% & 5.0\% & 10.0\%\\
\hline
\bf{kritická hodnota} & -3.43 & -3.12 & -2.86 & -2.57\\
\hline
\end{tabular}
\caption{Kritické hodnoty Dickey-Fullerova testu (bez lineárního trendu)}
\end{center}
\end{table}

\subsection{Rozšířený Dickey-Fullerův test}

Test jednotkového kořene je možné provádět také pro modely s komplikovanější dynamikou. $\{\Delta y_i\}$ 
v modelu (18.27) můžeme rozšířit přidáním členů $\Delta y_{t - h}$ na
\begin{equation}
\Delta y_t = \alpha + \theta y_{t-1} + \gamma_1 \Delta y_{t-1} + ... + \gamma_p \Delta y_{t - p} + e_t,
\end{equation}
kde $|\gamma_h| < 1$. To má za následek, že pro $H_0: \theta = 0$ sleduje $\{\Delta y_t\}$ stabilní AR(p) model a 
pro $H_1: \theta < 0$ sleduje $\{y_t\}$ stabilní AR(p + 1) model. 
Protože jsme do původního modelu připadli členy $\Delta y_{t - h}$ 
nazýváme tento test rozšířeným Dickey-Fullerovým testem. Logika 
tohoto testu je shodná s logikou původního Dickey-Fullerového 
testu. Jediným rozdílem jsou kritické hodnoty, které jsou dány 
tabulkou (18.2).

Přidání časově posunutých veličin do (18.29) je motivované snahou zbavit se sériové korelace v $\Delta y$. 
Čím více veličin přídáme, tím více ztratíme počátečních pozorování. Jestliže zahrneme příliš mnoho 
veličin, má výsledný malý vzorek za následek sníženou sílu testu. Pokud však přidáme příliš málo veličin, 
může být výsledek testu zavádějící, protože kritické hodnoty rozšířeného Dickey-Fullerova testu 
předpokládají dynamicky kompletní model. Počet časově posunutých veličin zahrnutých do modelu je často, 
kromě velikosti náhodného výběru, ovlivněn také frekvencí dat. Pro roční data jsou obvykle dostatečné jedna 
až dvě veličiny; v případě měsíčních dat zahrnujeme zpravidla dvanáct veličin.

\subsection{Lineární časový trend}

Stacionární proces s lineárním trendem\footnote{Jedná se o proces, jehož střední hodnota 
sleduje lineární trend, a který se po odstranění tohoto trendu 
stává I(0) procesem.} může být zaměněn za proces s jednotkovým 
kořenem, pokud tento trend nezahrneme do 
Dickey-Fullerova regresního modelu. Základní rovnici tak musíme upravit do tvaru
\begin{equation}
\Delta y_t = \alpha + \delta t + \theta y_{t - 1} + e_t,
\end{equation}
kde opět $H_0: \theta = 0$ a $H_1: \theta < 0$. Pokud zahrneme trend 
do regresního modelu, kritické hodnoty 
Dickey-Fullerova testu se změní, protože odstranění trendu udělá z procesu s jednotkovým kořenem proces, 
který více připomíná I(0) proces. Proto potřebujeme vyšší hodnoty $t$ statistiky k zamítnutí $H_0$.

\begin{table}
\begin{center}
\begin{tabular}{| l | c | c | c | c |}
\hline
\bf{hladina významnosti} & 1.0\% & 2.5\% & 5.0\% & 10.0\%\\
\hline
\bf{kritická hodnota} & -3.96 & -3.66 & -3.41 & -3.12\\
\hline
\end{tabular}
\caption{Kritické hodnoty Dickey-Fullerova testu (zahrnutý lineární trend)}
\end{center}
\end{table}

Mohlo by se zdát, že pro rozhodnutí o zahrnutí či nezahrnutí lineárního trendu do Dickey-Fullerova modelu 
stačí porovnat $t$ statistiku trendu s kritickou hodnotou standardního normálního $t$ rozdělení. Bohužel tato 
$t$ statistika nesleduje asymptoticky normální rozdělení s 
vyjímkou situace, kdy $|\rho| < 1$. Ačkoliv je asymptotické rozdělení této 
$t$ statistiky známé, je praxi používáno pouze vyjímečně. Při rozhodování, zda-li zařadit trend do 
Dickey-Fullerova regresního modelu či nikoliv, se nejčastěji spoléhá na intuici doplněnou o grafy časových řad.

\section{Zdánlivá regrese (spurious regression)}

Pojem zdánlivá korelace se používá pro popis situace, kdy jsou dvě náhodné veličiny společně korelované 
skrze třetí náhodnou veličinou. Pokud tuto třetí veličinu zafixujeme, je korelace mezi 
zkoumanými veličinami nulová. K tomu samému může dojít také v případě časových řad v souvislosti s I(0) veličinami.

Zdánlivou korelaci je možné nalézt i mezi časovými řadami, které mají rostoucí nebo klesající trend. 
Tento problém lze pak vyřešit zahrnutím časového trendu do regresního modelu.

Uvažujme dvě I(1) časové řady, které nesledují trend. Jednoduchý regresní model aplikovaný na tyto dvě řady 
pak často vede k významné $t$ statistice. Konkrétně uvažujme časové řady
\begin{equation}
x_t = x_{t - 1} + a_t
\end{equation}
a
\begin{equation}
y_t = y_{t - 1} + e_t,
\end{equation}
kde $t = 1, 2, ...$ a $\{a_t\}$ a $\{e_t\}$ jsou nezávislé chybové složky, které sledují shodné 
pravděpodobnostní rozdělení s nulovou střední hodnotou a rozptylem $\sigma_a^2$ a $\sigma_e^2$. Výše uvedené 
předpoklady implikují, že $\{a_t\}$ a $\{e_t\}$ jsou taktéž vzájemně nezávislé. Granger a Newbold (1974) 
ukázali, že ačkoliv jsou $y_t$ a $x_t$ nezávislé, regresní model mezi $y_t$ a $x_t$ indikuje statistiky 
významnou závislost.

Uvažujme jednoduchý regresní model
\begin{equation}
y_t + \beta_0 + \beta_1 x_t + u_t.
\end{equation}
Aby $t$ statistika odhadu $\hat{\beta}_t$ měla za předpokladu dostatečně velkého výběru přibližně 
normální rozdělení, mít mít chyba $\{u_t\}$ nulovou střední hodnotu a nesmí být sériově korelovaná. Nicméně 
protože za předpokladu $H_0: \beta_1 = 0$ máme $y_t = \beta_0 + u_t$, a protože $\{y_t\}$ je náhodná procházka s $y_0 = 0$, 
je rovnice (18.32) za předpokladu $H_0$ splněna pouze tehdy, jestliže $\beta_0 = 0$ a $u_i = y_i = \sum_{j = 1}^i 
e_j$.

Zahrnutí časového trendu do (18.32) vede ke stejným závěrům. Jestliže $y_t$ a $x_t$ jsou náhodné procházky s driftem a 
tento časový trend není zohledněn, má problém zdánlivé regrese zásadnější charakter. Stejné kvalitativní 
závěry také platí, jsou-li $\{a_t\}$ a $\{e_t\}$ obecné I(0) procesy spíše než i.i.d. posloupnosti.

Kromě toho, že obvyklá $t$ statistika limitně nesleduje standardní normální rozdělení\footnote{Ve skutečnosti 
$t$ statistika roste k nekone4nu s tím, jak se velikost náhodného vzorku $n$ blíží k nekonečnu.}, je také 
chování $R^2$ nestandardní. Namísto toho, aby $R^2$ mělo řádně definované $plim$, konverguje k náhodné 
veličině; $R^2$ je s vysokou pravděpodobností vysoké, i když jsou $\{y_t\}$ a $\{x_t\}$ nezávislé časové řady.

Identické závěry lze vznést také v případě vícero vysvětlujících veličin. Jestliže $\{y_t\}$ je I(1) a 
alespoň některé z vysvětlujích veličin jsou I(1), může se jednat o zdánlivou regresi.

\section{Kointegrace a modely korekce chyb (error correction models)}

\subsection{Kointegrace}

Jestliže $\{y_t: t = 0, 1, ...\}$ a $\{x_t: t = 0, 1, ...\}$ jsou dva I(1) procesy, pak obecně $y_t - \beta x_t$ je 
I(1) proces. Nicméně je možné, že pro určité $\beta \ne 0$ je $y_t - \beta x_t$ proces I(0), tj. proces s 
konstatní střední hodnotou, konstatním rozptylem a autokorelací, která závisí pouze na "časovém" posunu
mezi dvěma veličinami, a je asymptoticky nekorelovaný. Pokud takové $\beta$ existuje, říkáme, že $x$ a $y$ jsou 
kointegrované a $\beta$ nazýváme kointegračním parametrem.

Pro ilustraci uvažujme $\beta = 1$, $y_0 = x_0 = 0$ a $y_t = y_{t - 1} + r_t$ a $x_t = x_{t - 1} + v_t$, kde $\{r_t\}$ 
a $\{v_t\}$ jsou dva I(0) procesy s nulovou střední hodnotou. $y_t$ a $x_t$ mají tendenci se "potulovat" bez toho, 
aniž by se pravidelně vracely ke své počáteční nulové hodnotě. Naproti tomu $y_t - x_t$ je I(0) proces, který 
má nulovou střední hodnotu a k počáteční nulové hodnotě se vrací s jistou pravidelností\footnote{Pro ilustraci
ekonomického významu kointegrace uvažujme 3M EURIBOR a výnos pětiletého státního dluhopisu. V průběhu let se 
úroveň úrokových sazeb může značně lišit, nicméně rozdíl mezi oběma sazbami by měl vykazovat určitou stabilitu.}.

Pokud bychom měli k dispozici hodnotu parametru $\beta$, pak by testování kointegrace bylo relativně snadné - 
definovali bychom novou veličinu $s_t = y_t - \beta x_t$, na níž bychom aplikovali rozšířený Dickey-Fullerův 
test. Kointegrační parametr $\beta$ je však neznámý, a proto ho musíme nejprve odhadnout. Lze dokázat, že OLS 
odhad $\hat{\beta}$ z regrese
\begin{equation}
y_t = \hat{\alpha} + \hat{\beta} x_t
\end{equation}
je konzistentním odhadem kointegračního parametru $\beta$. Hlavním problémem však je, že při platné nulové hypotéze 
nejsou $y_t$ a $x_t$ kointegrované, což znamená, že se pro $H_0$ dopouštíme zdánlivé regrese. Naštěstí je 
možné tabelovat kritické hodnoty rozšířeného Dickey-Fullera testu pro parametr $\beta$ aplikovaného na rezidua 
z (18.33), tj. $\hat{u}_i = y_i - \hat{\alpha} - \hat{\beta} x_t$. Tento test nazýváme Engle-Grangerovým testem.

\begin{table}
\begin{center}
\begin{tabular}{| l | c | c | c | c |}
\hline
\bf{hladina významnosti} & 1.0\% & 2.5\% & 5.0\% & 10.0\%\\
\hline
\bf{kritická hodnota} & -3.90 & -3.59 & -3.34 & -3.04\\
\hline
\end{tabular}
\caption{Kritické hodnoty pro kointegrační test (bez lineárního trendu)}
\end{center}
\end{table}

V základním pojetí testu provádíme regresi $\Delta \hat{u}_t$ na $\hat{u}_{t - 1}$ a porovnáme $t$ statistiku pro 
$\hat{u}_{t - 1}$ s kritickými hodnotami v tabulce (18.3). Jestliže se $t$ statistika nachází pod tabelovanými 
kritickými hodnotami, máme důkaz, že pro určité $\beta$ je $y_t - \beta x_t$ proces I(0), tj. že $y_t$ a $x_t$ 
jsou kointegravané. Stejně jako v případě Dickey-Fullerova testu, můžeme také Engle-Grangerův test rozšířit 
o zpožděné veličiny $\Delta \hat{u}_t$, abychom tak zohlednili případnou sériovou korelaci. Jestliže $y_t$ a 
$x_t$ nejsou kointegrované, je regrese $y_t$ na $x_t$ zdánlivá a nemá žádný interpretovatelný význam.

Striktní definice kointegrace vyžaduje, aby $y_t - \beta x_t$ byl proces I(0) bez trendu. Abychom ilustrovali, co to 
znamená, uvažujme $y_t = \delta t + g_i$ a $x_t = \lambda t + h_t$, kde $\{g_t\}$ a $\{h_t\}$ jsou I(1) procesy. 
Jestliže jsou $y_t$ a $x_t$ kointegrované, pak musí existovat $\beta$ takové, že $g_t - \beta h_t$ je proces I(0). 
Pak ale
\begin{equation}
y_t - \beta x_t = (\delta - \beta \lambda)t + (g_t - \beta_t),
\end{equation}
což je obecně stacionární proces s 
trendem. Striktní forma kointegrace však existenci trendu nepřipouští, což implikuje $\delta = \beta \lambda$. 
Pro proces I(1) je tak možné, že stochastické části, tj. $g_t$ a $h_t$, jsou kointegrované, avšak parameter 
$\beta$, který zajišťuje, že $g_t - \beta h_t$ je I(0) proces, neeliminuje lineární časový trend. Tento 
problém lze vyřešit tím, že rozšířený Dickey-Fuller test aplikujeme na rezidua $\hat{u}_t$ regrese
\begin{equation}
\hat{y}_i = \hat{\alpha} + \hat{\eta} t + \hat{\beta} x_t.
\end{equation}

\begin{table}
\begin{center}
\begin{tabular}{| l | c | c | c | c |}
\hline
\bf{hladina významnosti} & 1.0\% & 2.5\% & 5.0\% & 10.0\%\\
\hline
\bf{kritická hodnota} & -4.32 & -4.03 & -3.78 & -3.50\\
\hline
\end{tabular}
\caption{Kritické hodnoty pro kointegrační test (zahrnutý lineární trend)}
\end{center}
\end{table}

Jestliže $y_t$ a $x_t$ jsou kointegrované I(1) procesy, pak platí

\begin{equation}
y_t = \alpha + \beta x_t + u_t,
\end{equation}

kde $u_t$ je I(0) proces s nulovou střední hodnotou. Obecně platí, že $\{u_i\}$ je sériově korelované, což 
však nemá vliv na konzistentnost odhadů. Nicméně protože $x_t$ je I(1) proces, obvyklé postupy pro testování 
parametrů nemusí být aplikovatelné.

Kointegrace mezi $y_t$ a $x_t$ nám neříká nic o vztahu mezi $\{x_t\}$ a $\{u_t\}$ - ty mohou být libovolně 
korelované. Dále, kromě požadavku, že $\{u_i\}$ je I(0) proces, kointegrace mezi $y_t$ a $x_t$ nevylučuje 
sériovou závislost $\{u_i\}$.

Hlavní nedostatek (18.36), který komplikuje testování parametrů nejvíce, tj. neexistenci striktní exogenity 
$\{x_i\}$, lze snadno odstranit. Protože $x_t$ je I(1) proces, sktriktní exogenita znamená, že $u_t$ je 
nekorelovan0 s $\Delta x_s$ pro libovolné $s$ a $t$. To lze, alespoň přibližně, zajistit s pomocí nových 
chybových veličin, kdy $u_t$ vyjádříme jako funkci $\Delta x_s$ pro všechna $s$ blízká $t$. Pro ilustraci uvažujme

\begin{equation}
u_t = \eta + \phi_0 \Delta x_t + \phi_1 \Delta x_{t - 1} + \phi_2 \Delta x_{t - 2} + \gamma_1 \Delta x_{t + 1} + \gamma_2 \Delta x_{t + 2} + e_t,
\end{equation}

kde $e_t$ je nekorelované s každým $\Delta x_s$ ve výše uvedené rovnici. Implicitně také předpokládáme, že $e_t$ je nekorelované s 
dalšími posuny $\Delta x_s$. Protože $e_t$ a $\Delta x_s$ jsou I(0) procesy, blíží se korelace mezi $e_i$ a 
$\Delta x_s$ nule s tím, jak narůstá $|s - t|$. Jestliže dosadíme (18.37) do (18.36), získáme
\begin{equation}
y_t = \alpha_0 + \beta x_t + \phi_0 \Delta x_t + \phi_1 \Delta x_{t - 1} + \phi_2 \Delta x_{t - 2} + \gamma_1 \Delta 
x_{t + 1} + \gamma_2 \Delta x_{t + 2} + e_t.
\end{equation}
Z konstrukce (18.38) vyplývá, že $x_t$ je v rámci této rovnice striktně exogenní. To je nezbytná podmínka pro 
získání přibližně normální $t$ statistiky pro $\hat{\beta}$. Pokud je $u_t$ nekorelované se všemi $\Delta 
x_s, s \ne t$, pak můžeme rovnici dále zredukovat na

\begin{equation}
y_t = \alpha_0 + \beta x_t + \phi_0 \Delta x_t + e_t,
\end{equation}

kde $\Delta x_t$ řeší jakokoliv současnou endogenitu mezi $x_t$ a $u_t$. Zda-li a kolik posunutých veličin 
vyloučit ze (18.37) je čistě empirická záležitost.

Jediným problémem (18.37) tak zůstává případná sériová korelace $\{e_t\}$, což lze např. vyřešit použitím 
robustní směrodatné odchylky.

Pojem kointegrace lze také rozšířit na více než dva procesy. Nicméně interpretace a případné odhady parametrů se
stávají mnohem složitějšími.

\section{Modely korekce chyb (error correction models)}

Jestliže $y_t$ a $x_t$ jsou I(1) procesy a nejsou kointegrované, můžeme se pokusit odhadnout dynamický model 
jejich prvních deferencí. Jako příklad uvažujme rovnici

\begin{equation}
\Delta y_t = \alpha_0 + \alpha_1 \Delta y_{t - 1} + \gamma_0 \Delta x_t + \gamma_1 \Delta x_{t - 1} + u_t,
\end{equation}

kde $u_i$ má nulovou střední hodnotu pro dané $\Delta x_t$, $\Delta y_{t - 1}$, $\Delta x_{t - 1}$ a jejich další 
posuny.

Pokud jsou $y_t$ a $x_t$ kointegrované parametrem $\beta$, můžeme definovat I(0) proces $s_t = y_t - \beta 
x_t$; pro zjednodušení předpokládejme, že $s_t$ má nulovou střední hodnotu. Proces $s_t$ přidáme do výše 
uvedené rovnice, čímž získáme

\begin{multline}
\Delta y_t = \alpha_0 + \alpha_1 \Delta y_{t - 1} + \gamma_0 \Delta x_t + \gamma_1 \Delta x_{t - 1} + \delta s_{t - 1} 
+ u_t\\
= \alpha_0 + \alpha_1 \Delta y_{t - 1} + \gamma_0 \Delta x_t + \gamma_1 \Delta x_{t - 1} + \delta(y_{t - 1} - \beta 
x_{t - 1}) + u_t.
\end{multline}

Člen $\delta(y_{t - 1} - \beta x_{x - 1})$ se nazývá korekce chyby (error correction term) a (18.41) se nazývá 
modelem korekce chyby.

Pro zjednodušení uvažujme model korekce chyby ve tvaru

\begin{equation}
\Delta y_t = \alpha_0 + \gamma_0 \Delta x_t + \delta(y_{t - 1} - \beta x_{t - 1}) + u_t,
\end{equation}

kde $\delta < 0$. Jestliže $y_{t - 1} > \beta x_{t - 1}$, pak hladina $y$ v předchozí periodě překročila 
rovnovážnou úroveň. Protože $\delta <0$, stahuje v následujících krocích model $y$ zpět k rovnovážné 
úrovni. Mechanismus pochopitelně funguje také obráceně.

V praxi musíme kontegrační faktor odhadnout. V tomto případě $s_{t - 1}$ nahradíme $\hat{s}_{t - 1} = y_{t - 1} 
- \hat{\beta} x_{t - 1}$, kde $\hat{\beta}$ je odhadem kointegračního faktoru $\beta$. Postup, ve kterém 
nahrazujeme kointegrační faktor $\beta$ jeho odhadem $\hat{\beta}$, se nazývá Engle-Grangerovou dvoustupňovou 
procedurou.

\section{Predikce}

Označme informaci, kterou jsme schopni pozorovat v čase $t$, jako $I_t$. Tato informace zahrnuje nejen $y_t$, ale 
také všechny předchozí hodnoty $y$ stejně jako hodnoty všech ostatních velčin v čase $t$ a před ním.

Označme predikci hodnoty $y_{t + 1}$ v čase $t$ jako $f_t$. Chybu predikce $f_t$ definujme jako $e_{t + 1} = y_{t + 
1} - f_t$. Pro posouzení kvality predikce je třeba definovat tzv. ztrátovou funkci. V případě predikce 
následující periody se nejčastěji používá druhá mocnina $e_{t + 1}$, kdy se snažíme minimalizovat

\begin{equation}
E[e^2_{t + 1} | I_t] = E[(y_{t + 1} - f_t) ^ 2 | I_t].
\end{equation}

Nejjednodušší prediktivní metodou je tzv. martingale. Proces $\{y_t\}$ je martingale, jestliže $E[y_{t + 1} | y_t, 
y_{t - 1}, ..., y_0] = y_t$ pro všechna $t \ge 0$.

O něco složitější příklad prediktivní metody je

\begin{equation}
E[y_{t + 1} | I_t] = \alpha y_t + \alpha (1 - \alpha) y_{t - 1} + ... + \alpha (1 - \alpha)^t y_0,
\end{equation}
kde $0 < \alpha < 1$. Tato metoda se nazývá metodou exponenciálního vyhlazování. Jestliže $f_0 = y_0$, pak pro 
$t \ge 1$ platí
\begin{equation}
f_t = \alpha y_t + (1 - \alpha) f_{t - 1}.
\end{equation}
Predikce hodnoty $y_{t + 1}$ je tedy váženým průměrem $y_t$ a minulé predikce $f_{t - 1}$ učiněné v čase $t - 1$.

V případě predikce přes několik period používáme notaci $f_{t,h}$, která označuje odhad hodnoty $y_{t + 
h}$ učiněný v čase $t$.

\subsection{Typy prediktivních modelů}

Uvažume jednoduchý model
\begin{equation}
y_t = \beta_0 + \beta_1 z_t + u_t
\end{equation}
a předpokládejme, že parametry $\beta_0$ a $\beta_1$ jsou známy. Pak platí
\begin{equation}
E[y_{t + 1}|I_t] = \beta_0 + \beta_1 z_{t + 1},
\end{equation}
kde $I_t$ obsahuje $z_{t + 1}, y_t, z_t, ..., y_1, z_1$. Jedná se tedy o tzv. podmíněnou predikci, protože je 
podmíněná znalostí hodnoty $z$ v čase $t + 1$. Nicméně znalost $z_{t + 1}$ v čase $t$ není obvyklá. Další 
problém s (18.46) jako modelem predikce je, že $E[u_{t + 1}|I_t] = 0$ vyžaduje nulovou sériovou korelaci $\{u_t\}$, 
což v praxi mnohdy není splněno. (18.46) tak musíme upravit do tvaru
\begin{equation}
E[y_{t + 1}|I_t] = \beta_0 + \beta_1 E[z_{t + 1}|I_t],
\end{equation}
tj. musíme nejprve odhadnout hodnotu $z$ v čase $t + 1$. Z tohoto důvodu hovoříme o tzv. nepodmíněné predikci.

Z výše uvedeného je zřejmé, že větší relevanci má model, který je založený na zpožděný veličinách, tj. například
\begin{equation}
y_t = \delta_0 + \alpha_1 y_{t - 1} + \gamma_1 z_{t - 1} + u_t,
\end{equation}
kde $E[u_t | I_{t - 1}] = 0$.

\subsection{Jednoperiodová predikce}

Uvažujme predikci hodnoty $y_{n+1}$ ve tvaru
\begin{equation}
\hat{f}_n = \hat{\delta}_0 + \hat{\alpha}_1 y_n + \hat{\gamma}_1 z_n
\end{equation}
a její chybu
\begin{equation}
\hat{e}_{n + 1} = y_{n + 1} - \hat{f}_n.
\end{equation}
Predikci $\hat{f}_n$ obvykle nazýváme bodovou predikcí. Vedle toho je možné zkonstruovat také intervalovou 
predikci, což je interval odhadu, kterým jsme se zabývali v kapitole 6.4. I když model nesplňuje 
klasické předpoklady lineárního modelu\footnote{Například pokud zahrnuje zpožděné veličiny, jako je tomu v případě 
(18.50).}, je intervalová predikce stále přibližně platná za předpokladu, že $u_i$ je pro dané $I_{t - 1}$ 
normálně rozdělené s nulovou střední hodnotou a konstantním rozptylem. Nechť $se(\hat{f}_n)$ je standardní 
směrodatná odchylka predikce a $\hat{\sigma}$ je směrodatná odchylka regrese\footnote{V kapitole 6.4 jsme 
$\hat{f}_n$ a $se(\hat{f}_n)$ získali jako průsečík a jeho směrodatnou odchylku v regresi $y_t$ na $y_{t - 1} - 
y_n$ a $z_{t - 1} - z_n$, kde $t = 1, 2, ..., n$.}. Pak
\begin{equation}
se(\hat{e}_{n + 1}) = \sqrt{[se(\hat{f})]^2 + \hat{\sigma}^2}
\end{equation}
a 95\% inteval predikce je
\begin{equation}
\hat{f}_n \pm 1.96 ~ se(\hat{e}_{n + 1}).
\end{equation}

Položme si otázku, zda-li po fixaci minulých hodnot $y$ v (18.49) vysvětlují minulé hodnoty $z$ 
vývoj $y$. V takovéto situaci říkáme, že existuje Grangera kauzalita mezi $z$ a $y$, jestliže
\begin{equation}
E[y_t|I_{t - 1}] \ne E[y_t|J_{t - 1}],
\end{equation}
kde $I_{t - 1}$ obsahuje informace o minulých hodnotách $y$ a $z$, kdežto $J_{t - 1}$ obsahuje informace pouze o 
minulých hodnotách $y$. Po specifikaci modelu a rozhodnutí, kolik zpožděných veličin $y$ má být zahrnuto v 
$E[y_t | y_{t - 1}, y_{t - 2}, ...]$, můžeme relativně snadno testovat nulovou hypotézu, neexistuje 
Grangerova kauzalita mezi $z$ a $y$. Pro ilustraci uvažujme, že $E[y_t | y_{t - 1}, y_{t - 2}, ...]$ závisí pouze na třech 
zpožděných veličinách, tj.
\begin{equation}
y_t = \delta_0 + \alpha_1 y_{t - 1} + \alpha_2 y_{t - 2} + \alpha_3 y_{t - 3} + u_t,
\end{equation}
kde $E[u_t | y_{t - 1}, y_{t - 2}, ...] = 0$. Pokud je nulová hypotéza platná, pak by jakákoliv zpožděná 
veličina $z$ přidaná do modelu měla být statisticky nevýznamná.

Jak ale v praxi určíme, kolik zpožděných veličin $y$ a $z$ má být zahrnuto do modelu? Nejprve začneme s 
odhadem autoregresivního modelu pro $y$ a pomocí $f$ a $F$ testů rozhodneme o počtu zpožděných vysvětlujících 
veličin. Po té, co je specifikován autoregresní model, testujeme přidání zpožděných veličin $z$. Volba 
počtu zpoždění veličiny $z$ je méně důležitá - jestliže neexistuje Grangerova kauzalita mezi $z$ a $y$, pak je 
libovolný soubor zpožděných veličin $z$ přidaný do modelu statisticky nevýznamný.

Uvažujme třetí časovou řadu $\{w_i\}$. říkáme, že mezi $z$ a $y$ existuje Grangera kauzalita podmíněně $w$, jestliže je 
splněno (18.54) s tím rozdílem, že $I_{t - 1}$ obsahuje informace o minulých hodnotách $y, z$ a $w$, kdežto 
$J_{t - 1}$ obsahuje informace pouze o minulých hodnotách $y$ a $w$. Je možné, že mezi $z$ a $y$ existuje Grangerova 
kauzalita, avšak mezi $z$ a $y$ neexistuje Grangerova kauzalita podmíněně $w$. Nulovou hypotézu, že 
mezi $z$ a $y$ neexistuje podmíněná Grangerova kauzalita, lze testovat pomocí statistické významnosti 
zpožděných veličin $z$ v modelu pro $y$, který kromě zpožděných veličin $z$ obsahuje také zpožděné veličiny $w$.

\subsection{Porovnání jednoperiodových predikcí}

Při porovnávání prediktivních metod se často používá ukazatel RMSE (root mean squared error)
\begin{equation}
RMSE = \sqrt{\frac{1}{m}\sum_{h = 0}^{m - 1} \hat{e}_{n + h + 1}^2},
\end{equation}
což je v podstatě výběrová směrodatná odchylka chyb predikce bez opravy na počet stupňů volnosti. Druhým 
relativně rozšířeným ukazatelem je MAE (mean absolute error)
\begin{equation}
MAE = \frac{1}{m}\sum_{h = 0}^{m - 1} |\hat{e}_{n + h + 1}|.
\end{equation}
