\chapter{Úvod}

\section{Příklady časových řad}

Časovou řadu (time series) lze chápat jako soubor pozorování, které jsou indexovány časem. V praxi je možné se setkat s nejrůznějšími typy časových řad. Nejčastěji se jedná o časové řady, které představují vývoj určité ekonomické veličiny v čase. Příkladem takové časové řady může být vývoj indexu cen akcií, hrubého domácí produktu či nezaměstnanosti. Další zdrojem časových řad jsou přírodní vědy jako je meteorologie (např. vývoj teploty či srážek) nebo geofyzika (např. monitoring seismických otřesů). Existují však i jiné obory lidské činnosti, které se zabývají analýzou časových řad jako jsou marketing (např. vývoj prodejů), demografie (např. vývoj porodnosti a úmrtnosti) nebo teorie kontroly (z analýzy časové řady kvality výstupu lze např. usuzovat na nutnost seřízení obráběcích strojů).

\section{Pojmosloví}

Jak již bylo zmíněno, časová řada je uspořádaným souborem pozorování. Typickou vlastností časové řady je to, že jednotlivá pozorování nejsou vzájemně nezávislá, a proto při jejich analýze musíme vzít v úvahu jeji pořadí.

Jestliže jsme schopni časovou řadu predikovat zcela přesně, jedná se o tzv. deterministickou časovou řadu. Ve většině případů však přesné predikce nejsme schopni. V tomto případě pak hovoříme o tzv. stochastické časové řadě - její budoucí hodnoty jsou minulými pozorováním predikovány pouze částečně.

\section{Cíle analýzy časových řad}

\subsection{Popis časové řady}

Prvním krokem při analýze časových řad je obvykle jejich grafické znázornění s cílem získat základní představu o jejich průběhu a případném časovém trendu či sezónnosti. Z grafu časové řady lze také poměrně snadno určit odlehlá pozorování, které mohou indikovat potřebu použití robustních statistických metod, a tzv. bodů zlomu v trendu časové řady. Grafické znázornění časové řady doplněné o její základní statistiky jako jsou průměrná hodnota, medián, maximální a minimální hodnota či směrodatná odchylka by měla být výchozím bodem pro následnou analýzu časové řady.

\subsection{Analýza časové řady}

Cílem analýzy časové řady je vysvětlení jejích změn pomocí jiné časové řady. Klasickým příkladem je snaha vysvětlit vývoj nezaměstnanosti pomocí změn hrubého domácího produktu. Tímto způsobem se snažíme o hlubší pochopení mechanismů, které stojí za touto časovou řadou.

\subsection{Predikce}

Velmi často také chceme být schopni predikovat vývoj určité časové řady. Tyto snahy hrají ústřední roli ekonomii, která inicializovala vznik nového oboru statistiky, tzv. ekonometrii, pod kterou časové řady spadají.