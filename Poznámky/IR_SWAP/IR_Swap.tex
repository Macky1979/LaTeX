\documentclass[a4paper]{book}
\usepackage[czech]{babel}
\usepackage[utf8]{inputenc}
\usepackage{pstricks}
\usepackage{amsmath}

\setlength{\unitlength}{1.0mm}

\begin{document}

\chapter{Úrokový swap}

\section{Teoretický úvod}

Úrokový swap je obchodem, kdy dochází k výměně peněžních toků mezi zúčastněnými stranami. Klasický úrokový swap, kdy dochází ke směně plovoucí úrokové sazby za fixní, lze modelovat pomocí imaginárního floatového a fixního dluhopisu. Vzhledem k tomu, že úrokový swap lze rozložit na jednodušší instrumenty, jedná se o syntetický instrument.\\

\noindent \textbf{Příklad:} Uvažujme tříletý úrokový swap s podkladovým kapitálem 100 MCZK. Jedna ze smluvních stran platí dvakrát ročně úroky odpovídající aktuální sazbě PRIBOR a zároveň obdrží od druhé strany jednou ročně fixní sazbu 4\% p.a. Úrokové platby jsou vztaženy k podkladovému kapitálu 100 MCZK.
\begin{center}
	\begin{pspicture}(0,0)(8,7.5)
		\rput(4,0.0){Rozklad úrokového swapu: (a) floatový dluhopis, (b) fixní dluhopis}
		\psline[arrows=->](0.5,4)(7.5,4)
		\rput(7.3,3.7){$t$}
		\rput(0.2,3.7){\tiny 0}
		\psline[arrows=>, linewidth=0.1mm, linestyle=dashed](0.5,7)(0.5,4)
		\rput(0.8,6.8){(a)}
		\psline[arrows=>, linewidth=0.1mm](0.5,1)(0.5,4)
		\rput(0.8,1.2){(b)}
		\psline[arrows=>, linewidth=0.1mm, linestyle=dashed](1.5,3)(1.5,4)
		\rput(2.2,3.7){\tiny 1Y}
		\psline[arrows=>, linewidth=0.1mm, linestyle=dashed](2.5,3)(2.5,4)
		\psline[arrows=>, linewidth=0.1mm](2.5,6)(2.5,4)
		\psline[arrows=>, linewidth=0.1mm, linestyle=dashed](3.5,3)(3.5,4)
		\rput(4.2,3.7){\tiny 2Y}
		\psline[arrows=>, linewidth=0.1mm, linestyle=dashed](4.5,3)(4.5,4)
		\psline[arrows=>, linewidth=0.1mm](4.5,6)(4.5,4)
		\psline[arrows=>, linewidth=0.1mm, linestyle=dashed](5.5,3)(5.5,4)
		\rput(6.1,3.7){\tiny 3Y}
		\psline[arrows=>, linewidth=0.1mm](6.5,7)(6.5,4)
		\psline[arrows=>, linewidth=0.1mm, linestyle=dashed](6.5,1)(6.5,4)
		\psline[arrows=>, linewidth=0.1mm, linestyle=dashed](6.4,3)(6.4,4)
		\psline[arrows=>, linewidth=0.1mm](6.4,6)(6.4,4)
	\end{pspicture}
\end{center}
Tento swap lze tedy "rozložit" na dva imaginární dluhopisy. První z nich je fixní s kupónovou sazbou 4\% p.a. a roční výplatou kupónů. Druhý dluhopis má kupónovou sazbu definovanou sazbou PRIBOR a půlroční výplatu kupónů. Tu část úrokového swapu, kterou představuje cash-flow generované fixním dluhopisem, nazýváme fixní nohou. Cash-flow generované imaginárním floatovým dluhopisem pak nazýváme floatovou nohou. Jednotlivé platby, je-li to možné, se vzájemně kompenzují a vyplácí se pouze rozdíl mezi oběma peněžními toky. Z toho principu mimojiné vyplývá, že podkladový kapitál se na začátku a na konci kontraktu nesměňuje. Z hlediska oceňování není kompenzace cash-flow důležitá.

\section{Matematické pozadí}

\subsection{Hodnota úrokového swapu v čase $t_0 = 0$}

Uvažujme jednoroční úrokový swap s podkladovým kapitálem $A$, v rámci kterého dochází k výměně floatové sazby $F$ za fixní sazbu $R$. Předpokládejme, že rok má 360 dní a definujme časový interval mezi $t_2$ a $t_1$ v ročním vyjádření jako 
\begin{equation*}
\alpha_{t_1, t_2} = \frac{t_2 - t_1}{360}
\end{equation*}
Dále definujme diskontní faktor $DF_{t_1, t_2}$ pro časové období $t_1$ až $t_2$ jako
\begin{equation*}
DF_{t_1, t_2} = \frac{1}{1 + F_{t_1, t_2} \alpha_{t_1, t_2}}
\end{equation*}
kde $F_{t_1, t_2}$ je forwardová úroková míra v čase $t_0$ platná pro časový interval $t_1$ až $t_2$. Předpokládejme, že pro forwardovou sazbu platí
\begin{equation*}
(1 + F_{t_0, t_2} \alpha_{t_0, t_2})  = (1 + F_{t_0, t_1} \alpha_{t_0, t_1})(1 + F_{t_1, t_2} \alpha_{t_1, t_2})
\end{equation*}

\subsubsection{Fixní noha}

Výplata $L_{fix}^{1Y}$ z fixní nohy na konci roku je definována jako
\begin{equation*}
L_{fix}^{1Y} = A R_{1Y} \alpha_{0,1Y}
\end{equation*}
kde $R_{1Y}$ představuje výši fixního kupónu jednoročního swapu. Současná hodnota fixní nohy je tedy definována jako
\begin{equation*}
PV(L_{fix}^{1Y}) = DF_{0, 1Y} L_{fix}^1 =  DF_{0, 1Y} A R_{1Y} \alpha_{0,1Y}
\end{equation*}

Obecný vzorec pro současnou hodnotu fixní nohy $n$-letého úrokového swapu v době uzavření kontraktu je
\begin{equation}
AR_{nY}(DF_{0,t_1} \alpha_{0,t_1} + DF_{0, t_2} \alpha_{t_1,t_2} + ... + DF_{0, t_n} \alpha_{t_{n-1},t_n})
\end{equation}

\subsubsection{Floatová noha}

Floatová noha se skládá ze dvou cash-flow. První cash-flow $L_{float}^{6M}$ generováno po půl roce trvání kontraktu je rovno
\begin{equation*}
L_{float}^{6M} = A F_{0,6M} \alpha_{0,6M}
\end{equation*}
Floatovou sazbu platnou za půl roku tedy odhadujeme pomocí forwardové sazby $F_{0,6M}$. Současná hodnota tohoto cash-flow je
\begin{equation*}
PV(L_{float}^{6M}) = DF_{0,6M} L_{float}^{6M} = DF_{0,6M} A F_{0,6M} \alpha_{0,6M} =
\end{equation*}
\begin{equation*}
= DF_{0,6M} A \bigg( \frac{DF_{0,0}}{DF_{0,6M}} - 1 \bigg) = DF_{0,6M} A \bigg( \frac{1}{DF_{0,6M}} - 1 \bigg) = A (1 - DF_{0,6M})
\end{equation*}
Obdobně současná hodnota druhého cash-flow $L_{float}^{1Y}$ generovaného na konci kontraktu je rovna
\begin{equation*}
PV(L_{float}^{1Y}) = DF_{0,1Y} A \bigg( \frac{DF_{0,6M}}{DF_{0,1Y}} - 1 \bigg) = A (DF_{0,6M} - DF_{0,1Y})
\end{equation*}
Současná hodnota celé floatové nohy je tedy
\begin{equation*}
PV(L_{float}^{6M}) + PV(L_{float}^{1Y}) = A (1 - DF_{0,6M}) + A (DF_{0,6M} - DF_{0,1Y}) = A (1 - DF_{0,1Y})
\end{equation*}

Obecný vzorec pro současnou hodnotu floatové nohy $n$-letého úrokového swapu v době uzavření kontraktu je
\begin{equation}
A(1 - DF_{0,nY})
\end{equation}

\subsubsection{Hodnota úrokového swapu}

Vzhledem k tomu, že při neexistenci arbitráže by měla být hodnota úrokového swapu v okamžiku jeho sjednání rovna nule, musí se současná hodnota fixní nohy rovnat současné hodnotě floatové nohy. Pro námi uvažovaný jednoroční úrokový swap tedy musí platit
\begin{equation*}
PV(L_{fix}^{1Y}) = PV(L_{float}^{6M}) + PV(L_{float}^{1Y}) 
\end{equation*}
\begin{equation}
DF_{0, 1Y} A R_{1Y} \alpha_{0,1Y} = A (1 - DF_{0,1Y})
\end{equation}

Dle (1.1) a (1.2) je obecná rovnice pro $n$-roční úrokový swap definována jako
\begin{equation*}
AR_{nY}(DF_{0,t_1} \alpha_{0,t_1} + DF_{t_1, t_2} \alpha_{0,t_2} + ... + DF_{0, t_n} \alpha_{t_{n-1},t_2}) = A(1 - DF_{0,nY})
\end{equation*}
Jestliže budeme uvažovat směnu podkladového kapitálu na začátku a na konci kontraktu, změní se nám tento obecný vzorec do podoby
\begin{equation*}
-A + AR_{nY}(DF_{0,t_1} \alpha_{0,t_1} + DF_{0, t_2} \alpha_{t_1,t_2} + ... + DF_{0, t_n} \alpha_{t_{n-1},t_n}) + A DF_{0, t_n} =
\end{equation*}
\begin{equation*}
= -A + A(1 - DF_{0,nY}) + A DF_{0, t_n}
\end{equation*}
\begin{equation*}
-A + AR_{nY}(DF_{0,t_1} \alpha_{0,t_1} + DF_{0, t_2} \alpha_{t_1,t_2} + ... + DF_{0, t_n} \alpha_{t_{n-1},t_n}) + A DF_{0, t_n} = 0
\end{equation*}
\begin{equation*}
R_{nY}(DF_{0,t_1} \alpha_{0,t_1} + DF_{t_1, t_2} \alpha_{0,t_2} + ... + DF_{0, t_n} \alpha_{t_{n-1},t_2}) + DF_{0, t_n} = 1
\end{equation*}
Tento vzorec v podstatě neříká nic jiného, než že v případě rovnovážné swapové sazby $R_{nY}$ je současná hodnota imaginárního fixního dluhopisu rovna jeho nominální hodnotě. Z toho vyplývá, že také hodnota floatové nohy je rovna nominální hodnotě podkladového aktiva\footnote{V opačném případě by nebyla zachována rovnost.}. 

\subsection{Hodnota úrokového swapu v čase $t$}

Výše uvedený postup lze aplikovat v případě, že chceme vypočítat hodnotu úrokového swapu v čase $t_0 = 0$. Hodnotu úrokového swapu v obecném čase $t$ lze opět vyjádřit jako rozdíl hodnoty fixní a floatové nohy diskontované k času $t$.

Současná hodnota fixní nohy v čase $t$ je za předpokladu roční výplaty rovna
\begin{equation*}
PV_t(L_{fix}^{t_1}) + PV_t(L_{fix}^{t_2}) + ... + PV_t(L_{fix}^{t_n}) = A R_{nY}(DF_{t, t_1} + DF_{t, t_2} + ... + DF_{t, t_n})
\end{equation*}
Současná hodnota floatové nohy v čase $t$ je pak rovna
\begin{equation*}
PV_t(L_{float}^{t_1}) + PV_t(L_{float}^{t_2}) + ... + PV_t(L_{float}^{t_n}) = A \bigg(\frac{DF_{t,t_1}}{DF_{t_1 - 1, t_1}} - DF_{t, nY} \bigg)
\end{equation*}
Hodnota úrokového swapu v čase $t$ je tedy rovna
\begin{equation*}
A R_{nY}(DF_{t, t_1} + DF_{t, t_2} + ... + DF_{t, t_n}) - A \bigg(\frac{DF_{t,t_1}}{DF_{t_1 - 1, t_1}} - DF_{t, nY} \bigg)
\end{equation*}
Ačkoliv by hodnota úrokového swapu v okamžiku jeho uzavření měla být nulová, neplatí tento předpoklad po dobu životnosti kontraktu. Hodnota swapu se mění s tím, jak se mění forwardové úrokové sazby.

\subsection{Výpočet diskontního faktoru}

Z (1.3) lze vypočíst diskontní faktor $DF_{0, 1Y}$. Po elemetnárních úpravách získáváme
\begin{equation*}
DF_{0, 1Y} = \frac{1}{1 + R_{1Y} \alpha_{0,1Y}}
\end{equation*}
Vzhledem k tomu, že platí $R_{1Y} = F_{0,1Y}$, je tento vzorec v souladu s výše uvedenou definicí diskontního faktoru.

Analogickým postupem lze z dvouletého swapu odvodit diskontní faktor $DF_{0,2Y}$ jako
\begin{equation*}
DF_{0,2Y} = \frac{1 - R_{2Y} \alpha_{0,1Y}DF_{0,1Y}}{1 + R_{2Y} \alpha_{1Y, 2Y}}
\end{equation*}
a diskontní faktor $DF_{0,3Y}$ z tříletého swapu jako
\begin{equation*}
DF_{0,3Y} = \frac{1 - R_{3Y}(\alpha_{0,1Y}DF_{0,1Y} + \alpha_{1Y,2Y}DF_{0,2Y})}{1 + R_{3Y} \alpha_{2Y, 3Y}}
\end{equation*}
Pro výpočet diskontního faktoru je tedy zapotřebí znalost diskotních faktorů pro předchozí období.

Obecný vzorec pro výpočet diskontního faktoru $DF_{0,nY}$ je
\begin{equation}
DF_{0,nY} = \frac{1 - R_{nY} \bigg( \sum_{t = 1Y}^{(n-1)Y} \alpha_{(t-1)Y,tY DF_{0,t}} \bigg)}{1 + R_{nY} \alpha_{(n-1)Y,nY}}
\end{equation}

\subsection{Zero kupón křivka}

Zero kupón křivku vypočteme z diskontních faktorů, které získáme dle (1.4). Vzorec pro výpočet diskotního faktoru, ze kterého se určí výnosové míry $Y_{0,t}$ této křivky, je definován jako
\begin{equation*}
DF_{0,t} = \frac{1}{(1 + Y_{0,t})^{\alpha_{t, t_0}}}
\end{equation*}
Elementární úpravou pak získáváme
\begin{equation*}
Y_{0,t} = \sqrt[\alpha_{t, t_0}]{\frac{1}{DF_{0,t}}} - 1
\end{equation*}
\end{document}
