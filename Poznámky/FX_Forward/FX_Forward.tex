\documentclass[a4paper]{book}
\usepackage[czech]{babel}
\usepackage[utf8]{inputenc}
\usepackage{pstricks}
\usepackage{amsmath}

\setlength{\unitlength}{1.0mm}

\begin{document}

\chapter{FX Forward}

\section{Úvod}

V rámci FX forwardu se smluvní strany dohodnou o směně dvou částek denominovaných v různých měnách. Směna probíhá v předem sjednaném budoucím čase $T$ a za předem sjednaný forwardový kurz $E^f_T$.\\

\noindent \textbf{Příklad:} KB se dohodla 23.10.2006 se svým klientem, že dne 23.10.2007 provedou směnu 1~000 USD za 22~012 CZK. Spotový kurz k 23.10.2006 byl 22.51 CZK/USD a forwardový kurz byl sjednán na 22.01 CZK/USD\footnote{$E^f_T = \frac{22~012}{1~000}=22.01$}.

\begin{center}
	\begin{pspicture}(0,0)(8,8)
		\rput(4,0.5){Výplata z FX forwardu}
		\rput(4,0){\tiny{(z pohledu strany, která v čase $T$ platí USD a získává CZK)}}

          \psline[linewidth=0.5mm](0.5,3)(7.5,3)
          \psline(0.7,2.9)(0.7,3.1)
		\psline[linestyle=dashed, arrows=->](7.2,3)(7.2,7)
		\psline[arrows=->](7.2,3)(7.2,1)
          
          \rput(6,7){\small{22~012 CZK}}
          \rput(6,1.3){\small{1~000 USD}}
          
		\rput(0.5,2.7){$T_0$}
		\rput(7,2.7){$T$}

	\end{pspicture}
\end{center}

\section{Výpočet}
Uvažujme obchodníka, který má k dispozici 1 USD. Předpokládejme, že za 1 rok bude chtít směnit USD na CZK. V tento okamžik má k dispozici dvě možnosti a to (a) uzavřít FX forward s forwardovou sazbou $E^f_T$, úročit USD sazbou $i_{usd}$ a po jednom roce provést konverzi USD do CZK nebo (b) spotově směnit USD na CZK za spotový kurz $E_0$ a CZK úročit sazbou $i_{czk}$ po dobu jednoho roku. Za předpokladu neexistence arbitráže musí být obě alternativy ekvivalentní. Obecně tedy platí
\begin{equation*}
(1 + i_{usd})^T E^f_T = E_0 (1 + i_{czk})^T
\end{equation*} 
Jedinou neznámou je forwardový kurz $E^f_T$. Ten lze vyjádřit
\begin{equation*}
E^f_T =  E_0 \frac{(1 + i_{czk})^T}{(1 + i_{usd})^T} 
\end{equation*}

\noindent \textbf{Příklad:} Rovnovážný forwardový kurz pro výše uvedený příklad za předpokladu, že $i_{usd}=5.38\%$ a $i_{czk}=3.05\%$, je
\begin{equation*}
E^f_T = 22.51 \frac{1 + 0.0305}{1 + 0.0538} = 22.012
\end{equation*}

\noindent \textbf{Poznámka:} Teoreticky by měl být FX forward uzavírán právě za rovnovážný kurz. V praxi však banky vychýlí kurz ve svůj prospěch a tímto způsobem realizují zisk z transakce. Například v případě, kdy by KB v čase $T$ prováděla nákup USD, nabídla by protistraně nižší než rovnovážný forwardový kurz.\\

Výraz
\begin{equation*}
\frac{(1 + i_{czk})^T}{(1 + i_{usd})^T}
\end{equation*}
je tzv. úrokový diferenciál. Jeho nosnou myšlenkou je, že vývoj měnových kurzů kompenzuje rozdíly v úrokových sazbách pro jednotlivé měny. Tímto způsobem je teoreticky zajištěn všem investorům stejný výnos bez ohledu na to, do které měny investují.

\subsection{Forwardové body}

V praxi se cena forwardu neuvádá ve formě forwardového kurz ale tzv. forwardových bodů. Forwardové body představují rozdíl mezi forwardovým a spotovým kurzem. V našem případě by se jednalo o -498 bodů.
\begin{equation*}
FW_bps = (E^f_T - E_0) \cdot 1~000 = (22.012 - 22.510) \cdot 1~000 = -498
\end{equation*}

\end{document}



