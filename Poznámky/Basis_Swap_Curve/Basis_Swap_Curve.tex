\documentclass[a4paper]{book}
\usepackage[czech]{babel}
\usepackage[utf8]{inputenc}
\usepackage{pstricks}
\usepackage{amsmath}

\setlength{\unitlength}{1.0mm}

\begin{document}

\chapter{Basis swapová křivka}

\section{Použití}

Basis-swap křivka je používána pro oceňování měnových úrokových swapů. Jedná se o swapovou křivku upravenou o tzv. basis swap spread. Důvodem je, že kdyby se použila standardní křivka, vycházela by tržní hodnota těchto swapů v době uzavření obchodu různá od nuly \footnote{Z definice swapových obchodů vyplývá za předpokladu neexistence arbitráže nulová tržní hodnota v okamžiku jejich sjednání.}. Toto vychýlení od nulové tržní hodnoty je pak odstraněno použitím spreadu. Ekonomickým zdůvodněním aplikace spreadů je rozdílná likvidita měn a rozdílné kreditní riziko subjektů, které se "skrývají" za příslušnou měnou. Basis swap spread tak vyjadřuje různou "kvalitu" měn. Basis swap spread je kotován trhem pro jednolivé měnové páry \footnote{Ve většině případů je jednou z měn USD.}.

\section{Technické pozadí}

V případě KB je podkladovou křivkou pro basis swapovou křivku FRA křivka. Tato je upravena o výše zmiňovaný basis swap spread. Tento spread je do systému Kondor+ natahován z excelového souboru \textit{Median4K+.xls} prostřednictvím DTS. Middle office po zafixování křivek použije open report \textit{Basis Swap Spreads Reconciliation} a příslušné spready zadá do systému Kondor+ ručně. Po té dojde k přepočtu  basis swapové křivky.

V současnosti jsou v systému nakonfigurovány basis swapové křivky pro CZK, SKK, USD, HUF, PLN a EUR. Poslední z křivek není používána pro oceňování žádného instrumentu a existuje pouze z historických důvodů. Basis swapové křivky jsou používány k oceňování měnových úrokových swapů (instrument IRS/CCS). EUR basis swapová křivka je vztažena k USD. Ostatní křivky jsou vztaženy k EUR.

\section{Teoretické pozadí}

Uvažujme měnový úrokový swap s splatností za $n$ let, v rámci kterého vyměňujeme plovoucí EUR sazbu za pevnou CZK sazbu. Tento swap má tedy dvě nohy - tzv. pevnou a plovoucí. Přepokládejme, že pokladovým kapitálem, ze kterého jsou počítány úrokové platby, je 1 EUR v případě plovoucí nohy popř. odpovídající CZK ekvivalent v případě pevné nohy. Dále předpokládejme, že KB bude platit pevnou CZK sazbu a obdrží plovoucí EUR sazbu. Na začátku kontraktu si KB smění s protistranou 1 EUR za CZK ekvivalent. Na konci životnosti kontraktu pak dochází ke zpětné směně kapitálu. Obě směny proběhnou za předem domluvený kurz {Jestliže tento domluvený kurz odpovídá spotovému kurzu v okamžiku sjednání kontraktu, je možné od této směny odhlédnout.}. V průběhu životnosti swapu proběhne $n$ plateb a to vždy na konci roku\footnote{Tento předpoklad sice neodpovídá realitě, avšak do určité míry zpřehledňuje navazující výpočty. Nejčastější periodou pro výměnu úrokových plateb jsou tři popř. šest měsíců.}. S poslední úrokovou platbou dochází také ke směně podkladového kapitálu.

Z definice swapu vyplývá, že tržní hodnota jeho hodnota plovoucí nohy je v okamžiku jeho uzavření 1 EUR. Zbývá tedy dopočítat pevnou CZK sazbu\footnote{Pohyblivá sazba je dána referenční sazbou a mění se v průběhu životnosti kontraktu.}. Jestliže přijmeme předpoklad, že předem domluvený kurz pro konverzi podkladového kapitálu v okamžiku sjednání obchodu byl roven spotovému kurzu a že tržní hodnota swapu se v okamžiku jeho uzavření rovná nule, musí platit
\begin{equation}
1 = \sum^n_{i=1}IR_{frw}^i \cdot \Delta t_{i-1,i} \cdot DF_i + DF_n
\end{equation}
kde $IR^i_{frw}$ představuje forwardovou sazbu pro období $\Delta t_{i-1,i}$, $DF_i$ představuje odpovídající diskontní faktor a $\Delta t_{i-1,i}$ časové období definováné jako $t_{i}-t_{i-1}$.

Jedinou neznámou ve výše uvedené rovnici jsou diskontní faktory. Kdybychom však pro jejich výpočet použili běžnou swapovou křivku, získali bychom pevnou úrokovou sazbu rozdílnou od sazby kotované na trhu\footnote{Tento závěr je ekvivalentní k tvrzení, že při použití běžné křivky pro účely ocenění tohoto měnového úrokového swapu bychom získali nenulovou tržní hodnotu v okamžiku jeho sjednání.}. Z tohoto důvodu je třeba příslušnou forwardovou sazbu upravit o tzv. basis swap spread. Rovnice (1.1) se tak mění na
\begin{equation*}
1 = \sum^n_{i=1}(IR_{frw}^i + M_n) \cdot \Delta t_{i-1,i} \cdot DF^M_i + DF^M_n
\end{equation*}
kde $M_n$ představuje kotovaný basis swap spread pro měnový úrokový swap s maturitou v $n$-tém roce a $DF^M_i$ odpovídající diskontní faktor.

První diskontní faktor tak lze snadno spočítat jako
\begin{equation*}
DF^{M_1}_1 = \frac{1}{1+(IR_{frw}^1 + M_1) \cdot \Delta t_{0,1}}
\end{equation*}
druhý diskotní faktor pak jako
\begin{equation*}
DF^{M_2}_2 = \frac{1-(IR_{frw}^1 + M_2)\cdot \Delta t_{0,1} \cdot DF^{M_1}_{1}}{1+(IR_{frw}^2 + M_2) \cdot \Delta t_{1,2}}
\end{equation*}
a tak dále. Protože pro výpočet konkrétního diskontního faktoru potřebujeme znát všechny předcházející diskontní faktory, nazýváme tento způsob výpočtu jako bootstrapping \footnote{Pojem "bootstrapping" vychází z anglického slova "bootstrap", což v předkladu znamená "jazyk boty". Za pojem "boostrapping" vděčíme baronu Prášilovi, který se v jedné ze svých historek chlubil, že unikl před jistou smrtí v močálu tak, že se z něj vytáhl za jazyky svých bot.}.

Obecný vzorec pro výpočet diskotního faktoru je
\begin{equation*}
DF^{M_n}_n = \frac{1-\sum^{n-1}_{i=1}(IR_{frw}^i + M_n) \cdot \Delta t_{i-1,i} \cdot DF^{M_i}_i}{1+(IR_{frw}^n + M_n) \cdot \Delta t_{i-1,i}}
\end{equation*}
Z diskontních faktorů je pak možné relativně snadno dopočítat odpovídající úrokové sazby. Například jednoroční sazba je dána vztahem
\begin{equation*}
IR_{bswp}^1 = \Bigg( \frac{1}{DF^{M_1}_1} - 1 \Bigg) \cdot \frac{1}{\Delta t_{0,1}}
\end{equation*}
Obecný vzorec pro výpočet úrokové sazby je
\begin{equation*}
IR_{bswp}^n = \frac{1-DF^{M_n}_n}{\sum^n_{i=1}\Delta t_{i-1,i} \cdot DF^{M_i}_i}
\end{equation*}
\end{document}



