\chapter{Základní pojmy}

Základem Bayesiánské statistiky je rovnice
\begin{equation}
P(\theta | data) = \frac{P(data | \theta) P(\theta)}{P(data)},
\end{equation}
kde $\theta$ představuje hledaný populační parametr jako např. pravděpodobnost narození jedince ženského pohlaví, pravděpodobnost vyléčení pacienta po podání nového léku atd. Cílem Bayesiánské statistiky je nalezení tzv. aposteriorního rozdělení tohoto parametru, tj. $P(\theta | data)$, což je pravděpodobnostní rozdělení $\theta$ podmíněné 