\documentclass[a4paper]{book}
\usepackage[czech]{babel}
\usepackage[utf8]{inputenc}
\usepackage{pstricks}
\usepackage{amsmath}
\usepackage{graphicx}

\setlength{\unitlength}{1.0mm}

\begin{document}

\title{Matematika finančních derivátů}
\author{John Wilmott, Sam Howison, Jeff Dewynne}
\date{1995}
\maketitle

\tableofcontents

\part{Opční teorie - základy}

\chapter{Úvod}

\section{Evropská kupní a prodejní opce}

Evropská kupní opce je kontrakt, jehož vlastník může v předem dohodnutém budoucím čase (tzv. maturitě nebo splatnosti opce) koupit dohodnutné možnoství podkladového aktiva (např. akcii, komoditu nebo dluhopis) za tzv. realizační cenu, která je taktéž předem dohodnuta. Druhá strana kontraktu, tzv. vypisovatel opce, musí, je-li k tomu vlastníkem opce vyzván, podkladové aktivum za dohodnutou cenu prodat. Právo rozhodnout, zda-li bude opce uplatněna, je tedy na straně jejího vlastníka. Vypisovatel opce je tak proti majiteli v nevýhodě, za což mu náleží finanční kompenzace - tzv. prémium. Prémium tak představuje ocenění práva majitele opce rozhodnout o jejím případném uplatnění. Prémium je vypláceno v okamžiku uzavření opce.

Evropská prodejní opce je identická s evropskou kupní opcí s tím rozdílem, že její majitel má právo podkladové aktivum prodat nikoliv nakoupit.

\section{Pákový efekt}

S opcemi je spojen tzv. pákový efekt. Ten umožňuje investorovi dosáhnout větší rizikové expozice portfolia, než kdyby investoval přímo do podkladového aktiva.

Mějme akcii, jejíž současná hodnota je 110 USD a kterou investor koupil před rokem za 100 USD. Investor tak dosáhl výnosu 10 USD. Uvažujme kupní opce na tuto akcii s realizační cenou 100 USD. Jestliže by se tato opce prodávala před rokem za prémium 5 USD, mohl investor namísto akcie nakoupit 20 opcí. Dnes by tak realizoval čistý zisk 100 USD. Jestliže by však současná cena akcie klesla na 90 USD, byla by ztráta v prvním případě pouze 10 USD zatímco v druhém případě celých 100 USD.

\section{Spekulativní a zajišťovací obchody}

Díky pákovému efektu jsou opce optimálním nástrojem pro spekulativní obchody - umožňují větší rizikovou expozici než by bylo možné při investování do podkladového aktiva. V případě, že spekulant správně odhadne budoucí vývoj na trhu, může dosáhnout několikanásobného zisku v porovnání s výnosem podkladového aktiva.

Jestliže chce investor spekulovat na pokles ceny opce může si koupit prodejní opci popř. vypsat kupní opci. V prvním případě nebude potenciální zisk shora omezen a maximální ztráta bude dána zaplaceným prémiem. V druhém případě naopak není omezena potenciální ztráta, avšak potenciální zisk shora omezen obdrženým prémiem. Další rozdíl je ten, že nákup opce vyžaduje prvnostní investici v podobně uhrazené prémie, kdežto prodej opce naopak představuje cash-in-flow.

V případě spekulace na růst ceny akcie by investor kupoval kupní opci nebo vypisoval prodejní opci.\\

Opce jsou také zajišťovacím nástrojem, tzv. slouží naopak ke snížení rizikové expozice portfolia. Jestliže např. investor má nakoupenou akcii, může potenciální ztrátu z této pozice eliminovat koupí prodejní opce.

\section{Ostatní typy opcí}

Výše uvažované opce byly evropské. Takováto opce může být uplatněna pouze v okamžiku své splatnosti. Vedle evropské opce existuje také opce americká, která může být uplatněna kdykoliv během své životnosti. Tato zdánlivě drobná odlišnost má zásadní význam pro ocenění opce.

Další rovinou rozlišení je dělení opcí podle způsobu stanovení výplaty. Vedle klasických opcí, kdy je výplata dána rozdílem realizační a aktuální ceny podkladového aktiva, existují tzv. exotické opce. V případě exotických opcí může výplata záviset na vývoji ceny podkladového aktiva v průběhu životnosti opce (např. asijská opce) nebo může být dána konkrétní částkou v závislosti na tom, zda-li cena podkladového aktiva protne určitou hranici (např. binární opce).

\section{Forwardové kontrakty a futures}

O opcích se často hovoří jako o derivátech. Derivátem je obecně finanční instrument, jehož cena je odvozena o jiného tzv. podkladového aktiva. Mezi deriváty spadají také tzv. forwardové kontrakty a futures. V obou případech se jedná o obchody, kdy má jedna zúčastněná strana povinnost prodat a druhá koupit v předem smluvený budoucí časový okamžik předem dohodnutý objem podkladového aktiva za předem dohodnutou cenu.

Forwardové kontrakty jsou tzv. over-the-counter (OTC) obchody. To znamená, že jsou uzavírány přímo mezi zúčastněnými subjekty bez účasti třetí osoby (klasicky banka a její klient). Výhodou forwardových kontraktů je flexibilita podmínek při sjednávání obchodu. Naproti tomu futures jsou standardizované obchody, které jsou prováděny přes prostředníka, kterým je burza. Předmětem standardizace je podkladové aktivum (obchody je možné sjednávat pouze na vybraná aktiva), minimální objem obchodu a datum splatnosti. Vedle toho musí zúčastněné subjekty u burzy udržovat hotovost na tzv. maržových účtech. Marže má za cíl pokrýt případné ztráty zúčastněných subjektů z titulu nepříznivého vývoje cen podkladového aktiva a snižuje tak kreditní riziko protistrany.

Vzhledem k tomu, že obě protistrany mají narozdíl od opce rovnocenné postavení, není ani jedna ze zůčastněných stran příjemcem prémia. Vstup do forwardového kontraktu popř. futures by tedy neměl obnášet žádné náklady\footnote{V případě forwardových kontraktů však banka účtuje klientovi provizi. Ta může mít podobu přímé platby nebo, což je běžnější, formu méně výhodné forwardové ceny, za kterou je na konci splatnosti podkladové aktivum prodáváno popř. nakupováno.}.

\section{Úrokové sazby a současná hodnota}

Položme si otázku, kolik bych musel nyní vložit na účet u banky, aby v čase $T$ byla výše depozita rovna částce $E$. Jestliže budeme uvažovat kontinuální úročení a konstatní úrokovou sazbu $r$, je růst depozita dán diferenciální rovnicí
\begin{equation*}
\frac{dM(t)}{M(t)}=r~dt
\end{equation*}
Tuto diferenciální rovnici lze vyřešit pomocí standardní metody separace proměnných.
\begin{equation*}
\frac{1}{M(t)}\frac{dM(t)}{dt} = r
\end{equation*}
\begin{equation*}
\int \frac{1}{M(t)}\frac{dM(t)}{dt} dt = \int r dt
\end{equation*}
\begin{equation*}
\int \frac{1}{M(t)}dM(t) = \int r dt
\end{equation*}
\begin{equation*}
\ln(M(t)) = rt + c
\end{equation*}
\begin{equation*}
M(t) = e^c e^{rt}
\end{equation*}
\begin{equation*}
M(t) = C e^{rt}
\end{equation*}
Protože v čase $t = 0$ musí platit $M(0) = C$, je výsledný tvar řešení
\begin{equation*}
M(t)=M(0)e^{rt}
\end{equation*}
kde $M(0)$ je počáteční výše depozita. Vzhledem k tomu, že $M(T) = E$ a hledanou veličinou je $M(0)$, platí
\begin{equation*}
M(0)=Ee^{-rT}
\end{equation*}
Jestliže bychom tedy dnes uložili do banky částku $Ee^{-rT}$, disponovali bychom v čase $T$ depozitem ve výši $E$. Současná hodnota částky $E$, kterou obdržíme v čase $T$, je tak rovna  $Ee^{-rT}$.

Kdyby úroková míra $r(t)$ byla funkcí času, bylo by možné výše uvedenou rovnici vyjádřit ve tvaru
\begin{equation*}
M(0)=Ee^{-\int_0^Tr(s)ds}
\end{equation*}

\chapter{Náhodná procházka}

\section{Úvod}

Základním kamenem oceňování opcí je předpoklad, že cena podkladového aktiva sleduje tzv. náhodnou procházku (random walk). V souladu s touto teorií je na cenu podkladového aktiva nahlíženo jako na náhodnou veličinu. U té sice nejsme schopni určit její hodnotu k určitému okamžiku v budoucnu, nicméně na základě historických pozorování jsme schopni zkonstruovat její pravděpodobnostní rozdělení. Pomocí tohoto rozdělení je pak možné např. určit, s jakou pravděpodobností cena překročí určitou úroveň.

Teorie náhodné procházky vychází z teorie efektivních trhů. Podle této teorie je cena podkladového aktiva formována novými informacemi, kdy jsou všechny veřejně dostupné informace okamžitě promítnuty v ceně podkladového aktiva. Současná cena ani její vývoj v minulosti tak již neobsahují žádnou ``cenotvornou'' informaci a nelze ji tak použít pro predikci budoucí ceny. Jak již bylo zmíněno, neznamená to, že by historická cena podkladového aktiva neměla vůbec žádnou vypovídací hodnotu. Historický vývoj totiž slouží jako základ pro odhad pravděpodobnostního rozdělení ceny podkladového aktiva.

\section{Modelování ceny podkladového aktiva}

Předpokládejme, že spotová cena podkladového aktiva je $S$. Uvažujme časový interval $dt$\footnote{V následujícím textu budeme používat označení $d \cdot$ pro nekonečně malé změny veličiny $\cdot$.}, během kterého se tato cena změní z $S$ na $S + dS$.
\begin{center}
	\begin{pspicture}(0,0)(9,6)
		\rput(4.5,0){Diskrétní verze náhodné procházky}

		\psline[arrows=->](1.0,1.0)(8.5,1.0)
		\psline[arrows=->](1.0,1.0)(1.0,5.5)
		\psline[linewidth=0.5mm](1.0,2.5)(2.5,3.0)(4.0,4.0)(5.5,3.5)(7.0,5.5)(8.5,5.7)

                \psline[linestyle=dotted](2.5,0.8)(2.5,3.0)
                \psline[linestyle=dotted](4.0,0.8)(4.0,4.0)

                \psline[linestyle=dotted](0.8,3.0)(2.5,3.0)
                \psline[linestyle=dotted](0.8,4.0)(4.0,4.0)

                \rput(8.5,0.8){\small{t}}
                \rput(0.6,5.5){\small{S(t)}}

                \rput(3.25,0.8){\tiny{dt}}
                \rput(0.70,3.5){\tiny{dS}}
	\end{pspicture}
\end{center}
Nejběžnějším přístupem k modelování $dS$ je její rozložení na dvě složky.

První složka je deterministická a její příspěvek k $dS/S$ je roven
\begin{equation*}
\mu dt
\end{equation*}
kde $\mu$ představuje průměrnou míru růstu uvažovaného podkladového aktiva. V jednodušších modelech je $\mu$ konstantní, ve složitějších je pak pak funkcí $S$ a $t$.

Druhá složka je stochastická a představuje změnu ceny aktiva v závislosti na externích faktorech (např. přísun nových informací). Její příspěvek k $dS/S$ je dán náhodným výběrem z normálního rozdělení s nulovou střední hodnotou.
\begin{equation*}
\sigma dX
\end{equation*}
Parametr $dX$ představuje náhodný výběr z normálního rozdělení a $\sigma$ směrodatnou odchylku výnosové míry uvažovaného podkladového aktiva (k tomuto parametru se vrátíme později).

Jestliže obě složky spojíme, získáváme stochastickou diferenciální rovnici
\begin{equation}
\frac{dS}{S}=\mu dt + \sigma dX
\end{equation}

V případě, že by $\sigma$ bylo rovno nule, získali bychom standardní diferenciální rovnici
\begin{equation*}
\frac{dS}{S}=\mu dt
\end{equation*}
jejímž řešením je
\begin{equation*}
S(t)=S(0)e^{\mu t}
\end{equation*}
kde $S(0)$ je počáteční cena podkladového aktiva. Jestliže je tedy $\sigma$ rovno nule, je cena podkladového aktiva deterministická a jsme schopni určit její hodnotu v libovolném časovém okamžiku.

\section{Wienerův proces}

Parametr $dX$, který představuje náhodnost ve vývoji budoucí ceny podkladového aktiva, je tzv. Wienerův proces. Tento proces má následující vlastnosti:
\begin{itemize}
\item $dX$ je náhodným výběrem z normálního rozdělení
\item střední hodnota $dX$ je nulová
\item rozptyl $dX$ je $dt$
\end{itemize}
Parametr $dX$ je tak možné rozepsat ve tvaru
\begin{equation*}
dX = \phi \sqrt{dt}
\end{equation*}
kde $\phi$ je náhodným výběrem z normovaného normálního rozdělení. Rovnice (2.1) tak přejde do tvaru
\begin{equation}
\frac{dS}{S}=\mu dt + \sigma \phi \sqrt{dt}
\end{equation}
Hustota pravděpodobnosti náhodné veličiny $\phi$ je
\begin{equation*}
\frac{1}{\sqrt{2 \pi}}e^{-\frac{1}{2}\phi^2}
\end{equation*}
Nechť $F$ je libovolná spojitá funkce. Definujme očekávanou hodnotu funkce $F$ jako
\begin{equation*}
\varepsilon[F(\cdot)]=\frac{1}{\sqrt{2 \pi}}\int_{-\infty}^{\infty}F(\phi)e^{-\frac{1}{2}\phi^2}d\phi
\end{equation*}
Pro $F(\phi) = \phi$ platí 
\begin{equation*}
\varepsilon[\phi]=0
\end{equation*}
\begin{equation*}
\varepsilon[\phi^2]=1
\end{equation*}

Parametr $dX$ je násobkem $\sqrt{dt}$, protože volba jiného řádu než $\sqrt{dt}$ by vedla k problémům pro $dt \rightarrow 0$. V tomto případě, který bude klíčový pro další analýzy, by totiž řešení bylo buďto nesmyslné nebo triviální.

Rovnice (2.1) má tu výhodu, že dobře odpovídá reálným datům pro akcie a akciové indexy\footnote{V případě měnových párů je však tato shoda poněkud horší a to zejména v dlouhodobém horizontu.}. Její aplikace je tedy obhajitelná nejen na teoretickém ale také praktickém základě.

Jestliže je cena podkladového aktiva dána rovnicí (2.1), má cena $S$ charakter náhodné veličiny, kterou lze popsat lognormálním rozdělením. Jestliže je relativní změna ceny podkladového aktiva vyjádřena pomocí logaritmu jako
\begin{equation*}
\ln \frac{S(t_{i+1})}{S(t_i)}
\end{equation*}
sleduje tato změna normální rozdělení. To lze dokázat následujícím způsobem. Jestliže náhodná veličina $x$ sleduje lognormální rozdělení, vyplývá z definice tohoto rozdělení, že náhodná veličina $\ln x$ sleduje normální rozdělení. Vzhledem k tomu, že
\begin{equation*}
\ln \frac{S(t_{i+1})}{S(t_i)} = \ln S(t_{i+1}) - \ln S(t_{i})
\end{equation*}
a skutečnosti, že lineární kombinací dvou náhodných veličin sledujících normální rozdělení získáme náhodnou veličinu, která opět sleduje normální rozdělení, je výše uvedené tvrzení pravdivé.

\section{Modelování ceny podkladového aktiva}

Pomocí rovnice (2.1) resp. (2.2) je také možné modelovat vývoj ceny podkladového aktiva. Z historických dat se nejprve vypočtou odhady parametrů $\mu$ a $\sigma^2$.
\begin{equation*}
\hat{m} = \frac{1}{n dt}\sum^{n-1}_{i=0}\frac{S_{i+1}-S_i}{S_i}
\end{equation*}
\begin{equation*}
\hat{\sigma}^2 = \frac{1}{(n-1) dt}\sum^{n-1}_{i=0}\Big(\frac{S_{i+1}-S_i}{S_i-\hat{m}}\Big)^2
\end{equation*}
V druhém kroce získáme pomocí náhodného výběru z normovaného normálního rozdělení hodnotu parametru $\phi$. V třetím kroce dosazením do (2.2) vypočteme relativní změnu podkladového aktiva v časovém intervalu $dt$. Opakovaním druhého a třetího kroku je možné modelovat vývoj ceny podkladového aktiva v čase.

\section{Markovova vlastnost}

Uvažujme na vlastnostmi rovnice (2.1). Tato rovnice se neodvolává na historické ceny. Cena podkladového aktiva $S + dS$ v čase $t + dt$ je výhradně závislá na spotové ceně $S$. Tuto nezávislost na minulém vývoji nazýváme Markovovou vlastností (Markov property). Dále uvažujme průměr náhodné veličiny $dS$. Platí
\begin{equation*}
\varepsilon[dS]=\varepsilon[\mu S dt + \sigma S dX]=\mu S dt
\end{equation*}
protože $\varepsilon[dX]=0$. V průměru je tak každá následující hodnota vyšší než ta předešlá o $\mu S dt$. Rozptyl náhodné veličiny $dS$ je pak roven
\begin{equation*}
D[dS]=\varepsilon[dS^2]-\varepsilon[dS]^2=\varepsilon[\sigma^2S^2dX^2]=\sigma^2S^2dt
\end{equation*}
Druhá odmocnina rozptylu je směrodatná odchylka, která je tak proporcionální k $\sigma$.

\section{It\^o lemma}

Odvození It\^o lemmy se opírá o mezivýsledek
\begin{equation}
 dX^2 = dt
\end{equation}
který je platný pro $dt \rightarrow 0$. Připomeňme, že $dX$ lze vyjádřit jako
\begin{equation*}
dX = \phi \sqrt{dt}
\end{equation*} 
Pro $dt \rightarrow 0$ je střední hodnota náhodné veličiny $dX^2$ rovna
\begin{equation*}
E[dX^2] = E[\phi^2 dt]
\end{equation*}
\begin{equation*}
E[dX^2] = dt E[\phi^2]
\end{equation*}
\begin{equation*}
E[dX^2] = dt
\end{equation*}
a její rozptyl roven
\begin{equation*}
D[dX^2] = E[(\phi^2 dt)^2] - E[\phi^2 dt]^2
\end{equation*}
\begin{equation*}
D[dX^2] = dt^2E[\phi^4] - dt^2E[\phi^2]
\end{equation*}
Vzhledem k tomu, že pro $dt \rightarrow 0$ je $dt^2 \approx 0$, platí 
\begin{equation*}
D[dX^2] = 0
\end{equation*}
Pro $dt \rightarrow 0$ tedy můžeme náhodnou veličinu $dX^2$ považovat za deterministickou a rovnu $dt$.

Nyní uvažujme hladkou funkci $f(S)$ a na okamžik zapomeňme, že $S$ je náhodná veličina. Je zřejmé, že změní-li se $S$ o $dS$, změní se také $f(S)$. Tuto změnu je možné vyjádřit pomocí Taylorova rozvoje jako
\begin{equation*}
f(S + dS) - f(S) = \frac{\partial f(S)}{\partial S}dS + \frac{1}{2}\frac{\partial^2f(S)}{\partial S^2}dS^2 + ...
\end{equation*}
\begin{equation}
df(S) = \frac{\partial f(S)}{\partial S}dS + \frac{1}{2}\frac{\partial^2f(S)}{\partial S^2}dS^2 + ...
\end{equation}
Nyní se vraťme zpět k rovnici (2.1). Dle této rovnice platí
\begin{equation*}
dS^2 = (\mu Sdt + \sigma S dX)^2
\end{equation*}
\begin{equation*}
dS^2 = \mu^2S^2dt^2 + 2\mu \sigma S^2dtdX + \sigma^2 S^2 dX^2
\end{equation*}
Jestliže použijeme mezivýsledek (2.3) a zanedbáme všechny členy řádu $dt^2$, přejde pro $dt \rightarrow 0$ výše uvedená rovnice do tvaru
\begin{equation*}
dS^2 = \sigma^2 S^2 dt
\end{equation*}
Dosazením do (2.4) a opětovným zanedbáním všech členů vyššího řádu než $dt$ získáváme
\begin{equation*}
df(S) = \frac{\partial f(S)}{\partial S}(\mu S dt + \sigma S dX) + \frac{1}{2}\frac{\partial^2f(S)}{\partial S^2}\sigma^2 S^2 dt
\end{equation*}
\begin{equation}
df(S) = \Big( \mu S\frac{\partial f(S)}{\partial S} + \frac{1}{2}\frac{\partial^2f(S)}{\partial S^2}\sigma^2 S^2 \Big)dt + \sigma S\frac{\partial f(S)}{\partial S}dX
\end{equation}
Rovnice (2.5) je základní podobnou It\^o lemmy, která se zabývá změnou funkce náhodné veličiny v důsledku infinityzimální změny náhodné veličiny samotné. To, že náhodná veličina $dX$ je řádu $dt$ je důležité pro analýzy, které budou následovat. Lze dokázat, že volba jiného řádu by vedla k nerealistickým vlastnostem modelu vývoje ceny podkladového aktiva pro $dt \rightarrow 0$. Jestliže by $dX \gg \sqrt{dt}$, byla by hodnota náhodné veličiny $dS$ rovna nule nebo nekonečnu. Je-li naopak $dX \ll \sqrt{dt}$, stane se z ceny podkladového aktiva deterministická veličina namísto stochastické.

Rovnice (2.5) se skládá ze stochatické složky, která je násobkem $dX$, a deterministické složky, která je násobkem $dt$. Z tohoto pohledu připomíná rovnice (2.5) rovnici (2.1). It\^o lemma také umožňuje zkoumání chování funkce $f(S)$, která, stejně jako cena podkladového aktiva, sleduje náhodnou procházku.

Rovnici (2.5) je možné dále zobecnit za předpokladu, že $f$ je nejen funkcí $S$, ale také funkcí $t$. S pomocí Taylorova rozvoje lze funkci $f(S,t)$ vyjádřit jako
\begin{equation*}
f(S + dS, t + dt) - f(S,t) = \frac{\partial f(S,t)}{\partial S}dS +  \frac{\partial f(S,t)}{\partial t}dt + 
\end{equation*}
\begin{equation*}
+ \frac{1}{2}\frac{\partial^2 f(S,t)}{\partial S^2}dS^2 + \frac{1}{2}\frac{\partial^2 f(S,t)}{\partial S^2}dt^2 + \frac{\partial^2 f(S,t)}{\partial S \partial t}dSdt + ...
\end{equation*}
Jestliže zanedbáme všechny členy, které mají vyšší řád než $dt$, lze s využitím (2.1) a (2.3) přepsat výše uvedenou rovnici do tvaru
\begin{equation}
df(S,t) = \Big(\mu S \frac{\partial f(S,t)}{\partial S} + \frac{1}{2}\sigma^2S^2\frac{\partial^2 f(S,t)}{\partial S^2} + \frac{\partial f(S,t)}{\partial t} \Big)dt +
\end{equation}
\begin{equation*}
+ \sigma S \frac{\partial f(S,t)}{\partial S}dX
\end{equation*}

Pro ilustraci It\^o lemmy uvažujme funkci
\begin{equation*}
f(S) = \ln S
\end{equation*}
Platí
\begin{equation*}
\frac{\partial f(S)}{\partial S} = \frac{1}{S}
\end{equation*}
a
\begin{equation*}
\frac{\partial^2 f(S)}{\partial S^2} = -\frac{1}{S^2}
\end{equation*}
Dosazením do (2.5) získáváme
\begin{equation*}
df(S)=(\mu - \frac{1}{2}\sigma^2)dt + \sigma dX
\end{equation*}

Rovnice $df(S)$ je stochastickou diferenciální rovnicí s konstatními koeficienty, což znamená, že $df(S)$ má normální rozdělení. Nyní se zaměřmě na samotnou funkci $f(S)$. Tu lze chápat jako součet dílčích změn $df(S)$\footnote{V limitě se tento součet stane integrálem.}. Vzhledem k tomu, že součtem náhodných proměnných, které se řídí normálním rozdělením, je opět náhodná proměnná s normálním rozdělením, má $f(S(t))-f(S(0))$ normální rozdělení se střední hodnotou $(\mu - \frac{1}{2}\sigma^2)$ a rozptylem $\sigma^2t$. Hustota pravděpodobnosti funkce $f(S)$ je tak rovna
\begin{equation}
\frac{1}{\sigma \sqrt{2 \pi t}}e^{-\frac{(f(S(t))-f(S(0))-(\mu - \frac{1}{2}\sigma^2)t)^2}{2 \sigma^2 t}}
\end{equation}
Jestliže $f(S) = \ln S$, lze snadno dokázat, že náhodnou veličinu $S$ lze popsat pomocí lognormálního rozdělení s hustotou pravděpodobnosti
\begin{equation*}
\frac{1}{\sigma S\sqrt{2 \pi t}}e^{-\frac{(\ln (f(S(t))/f(S(0)))-(\mu - \frac{1}{2}\sigma^2)t)^2}{2 \sigma^2 t}}
\end{equation*}

\section{Odstranění náhodnosti}

Náhodné složky obsažené v $S$ (rovnice (2.1)) a $f(S,t)$ (rovnice (2.5)) jsou odvozeny od náhodné veličiny $dX$. Toho je možné využít pro konstrukci veličiny $g(S,t)$, která bude v infinityzimálním časové periodě $dt$ deterministická. Této vlastnosti bude později využito při oceňování opcí. Nechť $\Delta$ je reálné číslo. Definujme funkci $g(S,t)$ jako
\begin{equation*}
g(S,t) = f(S,t)-\Delta S
\end{equation*}
Platí
\begin{equation*}
dg(S,t) = df(S,t)-\Delta dS
\end{equation*}
\begin{equation*}
dg(S,t) = \Bigg( \mu S \frac{\partial f(S,t)}{\partial S} + \frac{1}{2}\sigma^2S^2 \frac{\partial f(S,t)}{\partial t}\Bigg)dt + \sigma S \frac{\partial f(S,t)}{\partial S}dX -
\end{equation*}
\begin{equation*}
- \Delta(\sigma S dX + \mu S dt)
\end{equation*}
\begin{equation*}
dg(S,t) = \Bigg( \mu S \Big( \frac{\partial f(S,t)}{\partial S} - \Delta \Big) + \frac{1}{2}\sigma^2 S^2 \frac{\partial^2 f(S,t)}{\partial S^2} + \frac{\partial f(S,t)}{\partial t} \Bigg)dt +
\end{equation*}
\begin{equation*}
+ \sigma S \Bigg( \frac{\partial f(S,t)}{\partial S}- \Delta \Bigg)dX
\end{equation*}
Jestliže zvolíme $\Delta$ rovno
\begin{equation*}
\Delta = \frac{\partial f(S)}{\partial S}
\end{equation*}
zjednoduší se nám výše uvedená rovnice do tvaru
\begin{equation*}
dg(S,t) = \Bigg( \frac{1}{2}\sigma^2 S^2 \frac{\partial^2 f(S,t)}{\partial S^2} + \frac{\partial f(S,t)}{\partial t} \Bigg)dt
\end{equation*}
Člen s náhodnou veličinou $dX$ vypadl a $dg(S,t)$ se tak ze stochastické stala deterministickou veličinou. Pouze připomeňme, že výše uvedené odvození platí pouze v pro nekonečně malý časový interval délky $dt$.

\chapter{Black-Scholes model}

\section{Vnitřní a časová hodnota opce}

Před tím, než se budeme zabývat oceňováním opcí, uveďme zkratky, které budeme používat v celém následujícím textu.
\begin{itemize}
\item Hodnotu opce označujeme jako $V(S,t)$. V případě, že je třeba rozlišovat kupní a prodejní opci, budeme používat $C(S,t)$ pro kupní a $P(S,t)$ pro prodejní opci. Hodnota opce je funkcí současné ceny podkladového aktiva $S$ a času $t$, závisí však i na následujících faktorech:
\item $\sigma$ - směrodatná odchylka ceny podkladového aktiva
\item $E$ - realizační cena
\item $T$ - čas do splatnosti
\item $r$ - bezriziková úroková sazba
\end{itemize}
Nejprve uvažujme situaci v době splatnosti evropské kupní opce, tj. v čase kdy $t = T$. Jesliže $S > E$, je logické opci uplatnit - podkladové aktivum nakoupíme za cenu $E$ a obratem jej prodáme za cenu $S$. Majitel opce tak realizuje zisk ve výši $S-E$. Jestliže je naopak $S < E$, racionální investor tuto opci neuplatní. Hodnota evropské kupní opce v okamžiku její splatnosti je tak rovna
\begin{equation}
C(S,T) = \max(S-E,0)
\end{equation}
S tím, jak se blížíme datu splatnosti opce, lze očekávat, že se její hodnota bude blížit (3.1). Hodnotu kupní opce dle (3.1) nazýváme vnitřní hodnotou opce. Jedná se tedy o částku, kterou bychom získali uplatněním opce v době její splatnosti. Rozdíl mezi skutečnou a vnitřní hodnotou opce pak nazýváme časovou hodnotou opce.
\begin{center}
	\begin{pspicture}(0,0)(8.0,6.5)
		\rput(4.0,0.5){Vnitřní a časová hodnota evropské kupní opce}
                \rput(4.0,0.0){před splatností jako funkce $S$}

		\psline[arrows=->](0.5,1.5)(7.5,1.5)
		\psline[arrows=->](0.5,1.5)(0.5,6.0)

                \psline[linewidth=0.5mm](0.5,1.5)(3.0,1.5)(7.5,6.0)
                \pscurve(0.5,1.5)(2.5,1.6)(3.0,1.8)(4.0,2.7)(5.5,4.1)(7.4,6.0)

                \rput(0.5,1.2){\small{0}}
                \rput(3.0,1.2){\small{E}}
                \rput(7.5,1.2){\small{S}}
                \rput(0.2,6.0){\small{C}}
	\end{pspicture}
\end{center}

Analogicky k evropské kupní opce je vnitřní hodnota evropské prodejní opce dána rovnicí
\begin{equation*}
P(S,T) = \max(E-S,0)
\end{equation*}
\begin{center}
	\begin{pspicture}(0,0)(8.0,6.5)
		\rput(4.0,0.5){Vnitřní a časová hodnota evropské prodejní opce}
                \rput(4.0,0.0){před splatností jako funkce $S$}

		\psline[arrows=->](0.5,1.5)(7.5,1.5)
		\psline[arrows=->](0.5,1.5)(0.5,6.0)

                \psline[linewidth=0.5mm](0.5,5.5)(4.5,1.5)(7.5,1.5)
                \pscurve(0.5,5.2)(4.0,1.9)(4.5,1.7)(5.5,1.5)(7.5,1.5)

                \rput(0.5,1.2){\small{0}}
                \rput(4.5,1.2){\small{E}}
                \rput(7.5,1.2){\small{S}}
                \rput(0.2,6.0){\small{C}}
	\end{pspicture}
\end{center}
Zajímavostí prodejní opce oproti kupní opci je ta, že časová hodnota opce je pro dostatečně malá $S$ záporná. Hodnota opce je tak nižší než její vnitřní hodnota.

\section{Put-call parita}
Ačkoliv je prodejní opce na první pohled zcela odlišná od kupní opce, existuje mezi těmito typy opcí vztah, který se nazývá put-call paritou.

Uvažujme evropskou kupní a prodejní opci, které mají shodné podkladové aktivum, realizační cenu a zbytkovou splatnost. Dále uvažujme portfolio, které se skládá z podkladového aktiva, dlouhé pozice v prodejní a krátké pozici v kupní opci. Nechť $\Pi$ je hodnota tohoto portfolia.
\begin{equation*}
\Pi(t) = S + P(S,t) - C(S,t)
\end{equation*}
Výplata z portfolia v době splatnosti $T$ obou opcí je
\begin{equation*}
S + \max(E-S,0) - \max(S-E,0)
\end{equation*}
a je rovna $E$ bez ohledu na konečnou hodnotu podkladového aktiva. Toto portfolio je tedy bezrizikové\footnote{Portfolio je prosté tržních rizik. Investor však může utrpět ztrátu v důsledku realizace kreditního rizika. To však momentálně není předmětem našeho zájmu.}. Výnos z tohoto portfolia by tak měl být roven bezrizikové úrokové sazbě $r$. Hodnota portfolia v čase $t < T$ je rovna
\begin{equation}
S + P(S,t) - C(S,t) = Ee^{-r(T-t)}
\end{equation}
Rovnice (3.2) představuje put-call paritu.

\section{Analýza Black-Scholes modelu}

Před tím, než začneme se samotným popisem Black-Scholes modelu, uveďme předpoklady, na nichž je tento model založen.
\begin{itemize}
\item Cena podkladového aktiva sleduje lognormální rozdělení.
\item Bezriziková úroková sazba $r$ a volatilita podkladového aktiva $\sigma$ jsou známé funkce času po dobu životnosti opce.
\item Neexistují transakční náklady spojené se zajištěním portfolia.
\item Podkladové aktivum negeneruje po dobu životnosti opce žádné cash-flow (např. dividendy nebo úrokové platby). Od tohoto můžeme upustit za předpokladu, že výše cash-flow je dopředu známa a že je vypláceno k určitému časovému okamžiku nebo nebo spojitě po celou dobu životnosti opce.
\item Neexistuje možnost arbitráže. To znamená, že všechna bezriziková portfolia musí generovat výnos odpovídající bezrizikové sazbě.
\item S podkladovým aktivem je možné obchodovat nepřetržitě v libovolný časový okamžik.
\item Prodej nakrátko je povolen a pokladové aktivum je dokonale dělitelné. To znamená, že je možné prodat aktivum, které nevlastníme a že ho můžeme nakoupit popř. prodat libovolné množství (např $\sqrt{2}$ kusů akcií).
\end{itemize}

Uvažujme opci, jejíž hodnota $V(S,t)$ je závislá pouze na $S$ a $t$. S využitím It\^o lemmy lze $V(S,t)$ rozepsat jako
\begin{equation*}
dV = \sigma S \frac{\partial V}{\partial S}dX + \Bigg( \mu S \frac{\partial V}{\partial S} + \frac{1}{2}\sigma^2 S^2 \frac{\partial^2V}{\partial S^2} + \frac{\partial V}{\partial t} \Bigg)dt
\end{equation*}

Nyní uvažujme portfolio, které se skládá z jedné opce a $-\Delta$ jednotek podkladového aktiva. Hodnota portfolia $\Pi$ je rovna
\begin{equation}
\Pi = V -\Delta S
\end{equation}
a změna hodnoty portfolia $d \Pi$ v rámci jednoho časového kroku $dt$ je rovna
\begin{equation*}
d \Pi = dV -\Delta dS
\end{equation*}
S použitím (2.1) a (2.6) lze výše uvedenou rovnici přetransformovat do podoby
\begin{equation*}
d\Pi = \sigma S \Bigg( \frac{\partial V}{\partial S} - \Delta \Bigg)dX + \Bigg( \mu S \frac{\partial V}{\partial S} + \frac{1}{2}\sigma^2S^2\frac{\partial^2 V}{\partial S^2} + \frac{\partial V}{\partial t} - \mu \Delta S \Bigg)dt
\end{equation*}
Jak jsme již ukázali dříve, náhodnost je možné eliminovat tak, že zvolíme
\begin{equation}
\Delta = \frac{\partial V}{\partial S}
\end{equation}
Poznamenejme, že $\partial V/ \partial S$ je hodnotou $\Pi$ pro začátek časového kroku $dt$. V tomto případě se hodnota portfolia stane pro infinitymezální časový krok $dt$ deterministická.
\begin{equation*}
d \Pi = \Bigg( \frac{\partial V}{\partial t}  + \frac{1}{2}\sigma^2S^2\frac{\partial^2V}{\partial S^2}\Bigg)dt
\end{equation*}
Vzhledem k tomu, že výše uvedené portfolio je bezrizikové, musí generovat výnos odpovídající bezrizikové úrokové sazbě. Platí tedy
\begin{equation*}
r \Pi dt  = \Bigg( \frac{\partial V}{\partial t}  + \frac{1}{2}\sigma^2S^2\frac{\partial^2V}{\partial S^2}\Bigg)dt
\end{equation*}
Substitucí (3.3) a (3.4) do výše uvedené rovnice získáváme diferenciální rovnici
\begin{equation}
\frac{\partial V}{\partial t} + \frac{1}{2}\sigma^2 S^2 \frac{\partial^2 V}{\partial S^2} + rS \frac{\partial V}{\partial S} - rV = 0
\end{equation}
Tato diferenciální rovnice tvoří jádro Black-Scholes modelu. Při splnění výše uvedených předpokladů musí libovolný finanční derivát, jehož hodnota je závislá pouze na $S$ a $t$, splňovat diferenciální rovnici (3.5). Na této rovnici je také zajímavé, že její hodnota není funkcí $\mu$. Jediným parametrem z výchozí rovnice (2.1), který má vliv na hodnotu opce $V$, je tak volatilita ceny podkladového aktiva $\sigma$.

\section{Black-Scholes rovnice}

Nejčastějším typem parciální diferenciální rovnice ve finanční matematice je tzv. parabolická rovnice. Parabolická rovnice pro funkci $V(S,t)$ definuje vztah mezi mezi $V$ a jeho parciálními derivacemi vzhledem k nezávislým proměnným $S$ a $t$. V nejjednodušším případě je nejvyšší parciální derivace vzhledem k $S$ druhého řádu a nejvyšší derivace vzhledem k $t$ je prvního řádu. Rovnice (3.5) tedy splňuje tuto definici. Jestliže je rovnice navíc lineární a znaménka jednotlivých derivací jsou po převedení na jednu stranu shodná, jedná se o zpětnou parabolickou diferenciální rovnici. V opačném případě se jedná o dopřednou parabolickou rovnici. Rovnice (3.5) je tedy zpětnou parabolickou diferenciální rovnicí.

K jednoznačnému určení hodnoty opce nestačí pouze rovnice (3.5), protože ta sama o sobě jednoznačné řešení nemá. Proto je třeba definovat podmínky, které daná opce musí splňovat. Tyto podmínky charakterizující hodnotu opce pro vybranou podmnožinu jejího definičního oboru. Spolu s diferenciální rovnicí (3.5) jednoznačně určují hodnotu uvažované opce. Standardně se definují dvě tzv. hraniční podmínky pro parametr $S$ a jedna tzv. konečná podmínka pro parametr $t$. Obecný zápis podmínek pro $S$ je
\begin{equation*} 
V(S,t)=V_a(t), ~~~ S = a
\end{equation*}
a
\begin{equation*} 
V(S,t)=V_b(t), ~~~ S = b
\end{equation*}
kde $V_a$ a $V_b$ jsou funkcemi proměnné $t$. Protože (3.5) je zpětná parabolická diferenciální rovnice, musíme specifikovat podmínku pro $t = T$. Tato podmínka má tvar
\begin{equation*}
V(S,t)= V_T(S)
\end{equation*}
Rovnici (3.5) tak řešíme ``pozpátku'' směrem k času $t = 0$. Jestliže by se jednalo o dopřednou parabolickou diferenciální rovnici, stanovovali bychom tzv. počáteční podmínku pro $t = 0$. Dopřednou parabolickou diferenciální rovnici lze snadno změnit na zpětnou pouhou změnou proměnných $t' = -t$ a naopak. Oba typy jsou tedy z matematického pohledu identické a je běžné, že se zpětná parabolická diferenciální rovnice před tím, než se započne s její analýzou, přetransformuje na dopřednou.

\section{Podmínky pro evropské opce}

\subsection{Evropská kupní opce}

Jak bylo řečeno v předchozí kapitole, aby rovnice (3.5) měla jedinečné řešení, je třeba ji doplnit sadou tří podmínek - jednou pro parametr $t$ a dvěmi pro parametr $S$.

Uvažujme evropskou kupní opci, jejíž hodnotu budeme značit $C(S,t)$, realizační hodnotu $E$ a splatnost $T$. Podmínka pro $t$ je dána hodnotou opce v době její splatnosti $T$.
\begin{equation}
C(S,T) = \max(S-E,0)
\end{equation}
Podmínky pro $S$ definujeme pro krajní hodnoty této proměnné, tj. pro $S=0$ a $S \rightarrow \infty$. Je-li $S = 0$, stává se proces popsaný rovnicí (2.1) deterministickým a $S$ je nulové pro libovolný budoucí časový okamžik. Evropská kupní opce je tedy bezcenná.
\begin{equation}
C(0,t) = 0
\end{equation}
Jeslitliže se naopak $S$ blíží nekonečnu, je prakticky jisté, že opce bude uplatněna. Realizační cena $E$ je navíc v porovnání s $S$ zanedbatelná. Proto platí
\begin{equation}
C(S \rightarrow \infty,t) \sim S
\end{equation}

\subsection{Evropská prodejní opce}

V případě evropské prodejní opce jsou podmínky analogické. Podmínka vzhledem k času $t$ je opět stanovena k datu splatnosti opce $T$ a má tvar
\begin{equation}
P(S,T) = \max(E-S, 0)
\end{equation}
Stejně jako v případě evropské kupní opce jsou podmínky pro $S$ stanoveny pro krajní hodnoty $S=0$ a $S \rightarrow 0$. Vzhledem k charakteru výplatního profilu opce mají tyto podmínky podobu
\begin{equation}
P(0,t)=Ee^{-r(T-t)}
\end{equation}
a
\begin{equation}
P(S \rightarrow \infty,t) \rightarrow 0
\end{equation}

\section{Rovnice Black-Scholes modelu}

Jestliže jsou $\sigma$ a $r$ konstatní, má řešení diferenciální rovnice (3.5) za podmínek (3.6), (3.7) a (3.8) pro evropskou kupní opci tvar
\begin{equation*}
C(S,t) = S N(d_1) - Ee^{-r(T-t)}N(d_2)
\end{equation*}
kde $N(\cdot)$ představuje kumulativní distribuční funkci normovaného normálního rozdělení. Parametry $d_1$ a $d_2$ této funkce jsou definovány jako
\begin{equation*}
d_1 = \frac{\ln \frac{S}{E}+(r+\frac{1}{2}\sigma^2)(T-t)}{\sigma \sqrt{T-t}}
\end{equation*}
\begin{equation*}
d_2 = \frac{\ln \frac{S}{E}+(r-\frac{1}{2}\sigma^2)(T-t)}{\sigma \sqrt{T-t}}
\end{equation*}
Pro evropskou prodejní opci a jí odpovídající hraniční podmínky (3.9), (3.10) a (3.11) má řešení rovnice (3.5) tvar
\begin{equation*}
P(S,t)=Ee^{-r(T-t)}N(-d_2) - SN(-d_1)
\end{equation*}
kde $d_1$ a $d_2$ jsou definovány stejně jako v předchozím případě.

Následující obrázky ukazují, jak se vyvíjí hodnota evropské opce s tím, jak se $t$ blíží $T$.
\begin{center}
	\begin{pspicture}(0,0)(8.0,6.5)
		\rput(4.0,0.5){Hodnota evropské kupní opce jako funkce}
                \rput(4.0,0.0){parametru $S$ pro různá $t$}

		\psline[arrows=->](0.5,1.5)(7.5,1.5)
		\psline[arrows=->](0.5,1.5)(0.5,6.0)

                \psline[linewidth=0.5mm](0.5,1.5)(3.0,1.5)(7.5,6.0)
                \pscurve(1.5,1.5)(2.5,1.60)(5.0,3.65)(7.3,5.95)
                \pscurve(1.2,1.5)(2.5,1.75)(5.0,3.80)(7.3,6.15)
                \pscurve(1.0,1.5)(2.5,1.90)(5.0,3.95)(7.3,6.30)


                \rput(0.5,1.2){\small{0}}
                \rput(3.0,1.2){\small{E}}
                \rput(7.5,1.2){\small{S}}
                \rput(0.2,6.0){\small{C}}

                \psline[arrows=->](6.2,5.7)(7.2,5.7)
                \psline[arrows=->](6.2,5.9)(7.2,5.9)
                \psline[arrows=->](6.2,6.1)(7.2,6.1)
                \psline[arrows=->](6.2,6.3)(7.2,6.3)

                \rput(5.3,5.7){\tiny{$T-t = 0.0$}}
                \rput(5.3,5.9){\tiny{$T-t = 0.5$}}
                \rput(5.3,6.1){\tiny{$T-t = 1.0$}}
                \rput(5.3,6.3){\tiny{$T-t = 1.5$}}

	\end{pspicture}
\end{center}
\begin{center}
	\begin{pspicture}(0,0)(8.0,6.5)
		\rput(4.0,0.5){Hodnota evropské prodejní opce jako funkce}
                \rput(4.0,0.0){parametru $S$ pro různá $t$}

		\psline[arrows=->](0.5,1.5)(7.5,1.5)
		\psline[arrows=->](0.5,1.5)(0.5,6.0)

                \psline[linewidth=0.5mm](0.5,5.5)(4.5,1.5)(7.5,1.5)
                \pscurve(0.5,5.30)(2.5,3.30)(3.80,2.07)(4.40,1.70)(7.0,1.5)
                \pscurve(0.5,5.15)(2.5,3.10)(3.80,1.95)(4.40,1.70)(7.0,1.5)
                \pscurve(0.5,4.95)(2.5,2.90)(3.80,1.85)(4.40,1.70)(7.0,1.5)

                \rput(0.5,1.2){\small{0}}
                \rput(4.5,1.2){\small{E}}
                \rput(7.5,1.2){\small{S}}
                \rput(0.2,6.0){\small{C}}

                \psline[arrows=->](2.0,4.7)(0.75,4.7)
                \psline[arrows=->](2.0,4.9)(0.70,4.9)
                \psline[arrows=->](2.0,5.1)(0.70,5.1)
                \psline[arrows=->](2.0,5.3)(0.75,5.3)

                \rput(2.8,4.7){\tiny{$T-t = 1.5$}}
                \rput(2.8,4.9){\tiny{$T-t = 1.0$}}
                \rput(2.8,5.1){\tiny{$T-t = 0.5$}}
                \rput(2.8,5.3){\tiny{$T-t = 0.0$}}

	\end{pspicture}
\end{center}
Z výše uvedených obrázků je zřejmé, že s tím, jak se $t \rightarrow T$, blíží se hodnota opce vnitřní hodnotě opce, tj. časová hodnota opce se blíží nule.

\subsection{Delta opce}

Řecké písmeno delta vyjadřuje citlivost ceny opce na změnu ceny podkladového aktiva v časovém intervalu $dt \rightarrow 0$. Delta evropské kupní opce je definováno jako
\begin{equation*}
\Delta = \frac{\partial C}{\partial S} = N(d_1)
\end{equation*}
a evropské prodejní opce jako
\begin{equation*}
\Delta = \frac{\partial P}{\partial S} = N(d_1)-1
\end{equation*}
Stejně jako pro opce samotné platí také pro jejich delty put-call parita.
\begin{equation*}
S = P(S,t) - C(S,t)
\end{equation*}
\begin{equation*}
\frac{\partial S}{\partial S} = \frac{\partial P(S,t)}{\partial S} - \frac{\partial C(S,t)}{\partial S}
\end{equation*}
\begin{equation*}
1 = 1 - N(d_1) - N(d_1)
\end{equation*}
\begin{equation*}
0 = 0
\end{equation*}
\begin{center}
	\begin{pspicture}(0,0)(7.0,7.0)
		\rput(3.5,0.5){Delta evropské kupní opce jako funkce}
                \rput(3.5,0.0){parametru $S$ pro různá $t$}

		\psline[arrows=->](0.5,1.5)(6.5,1.5)
		\psline[arrows=->](0.5,1.5)(0.5,6.0)

                \psline[linewidth=0.5mm](0.5,1.5)(4.0,1.5)(4.0,5.0)(6.0,5.0)
                \pscurve(2.0,1.5)(2.9,2.00)(4.5,4.5)(5.20,4.85)(6.0,5.0)
                \pscurve(2.5,1.5)(3.1,1.90)(4.5,4.4)(5.20,4.90)(6.0,5.0)
                \pscurve(3.0,1.5)(3.3,1.80)(4.5,4.3)(5.20,4.95)(6.0,5.0)

                \rput(0.5,1.2){\small{0}}
                \rput(4.0,1.2){\small{E}}
                \rput(6.5,1.2){\small{S}}
                \rput(0.2,6.0){\small{C}}

                \psline[arrows=->](3.0,3.0)(4.05,3.0)
                \psline[arrows=->](3.0,3.2)(4.00,3.2)
                \psline[arrows=->](3.0,3.4)(3.95,3.4)
                \psline[arrows=->](3.0,3.6)(3.90,3.6)

                \rput(2.2,3.0){\tiny{$T-t = 0.0$}}
                \rput(2.2,3.2){\tiny{$T-t = 0.5$}}
                \rput(2.2,3.4){\tiny{$T-t = 1.0$}}
                \rput(2.2,3.6){\tiny{$T-t = 1.5$}}
                
	\end{pspicture}
\end{center}
Výše uvedený obrázek zachycuje hodnotu delty evropské kupní opce jako funkce parametru $S$ pro různé hodnoty parametru $t$. Delta vždy nabývá hodnot z intervalu nula až jedna. S tím, jak se $t \rightarrow T$, stává se její hodnota stále více citlivá na změnu $S$. Jesltiže si uvědomíme, že delta vyjadřuje změnu hodnoty opci v závislosti na změně $S$, není toto zjištění překvapivé. S tím, jak se zbytková splatnost opce blíží nule, klesá pravděpodobnost, že se cena podkladového aktiva v době splanosti opce bude významněji odlišovat od spotové ceny.

\subsection{Delta zajištění}

Uvažujme vypisovatele evropské kupní opce, který, bude-li k tomu vyzván, musí v době splatnosti opce dodat podkladové aktivum. Předpokládejme, že tento investor využívá tzv. delta zajištění. Invetor tedy drží delta jednotek podkladového aktiva, protože v infinitimezálním časovém okamžiku platí
\begin{equation*}
 C - \Delta S = 0
\end{equation*}
V době splatnosti bude investor držet správné množství podkladového aktiva. V případě, že bude opce uplatněna, bude její delta rovna jedné a investor tak bude držet jednu jednotku podkladového aktiva. V opačném případě bude delta opce rovna nule a investor tak nebude držet žádné podkladové aktivum. V ideálním případě by investor při vypsání opce nakoupil podkladové aktivum v objemu odpovídající počáteční deltě opce a následně tuto pozici postupně navyšoval resp. redukoval až do splatnosti opce s tím, jak se měnila delta opce. Problematická však může být např. situace, kdy by cena podkladového aktiva vzrostla, klesla a opět vzrostla. V tomto případě by investor pozici v podkladovém aktivu nejprve navýšil, po té zredukovat a následně opět navýšil. Navíc pro $t \rightarrow T$ v situaci, kdy by se spotová cena podkladového aktiva $S$ nacházela v  blízkosti realizační ceny $E$, se může delta opce změnit z čísla blízkého nule na číslo blízké jedné popř. naopak a to i několikrát po sobě. S ohledem na možné transakční náklady se tedy delta zajištění pro praxi příliš nehodí.

\section{Zajištění v praxi}

Zajištění spočívá ve snížení senzitivity portfolia na pohyb ceny podkladového aktiva a to tak, že investor zaujme opačnou pozici v jiném finančním instrumentu. Dva extrémní příklady jsme si již ukázali - v obou byla senzitivita zredukována na nulu. V prvním případě jsme vycházeli z put-call parity a v druhém případě se jednalo o delta zajištění. Existují však i jiné způsoby zajištění, které kromě delty zahrnují také jiná tzv. řecká písmena.

Pouze připomeňme, že delta portfolia je dána vztahem
\begin{equation*}
\Delta = \frac{\partial \Pi}{\partial S}
\end{equation*}
Portfolio, které se skládá z opce a delta jednotek podkladového aktiva, nazýváme delta neutrální. Senzitivita tohoto portfolia na změnu ceny podkladového aktiva je v nekonečně krátkém časovém okamžiku nulová.

Pomocí delta zajištění jsme z větší části eliminovali citlivost hodnoty portfolia na změnu ceny podkladového aktiva. Nicnémě i po delta zajištění zůstává portfolio marginálně citlivé na změnu ceny podkladového aktiva. Důvodem je nelinearita delty. K odstranění této zbytkové citlivosti se používá tzv. gamma zajištění.
\begin{equation*}
\Gamma = \frac{\partial^2 \Pi}{\partial S^2}
\end{equation*}
Změna hodnoty portfolia v čase je vyjádřena řeckým písmenem $\Theta$.
\begin{equation*}
\Theta = \frac{\partial \Pi}{\partial t}
\end{equation*}
Senzitivita portfolia na volatilitu ceny podkladového aktiva je dána řeckým písmenem vega
\begin{equation*}
\nu = \frac{\partial \Pi}{\partial \sigma}
\end{equation*}
a konečně pro měření citlivosti hodnoty portfolia na změnu bezrizikové úrokové míry se využívá řecké písmeno $\rho$.
\begin{equation*}
\rho = \frac{\partial \Pi}{\partial r}
\end{equation*}
 
\subsection{Imunizace portfolia}

Aby portfolio nebylo citlivé na změny vybraných veličin\footnote{Mezi takovéto veličinu může patřit např. změna ceny podkladového aktiva $dS$, časový posun $dt$, změna volatility podkladového aktiva $d \sigma$ nebo změna bezrizikové úrokové sazby $dr$.}, musí kromě podkladového aktiva zahrnovat také finanční deriváty odvozené od uvažovaného podkladového aktiva. Takovéto portfolio označujeme jako imunizované. Abychom sestavili portfolio imunizované vůči $n$ veličinám, musí toto portfolio kromě podkladového aktiva obsahovat také nejméně $n$ finančních derivátů. Matematickým vyjádřením tohoto problému je pak $n$ rovnic o $n$ neznámých. Následující příklad ilustruje imunizaci portfolia vůči $dS$ a $dt$.

Uvažujme portfolio, které se skládá ze tří finančních instrumentů - podkladového aktiva a dvou opcí $V_1$ a $V_2$. Označme deltu první resp. druhé opce jako $\Delta_1$ resp. $\Delta_2$, thetu první resp. druhé opce jako $\Theta_1$ resp. $\Theta_2$ a gammu první resp. druhé opce jako $\Gamma_1$ resp. $\Gamma_2$. Hodnotu portfolia definujme jako
\begin{equation*}
\Pi = S + a_1V_1 + a_2V_2
\end{equation*}
Je-li finanční derivát funkcí $S$ a $t$, platí pro něj dle It\^o lemmy následující vztah
\begin{equation*}
dV = \Bigg( \mu S \frac{\partial V}{\partial S} + \frac{1}{2}\sigma^2S^2\frac{\partial^2 V}{\partial S^2}+\frac{\partial S}{\partial t} \Bigg) dt + \sigma S \frac{\partial V}{\partial S}dX
\end{equation*}
S využitím řeckých písmen lze tento vztah přepsat do tvaru
\begin{equation*}
dV = \Bigg( \mu S \Delta + \frac{1}{2}\sigma^2S^2 \Gamma + \Theta \Bigg) dt + \sigma S \Delta dX
\end{equation*}
Změna ceny podkladového aktiva je dána rovnicí (2.1).
\begin{equation*}
dS = \mu S dt + \sigma S dX
\end{equation*}
Aby bylo portfolio inumizované, musí platit
\begin{equation*}
dS  + a_1 dV_1 + a_2 dV_2 = 0
\end{equation*}
S využitím výše uvedených rovnic lze tento vztah dále upravit do tvaru
\begin{equation*}
\mu dt + \sigma dX + a_1 \Bigg( \bigg(\mu \Delta_1 + \frac{1}{2}\sigma^2 S \Gamma_1 + \Theta_1 \bigg)dt + \sigma \Delta_1 dX \Bigg) + a_2 \Bigg( \bigg(\mu \Delta_2 + \frac{1}{2}\sigma^2 S \Gamma_2 + \Theta_2 \bigg)dt + \sigma \Delta_2 dX \Bigg) = 0
\end{equation*}
Abychom eliminovali náhodnou složku, musí platit
\begin{equation*}
\sigma dX + a_1 \sigma \Delta_1 dX + a_2 \sigma \Delta_2 = 0 
\end{equation*}
\begin{equation*}
1 + a_1 \Delta_1 + a_2 \Delta_2 
\end{equation*}
\begin{equation}
a_2 = - \frac{1 + a_1 \Delta_1}{\Delta_2}
\end{equation}
Abychom imunizovali portfolio proti ztrátě jeho hodnoty v důsledku prostého plynutí času, musí platit
\begin{equation*}
\mu dt + a_1 \bigg( \Delta_1 + \frac{1}{2}\sigma^2S \Gamma_1 + \Theta_1 \bigg) dt + a_2 \bigg( \Delta_2 + \frac{1}{2}\sigma^2S \Gamma_2 + \Theta_2 \bigg) dt = 0
\end{equation*}
\begin{equation}
\mu - 1 + \frac{1}{2}\sigma^2(a_1 \Gamma_1 + a_2 \Gamma_2) + (a_1 \Theta_1 + a_2 \Theta_2) = 0
\end{equation}
Dosazením (3.12) do (3.13) a převedením $a_1$ na jednu stranu rovnice získáváme
\begin{equation*}
a_1 = \frac{1 - \mu + \frac{1}{\Delta_2}\bigg( \frac{1}{2} \sigma^2 S + \Theta_2 \bigg)}{\frac{1}{2} \sigma^2 S (\Gamma_1 - \frac{\Delta_1}{\Delta_2}\Gamma_2) + \bigg( \Theta_1 - \frac{\Delta_1}{\Delta_2} \Theta_2 \bigg)}
\end{equation*}
Parametru $a_2$ je pak dána rovnicí (3.12).

Aby byla hodnota portfolia $\Pi$ imunní vůči změnám $dS$ a $dt$, musí se skládat z jedné jednotky podkladového aktiva, $a_1$ jednotek opce $V_1$ a $a_2$ jednotek opce $V_2$.

\section{Volatilita ceny podkladového aktiva}

Až dosud jsme předpokládáli, že na základě historických dat odhadneme hodnoty vstupních parametrů, které dosadíme do Black-Scholes modelu a následně vypočteme hodnotu opce.

Nezbytným vstupním parametrem Black-Scholes modelu je bezriziková úroková sazba. Tato sazba je kotována trhem pro různé splatnosti. Pro výběr konkrétní úrokové sazby je rozhodující zbytková splatnost opce. Tento parametr nepřestavuje z praktického hlediska problém.

Problematickým parametrem je však volatilita ceny podkladového aktiva. Předpoklad, že volatilita je konstatní v čase, je zavádějící. Volatilita vypočtená na základě historických dat je totiž značně ovlivněna výběrem časové řady, a proto ji nelze použít jako odhad současné nebo dokonce budoucí volatility. Přímý výpočet volatility v praxi je tak velmi problematický. Volatilita pro evropské opce je však kotována trhem. Trh tuto volatilitu tedy ``zná''. Volatility jsou kotovány pro rozdílnou splatnost a realizační cenu popř. deltu opce\footnote{Realizační cenu lze z delty příslušné opce při znalosti ostatních vstupních parametrů snadno dopočítat. Z tohoto pohledu je tedy lhostejno, zda-li je volatilita kotována ve vztahu k realizační ceně nebo deltě opce.} v rozdělení na kupní a prodejní opce. Příklad takovéto kotace je uveden na obrázku \ref{intel_volatility}.

\begin{figure}
  \includegraphics[bb=0 0 350 250]{intel.bmp}
  \caption{Kotace implikované volatility (sloupec IVM) pro akcie společnosti Intel v závisloti na realizační ceně resp. deltě opce (sloupec DM) a době do splatnosti opce (zdroj: Bloomberg)}
  \label{intel_volatility}
\end{figure}

Zaměřme se na kotovanou volatilitu pro evropskou kupní opci se splatností 22.listopadu 2008. Kotovaná volatilita klesá z 363.36\% pro realizační cenu 9 USD až na 129.09\% pro 13 USD a pak opět roste na 462.81\% pro realizační cenu 23 USD. Jestliže bychom tedy vynesli tuto volatitu do grafu proti realizační ceně, získali bychom graf ve tvaru písmene ``U''. Pro tento graf se používá pojem ``volatility smile'', který má vyjadřovat typický vztah mezi realizační cenou a kotovanou volatilitou.

S pomocí kotované volatility a úrokové sazby je možné vypočíst hodnotu opce. Cena opce tak není kotována přímo, ale nepřímo skrze kotaci příslušné volatility. Jestliže bychom namísto volatility měli kotovanou přímo cenu opce (např. u OTC obchodů), je možné volatilitu zpětně dopočítat. Takto vypočtenou volatilitu pak označujeme jako implikovanou volatilitu.

Odlišné kotace volatilit pro jednotlivé realizační ceny odráží skutečnost, že cena podkladového aktiva ve skutečnosti nesleduje lognormální rozdělení. Lognormální rozdělení je pouhou aproximací pravděpodobnostního rozdělení, které očekává trh.

Vzhledem k tomu, že volatilita je kotována také pro různé splatnosti, mění se pravděpodobnostní rozdělení očekávané trhem také v čase. V případě kotací volatilit v závislosti na realizační ceně resp. deltě a zbytkové splatnosti hovoříme o tzv. ``volatility surface''.

\chapter{Parciální diferenciální rovnice}

\section{Difúzní rovnice}

Teplotní nebo také difúzní rovnice
\begin{equation}
\frac{\partial u}{\partial \tau} = \frac{\partial^2 u}{\partial x^2}
\end{equation}
popisuje šíření tepla v jednorozměrném spojitém médiu, kde $u(u, \tau)$ představuje teplotu v tyči z homegenního materiálu jejíž konce jsou dokonale izolovány. Teplota této tyče je tak pouze funkcí vzdálenosti $x$ od zdroje tepla a času $\tau$. Výše popsaná difúzní rovnice splňuje následující podmínky:
\begin{itemize}
\item Difúzní rovnice je lineární rovnicí. To znamená, že jsou-li $u_1$ a $u_2$ řešením, je řešením také jejich libovolná lineární kombinace $c_1 u_1 + c_2 u_2$.
\item Vzhledem k tomu, že nejvyšší derivací je člen $\frac{\partial^2 u}{\partial x^2}$, jedná se diferenciální rovnici druhého řádu.
\item Jedná se o parabolickou rovnici a její charakteristiky jsou dány $\tau = k$, kde $k$ je libovolná konstanta\footnote{Pojem charakteristika bude vysvětlen dále.}.  Informace se tak šíří podél charakteristiky v prostoru $u(x, \tau)$ a každá změna $u$ v určitém bodě se okamžitě projeví ve všech ostatních bodech.
\item Řešením této rovnice jsou analytické funkce proměnné $x$. Pro libovolnou hodnotu $\tau$ větší než počáteční čas $t$ existuje pro každou funkci $u(x, \tau)$ proměnné $x$ konvergentní polynomická řada ve tvaru
\begin{equation}
a_0 + a_1(x - x_0) + a_2(x - x_0)^2 + ...
\end{equation}
Z praktického hlediska tedy můžeme pro $\tau > 0$ považovat řešení difúzní rovnice za tak hladkou funkci, jak jen funkce může být\footnote{Jak již bylo řečeno výše, řešení této rovnice lze aproximovat analytickou funkcí, která je ``dokonale'' hladkou funkcí.}. Může však existovat nespojitost v čase jako důsledek existence hraničních podmínek. To je důsledek toho, že se informace šíří nekonečně rychle podél charakteristiky $\tau = k$, kde $k$ je libovolná konstanta.
\end{itemize}

Z fyzikální pohledu je tepelná difúze ``vyhlazovací'' proces - teplo se šíří z horké do chladné části a dochází tak k vyrovnávání teplot v námi uvažované modelové tyči. Difúzní rovnice je matematickým modelem tohoto procesu. Lze dokázat, že ačkoliv výchozí hodnoty mohou být skokové, existuje na oboru hodnot $-\infty < x < \infty$ pro počáteční podmínku $u(x, 0) = u_0(x)$ a hraniční podmínky $u(x \rightarrow \pm \infty , \tau) \rightarrow 0$ analytické řešení difúzní rovnice (4.1) pro všechna $\tau > 0$.

Jako ilustrace všech těchto vlastností může posloužit např. řešení
\begin{equation}
u(x, \tau) = \frac{1}{2 \sqrt{\pi \tau}}e^{\frac{-x^2}{4 \tau}}
\end{equation}
pro
$-\infty < x < \infty$ a $\tau > 0$. Pro $\tau > 0$ se jedná o hladkou Gausovu křivku, avšak pro $\tau = 0$ je (4.2) přejde v tzv. delta funkci
\begin{equation*}
u_{\delta}(x,0) = \delta(x) 
\end{equation*}
Delta funkce $u_{\delta}(x,0)$ je charakteristická tím, že její hodnota je pro $x \neq 0$ nulová a pro $x = 0$ se blíží nekonečnu, avšak integrál této funkce je vždy roven jedné. Následující obrázek představuje $u(x, \tau)$ pro různé hodnoty $\tau$.
\begin{center}
	\begin{pspicture}(0,0)(10.0,7.0)
		\rput(5.0,0.5){Fundamentální řešení difúzní rovnice}

		\psline(0.5,1.0)(9.5,1.0)
		\psline(5.0,1.0)(5.0,6.5)

                \pscurve(1.0,1.0)(3.0,1.2)(5.0,2.2)(7.0,1.2)(9.0,1.0)
                \pscurve[curvature=0.7 0.1 0](2.5,1.0)(4.0,1.2)(5.0,4.0)(6.0,1.2)(7.5,1.0)
                \pscurve(3.2,1.0)(3.5,1.0)(4.6,1.1)(4.8,1.5)
                \pscurve(5.2,1.5)(5.4,1.1)(6.5,1.0)(6.8,1.0)
                \pscurve[curvature=0.1 0.1 0](4.8,1.5)(5.0,6.0)(5.2,1.5)

                \rput(4.8,6.5){\small{$u_{\delta}$}}
                
                \psline[arrows=->](4.3,5.0)(4.8,5.0)
                \psline[arrows=->](4.05,3.5)(4.55,3.5)
                \psline[arrows=->](3.0,1.5)(3.5,1.5)

                \rput(3.8,5.0){\tiny{$\tau = 0.2$}}
                \rput(3.5,3.5){\tiny{$\tau = 1.0$}}
                \rput(2.5,1.5){\tiny{$\tau = 5.0$}}

	\end{pspicture}
\end{center}
Počáteční hodnota delta funkce $u_{\delta}(x, 0)$ říká, že veškeré teplo je napočátku, tj. v čase $\tau = 0$, koncentrováno v bodě $x = 0$. Delta funkce modeluje šíření tepla materiálem  pro $\tau > 0$  a je fundamentálním řešením difúzní rovnice (4.1). Delta funkce také ilustruje nekonečně rychlé šíření tepla zmiňované výše. Pro $\tau = 0$ je řešení (4.3) nulové pro všechna $x \neq 0$, avšak pro libovolné $\tau > 0$ jakkoliv malé a libovolné $x$ jakkoliv velké je $u_{\delta}(x, \tau) > 0$. Teplo, které bylo původně koncentrováno do jednoho bodu, se tak okamžitě rozšířilo podél celé délky tyče.

\subsection{Technická poznámka: Charakteristiky lineární parciální diferenciální rovnice druhého řádu}

Charakteristiky lineární parciální diferenciální rovnice druhého řádu je možné chápat jako křivky, podél kterých se může šířit informace definovaná funkcí $u$, nebo také jako křivky, napříč kterými se může vyskytovat nespojitost druhé derivace funkce $u$. Předpokládejme, že $u(x, \tau)$ splňuje obecnou lineární diferenciální rovnici druhého řádu.
\begin{equation*}
a(x, \tau)\frac{\partial^2 u}{\partial x^2} + b(x, \tau)\frac{\partial^2 u}{\partial x \partial \tau} + c(x,\tau)\frac{\partial^2 u}{\partial \tau^2} + d(x, \tau)\frac{\partial u}{\partial x} + 
\end{equation*}
\begin{equation*}
+ e(x, \tau)\frac{\partial u}{\partial \tau} + f(a, \tau)u + g(x, \tau) = 0
\end{equation*}
V případě, že existuje charakteristika definovaná ve tvaru $x = x(\epsilon)$ a $\tau = \tau(\epsilon)$, je možné výše uvedenou diferenciální rovnici za předpokladu splnění podmínky
\begin{equation*}
a(x, \tau) \bigg( \frac{d \tau}{d \epsilon} \bigg)^2 - b(x, \tau)\frac{d \tau}{d \epsilon} \frac{d x}{d \epsilon} + c(x, \tau) \bigg( \frac{dx}{d \epsilon} \bigg)^2 = 0
\end{equation*}
 vyjádřit pomocí směrových derivací.

Nyní vyvstává otázka, zda-li výše uvedená kvadratická rovnice má (a) dva reálné kořeny, (b) jeden reálný kořen nebo (c) žádný kořen z množiny reálných čísel. Těmto řešením odpovídají situace, kdy je diskriminant $b^2 - 4ac$ kvadratické rovnice (a) větší než nula, (b) roven nule nebo (c) menší než nula. V prvním případě existují dvě charakteristiky, které nazýváme hyperbolické - typickým příkladem je např. šíření kapaliny; ve financích se tento typ diferenciální rovnice prakticky nevyskytuje. Je-li diskriminant roven nule, označujeme charakteristiku jako parabolickou. Tento typ diferenciální rovnice je ve financích nejběžnější a také všechny diferenciální rovnice v této knize jsou parabolické. V posledním třetím případě, kdy kořeny nejsou z množiny reálných čísel, hovoříme o eliptické charakteristice. Ve finanční matematice je možné se s tímto typem rovnice setkat např. u doživotních opcí ve vícefaktorových modelech. Tuto oblast však naše kniha nepokrývá.

Vzhledem k tomu, že jsou parametry $a$, $b$ a $c$ funkcí $x$ a $\tau$, může se typ diferenciální rovnice měnit. Diferenciální rovnice (3.5), kterou jsme odvodili v předchozí kapitole, je parabolická pro $S > 0$. Tato skutečnost má zásadní důsledek - $S = 0$ představuje hranici, kterou nemůže informace protnout.

\subsection{Technická poznámka: Delta funkce a stranová funkce}

Delta funkci $\delta(x)$ je vhodné spíše než jako klasickou funkci chápat jako ``obecnou'' funkci. Delta funkce je formálně definovaná jako lineární mapa, nicméně její existence je dána zejména potřebou matematického popisu limity funkce, která, ačkoliv je omezována na pořád menší interval, zůstává konečná.

Předpokládejme, že investice generuje v časovém intervalu $dt$ peněžní prostředky ve výši $f(t)dt$, kde funkce $f(t)$ je definována následovně
\begin{equation*}
f(t)= \frac{1}{2 \epsilon},~|t| \le \epsilon
\end{equation*} 
\begin{equation*}
f(t)= 0,~|t| > \epsilon
\end{equation*}
Následující graf zobrazuje funkci pro různé hodnoty $\epsilon$. 
\begin{center}
	\begin{pspicture}(0,0)(10.0,7.0)
		\rput(5.0,0.0){Tři členy limitní řady delta funkce}

		\psline(0.5,1.0)(9.5,1.0)
		\psline(5.0,1.0)(5.0,6.5)

                \psline(1.0,1.0)(3.7,1.0)(3.7,2.5)(6.3,2.5)(6.3,1.0)(9.0,1.0)
                \psline(1.0,1.0)(4.2,1.0)(4.2,4.0)(5.8,4.0)(5.8,1.0)(9.0,1.0)
                \psline[linewidth=0.5mm](1.0,1.0)(4.7,1.0)(4.7,6.0)(5.3,6.0)(5.3,1.0)(9.0,1.0)

                \psline[arrows=<->](4.7,0.7)(5.3,0.7)
                \psline[arrows=<->](3.2,1.0)(3.2,6.0)

                \rput(5.0,0.5){\tiny{$2\epsilon$}}
                \rput(2.9,3.5){\tiny{$\frac{1}{2 \epsilon}$}}

                \rput(9.5,0.7){\small{t}}

	\end{pspicture}
\end{center}
S tím, jak klesá $\epsilon$, se funkce stává vyšší a užší. Je zřejmé, že celková výplata generovaná uvažovanou investicí je
\begin{equation*}
\int^{\infty}_{-\infty}f(t)dt
\end{equation*}
Tento integrál je roven jedné bez ohledu na hodnotu $\epsilon$, avšak pro všechna $t \neq 0$ se $f(t)$ blíží hodnotě nula s tím jak $\epsilon \rightarrow 0$. Tímto způsobem lze neformálně definovat delta funkci $\delta(t)$ - pro $\epsilon \rightarrow 0$ se jedná o ``limitu'' libovolné jednoparametrové ``rodiny'' funkcí $\delta_{\epsilon}(t)$ s následujícími vlastnostmi
\begin{itemize}
\item $\delta_{\epsilon}(t)$ je po částech hladká pro libovolné $\epsilon$
\item $\int^{\infty}_{-\infty} \delta_{\epsilon}(t)dt = 1$
\item $\underset{\epsilon \rightarrow 0} {\lim} \delta_{\epsilon}(t) = 0$ pro všechna $t \neq 0$
\end{itemize}
Takováto řada funkcí se nazývá delta řada. Výše popsaná funkce $f(t)$ je jednou z nich. Další je např. funkce
\begin{equation*}
\delta_{\epsilon}(x) = \frac{1}{2 \sqrt{\pi \epsilon}}e^{-\frac{x^2}{4 \epsilon}}
\end{equation*}
která namísto nezávislé veličiny $t$ používá veličinu $x$. Jestliže nahradíme $\epsilon$ veličinou $\tau$, dostáváme rovnici (4.3). Lze dokázat, že tato rovnice má integrál roven jedné a že, podobně jako $f(t)$, je rovna nule pro $x \neq 0$ a $\epsilon \rightarrow 0$. Pro $x = 0$ a $\epsilon \rightarrow 0$ roste její hodnota k nekonečnu.

Je-li $\phi(x)$ hladkou funkcí, platí
\begin{equation*}
\int^{\infty}_{-\infty}\delta(x)\phi(x)dx = \underset{\epsilon \rightarrow 0}\lim \int^{\infty}_{-\infty} \delta_{\epsilon}(x)\phi(x)dx = \phi(0)
\end{equation*}
Výše uvedený vztah definuje delta funkci jako spojitou lineární mapu z hladké funkce $\phi(x)$ do množiny reálných čísel, konkrétně do hodnoty $\phi(0)$. Je zřejmé, že pro libovolné $a,b > 0$ platí
\begin{equation*}
\int^b_{-a} \delta{x}\phi{x}dx = \phi(0)
\end{equation*}
a pro livolné $x_0$ platí
\begin{equation*}
\int^{\infty}_{-\infty}\delta(x - x_0)\phi(x)dx = \phi(x_0)
\end{equation*}
Násobením $\phi$ členem $\delta(x - x_0)$ a následným integrováním, tak ``vybereme'' hodnotu funkce $\phi$ v bodě $x_0$. Dále platí
\begin{equation*}
\int^x_{-\infty}\delta(s)ds = \mathcal{H}(x)
\end{equation*}
kde $\mathcal{H}(x)$ je tzv. stranová funkce definovaná jako
\begin{equation*}
\mathcal{H}(x) = 0,~x < 0
\end{equation*}
\begin{equation*}
\mathcal{H}(x) = 1,~x \ge 0
\end{equation*}
Inverzně lze delta funkci definovat ze stranové funkce.
\begin{equation*}
\mathcal{H}'(x) = \delta(x)
\end{equation*}
Výše uvedený vztah také ilustruje, že derivace funkce, která je v určitém bodě z důvodu ``skoku'' nespojitá, je v tomto době rovna součinu delty funkce a příslušného ``skoku''.

\section{Počáteční podmínky}

\subsection{Počáteční podmínky a konečný interval}

Přepokládejme, že chceme vyřešit diferenciální rovnici
\begin{equation*}
\frac{\partial u}{\partial r} = \frac{\partial^2 u}{\partial x^2}
\end{equation*}
představující šíření tepla v tyči o délce $2L$ pro konečný interval $-L < x < L$ a $\tau > 0$.

Je zřejmé, že nejprve musíme specifikovat počáteční teplotu $u(x, 0) = u_0(x)$ pro $-L < x < L$. Dále se zdá rozumné přepodkládat, že pro stanovení hodnoty $u(x, \tau)$ stačí, budeme-li znát (a) teplotu na obou koncích tyče nebo (b) tepelné toky na těchto koncích. Tomu odpovídá matematický zápis
\begin{equation*}
u(-L, \tau) = g_-(\tau),~u(L, \tau)=g_+(\tau)
\end{equation*}
resp.
\begin{equation*}
-\frac{\partial u}{\partial x}(-L, \tau) = h_-(\tau),~\frac{\partial u}{\partial x}(L, \tau) = h_+(\tau)
\end{equation*}
V prvním případě to tak jsou teploty a v druhém případě teplné toky, které jsou definované pro $x = -L$ a $x = L$.

\subsection{Počáteční podmínky a nekonečný interval}

Nyní uvažujme, že se teplo šíří v imaginární nekonečně dlouhé tyči. Také v tomto případě je důležité definovat, jak se $u$ chová na ``konci'' tyče, tj. pro $L \rightarrow \pm \infty$. S nekonečně dlouhou tyčí jsou spojené jisté technické problémy, nicméně zjednodušeně řečeno platí, že neroste-li $u$ příliš rychle, existuje jednoznačné řešení závislé na počáteční hodnotě $u_0(x)$. Diferenciální rovnice
\begin{equation}
\frac{\partial u}{\partial \tau} = \frac{\partial^2 u}{\partial x^2} ~~~ -\infty < x < \infty, ~ \tau > 0
\end{equation}
 je tak pro
\begin{equation*}
u(x,0) = u_0(x)
\end{equation*}
a za podmínek
\begin{center}
$u(x, 0)$ je funkce s konečným počtem skokových nespojitostí
\end{center}
\begin{equation}
\underset{|x| \rightarrow \infty} \lim u_0(x)e^{-ax^2} = 0 ~~~ a > 0
\end{equation}
\begin{equation}
\underset{|x| \rightarrow \infty} \lim u(x, \tau)e^{-ax^2} = 0 ~~~ a > 0, \tau > 0
\end{equation}
z matematického pohledu správně formulovaným problémem.

\subsection{Počáteční podmínky a semi-konečný interval}

V některých případech je interval pro $x$ z jedné strany ohraničen reálným číslem a z druhé strany nekonečnem, což vyžaduje kombinaci obou výše popsaných postupů. Tato situace je typická pro bariérové opce. Jestliže chceme např. vyřešit diferenciální rovnici (4.3) pro $0 < x < \infty$ a $\tau > 0$, je třeba definovat dostatečně hladkou funkci $u_0(x)$ pro $0 < x < -\infty$, dostatečně hladkou hladkou hraniční podmínku $u(0, \tau) = g_0(\tau)$ pro $x = 0$ a ``růstové'' podmínky (4.4) a (4.5) pro $x \rightarrow \infty$. To zaručuje, že (4.3) je z matematického hlediska správně definovaným problémem.

\ section{Zpětná a dopředná diferenciální rovnice}

Až dosud jsme se zabývali dopřednou diferenciální rovnicí
\begin{equation*}
\frac{\partial u}{\partial \tau} = \frac{\partial^2 u}{\partial x^2}
\end{equation*}
doplněnou o podmínku pro $\tau = 0$. Uvažujme zpětnou diferenciální rovnici
\begin{equation}
\frac{\partial u}{\partial \tau} = -\frac{\partial^2 u}{\partial x^2}
\end{equation}
odvozenou od dopředné diferenciální rovnice pomocí substituce $\tau_0 - \tau$, kde $\tau_0$ představuje konstantu. Jestliže bychom chtěli řešit tuto rovnici na množině stejných podmínek jako v případě dopředné diferenciální rovnice, zjistili bychom, že není z matematického pohledu správně definovaným problémem. Ve většině případů by totiž tento problém neměl řešení a pokud by existovala taková funkce $u$, konvergovala by její hodnota v konečném čase k nekonečnu. Dobrým příkladem je fundamentální řešení difúzní rovnice (4.2). V čase $\tau_0$ má řešení tvar
\begin{equation*}
\frac{1}{2\sqrt{\pi \tau_0}}e^{-\frac{x^2}{4 \tau_0}}
\end{equation*}
Jestliže tuto rovnici použijeme jako počáteční podmínku $u_0(x)$ pro rovnici (4.6), má řešení tvar
\begin{equation*}
\frac{1}{2\sqrt{\pi (\tau_0 - \tau)}}e^{-\frac{x^2}{4 (\tau_0 - \tau)}}
\end{equation*}
Výše uvedená rovnice se stává singulární pro $\tau = \tau_0$, kdy je rovna delta funkci $\delta(x)$. Navíc tato funkce přestává být reálnou pro $\tau > \tau_0$.

Z fyzikálního pohledu má výše popsané logicky snadno uchopitelnou interpretaci. Dopředná difúzní rovnice modeluje šíření tepla z počátečních hodnot. Zpětná difúzní rovnice je naopak o určení počátečních hodnot, ze kterých se vyvinulo konečné rozdělení tepla a jedná se tedy v porovnání s dopřednou difúzní funkcí o reverzní proces. Dopředná difúzní rovnice ``vyhlazuje'' původní nerovnoměrné rozdělení tepla. Naopak zpětná difúzní rovnice odvozuje z původně hladkého rozdělení výchozí nerovnoměrné rozdělení tepla v imaginární tyči. Alternativním přístupem je chápat dopřednou difúzní rovnici jako proces, v rámci kterého proudí teplo z částí s vyšší teplotou do částí s nižší teplotou. Zpětná difúzní rovnice popisuje modelovou situaci, kdy teplo proudí z chladnějších částí imaginární tyče do teplejších částí tyče. Studené části imaginární tyče se tak stávají ještě studenějšími zatímco teplota teplejších částí může v rámci modelu růst až k nekonečnu.

Navzdory výše řečenému je možné, aby byl problém (4.6) z matematického hlediska správně definován. Je tak možné řešit (4.6) pro $0 < \tau < \tau_0$ a dané $u(\tau_0)$. Důkaz lze provést převedením (4.6) na dopřednou difúzní rovnici pomocí substituce $\tau_0 - \tau$ za $\tau$. 

\chapter{Rovnice Black-Scholes modelu}

\section{Podobnostní řešení}

Řešení $u(x, \tau)$ parciální diferenční rovnice může společně s počátečními a hraničními podmínkami záviset na určité kombinaci nezávislých proměnných $x$ a $\tau$. V tomto případě je možné problém vyjádřit jako klasickou diferenciální rovnici, kde jedinou nezávislou proměnnou je uvažovaná kombinace. Řešení této klasické diferenciální rovnice pak nazýváme podobnostním řešením ve vztahu k původní parciální diferenční rovnici. Matematická argumentace obhajující existenci takovéhoto řešení překračuje rámec této knihy. Problematika je částečně osvětlena v technické poznámce této kapitoly.\\

\noindent \textbf{Příklad:} Předpokládejme, že $u(x, \tau)$ splňuje pariciální diferenční rovnici pro $x > 0$
\begin{equation}
\frac{\partial u}{\partial \tau} = \frac{\partial^2 u}{\partial x^2}, ~~ x > 0, \tau > 0
\end{equation}
při počáteční podmínce
\begin{equation}
u(x, 0) = 0
\end{equation}
a hraničních podmínkách
\begin{equation}
u(0, \tau) = 1, ~~ x = 0
\end{equation}
\begin{equation}
u(x, \tau) \rightarrow 0, ~~ x \rightarrow 0
\end{equation}
Rovnice (5.1) modeluje šíření tepla v dlouhé tyči. Počáteční teplota je nulová, přičemž následně je teplota na jednom konci tyče náhle zvýšena na jedna a udržována na této hodnotě. Dle argumentů uvedených v technické poznámce hledáme řešení, pro které je $u(x, \tau)$ závislé na $x$ a $\tau$ skrze kombinaci $\xi = x / \sqrt{\tau}$ a tudíž $u(x, \tau) = U(\xi)$. Derivováním lze dokázat, že platí
\begin{equation*}
\frac{\partial u}{\partial \tau} = \frac{\partial U}{\partial \xi}\frac{\partial \xi}{\partial \tau}
\end{equation*}
\begin{equation*}
\frac{\partial u}{\partial \tau} = -\frac{1}{2 \tau} \xi \frac{\partial U}{\partial \xi}
\end{equation*}
a
\begin{equation*}
\frac{\partial^2 u}{\partial x^2} = \frac{\partial \frac{\partial u}{\partial x}}{\partial x}
\end{equation*}
\begin{equation*}
\frac{\partial^2 u}{\partial x^2} = \frac{\partial \frac{\partial U}{\partial \xi}\frac{\partial \xi}{\partial x}}{\partial \xi}\frac{\partial \xi}{\partial x}
\end{equation*}
\begin{equation*}
\frac{\partial^2 u}{\partial x^2} = \frac{\partial^2 U}{\partial \xi^2}\bigg( \frac{\partial \xi}{\partial x}\bigg)^2
\end{equation*}
\begin{equation*}
\frac{\partial^2 u}{\partial x^2} = \frac{1}{\tau}\frac{\partial^2 U}{\partial \xi^2}
\end{equation*}
Dosazením do původní diferenciální rovnice (5.1), tak získáváme
\begin{equation*}
\frac{\partial^2 U}{\partial \xi^2} - \frac{1}{2} \xi \frac{\partial U}{\partial \xi} = 0
\end{equation*}
Podmínky (5.2), (5.3) a (5.4) se přetransformují do podoby
\begin{equation}
U(0) = 1, ~~ U(\infty) = 0
\end{equation}
přičemž druhá z rovnic zahrnuje podmínku (5.2) a (5.4). Pomocí separace proměnných je možné získat řešení pro $U'(\xi)$.
\begin{equation*}
U''(\xi) = -\frac{1}{2}\xi U'(\xi)
\end{equation*}
\begin{equation*}
\frac{1}{U'(\xi)} U''(\xi)= -\frac{1}{2}\xi
\end{equation*}
\begin{equation*}
\int \frac{1}{U'(s)} U''(s) ds = -\frac{1}{4}\xi^2 + c
\end{equation*}
\begin{equation*}
\int \frac{1}{U'(s)} d U'(s) = -\frac{1}{4}\xi^2 + c
\end{equation*}
\begin{equation*}
\ln U'(\xi) = -\frac{1}{4}\xi^2 + c
\end{equation*}
\begin{equation*}
U'(\xi) = Ce^{-\frac{1}{4}\xi^2}
\end{equation*}
Následným integrováním výrazu
\begin{equation*}
U(\xi) = C \int^{\xi}_0 e^{-\frac{s^2}{4}}ds + D
\end{equation*}
s využitím vztahu $\int^{\xi}_0 = \int^{\infty}_0 - \int^{\infty}_{\xi}$ a standardní rovnice
\begin{equation*}
\int^{\infty}_0 e^{-\frac{s^2}{4}}ds = \sqrt{\pi}
\end{equation*}
získáme
\begin{equation*}
U(\xi) = C \bigg(\sqrt{\pi}  - \int^{\infty}_{\xi} e^{-\frac{s^2}{4}}ds \bigg) + D
\end{equation*}
S pomocí hraničních podmínek (5.5) lze určit hodnotu konstanty $D$
\begin{equation*}
U(0) = C(\sqrt{\pi} - \sqrt{\pi}) + D = 1
\end{equation*}
\begin{equation*}
D = -1
\end{equation*}
a $C$
\begin{equation*}
U(0) = C(\sqrt{\pi} - 0) - 1 = 0
\end{equation*}
\begin{equation*}
C = \frac{1}{\sqrt{\pi}}
\end{equation*}
Výsledný tvar $U(\xi)$ je tedy
\begin{equation*}
U(\xi) = \frac{1}{\sqrt{\pi}} \int^{\infty}_{\xi} e^{-\frac{s^2}{4}}ds
\end{equation*}
a tvar $u(x,\tau)$ pak
\begin{equation*}
U(\xi) = \frac{1}{\sqrt{\pi}} \int^{\infty}_{x /\sqrt{\tau}} e^{-\frac{s^2}{4}}ds
\end{equation*}
Lze snadno ověřit, že tato rovnice splňuje definici problému (5.1) až (5.4) a že tím pádem řešení závisí pouze na $\frac{x}{\sqrt{\tau}}$.\\

\noindent \textbf{Příklad:} V tomto příkladě odvodíme fundamentální řešení pro $u_{\delta}(x, \tau)$. Nechť
\begin{equation*}
u_{\delta}(x, \tau) = \tau^{-\frac{1}{2}}U(\xi)
\end{equation*}
kde opět
\begin{equation*}
\xi = \frac{x}{\sqrt{\tau}}
\end{equation*}
Člen $\tau^{-\frac{1}{2}}$ má zajistit, že $\int^{\infty}_{-\infty}u_{\delta}(x,\tau)dx$ je konstatní pro všechna $\tau$, což lze dokázat přímým výpočtem. Postupnými úpravami získáme
\begin{equation*}
\frac{\partial u_{\delta}}{\partial \tau} = \frac{\partial \tau^{\frac{1}{2}U_{\delta}}}{\partial \tau}
\end{equation*}
\begin{equation*}
\frac{\partial u_{\delta}}{\partial \tau} = -\frac{1}{2}\tau^{-\frac{3}{2}}U_{\delta} - \frac{1}{2}\tau^{-\frac{1}{2}} \xi \frac{\partial U_{\delta}}{\partial \xi}\frac{\partial \xi}{\partial \tau}
\end{equation*}
\begin{equation*}
\frac{\partial u_{\delta}}{\partial \tau} = -\frac{1}{2}\tau^{-\frac{3}{2}}U_{\delta} - \frac{1}{2}\tau^{-\frac{3}{2}} \xi \frac{\partial U_{\delta}}{\partial \xi}
\end{equation*}
\begin{equation*}
\frac{\partial u_{\delta}}{\partial \tau} = -\frac{1}{2}\tau^{-\frac{3}{2}} \bigg( U_{\delta} + \xi \frac{\partial U_{\delta}}{\partial \xi} \bigg)
\end{equation*}
\begin{equation*}
\frac{\partial u_{\delta}}{\partial \tau} = -\frac{1}{2}\tau^{-\frac{3}{2}} \frac{\partial \xi U_{\delta}}{\partial \xi}
\end{equation*}
resp.
\begin{equation*}
\frac{\partial^2 u_{\delta}}{\partial x^2} = \tau^{-\frac{1}{2}}\frac{\frac{\partial U_{\delta}}{\partial \xi} \frac{\partial \xi}{\partial x}}{\partial \xi} \frac{\partial \xi}{\partial x}
\end{equation*}
\begin{equation*}
\frac{\partial^2 u_{\delta}}{\partial x^2} = \tau^{-\frac{1}{2}}\frac{\partial^2 U_{\delta}}{\partial \xi^2} \bigg( \frac{\partial \xi}{\partial x} \bigg)^2
\end{equation*}
\begin{equation*}
\frac{\partial^2 u_{\delta}}{\partial x^2} = \tau^{-\frac{3}{2}}\frac{\partial^2 U_{\delta}}{\partial \xi^2}
\end{equation*}
Protože výchozí difúzní rovnice má stejně jako v předchozím příkladě tvar
\begin{equation*}
\frac{\partial u}{\partial \tau} = \frac{\partial^2 u}{\partial x^2}, ~~ x > 0, \tau > 0
\end{equation*}
budeme hledat řešení diferenciální rovnice
\begin{equation}
\frac{\partial^2 U_{\delta}}{\partial \xi^2} + \frac{1}{2}\frac{\partial \xi U_{\delta}}{\partial \xi} = 0
\end{equation}
Integrováním této rovnice přes $\xi$ snížíme její řád o jeden stupeň.
\begin{equation*}
\frac{\partial U_{\delta}}{\partial \xi} + \frac{1}{2} \xi U_{\delta} + d = 0
\end{equation*}
Diferenciální rovnice
\begin{equation*}
\frac{\partial U_{\delta}}{\partial \xi} + \frac{1}{2} \xi U_{\delta} = 0
\end{equation*}
má stejně jako v prvním příkladě řešení
\begin{equation*}
U_{\delta} = Ce^{-\frac{1}{4}\xi^2}
\end{equation*}
Výsledný tvar řešení rovnice (5.6) má tedy podobu
\begin{equation*}
U_{\delta}(\xi) = Ce^{-\frac{1}{4}\xi^2} + D
\end{equation*}
Zvolíme-li $D=0$ a znormujeme-li řešení pomocí $C = \frac{1}{2\sqrt{\pi}}$, tak aby platilo
\begin{equation*}
\int^{\infty}_{-\infty}u_{\delta} dx = 1
\end{equation*}
získáme fundamentální řešení
\begin{equation*}
u_{\delta}(x, \tau) = \frac{1}{2\sqrt{\pi \tau}}e^{-\frac{x^2}{4 \tau}}
\end{equation*}
což odpovídá rovnici (4.2).

\section{Technická poznámka: Invariance a podobnostní řešení}

Klíčem k podobnostnímu řešení je, že jak diferenciální rovnice tak počáteční a hraniční podmínky jsou invariantní pro $x \mapsto \lambda x$ a $\tau \mapsto \lambda^2 \tau$, kde $\lambda$ je libovolné reálné číslo. Tato transformace se nazývá jednoparametrovou grupovou transformací a její invarianci je možné ověřit pomocí nových proměnných $X = \lambda x$ a $T = \lambda^2 \tau$, kde funkce $u$ splňuje podmínku
\begin{equation*}
\frac{\partial u}{\partial T} = \frac{\partial^2 u}{\partial X^2}
\end{equation*}
V prvním z příkladů uváděných v předchozí kapitole se podmínky přestransformují do podoby $u(X,0)=0$ a $u(0,T)=1$ pro libovolné $\lambda$. Navíc $\frac{x}{\sqrt{\tau}} = \frac{X}{\sqrt{T}}$ je jedinou kombinací $X$ a $T$, která není funkcí $\lambda$ a řešení je tak závislé pouze na $\frac{x}{\sqrt{\tau}}$. Pro použitelnost metody podobnostního řešení je nezbytné, aby diferenciální rovnice a počáteční a hraniční podmínky byly invariantní pro výše uvažovanou transformaci. V druhém z příkladů předchozí kapitoly funkce proměnné $\tau$, v tomto konkrétním případě se jednalo o funkci $\tau^{-\frac{1}{2}}$, násobí funkci $U_{\delta}(\xi)$, protože difúzní rovnice je vzhledem ke své linearitě také invariantní pro jednoparametrovou grupu $u \mapsto \mu u$. Dobrým praktickým testem pro nalezení podobnostního řešení je vyzkoušet $u = \tau^{\alpha} f(x/ \tau^{\beta})$ v naději, že $x$ a $\tau$ zůstanou v rovnicích pouze pro kombinaci $\xi = x/ \tau^{\beta}$. V prvním z příkladů předchozí kapitoly je výsledkem uplatnění tohoto postupu $\alpha = 0$ získané z hraniční podmínky v $x = 0$ a $\beta = \frac{1}{2}$ získané z difúzní rovnice. V druhém příkladě vychází $\alpha = -\frac{1}{2}$, protože chceme, aby byl intergrál funkce $u(x, \tau)$ přes $x$ nezávislý na $\tau$, a $\beta$ je opět rovno $\frac{1}{2}$.

\section{Problém počáteční podmínky a difúzní rovnice}

Fundamentální řešení difúzní rovnice je možné použít pro odvození explicitního řešení problému definovaného rovnicemi (4.3) - (4.5). V rámci tohoto problému řešíme difúzní rovnici pro $-\infty < x < \infty$ a $\tau > 0$ za podmínky $u(x, 0) = u_0(x)$ a za podmínky "přijatelného" růstu pro $x = \pm \infty$. Klíčem k řešení je skutečnost, že počáteční data je možné vyjádřit ve tvaru
\begin{equation*}
u_0(x) = \int^{\infty}_{-\infty}u_0(\xi)\delta(\xi - x)d\xi
\end{equation*}
kde $\delta(\cdot)$ je delta funkcí. Připomeňme, že fundamentální řešení difúzní rovnice
\begin{equation*}
u_{\delta}(s, \tau) = \frac{1}{2 \sqrt{\pi \tau}}e^{-\frac{s^2}{4 \tau}}
\end{equation*}
má počáteční hodnotu $u_{\delta}(s,0) = \delta(s)$. Protože $u_{\delta}(s-x,\tau) = u_{\delta}(x-s, \tau)$ je při použití $x$ popř. $s$ jako nezávislé veličiny funkce
\begin{equation*}
u_{\delta}(s-x, \tau) = \frac{1}{2 \sqrt{\pi \tau}}e^{-\frac{(s-x)^2}{4 \tau}}
\end{equation*}
řešením difúzní rovnice s počáteční hodnotou $u_{\delta}(s-x,0) = \delta(s-x)$. Pro libovolnou konstantu $s$ tedy funkce
\begin{equation*}
u_0(s)u_{\delta}(s-x, \tau)
\end{equation*}
proměnných $x$ a $\tau$ splňuje difúzní rovnici $\frac{\partial u}{\partial \tau} = \frac{\partial^2 u}{\partial x^2}$ a má počáteční hodnotu $u_0(s)\delta(s-x)$. Vzhledem k tomu, že je difúzní rovnice lineární, je možné jednotlivá řešení skládat pomocí superpozice. Pokud tak učiníme integrováním přes $s$ pro $-\infty < s < \infty$, získáme řešení difúzní rovnice
\begin{equation}
u(x, \tau) = \frac{1}{2 \sqrt{\pi \tau}} \int^{\infty}_{-\infty} u_0(s)e^{-\frac{(x-\tau)^2}{4 \tau}}
\end{equation}
s počáteční hodnotou
\begin{equation*}
u(x,0) = \int^{\infty}_{-\infty}u_0(s)\delta(s-x)ds = u_0(x).
\end{equation*}
Výše uvedená rovnice je tedy hledaným explicitním řešením problému definovaného rovnicemi (4.3) - (4.5). Lze dokázat, že toto řešení je jedinečné. Výše uvedený postup není jediný, jak lze odvodit toto řešení - alternativu představuje Fourierova transformace.

Rovnice (5.7) má následující fyzikální interpretaci. Připomeňme, že fundamentální řešení difúzní rovnice popisuje šíření tepla z výchozího tepelného bodu, kdy v čase $\tau = 0$ je všechno teplo koncentrováno v počátku. Matematicky je tento bod vyjádřen pomocí delta funkce. Nyní uvažujme počáteční tepelné rozdělení $u_0(x)$, které skládá z řady teplených bodů, přičemž tepelný bod v $x = s$ má hodnotu $u_0(s)ds$. Z každého bodu se šíří teplo - výsledné tepelné rozdělení tak odpovídá fundamentálnímu řešení vynásobenému $u_0(s)$ s $x$ nahrazeným $x-s$. Protože je difúzní rovnice lineární, lze výsledné tepelné rozdělení získat superpozicí jednotlivých bodů. V limitě je pak tento součet nahrazen integrálem (5.7).

\section{Řešení diferenciální rovnice Black-Scholes modelu}

Diferenciální rovnice Black-Scholes modelu a podmínky pro evropskou kupní opci s hodnotou $C(S,t)$, jsou
\begin{equation}
\frac{\partial C}{\partial t} + \frac{1}{2}\sigma^2S^2\frac{\partial^2 C}{\partial S^2} + rS \frac{\partial C}{\partial S} - rC = 0
\end{equation}
\begin{equation*}
C(0,t) = 0, ~~ C(S \rightarrow \infty,t) \sim S
\end{equation*}
\begin{equation*}
C(S,T) = \max(S-E,0)
\end{equation*}
Rovnice (5.8) na první pohled nepřipomíná difúzní rovnici a navíc se jedná zpětnou diferenciální rovnici s ``konečnou'' podmínkou stanovenou pro $t = T$. Definujme
\begin{equation*}
S = Ee^x, ~~ t = T - \frac{\tau}{\frac{1}{2}\sigma^2}, ~~ C = Ev(x, \tau)
\end{equation*}
Postupnými úpravami jednotlivých členů rovnice (5.8) získáme
\begin{equation*}
\frac{\partial C(S,t)}{\partial t} = \frac{Ev(x, \tau)}{\partial \tau}\frac{\partial \tau}{\partial t} = E \frac{\partial v(x,\tau)}{\partial \tau} \frac{1}{\frac{\partial t}{\partial \tau}} = -\frac{1}{2}\sigma^2 E \frac{\partial v(x, \tau)}{\partial \tau}
\end{equation*}
\begin{equation*}
\frac{1}{2}\sigma^2 S^2 \frac{\partial^2 C(S,t)}{\partial S^2} = \frac{1}{2}\sigma^2 E^2 ( e^x)^2 \frac{\frac{E v(x, \tau)}{\partial x} \frac{1}{\frac{\partial S}{\partial x}}}{\partial x} \frac{1}{\frac{\partial S}{\partial x}} = \frac{1}{2}\sigma^2E \bigg( \frac{\partial^2 v}{\partial x^2} - \frac{\partial v}{\partial x} \bigg)
\end{equation*}
\begin{equation*}
rS\frac{\partial C(S,t)}{\partial S} = r E e^x \frac{\partial E v(x,\tau)}{\partial x} \frac{1}{\frac{\partial S}{\partial x}} = r E \frac{\partial v(x, \tau)}{\partial x}
\end{equation*}
\begin{equation*}
rC(S,t) = rEv(x, \tau)
\end{equation*}
Výsledná rovnice má tedy po výše uvedených úpravách tvar
\begin{equation}
\frac{\partial v}{\partial \tau} = \frac{\partial^2 v}{\partial x^2} + (k - 1)\frac{\partial v}{\partial x} - kv
\end{equation}
kde $k = \frac{r}{\frac{1}{2}\sigma^2}$. Vzhledem k tomu, že $\tau = 0$ pro $t = T$, přetransformuje se ``konečná'' podmínka do podoby počáteční podmínky
\begin{equation*}
v(x,0) = \max(e^x - 1,0)
\end{equation*}

Rovnice (5.9) obsahuje na první pohled pouze jeden bezrozměrný parametr $k = \frac{r}{\frac{1}{2}\sigma^2}$, ačkoliv v původní rovnici figurovaly čtyři rozměrové veličiny, tj. $E$, $T$, $\sigma^2$ a $r$. Ve skutečnosti v (5.9) figuruje ještě bezrozměrný parametr $\frac{1}{2}\sigma^2T$, který představuje bezrozměrný čas do splatnosti. Tyto dva bezrozměrné parametry jsou jediné nezávislé parametry problému (4.3) - (4.5). Ostatní parametry byly "zavlečeny" v průběhu transformace původního problému, tj. řadou sousledných aritmetických operací.

Rovnice (5.9) nyní mnohem více připomíná difúzní rovnici, na kterou je jí převést pouhou změnou veličiny. Použijeme-li substituci
\begin{equation*}
v = e^{\alpha x + \beta \tau}u(x, \tau)
\end{equation*}
získáme derivováním (5.9) vztah
\begin{equation*}
\beta u + \frac{\partial u}{\partial \tau} = \alpha^2 u + 2 \alpha \frac{\partial u}{\partial x} + \frac{\partial^2 u}{\partial x^2} + (k - 1)\bigg( au + \frac{\partial u}{\partial x} \bigg) - ku
\end{equation*}
Funkci $u$ je možné eliminovat, jestliže zvolíme $\beta = \alpha^2 + (k - 1)\alpha - k$. Podobně je možné se zbavit $\frac{\partial u}{\partial x}$, platí-li $2\alpha + (k - 1) = 0$. Řešením těchto dvou rovnic získáváme
\begin{equation*}
\alpha = -\frac{1}{2}(k - 1), ~~ \beta = -\frac{1}{4}(k + 1)^2
\end{equation*}
Platí tedy
\begin{equation*}
v = e^{-\frac{1}{2}(k - 1)x - \frac{1}{4}(k + 1)^2 \tau}u(x, \tau)
\end{equation*}
kde
\begin{equation*}
\frac{\partial u}{\partial \tau} = \frac{\partial^2 u}{\partial x^2},~~ -\infty < x < \infty,~ \tau > 0
\end{equation*}
s podmínkou
\begin{equation}
u(x, 0) = u_0(x) = \max \bigg( e^{\frac{1}{2}(k-1)x} - e^{\frac{1}{2}(k+1)x}, 0 \bigg)
\end{equation}
Tímto poněkud zdlouhavým způsobem jsme se dobrali rovnice pro výplatu evropské kupní opce. Řešením difúzní rovnice je (5.7)
\begin{equation}
u(x, \tau) = \frac{1}{2\sqrt{\pi \tau}} \int^{\infty}_{-\infty} u_0(s)e^{-\frac{(x-\tau)^2}{4 \tau}} ds
\end{equation}
kde $u_0(x)$ je dáno rovnicí (5.10).

Nyní pouze zbývá upravit intergrál (5.11). Nejprve provedeme substituci
\begin{equation}
x' = \frac{s-x}{\sqrt{2 \tau}}
\end{equation}
která nám výchozí rovnici (5.11) upraví do podoby
\begin{equation}
u(x, \tau) = \frac{1}{\sqrt{2 \pi}}\int^{\infty}_{-\infty}u_0(x'\sqrt{2 \tau} + x)e^{-\frac{1}{2}{x'}^2} dx'
\end{equation}
Protože $k \ge 0$\footnote{Připomeňmě, že $k = \frac{r}{\frac{1}{2}\sigma^2}$. Parametr $k$ by tak mohl být záporný pouze v případě záporné bezrizikové úrokové sazby.}, platí s ohledem na (5.10)
\begin{equation*}
u_0(x) = e^{\frac{1}{2}(k + 1)x} - e^{\frac{1}{2}(k - 1)x}, ~~ x \ge 0
\end{equation*}
\begin{equation*}
u_0(x) = 0, ~~ x < 0
\end{equation*}
Rovnici (5.13) lze tedy dále upravit do tvaru
\begin{equation*}
u(x, \tau) = \frac{1}{\sqrt{2 \pi}}\int^{\infty}_{-x/\sqrt{2 \tau}}e^{\frac{1}{2}(k + 1)(x + x'\sqrt{2 \tau})}e^{-\frac{1}{2}{x'}^2}dx' - \frac{1}{\sqrt{2 \pi}}\int^{\infty}_{-x/\sqrt{2 \tau}}e^{\frac{1}{2}(k - 1)(x + x'\sqrt{2 \tau})}e^{-\frac{1}{2}{x'}^2}dx'
\end{equation*}
\begin{equation*}
u(x, \tau) = I_1 - I_2
\end{equation*}
V dalším kroce upravme člen $I_1$.
\begin{equation*}
I_1 = \frac{1}{\sqrt{2 \pi}} \int^{\infty}_{-x/\sqrt{2 \tau}}e^{\frac{1}{2}(k+1)(x + x'\sqrt{2 \tau})-\frac{1}{2}x'^2 dx'}
\end{equation*}
\begin{equation*}
I_1 = \frac{e^{\frac{1}{2}(k+1)x}}{\sqrt{2 \pi}} \int^{\infty}_{-x/\sqrt{2 \tau}}e^{\frac{1}{2}(k + 1)\tau}e^{-\frac{1}{2}(x' - \frac{1}{2}(k + 1)\sqrt{2 \tau})^2}dx'
\end{equation*}
Nyní provedeme substituci $\rho = x' - \frac{1}{2}(k + 1)\sqrt{2 \tau}$ a pokračujeme v úpravách.
\begin{equation*}
I_1 = \frac{e^{\frac{1}{2}(k + 1)x + \frac{1}{2}(k + 1)^2 \tau}}{\sqrt{2 \pi}} \int^{\infty}_{-x/\sqrt{2 \tau} - \frac{1}{2}(k+1)\sqrt{2 \tau}}e^{-\frac{1}{2}\rho^2}d \rho
\end{equation*}
Protože je $e^{-\frac{1}{2}x^2}$ sudou funkcí, platí
\begin{equation*}
\int^{\infty}_{-x/\sqrt{2 \tau} - \frac{1}{2}(k+1)\sqrt{2 \tau}}e^{-\frac{1}{2}\rho^2}d \rho = \int^{x/\sqrt{2 \tau}}_{-\infty} e^{-\frac{1}{2}(k+1)\sqrt{2 \tau}}e^{-\frac{1}{2}\rho^2}d \rho
\end{equation*}
a $I_1$ je tak možné vyjádřit ve formě
\begin{equation*}
I_1 = e^{\frac{1}{2}(k+1)x + \frac{1}{4}(k + 1)^2 \tau}N(d_1)
\end{equation*}
kde
\begin{equation*}
d_1 = \frac{x}{\sqrt{2 \tau}} + \frac{1}{2}(k + 1)\sqrt{2 \tau}
\end{equation*}
a
\begin{equation*}
N(d_1) = \frac{1}{\sqrt{2 \tau}}\int^{d_1}_{-\infty}e^{-\frac{1}{2}s^2 ds}
\end{equation*}
představuje kumulativní distribuční funkci normálního rozdělení. Úprava $I_2$ je identická s výjimkou toho, že $(k + 1)$ je nahrazeno $(k - 1)$.
\begin{equation*}
I_1 = e^{\frac{1}{2}(k - 1)x + \frac{1}{4}(k - 1)^2 \tau}N(d_2)
\end{equation*}
\begin{equation*}
d_2 = \frac{x}{\sqrt{2 \tau}} + \frac{1}{2}(k - 1)\sqrt{2 \tau}
\end{equation*}
Připomeňme si výše odvozenou rovnici
\begin{equation*}
v(x, \tau) = e^{-\frac{1}{2}(k - 1)x - \frac{1}{4}(k + 1)^2 \tau}u(x, \tau)
\end{equation*}
a substituce $x = \ln \frac{S}{E}$, $\tau = \frac{1}{2}\sigma^2(T - t)$ a $C = Ev(x,\tau)$. S jejich pomocí lze elemetárními aritmetickými operacemi odvodit výslednou rovnici pro hodnotu evropské kupní opce jako
\begin{equation*}
C(S,t) = SN(d_1) - Ee^{-r(T - t)}N(d_2)
\end{equation*}
kde
\begin{equation*}
d_1 = \frac{\ln\frac{S}{E} + (r + \frac{1}{2}\sigma^2)(T - t)}{\sigma \sqrt{T - t}}
\end{equation*}
\begin{equation*}
d_2 = \frac{\ln\frac{S}{E} + (r - \frac{1}{2}\sigma^2)(T - t)}{\sigma \sqrt{T - t}}
\end{equation*}
Odpovídající rovnici pro evropskou prodejní opci lze snadno odvodit pomocí put-call parity
\begin{equation*}
C - P = S - Ee^{-r(T - t)}
\end{equation*}
a vztahu $N(-d) = 1 - N(d)$. Postupnými elementárními úpravami získáme
\begin{equation*}
P(S,t) = Ee^{-r(T - t)}N(-d_2) - SN(-d_1)
\end{equation*}
Deltu kupní a prodejní opce lze vypočíst pomocí derivace. V případě evropské kupní opce má delta tvar
\begin{equation*}
\Delta = \frac{\partial C}{\partial S}
\end{equation*}
\begin{equation*}
\Delta = N(d_1) + S \frac{\partial N(d_1)}{\partial S} - Ee^{-r(T-t)}\frac{\partial N(d_1)}{\partial S}
\end{equation*}
\begin{equation*}
\Delta = N(d_1) + S \frac{\partial N(d_1)}{\partial d_1}\frac{\partial d_1}{\partial S} - Ee^{-r(T-t)}\frac{\partial N(d_2)}{\partial d_2}\frac{\partial d_2}{\partial S}
\end{equation*}
\begin{equation*}
\Delta = N(d_1) + \frac{S \frac{\partial N(d_1)}{\partial d_1} - Ee^{-r(T-t)}\frac{\partial N(d_2)}{\partial d_2} \frac{\partial d_2}{\partial S}}{S \sigma \sqrt{T - t}}
\end{equation*}
\begin{equation*}
\Delta = N(d_1)
\end{equation*}
Poslední z výše uvedených úprav vychází ze vztahu
\begin{equation*}
S \frac{\partial N(d_1)}{\partial d_1} = Ee^{-r(T-t)}\frac{\partial N(d_2)}{\partial d_2}
\end{equation*}
Prokázání této rovnosti je však poměrně složité. Deltu prodejní opce lze odvodit pomocí put-call parity jako
\begin{equation*}
\Delta = N(d_1) - 1
\end{equation*}

\section{Technická poznámka: Bezrozměrné veličiny}

Diferenciální rovnice modelující fyzikální a finanční procesy často obsahují mnoho parametrů jako jsou příměsi materiálu a jeho teplotní vodivost nebo konstanty podkladového stochastického modelu (např. výnosová míra a volatilita). První krokem při řešení těchto rovnic je indexace jejich parametrů pomocí tzv. ``typických hodnot'' s cílem sesbírat parametry co nejvíce dohromady. V předešlé kapitole jsme parametry $S$ a $V$ indexovali pomocí $E$, což byla jediná a priori ``typická veličina''. Ačkoliv je možné $S$ měřit v CZK, EUR či USD, indexovaná veličina $x$ neměla žádné měřítko. To je důležité vzhledem k tomu, že rozvoj $e^S = 1 + S + \frac{1}{2}S^2 + ...$ postrádá smysl, je-li $S$ rozměrnou veličinou\footnote{Uvědomte si, že ačkoliv absolutní změna $dS$ ceny podkladového aktiva je rozměrnou veličinou, její relativní změna $\frac{dS}{S}$ je bezrozměrnou veličinou.}. Podobně jako $x$ také indexovaná veličina $v$ byla bezrozměrná.

Po tomto kroku je možné sesbírat zbývající parametry do tzv. bezrozměrných grup, pro které se používá také označení bezrozměrné parametry. Tímto zjistíme skutečný počet nezávislých konstant v hledaném řešení. Jestliže je hodnota jednoho z výsledných bezrozměrných parametrů příliš velká nebo naopak příliš malá, je možné tohoto využít pro přibližné řešení. Tuto aproximaci nazýváme asymptotickou expanzí a příslušnou teorii, která se snaží nalézt techniky pro odvození přibližných řešení, pak asymptotickou analýzou.

V Black-Scholes rovnici jsou parametry $r$ a $\sigma^2$ vztaženy k veličině \textit{čas}$^{-1}$. Parametry $k = \frac{r}{\frac{1}{2}\sigma^2}$ a $\frac{1}{2}\sigma^2 T$ jsou tak bezrozměrné a jedná se o jediné bezrozměrné parametry základního problému evropské kupní a prodejní opce.

\section{Binární opce}

Ačkoliv jsme se až dosud zabývali pouze plain-vanilla evropskou kupní a prodejní opcí, výplatní profil opce byl důležitý až posledních krocích úpravy. Funkce $u_0(s)$ v rovnici (5.11) může být kombinací libovolných opcí. Linearita Black-Scholes rovnice totiž umožňuje oceňovat portfolia opcí pomocí superpozice. Díky tomu je možné oceňovat jednotlivé opční strategie jako je např. straddle nebo stangle. Navíc výsledná výplata nemusí být pouze kombinací evropských kupních či prodejních opcí, ale můžeme uvažovat libovolnou funkci $S$.

Uvažujme výplatní funkci $\Lambda(S)$ v čase $T$ a hodnotu opce $V(S,t)$ v čase $t$. Platí $V(S,T) = \Lambda(S)$. Nejprve odvodíme funkci $u_0(x)$ odpovídající $\Lambda(S)$ po transformaci, kterou jsme použili výše. To znamená, že položíme $S = Ee^x$ a $V(S,t) = Ee^{\alpha x + \beta \tau}u(x, \tau)$. Z výplaty vyplývá $V(S,T) = \Lambda(S) = Ee^{\alpha x}u_0{x}$. S pomocí rovnice (5.11) je $V(S,t)$ možné vyjádřit jako
\begin{equation}
V(S,t) = \frac{e^{-r(T-t)}}{\sigma \sqrt{2 \pi (T - t)}} \int^{\infty}_{0} \Lambda(S')e^{-\frac{1}{2}\big( \frac{\ln(S'/S)-(r - 1/2 \sigma^2)(T-t)}{\sigma \sqrt{T-t}} \big)^2}\frac{dS'}{S'}
\end{equation}
Rovnice (5.14) zahrnuje mimojiné také evropské plain-vanilla kupní a prodejní opce. Deltu lze odvodit derivací (5.14) dle $S$. Při odvozování (5.14) jsme předpokládali, že $\sigma$ a $r$ jsou konstanty a že podkladové aktivum negeneruje žádné cash-flow.

Uvažujme kupní binární opci typu cash-or-nothing. V případě, že je spotová cena v době maturity vyšší než realizační cena, vyplácí opce částku $H$. V opačném případě je výplata nulová. Je-li $\Lambda(S)$ výplatní funkcí v době splatnosti opce, platí
\begin{equation*}
\Lambda(S) = B H(S - E)
\end{equation*}
S využítím rovnice (5.14) a vlastností stranové funkce $\mathcal{H}(x)$ lze odvodit cenu této opce následujícím způsobem.
\begin{equation*}
V(S,t) = \frac{e^{-r(T-t)}}{\sigma \sqrt{2 \pi (T - t)}} \int^{\infty}_{0} B H(S' - E)e^{-\frac{1}{2}\big( \frac{\ln(S'/S)-(r - 1/2 \sigma^2)(T-t)}{\sigma \sqrt{T-t}} \big)^2}\frac{dS'}{S'}
\end{equation*}
\begin{equation*}
V(S,t) = \frac{B e^{-r(T-t)}}{\sqrt{2 \pi}} \int^{\infty}_{E} \frac{1}{S' \sigma \sqrt{T - t}}e^{-\frac{1}{2}\big( \frac{\ln(S'/S)-(r - 1/2 \sigma^2)(T-t)}{\sigma \sqrt{T-t}} \big)^2} dS'
\end{equation*}
Aplikujme substituci $\rho = \frac{\ln(S'/S)-(r - 1/2 \sigma^2)(T-t)}{\sigma \sqrt{T-t}}$. Je-li $S' = E$, pak $\rho = \frac{\ln(E/S)-(r - 1/2 \sigma^2)(T-t)}{\sigma \sqrt{T-t}}$. Pro derivaci platí $d \rho = \frac{1}{S' \sigma \sqrt{T - t}} d S'$. Výše uvednou rovnice lze tedy dále upravit do tvaru
\begin{equation*}
V(S,t) = \frac{B e^{-r(T-t)}}{\sqrt{2 \pi}} \int^{\infty}_{\frac{\ln(E/S)-(r - 1/2 \sigma^2)(T-t)}{\sigma \sqrt{T-t}}}e^{-\frac{1}{2}\rho^2} d\rho
\end{equation*}
Protože je funkce $e^{-\frac{1}{2}\rho^2}$ sudá, platí
\begin{equation*}
V(S,t) = \frac{B e^{-r(T-t)}}{\sqrt{2 \pi}} \int^{\frac{\ln(S/E)+(r - 1/2 \sigma^2)(T-t)}{\sigma \sqrt{T-t}}}_{-\infty} e^{-\frac{1}{2}\rho^2} d\rho
\end{equation*}
\begin{equation*}
V(S,t) = B e^{-r(T-t)}N(d_2)
\end{equation*}
kde
\begin{equation*}
d_2 = \frac{\ln(S/E)+(r - \frac{1}{2} \sigma^2)(T-t)}{\sigma \sqrt{T-t}}
\end{equation*}

Ačkoliv je ocenění binární opce relativně snadné, je poměrně složité tuto opci zajistit těsně před jejich splatností. To je způsobeno nespojitostí výplatní funkce. Diferencováním funkce $H(S-E)$ vzhledem k $S$ zjistíme, že s $t \rightarrow T$ se delta této opce blíží $B \delta(S -E)$. Mimo bod $S = E$ je tato funkce rovna nule. Jesliže se tedy $S$ nachází v okolí $E$, je vysoká pravděpodobnost, že cena v průběhu zbytkové splatnosti protne realizační cenu a to i několikrát. Delta opce se tedy může v relativně krátké době změnit z hodnot blízkých nule hodnotu blízkou $B$ a naopak. Odvození Black-Scholes modelu předpokládá, že opce je kontinuálně zajištěna podkladovým aktivem jehož množství odpovídá její deltě. V případě binárních opcí je však tento postup značně nepraktický. To vede k otázce, zda-li je použití Black-Scholes modelu pro oceňování binárních opcí adekvátní.

\section{Rizikově neutrální svět}

Poněkud odlišným pohledem na oceňování opcí je aplikace rizikově neutrálního přístupu. Připomeňme, že míra růstu $\mu$ se ve výsledné rovnici (3.5) Black-Scholes modelu nevyskytuje. Ačkoliv tedy se cena opce odvíjí od volatility ceny podkladového aktiva, je míra jejího růstu irelevantní. Různí investoři tak mohou mít diametrálně rozlišnou představu o míře růstu $\mu$ a přesto se shodnou na ceně opce. Navíc také sklon k riziku jednotlivých investorů nehraje při oceňování opcí roli. Důvod je ten, že riziko obsažené v opci může být (alespoň teoreticky) odstraněno prostřednictvím zajištění. Výnosová míra přesahující bezrizikovou úrokou míru tedy není opodstatněná. Základní myšlenkou Black-Scholes modelu je, že je možné vytvořit bezrizikové portfolio skládající se z podkladového aktiva a derivátu, v našem konkrétním případě opce. V tomto případě je možné derivát oceňovat, jako by všechny náhodné procesy byly rizikově neutrální. To znamená, že parametr $\mu$ v rovnici (2.1) může být nahrazen bezrizikovou mírou $r$. Opce je pak oceňována současnou hodnotou očekávané výplaty v době splatnosti s využitím výše popsané modifikace náhodné procházky. Schéma postupu je následující.

Nejprve připomeňme, že současná hodnota částky $A$ v čase $T$ je rovna $Ae^{-r(T-t)}$. Přesuňme se do rizikově neutrálního světa a v rovnici náhodné procházky, která modeluje cenu podkladového aktiva $S$, nahraďme parametr $\mu$ parametrem $r$. Na základě takto modifikované náhodné procházky vypočteme hustotu pravděpodobnosti budoucích hodnot $S$. Ta je dána rovnicí (2.7)\footnote{Je důležité si uvědomit, že tato modifikovaná hustota pravděpodobnosti není skutečnou hustotou pravděpodobnosti ceny podkladového aktiva $S$.}. V dalším kroce vypočtěme s využitím modifikované hustoty pravděpodobnosti očekávané hodnoty výplatní funkce $\Lambda(S)$ - jednotlivé hodnoty $\Lambda(S)$ vynásobíme rizikově neutrální hustotou pravděpodobnosti a tento součin integrujeme přes všechny hodnoty, kterých může $S$ v čase $T$ nabývat, tj. od nuly do nekonečna. Výsledný integrál, který představuje očekávanou výplatu opce v čase $T$, je třeba diskontovat, abychom získali současnou hodnotu v čase $t$. Výsledným vzorcem je rovnice (5.14). Derivací je možné dokázat, že tato rovnice splňuje (3.5). V případě, že výplatní funkce není komplikovaná, je možné z (5.14) odvodit rovnici pro výpočet ceny příslušného typu opce, tak jak jsme to demonstrovali např. na binární opci.

Ačkoliv je myšlenka nahrazení parametru $\mu$ parametrem $r$ velice elegantní, má několiv významných negativ. V první řadě vyžaduje znalost hustoty pravděpodobnosti budoucích hodnot podkladového aktiva (v případě rizikově neturálního světa). Pro náhodnou procházku s konstatními parametry není toto příliš komplikované. V případě složitějších modelů je třeba nejprve tuto funkci odvodit a teprve po té je možné přistoupit k integrování s cílem odvodit očekávaný výnos. Další nevýhodou je to, že riziková neutralita často vede k mýlkám typu
\begin{itemize}
\item Lze dokázat, že $\mu = r$.
\item Delta opce vyjadřuje pravděpodobnost, že tato opce vyexpiruje jako at-the-money.
\end{itemize}
Jestliže by první věta byla pravdivá, pak by všechna aktiva generovala výnos odpovídající bezrizikové úrokové míře. Při vyšší volatilitě by pak neexistoval důvod, proč by investoři měli preferovat akcie před státními dluhopisy. Jestliže by platilo $\mu = r$, byla by druhá věta pravdivá. Pravděpodobnost, že $S > E$ v čase $t = T$, může být odvozena pomocí výpočtu očekávané hodnoty výrazu $H(S -E)$. Pro tento výpočet je nezbytná znalost parametru $\mu$.

Koncept rizikové neutrality není příliš intiutivní, což je zdrojem výše uvedených mýlek. Nicméně klíčové kroky při dovození Black-Scholes rovnice, totiž neexistence arbitráže a předpoklad, že bezrizikové portfolio generuje výnos odpovídající bezrizikové úrokové míře, intuitivní jsou. 

\chapter{Modifikace Black-Scholes modelu}

\section{Opce na akcie generující dividendový výnos}

Řada aktiv generuje výnos v podobě výplaty dividendy nebo úroku. Příkladem takovýchto aktiv je dluhopis nebo akcie. V této kapitole se budeme zabývat z toho plynoucími modifikacemi Black-Scholes modelu. Modifikace budeme ilustrovat na akcii jako podkladovém aktivu pro evropskou kupní opci.

Je třeba si uvědomit, že výplata dividend ovlivňuje cenu podkladové akcie a tím pádem také hodnotu odpovídající opce. Při stanovení dopadu dividendy na hodnotu opce jsou rozhodující dva faktory
\begin{itemize}
\item frekvence výplaty dividendy
\item výše dividendy
\end{itemize}
Co se frekvence výplaty týče, zaměříme se dva základní modely a to
\begin{itemize}
\item spojitou výplatu dividend (jedná se o analogii k spojitému úročení)
\item diskrétní výplatu dividend, kdy jsou dividendy vypláceny vždy k určitému předem známému časovému okamžiku
\end{itemize}
Výši dividendy je možné chápat jako stochastickou nebo jako deterministickou veličinu. V následujícím textu se zaměříme pouze na druhou z možností. Budeme tedy předpokládat, že výše dividendy je vždy dopředu známa. Vzhledem k tomu, že řada akciových společností udržuje konstatní dividendovou výplatní politiku, je tento předpoklad přijetelný také z praktického hlediska.

\subsection{Spojitá výplata dividend}

Předpokládejme, že v čase $dt$ generuje podkladová akcie dividendu $D_0Sdt$, kde $D_0$ představuje konstantu. Opomineme-li vazbu skrze $S$, je tato výplata nezávislá na čase $t$. Dividendový výnos je definovaný jako část ceny podkladové akcie vyplácené za jednotku času. Tento model je vhodný nejen pro opce na akciové indexy ale také pro krátkodobé měnové opce\footnote{Je značně diskutabilní, zda-li (2.1) představuje přijatelný model pro vývoj měnového kurzu v dlouhém časovém období.}, kde $D_0 = r_f$ představuje spojitou úrokovou mírou pro ``zahraniční'' měnu po dobu životnosti uvažované opce.

Nejprve uvažujme efekt výplaty dividendy na cenu podkladové akcie. Jestliže má být vyloučena možnost arbitráže, musí při každém časovém kroku délky $dt$ tato cena klesnout o výši vyplacené dividendy. Náhodná procházka (2.1) se tak modifikuje do podoby
\begin{equation}
dS = \sigma S dX + (\mu - D_0)S dt
\end{equation}
Ačkoliv původní Black-Scholes rovnice (3.5) nezahrnuje koeficient parametru $dt$ rovnice (2.1), bude mít $D_0$ vliv na její podobu. Připomeňme, že rovnice (3.5) vychází z myšlenky zajištěného portfolia. V rámci tohoto bezrizikového portfolia držíme $-\Delta$ jednotek podkladové akcie, z nichž každá vyplácí dividendu ve výši $D_0 S dt$. Hodnota uvažovaného portfolia se proto změní o $-D_0 S \Delta dt$. Celková změna hodnoty portfolia je tak dána rovnicí
\begin{equation*}
d\Pi = dV - \Delta dS - D_0 S \Delta dt
\end{equation*}
a rovnice (3.5) přejde do tvaru
\begin{equation}
\frac{\partial V}{\partial t} + \frac{1}{2} \sigma^2 S^2 \frac{\partial^2 V}{\partial S^2} + (r - D_0)S \frac{\partial V}{\partial S} - rV = 0
\end{equation}
V případě evropské kupní opce má konečná podmínka stále podobu
\begin{equation*}
C(S,T) = \max(S - E, 0)
\end{equation*}
stejně jako hraniční podmínka pro $S = 0$, která má tvar
\begin{equation*}
C(0,t) = 0
\end{equation*}
Jedinou změnou je hraniční podmínka pro $S \rightarrow \infty$, která má nově tvar
\begin{equation*}
C(S,t) \sim Se^{-D_0(T - t)}
\end{equation*}
Tato změna vyjadřuje skutečnost, že pro $S \rightarrow \infty$ hodnota opce odpovídá hodnotě podkladového aktiva bez dividendového výnosu.

Hodnotu kupní opce můžeme určit stejně jako předchozí kapitole - tj. zredukovat rovnici (6.2) do podoby difúzní rovnice a tu následně s pomocí počáteční a hraničních podmínek řešit. Nicméně existuje rychlejší řešení. Stačí si uvědomit, že pomocí substituce
\begin{equation}
C(S,t) = e^{-D_0(T-t)}C_1(S,t)
\end{equation}
lze rovnici
\begin{equation*}
\frac{\partial C}{\partial t} + \frac{1}{2} \sigma^2 S^2 \frac{\partial^2 C}{\partial S^2} + (r - D_0)S \frac{\partial C}{\partial S} - rC = 0
\end{equation*}
upravit do podoby
\begin{equation*}
D_0e^{-D_0(T-t)}C_1 + e^{-D_0(T-t)}\frac{\partial C_1}{\partial t} + \frac{1}{2}\sigma^2 S^2 \frac{\partial^2 C_1}{\partial S^2} + (r - D_0)e^{-D_0(T-t)}S\frac{\partial C_1}{\partial S} - r e^{-D_0(T-t)}C_1 = 0
\end{equation*}
\begin{equation*}
\frac{\partial C_1}{\partial t} + \frac{1}{2}\sigma^2 S^2 \frac{\partial^2 C_1}{\partial S^2} + (r - D_0)S\frac{\partial C_1}{\partial S} - (r - D_0)C_1 = 0
\end{equation*}
Použijeme-li $r - D_0$ namísto $r$, splňuje derivát $C_1(S,t)$ původní Black-Scholes rovnici (3.5), přičemž počáteční a hraniční podmínky zůstanou substitucí (6.3) nedotčeny. Také hodnota derivátu $C_1(S,t)$ v čase splatnosti $T$ je rovna hodnotě původní kupní opce $C(S,t)$. Hodnota evropské kupní opce je tak dána rovnicí
\begin{equation*}
C(S,t) = e^{-D_0(T-t)}SN(d_{10})-Ee^{-r(T-t)}N(d_{20})
\end{equation*}
kde
\begin{equation*}
d_{10} = \frac{\ln{\frac{S}{E}}+(r - D_0 + \frac{1}{2}\sigma^2)(T-t)}{\sigma \sqrt{T - t}}
\end{equation*}
\begin{equation*}
d_{20} = d_{10} - \sigma \sqrt{T - t}
\end{equation*}
\begin{center}
	\begin{pspicture}(0,0)(8.0,6.5)
		\rput(4.0,0.5){Hodnota evropské kupní opce s nulovým (horní křivka) a}
        \rput(4.0,0.0){nenulovým (dolní křivka) dividendovým výnosem}

		\psline[arrows=->](0.5,1.5)(7.5,1.5)
		\psline[arrows=->](0.5,1.5)(0.5,6.0)

        \psline(0.5,1.5)(3.0,1.5)(7.5,6.0)
        \pscurve(1.0,1.5)(2.5,1.80)(5.0,3.70)(7.3,5.85)
        \pscurve(1.0,1.5)(2.5,1.95)(5.0,3.82)(7.3,5.95)

        \rput(0.5,1.2){\small{0}}
        \rput(3.0,1.2){\small{E}}
        \rput(7.5,1.2){\small{S}}
        \rput(0.2,6.0){\small{C}}

	\end{pspicture}
\end{center}

\subsection{Diskrétní výplata dividend}

Předpokládejme, že podkladová akcie generuje po dobu životnosti opce pouze jednu dividendu. Dále předpokládejme, že tato dividenda bude vyplacena v čase $t_d$ a výše dividendového výnosu bude $d_y$. Majitel akcie tak v čase $t_d$ obdrží dividendu ve výši $d_y S$, kde $S$ představuje cenu akcie těsně před její výplatou.

Nejprve opět uvažujme dopad výplaty dividendy na cenu akcie. Její cena v čase $t_d^-$ těsně před výplatou dividendy se nemůže rovnat její ceně v čase $t_d^+$ těsně po výplatě dividendy. Kdyby tomu tak bylo, mohl by investor nakoupit podkladovou akcii těsně před výplatou dividendy, vyinkasovat dividendu a následně akcii prodat. Pomineme-li další faktory jako např. daně, je zřejmé, že cena akcie musí klesnout o částku odpovídající vyplácené dividendě. Proto platí
\begin{equation}
S(t_d^+) = S(t_d^-) - d_yS(t_d^-) = S(t_d^-)(1-d_y)
\end{equation}

\subsubsection{Skoková podmínka}

Prokázali jsme, že diskrétní výplata dividend vede k cenovému skoku podkladové akcie v okamžiku výplaty dividendy. Dalším krokem bude zjistit, jaký vliv má tento skok na hodnotu opce. Tato otázka nás přivádí k tzv. skokové podmínce.

O skokové podmínce hovoříme v souvislosti s nespojitou změnou v některé z nezávislých veličin, které ovlivňují hodnotu uvažovaného finančního derivátu. Mimo den, kdy je vyplácena dividenda, se hodnota opce mění v důsledku náhodných změn ceny podkladového aktiva, které jsou spojité v čase. V den výplaty dividendy se však tato cena změní skokově dle (6.4). Abychom zabránili arbitráži, musí být hodnota opce narozdíl od podkladového aktiva spojitou funkcí času i přes datum výplaty dividendy. Protože vlastník opce není příjemcem dividendy, musí být hodnota opce těsně před a těsně po výplatě dividendy stejná. Tato jednoduchá úvaha vede ke skokové podmínce
\begin{equation}
V(S(t_d^-), t_d^-) = V(S(t_d^+), t_d^+)
\end{equation}
Hodnota opce tak musí být spojitá v čase pro libovolný model vývoje ceny podkladového aktiva.

V rámci této knihy uvažujeme model hodnoty opce založený na diferenciální rovnici s $S$ a $t$ jako nezávislými veličinami. Protože potřebujeme brát v potaz všechny možné modely pro vývoj $S$, nechápeme cenu podkladového aktiva jako funkci $t$ tak, jak je popsaná rovnicí (6.4). Uvažujme o hodnotě opce $V$ jako o funkci veličiny $S$. Položme si otázku, jak se změní cena opce $S$ přes den výplaty dividendy. Odpovědí je, že se hodnota opce s ohledem na $V$ změní nespojitě dle (6.4) s veličinami $S(t_d^+)$ a $S(t_d^-)$ ``svázanými'' podmínkou (6.5). Výsledkem těchto úvah je rovnice
\begin{equation}
V(S,t_d^-) = V(S(1-d_y),t_d^+)
\end{equation}
která vyjadřuje fakt, že hodnota opce těsně před vyplatou dividendy pro cenu podkladového aktiva $S$ se rovná hodnotě opce těsně po výplatě dividendy pro cenu podkladového aktiva $S(1 - d_y)$. Kdyby bylo $S$ konstatní, změnila by se hodnota opce přes den výplaty dividendy nespojitě. Nicméně (6.6) odpovídá předpokladu, že hodnota opce je spojitá v čase pro libovolný model náhodné procházky podkladového aktiva. Spojitost hodnoty opce v čase neznamená, že není ovlivněna výplatou dividendy. Vliv skokové podmínky (6.6) je však rozložen do celé životnosti opce.

Ačkoliv se přes den výplaty dividendy nezmění hodnota opce, změní se její delta. V případě zajištění je třeba provést provést úpravu struktury portfolia.

\subsection{Kupní opce s jednou dividendovou platbou}

Zkusme ocenit evropskou kupní opci, kde podkladová akcie generuje po dobu životnosti opce jednu dividendovou platbu. Protože je Black-Scholes rovnice zpětně parabolická, postupujeme od okamžiku splatnosti opce směrem k časovému okamžiku, ke kterému chceme opci ocenit. V případě, že je vyplácena dividenda, je postup následující
\begin{itemize}
\item řešíme Black-Scholes rovnici od okamžiku splatnosti opce do časového okamžiku těsně před výplatou dividendy, tj. do času $t_d^+$
\item aplikujeme skokovou podmínku (6.6) přes časový okamžik $t_d$, abychom našli hodnotu opce v $t_d^-$
\item řešíme Black-Scholes rovnici od $t_d^-$ směrem k časovému okamžiku ocenění opce s využitím výše odvozené hodnoty opce jako výchozích dat
\end{itemize}
V souladu s tímto postupem tak řešíme Black-Scholes rovnici dvakrát a to pro časový interval $T > t > t_d$ a následně pro časový interval $t_d > t > 0$, kde 0 představuje okamžik ocenění opce, tj. ve většině případů dnešek. Hodnoty opce v $t_d^{\pm}$ jsou provázány skrze (6.6). Velmi podobný přístup je volen při oceňování některých exotických opcí.

Označme námi uvažovanou kupní opci jako $C_d(S,t)$. Dále použijme označení $C(S, t, E)$ pro hodnotu plain-vanilla evropské kupní opcen s realizační cenou $E$. Po výplatě dividendy je opce $C_d(S,t)$ shodná s opcí $C(S,t,E)$.
\begin{equation*}
C_d(S,t) = C(S,t,E)~~~t_d^+ \le t \le T
\end{equation*}
Nyní použijeme vztah (6.6) a získáme
\begin{equation*}
C_d(S, t_d^-) = C_d(S(1-d_y),t_d^+) = C(S(1-d_y),t_d^+,E)
\end{equation*}
V této fázi můžeme použít standardní rovnici pro výpočet hodnoty evropské kupní opce. Zaměřme se nyní na $C(S(1 - d_y),t,E)$. V době splatnosti má tento derivát hodnotu
\begin{equation*}
C(S(1-d_y),T,E) = \max(S(1 - d_y) - E, 0) = (1 - d_y)\max(S - E(1 - d_y)^{-1},0)
\end{equation*}
Proto je možné $C_d(S,t)$ vyjádřit jako
\begin{equation*}
C_d(S,t) = (1 - d_y)C(S, t, E(1 - d_y)^{-1})
\end{equation*}
Výplata dividendy tak snižuje hodnotu opce. Logická interpretace je taková, že majitel opce dividendu nezíská, avšak hodnota podkladového aktiva se v důsledku její výplaty sníží.

\subsection{Technická poznámka: Sjednocení spojité a diskrétní výplaty dividend}

Předpokládejme, že výplata dividendy je obecnou funkcí $S$ a $t$ ve tvaru $D(S,t)$. V případě spojité výplaty dividendy má tato funkce tvar $D(S,t) = D_0S$ zatímco v případě diskrétní výplaty tvar $D(S,t) = D_{\delta}S \delta(t - t_d)$, kde $D_{\delta}$ představuje konstantu jejíž vazbu na $d_y$ objasníme níže. Náhodná procházka modelující vývoj ceny podkladového aktiva má tvar
\begin{equation*}
dS = (\mu S - D(S,t))dt + \sigma S dX
\end{equation*}
Je-li výplata dividend diskrétní, přejde tato rovnice do tvaru
\begin{equation*}
dS = (\mu S - D_{\delta}S \delta(t - t_d))dt + \sigma S dX
\end{equation*}
Intergrováním přes časový okamžik výplaty dividendy získáváme
\begin{equation*}
\int_{S(t_d^-)}^{S(t_d^+)} \frac{dS}{S} = \int_{t_d^-}^{t_d^+}\mu dt - D_{\delta}\int_{t_d^-}^{t_d^+} + \int_{t_d^-}^{t_d^+}\sigma dX
\end{equation*}
Protože $t_d^-$ a $t_d^+$ se liší pouze infinitezimálně, je jediným nenulovým členem výše uvedeného integrálu člen obsahující delta funkci. Integrál tak přejde to tvaru
\begin{equation*}
\int_{S(t_d^-)}^{S(t_d^+)} \frac{dS}{S} = -D_{\delta} \int_{t_d^-}^{t_d^+}\delta(t - t_d)dt
\end{equation*}
\begin{equation}
\ln \frac{S(t_d^+)}{S(t_d^-)} = -D_{\delta}
\end{equation}
Proto je v případě výplaty dividendy definované jako $D_{\delta}S \delta (t - t_d)$ cena podkladového aktiva diskontována členem $e^{D_{\delta}H(t - t_d)}$ přičemž $D_{\delta} = -\ln(d_y)$.

Pro libovolný model vývoje ceny podkladového aktiva je hodnota opce spojitá, a proto musí být opět splněna skoková podmínka
\begin{equation*}
V(S(t_d^+), t_d^+) = V(S(t_d^-), t_d^-)
\end{equation*}
 kde $S(t_d^+)$ a $S(t_d^-)$ jsou propojeny skrze (6.7).

\section{Forwardové a futures kontrakty}

Oceňování forwardových a futures kontraktů je v porovnání s opcemi mnohem jednodušší. Veškeré riziko je totiž možné zajistit jedinou operací na začátku kontraktu\footnote{V případě opcí je nutné zajištění neustále ``korigovat'', což v případě forwardových a futures kontraktů není zapotřebí.}. Pro samotné ocenění pak není rozhodující vývoj ceny pokladového aktiva - jediným nutným předpokladem je znalost budoucího vývoje úrokových sazeb. Protože jsou však forwardové a futures kontrakty stejně jako opce finančními deriváty, je možné také pro jejich ocenění použít Black-Scholes model. V této kapitole se budeme zabývat oceněním forwardových kontraktů\footnote{Přimeňme, že z hlediska ocenění není mezi forwardovým a futures kontraktem zásadnější rozdíl.}.

Uvažujme forwardový kontrakt, který je uzavřen v čase $t$ při spotové ceně podkladového aktiva $S(t)$ a forwardové ceně $F$. Naším úkolem tak bude nalézt vztah mezi $S(t)$ a takovou forwardovou cenou $F$, pro kterou bude hodnota kontraktu v čase $t$ nulová. V rámci naší analýzy budeme předpokládat konstantní úrokové sazby po dobu životnosti forwardového kontraktu.

Existuje několik způsobů, jak ocenit forwardový kontrakt. Základním přístupem je ocenění pomocí předpokladu neexistence arbitráže. Uvažujme investora, který je krátký ve forwardovém kontraktu. Tento investor tedy bude muset v době splatnosti, tj. v čase $T$, dodat podkladové aktivum za cenu $F$. Možností, jak se zajistit proti riziku tohoto typu kontraktu, je nakoupit v čase $t$ podkladové aktivum za cenu $S(t)$ a toto aktivum držet až do časového okamžiku $T$. V čase $T$ získá investor za toto aktivum $F$ peněžních jednotek. Současná hodnota této částky je $Fe^{-r(T-t)}$. ``Spravedlivá'' forwardová cena je tak
\begin{equation*}
F = S(t)e^{r(T-t)}
\end{equation*}

Dalším způsobem ocenění forwardového kontraktu je jeho ``rozložení'' na dlouhou pozici v evropské kupní opci a krátkou pozici v evropské prodejní opci, kde obě opce musí mít shodnou realizační cenu a čas do splatnosti jako původní forwardový kontrakt\footnote{Tato dekompozice forwardového kontraktu na dvojici opcí je reformulací put-call parity.}. Vzhledem k tomu, že hodnota forwardového kontraktu je v okamžiku jeho uzavření nulová, musí pro realizační cenu opcí $E$ platit
\begin{equation*}
S(t) - Ee^{-r(T-t)} = 0
\end{equation*}
Vzhledem k tomu, že $E = F$, získáváme
\begin{equation*}
F = S(t)e^{r(T-t)}
\end{equation*}

Poslední způsobem ocenění forwardového kontraktu je pomocí Black-Scholes modelu. Vzhledem k tomu, že forwardový kontrakt je finančním derivátem, musí splňovat rovnici (3.5). Výplata tohoto derivátu v čase $T$ je rovna $S-F$. Klíčem k ocenění forwardového kontraktu je pak rovnice (5.14).
\begin{equation}
V(S,t) = \frac{e^{-r(T - t)}}{\sigma \sqrt{2 \pi (T - t)}} \int^{\infty}_0 (S' - F)e^{-\frac{1}{2} \big( \frac{\ln (S'/S) - (r - \frac{1}{2}\sigma^2)(T - t)}{\sigma \sqrt{T - t}} \big)^2} \frac{1}{S'}dS'
\end{equation}
Nyní si připomeňme, že hustota pravděpodobnosti ceny podkladového aktiva $S'$ v čase $T$ je
\begin{equation*}
\frac{1}{\sigma \sqrt{2 \pi (T - t)}} e^{-\frac{1}{2} \big( \frac{\ln (S'/S) - (\mu - \frac{1}{2}\sigma^2)(T - t)}{\sigma \sqrt{T - t}} \big)^2}
\end{equation*}
a že se pohybujeme v rizikově neutrálním světě, který umožňuje úplnou eliminaci rizika. Proto můžeme v náhodné procházce podkladového aktiva $S$ nahradit $\mu$ bezrizikovou úrokovou sazbou $r$. Hustota pravděpodobnosti se tak modifikuje do podoby
\begin{equation*}
\frac{1}{\sigma \sqrt{2 \pi (T - t)}} e^{-\frac{1}{2} \big( \frac{\ln (S'/S) - (r - \frac{1}{2}\sigma^2)(T - t)}{\sigma \sqrt{T - t}} \big)^2}
\end{equation*}
Pouze připomeňme, že se jedná o hustotu pravděpodobnosti náhodné veličiny $S'$ v rizikově neutrálním nikoliv reálném světě. Očekávanou hodnotu náhodné veličiny $S'$ v rizikově neutrálním světě v čase $T$ je tak dána rovnicí
\begin{equation*}
E[S'] = \frac{1}{\sigma \sqrt{2 \pi (T - t)}}\int^{\infty}_0 e^{-\frac{1}{2} \big( \frac{\ln (S'/S) - (r - \frac{1}{2}\sigma^2)(T - t)}{\sigma \sqrt{T - t}} \big)^2}dS'
\end{equation*}
V rizikově neutrálním světě musí platit
\begin{equation*}
E[S'] = S(t)e^{r(T-t)}
\end{equation*}
Rovnici (6.8) tak lze upravit do tvaru
\begin{equation*}
V(S,t) = S(t) - F e^{-r(T - t)}\frac{1}{S ' \sigma \sqrt{(T - t)}}\frac{1}{\sqrt{2 \pi}} \int^{\infty}_0 e^{-\frac{1}{2} \big( \frac{\ln (S'/S) - (r - \frac{1}{2}\sigma^2)(T - t)}{\sigma \sqrt{T - t}} \big)^2} dS'
\end{equation*}
\begin{equation*}
V(S,t) = S(t) - e^{-r(T - t)} F \frac{1}{\sqrt{2 \pi}} \int^{\infty}_{-\infty}e^{-\rho^2}d\rho
\end{equation*}
\begin{equation*}
V(S,t) = S(t) - e^{-r(T - t)} F N(\infty)
\end{equation*}
\begin{equation*}
V(S,t) = S(t) - e^{-r(T - t)} F
\end{equation*}
Protože pro forwardový kontrakt platí, že jeho hodnota je v okamžiku uzavření $t$ nulová, získáváme
\begin{equation*}
F = e^{r(T - t)} S(t)
\end{equation*}
Až dosud jsme předpokládali, že podkladové aktivum nevyplácí dividendy. Jestliže podkladové aktivum generuje konstatní dividendový výnos $D_0$, modifikuje se výše uvedená rovnice do tvaru
\begin{equation*}
F = e^{(r - D_0)(T - t)} S(t)
\end{equation*}
Parametr $D_0$ může být i záporný - např. zlato vyžaduje úhradu pojištění a nákladů spojených s uskladněním.

\section{Opce na futures}

Řada opcí má jako podkladové aktivum futures kontrakty. Důvod je ten, že v řadě případů jsou futures kontrakty v porovnání s podkladovým aktivem likvidnější a jsou s nimi spojeny nižší transakční náklady\footnote{Typickým příkladem jsou futures kontrakty na ropu.}. Hodnota opcí na futures kontrakty je funkcí $F$ a $t$, tj. má tvar $V(F,t)$. Protože
\begin{equation*}
F = Se^{r(T - t)}
\end{equation*}
je možné odvodit parciální diferenciální rovnici pro $V(F,t)$ na základě standardní Black-Scholes rovnice a to tak, že $S$ nahradíme $Fe^{-r(T - t)}$. V rovnici (3.5) nahradíme člen
\begin{equation*}
\frac{\partial V}{\partial S}
\end{equation*}
výrazem
\begin{equation*}
\frac{\partial V}{\partial F}\frac{\partial F}{\partial S} = e^r(T - t)\frac{\partial V}{\partial F}
\end{equation*}
člen
\begin{equation*}
\frac{\partial^2 V}{\partial S^2}
\end{equation*}
výrazem
\begin{equation*}
\frac{\frac{\partial V}{\partial S}}{\partial S} = \frac{\frac{\partial V}{\partial F}\frac{\partial F}{\partial S}}{\partial F}\frac{\partial F}{\partial S} = (e^{r(T - t)})^2 \frac{\partial^2 V}{\partial F^2}
\end{equation*}
a konečně člen
\begin{equation*}
\frac{\partial V}{\partial t}
\end{equation*}
výrazem
\begin{equation*}
\frac{\partial V}{\partial t} + \frac{\partial V}{\partial F}\frac{\partial F}{\partial t} = \frac{\partial V}{\partial t} - rF\frac{\partial V}{\partial F}
\end{equation*}
Rovnice (3.5) se tak zmodifikuje do podoby
\begin{equation}
\frac{\partial V}{\partial t} + \frac{1}{2}\sigma^2\frac{\partial^2 V}{\partial F^2} - rV = 0
\end{equation}

Rovnici (6.9) je možné odvodit také přímo pomocí It\^o lemmy (2.5). Nejprve uvažujme nádhodnou procházku pro $F$. Vzhledem k tomu, že $F = Se^{r(T - t)}$, platí
\begin{equation*}
dF = e^{r(T-t)}dS - e^{r(T - t)}rSdt
\end{equation*}
\begin{equation*}
dF = (\sigma S dX + \mu S dt)e^{r(T - t)} - r Fdt
\end{equation*}
\begin{equation}
dF = (\mu - r)F dt + \sigma F dX
\end{equation}
Podle It\^o lemmy platí pro $dV$
\begin{equation}
dV = \sigma F \frac{\partial V}{\partial F}dX + \bigg( (\mu - r) F \frac{\partial V}{\partial F} + \frac{1}{2}\sigma^2 F^2\frac{\partial^2 V}{\partial F^2} + \frac{\partial V}{\partial t} \bigg)dt
\end{equation}
Dále uvažujme bezrizkové portfolio $\Pi = V - F$. Změna hodnoty portfolia $d \Pi$ je tak s ohledem na (6.10) a (6.11) rovna
\begin{equation*}
d \Pi = dV - \Delta dF
\end{equation*}
\begin{equation*}
d \Pi = \sigma F \frac{\partial V}{\partial F}dX + \bigg( (\mu - r) F \frac{\partial V}{\partial F} + \frac{1}{2}\sigma^2 F^2\frac{\partial^2 V}{\partial F^2} + \frac{\partial V}{\partial t} \bigg)dt - \Delta(\sigma F dX + (\mu -r)Fdt)
\end{equation*}
Jestliže zvolíme $\Delta = \frac{\partial V}{\partial F}$, zbavíme se náhodné složky $dX$. Výsledný tvar pro $d\Pi$ je tak
\begin{equation*}
d \Pi = \bigg( \frac{1}{2}\sigma^2 F^2\frac{\partial^2 V}{\partial F^2} + \frac{\partial V}{\partial t} \bigg)dt
\end{equation*}
K sestavení portfolia $\Pi$ v čase $t$ je zapotřebí částka $V$. Vzhledem k tomu, že je portfolio $\Pi$ bezrizikové, musí platit
\begin{equation*}
d \Pi = r V dt
\end{equation*}
\begin{equation*}
\bigg( \frac{1}{2}\sigma^2 F^2\frac{\partial^2 V}{\partial F^2} + \frac{\partial V}{\partial t} \bigg)dt - r V dt = 0
\end{equation*}
\begin{equation*}
\frac{\partial V}{\partial t} + \frac{1}{2}\sigma^2 F^2\frac{\partial^2 V}{\partial F^2} - r V = 0
\end{equation*}
Tímto způsobem jsme odvodili rovnici (6.9).

Protože (6.9) odpovídá Black-Scholes rovnici pro podkladové aktivum generující výnosovou míru $r$, může výše odvozený výsledek využít pro ocenění opce na futures kontrakt. V případě evropské kupní opce je její hodnota dána rovnicí
\begin{equation*}
C(F,t) = e^{-r(T - t)}(FN(d_1) - EN(d_2))
\end{equation*}

\section{Parametry Black-Scholes modelu jako funkce času}

Až dosud jsme předpokládali, že bezriziková sazba $r$ a volatilita ceny podkladového aktiva $\sigma^2$ jsou konstanty. V následujícím textu budeme předpokládat, že se jedná o funkce času, jejichž průběh je znám\footnote{Bezriziková sazba $r$ a volatilita ceny podkladového aktiva $\sigma^2$ jako stochatické veličiny by vedly z analytického hlediska k příliš komplikovaným modelům.}.

Jestliže nahradíme konstanty $r$ a $\sigma^2$ funkcemi $r(t)$ a $\sigma(t)^2$, zůstane základní rovnice (3.5) Black-Scholes modelu nezměněna stejně jako ``konečné'' a hraniční podmínky. Vyjímkou je hraniční podmínka pro evropskou prodejní opci, která se z
\begin{equation*}
P(0,t) = Ee^{-r(T - t)}
\end{equation*}
změní na
\begin{equation*}
P(0,t) = Ee^{-\int_t^T r(\tau) d \tau}
\end{equation*}
Modifikovaná rovnice (3.5) má podobu
\begin{equation}
\frac{\partial V}{\partial t} + \frac{1}{2}\sigma(t)^2 S^2\frac{\partial^2 V}{\partial S^2} + r(t)S \frac{\partial V}{\partial S} - r(t)V = 0
\end{equation}
Proveďme následující substituce
\begin{equation*}
\bar{S} = Se^{\alpha(t)}
\end{equation*}
\begin{equation*}
\bar{V} = Ve^{\beta(t)}
\end{equation*}
\begin{equation*}
\bar{t} = \gamma (t)
\end{equation*}
kde $\alpha$, $\beta$ a $\gamma$ vybereme tak, abychom v rovnici (6.12) eliminovali všechny koeficienty, které jsou funkcí času. S přihlédnutím k výše uvedeným substitucím se (6.12) přetransformuje do tvaru
\begin{equation}
\dot{\gamma}(t)\frac{\partial \bar{V}}{\partial \bar{t}} + \frac{1}{2}\sigma(t)^2 \bar{S}^2 \frac{\partial^2 \bar{V}}{\partial \bar{S}^2} + (r(t) + \dot{\alpha}(t))\bar{S} \frac{\partial \bar{V}}{\partial \bar{S}} - (r(t) + \dot{\beta}(t))\bar{V}
\end{equation}
kde $\cdot = d /dt$. Jak již bylo zmíněno výše, lze v rovnici (6.13) eliminovat členy obsahující $\bar{V}$, $\partial \bar{V} / \partial \bar{S}$ a $\partial^2 \bar{V} / \partial \bar{S}^2$ pomocí vhodné volby parametrů $\alpha$, $\beta$ a $\gamma$. Použijeme-li substituce
\begin{equation*}
\alpha(t) = \int_t^Tr(\tau) d \tau
\end{equation*}
\begin{equation*}
\beta(t) = \int_t^Tr(\tau) d \tau
\end{equation*}
\begin{equation*}
\gamma(t) = \int_t^T \sigma(\tau)^2 d \tau
\end{equation*}
zjednodušší se (6.13) na
\begin{equation}
\frac{\partial \bar{V}}{\partial \bar{t}} = \frac{1}{2} \bar{S}^2 \frac{\partial^2 \bar{V}}{\partial \bar{S}^2}
\end{equation}
Tato rovnice neobsahuje žádné členy, které by byly funkcí času a postrádá jakýkoliv odkaz na $r$ a $\sigma^2$. Jestliže je $\bar{V}(\bar{S}, \bar{t})$ je řešením (6.14), pak dopovídající řešení původní rovnice (6.12) má tvar
\begin{equation}
V = e^{-\beta(t)} \bar{V}(Se^{\alpha(t)}, \gamma{t})
\end{equation}
Označme libovolné řešení Black-Scholes diferenciální rovnice jako $V_{BS}$ pro konstantní $r$ a $\sigma^2$ a nulový dividendový výnos. S ohledem na výše odvozené je zřejmé, že toto řešení lze vyjádřit jako
\begin{equation}
V_{BS} = e^{-r(T - t)}\bar{V}_{BS}(Se^{r(T - t)},\sigma^2(T-t))
\end{equation}
Je užitečné si uvědomit, že rovnici (6.15) lze z rovnice (6.16) získat pomocí substitucí
\begin{equation*}
r = \frac{1}{T - t} \int^T_t r(\tau) d \tau
\end{equation*}
\begin{equation*}
\sigma^2 = \frac{1}{T - t} \int^T_t \sigma(\tau)^2 d \tau
\end{equation*}
Z těchto substitucí je vyplývá, že v klasickém Black-Scholes modelu s konstantní bezrizikovou úrokovou sazbou a volatilitou mají tyto parametry charakter průměrných hodnot po dobu zbytkové životnosti příslušné opce.

\chapter{Americké opce}

\section{Úvod}

Připomeňme, že americké opce se od evropských liší tím, že mohou být uplatněny kdykoliv v průběhu své životnosti. Protože má její majitel v porovnání s evropskou opcí větší práva, měla by být hodnota americké opce vyšší.
\begin{center}
	\begin{pspicture}(0,0)(8.0,6.5)
		\rput(4.0,0.5){Vnitřní a časová hodnota evropské prodejní opce}
        \rput(4.0,0.0){před splatností jako funkce $S$}

		\psline[arrows=->](0.5,1.5)(7.5,1.5)
		\psline[arrows=->](0.5,1.5)(0.5,6.0)

        \psline(0.5,5.5)(4.5,1.5)
        \pscurve[linewidth=0.5mm](0.5,5.2)(4.0,1.9)(4.5,1.7)(5.5,1.6)(6.5,1.55)

        \rput(0.5,1.2){\small{0}}
        \rput(4.5,1.2){\small{E}}
        \rput(7.5,1.2){\small{S}}
        \rput(0.2,6.0){\small{P}}
	\end{pspicture}
\end{center}
Následující argument vycházející z neexistence arbitráže dokazuje, že hodnota americké prodejní opce musí být vyšší než hodnota odpovídající evropské opce. Jak je patrné z výše uvedeného obrázku, pro dostatečně malá $S$ platí, že hodnota evropské opce je nižší než její vnitřní hodnota. Předpokládejme, že $S$ splňuje tuto podmínku, tj.
\begin{equation*}
P(S,t) < \max(E - S, 0)
\end{equation*}
a uvažujme, co by stalo v případě předčasného uplatnění opce. Protože $S < E$, lze výše uvedenou nerovnost zjednodušit do podoby $P(S,t) < E - S$. Investor by mohl na trhu koupit prodejní opci za cenu $P$, podkladové aktivum za cenu $S$ a následně opci uplatnit a podkladové aktivum prodat za cenu $E$. Jeho bezrizikový výnos by tak byl roven $E - S - P$. Vzhledem k tomu, že tato situace může existovat pouze po nekonečně krátký časový okamžik, musí pro americkou prodejní opci platit nerovnost
\begin{equation*}
P(S,t) \ge \max(E - S, 0)
\end{equation*}
Americká a evropská prodejní opce tak musí být odlišnou hodnotu.

Další agrument se zabývá hodnotou americké kupní opce na podkladové aktivum, které generuje nenulový dividendový výnos $D_0$. Připomeňme, že pro dostatečně velká $S$ hodnota evropské kupní opce přibližně splňuje podmínku
\begin{equation*}
C(S,t) \sim Se^{-D_0(T - t)}
\end{equation*}
Pro dostatečně velká $S$ a nenulové $D_0$ tak platí
\begin{equation*}
C(S,t) < \max(S - E,0)
\end{equation*}
Protože $S > E$, lze tuto nerovnost dále zjednodušit na tvar $C(S,t) < S - E$. Jestliže by měl investor možnost předčasného uplatnění opce, mohl by nakoupit opci za cenu $C$, tuto opci uplatnit a za cenu $E$ koupit podkladové aktivum a následně toto aktivum prodat za $S$. Bezrizikový výnos, který tak investor získá je $S - E - C$. Vzhledem k tomu, že tato arbitráž nemůže trvat po delší dobu, musí pro americkou kupní opci platit
\begin{equation*}
C(S,t) \ge \max(S - E, 0)
\end{equation*}

V obou výše uvedených příkladech musí existovat hodnoty $S$, pro které je z pohledu majitele optimální opci předčasně uplatnit. V opačném případě by nebyla americká opce nikdy předčasně uplatněna a její hodnota by byla shodná s evropskou opcí. Ocenění americké opce je tak v porovnání s evropskou opcí komplikovanější, protože v každém časovém okamžiku musíme nejen určit hodnotu opce, ale také rozhodnout, zda-li je optimální tuto opci předčasně uplatit. V souvislosti s touto problematikou pak hovoříme o tzv. problému volné hraniční podmínky. Pro každý časový okamžik $t$ existuje určitá hodnota $S$, která představuje hranici mezi dvěma regiony - (a) pro ceny podkladového aktiva menší než tato hraniční cena je optimální opci předčasně uplatnit a (b) pro ceny větší než tato hraniční cena je naopak racionální opci nadále držet\footnote{V praxi může existovat vícero takovýchto hraničních cen $S$. Prozatím však budeme předpokládat, že existuje pouze jedna takováto cena.}. Tuto hraniční cenu značíme jako $S_f(t)$. Tuto cenu předem neznáme, a proto v porovnání s evropskými opcemi postrádáme jednu z informací.

Analýzu problému volné hraniční podmínky, který je klíčem k oceňování amerických opcí, začneme představením tzv. problému překážky. Důvodem je, že americké opce a problém překážky pojí řada podobných vlastností.

\section{Problém překážky}

Ve své nejjednodušší verzi je problém překážky definován následovně. Uvažujme pružnou strunu nataženou mezi body $A$ a $B$ a spojitý objekt\footnote{Spojitost v tomto případě chápeme v matematickém slova smyslu.}, který pod danou strunou prochází a napíná ji tak.
 \begin{figure}
  \begin{center}
	\begin{pspicture}(0,0)(9.0,4.5)

		\psline(0.5,0.5)(8.5,0.5)

        \pscurve[linewidth=0.5mm](2.0,0.5)(3.5,3.5)(4.5,4.0)(5.5,3.5)(6.5,0.5)
        \psline(1.0,0.5)(3.5,3.5)
        \psline(5.5,3.5)(8.0,0.5)

        \rput(1.0,0.0){\small{$A$}}
        \rput(8.0,0.0){\small{$B$}}
        \rput(3.2,3.8){\small{$P$}}
        \rput(5.6,3.8){\small{$Q$}}
	\end{pspicture}
 \end{center}
 \begin{center}
  \caption{\label{obstacle} Ilustrace klasického problému překážky}
 \end{center}
\end{figure}
Vymezení části povrchu překážky, na které dochází ke kontaktu se strunou, předem neznáme. Víme však, že struna se buďto dotýká překážky, čímž je její pozice dána, nebo představuje tečnu k překážce vedené z bodu $A$ popř. $B$ a musí splňovat rovnici pohybu. Struna musí dále splňovat dvě podmínky. První podmínka říká, že struna leží na překážce\footnote{V tomto případě se tedy jedná o bod z oblasti dotyku mezi strunou a překážkou.} nebo nad ní\footnote{V tomto případě se zase jedná o body, kdy struna plní funkci tečny k překážce.}. Z této podmínky v kombinaci s pohybovou rovnicí vyplývá, že křivost struny musí být záporná nebo nulová. Druhou podmínkou je, že směrnice struny musí být spojitá. To je zřejmé s vyjímkou bodů, kde struna prvně ztrácí kontakt s překážkou. Vzhledem k tomu, že překážka je spojitá, musí být spojitá také směrnice struny. Struna tedy musí splňovat následující čtveřici podmínek
\begin{itemize}
\item struna musí být na nebo nad překážkou
\item struna musí mít zápornou nebo nulovou křivost
\item struna musí být spojitá
\item směrnice struny musí být spojitá
\end{itemize}
Jsou-li výše uvedené podmínky splněny, lze dokázat, že problém překážky má právě jedno řešení. Ačkoliv je struna a její směrnice spojitá, může její křivost (a tím pádem také její druhá derivace) vykazovat nespojitosti.

\section{Americká opce jako problém obecné hraniční podmínky}

Lze dokázat, že problém ocenění americké opce lze jednoznačně formulovat sérií podmínek podobných těm, které jsme uváděli v případě problému překážky. Jedná se o následující podmínky
\begin{itemize}
\item hodnota opce musí být větší nebo rovna vnitřní hodnotě opce
\item Black-Scholes rovnice je nahrazena nerovností (viz. dále)
\item hodnota opce musí být spojitou funkcí ceny podkladového aktiva $S$
\item delta opce, tj. její směrnice, musí být spojitá
\end{itemize}
První z těchto podmínek vylučuje možnost arbitráže. Bezrizikový výnos z případného předčasného uplatnění opce musí být nulový nebo záporný\footnote{To ovšem neznamená, že by americká opce neměla být nikdy předčasně uplatněna.}. Jestliže je tedy hodnota opce rovna její vnitřní hodnotě, je optimální opci uplatnit; v případě, že je hodnota opce vyšší, splňuje Black-Scholes rovnici a je naopak optimální ji držet. První dvě podmínky tak lze spojit do jedné nerovnosti. Jedná se o Black-Scholes nerovnost a tato druhá podmínka tak v sobě zahrnuje také první z podmínek.

Třetí podmínka požadující spojitost hodnoty opce jako funkce $S$, opět vychází z podmínky neexistence arbitráže. Jestliže by hodnota opce byla nespojitou funkcí $S$, pak by bylo možné sestavit takové portfolio, které by generovalo bezrizikový výnos v bodě nespojitosti.

Stejně jako v případě problému překážky ani v případě americké opce neznáme dopředu hodnotu $S_f$. Abychom mohli jednoznačně určit hodnotu opce, je tak nutné pro $S_f$ stanovit dvě dodatečné podmínky. První podmínkou je požadavek spojitosti hodnoty americké opce jako funkce $S$. Druhou podmínkou výše definovaná čtvrtá podmínka požadující, aby delta příslušné opce byla také spojitou funkcí $S$. Odvození této podmínky je však relativně obtížné a přesný postup přesahuje rozsah této knihy.

\subsection{Americká prodejní opce}

Uvažujme americkou prodejní opci s hodnotou $P(S,t)$, vnitřní hodnotu této opce definovanou jako $\max(E - S, 0)$ a hraniční cenu $S_f$. Je-li cena podkladového aktiva nižší než $S_f$, je racionální opci uplatnit; v opačném případě je naopak racionální opci držet. Vzhledem k nerovnosti $P(S,t) \ge \max(E - S, 0)$ je tedy bod $S_f$ bodem, kde se funkce $P(S,t)$ dokýká funkce $\max(E - S, 0)$. Protože $S_f > E$, je směrnice funkce $\max(E - S, 0)$ v tomto bodě rovna -1. Směrnice hodnoty opce\footnote{Připomeňme, že směrnice hodnoty opce je její deltou.} definovaná jako $\partial P / \partial S$ může mít v tomto bodě hodnotu\footnote{Od možnosti, že směrnice hodnoty opce v tomto bodě není definovaná, prozatím odhlédněme.}
\begin{itemize}
\item menší než -1
\item větší než -1
\item rovnu -1
\end{itemize}
Lze dokázat, že první dvě možnosti nejsou slučitelné s teorií oceňování americké opce.

Nejprve uvažujme situaci $\partial P / \partial S < -1$, kterou ilustruje křivka (a) níže uvedeného obrázku. Hodnota opce $P(S,t)$ protíná vnitřní hodnotu opce $\max(E - S, 0)$ v bodě $S = S_f^{(a)}$. Jestliže se $S$ infinitizimálně zvýší, hodnota opce $P(S,t)$ klesne pod vnitřní hodnotu opce, protože směrnice hodnoty opce je menší než -1, kdežto směrnice vnitřní hodnoty opce je rovna -1. To však popírá podmínku $P(S,t) \ge \max(E - S, 0)$, a proto můžeme tuto možnost vyloučit.

Dále uvažujme druhou variantu $\partial P / \partial S > -1$, kterou ilustruje křivka (b). V tomto případě lze dokázat, že hodnota opce není z pohledu jejího vlastníka optimální. Nepředstavuje totiž nejvyšší možnou hodnotu, která je konzistentní s Black-Scholes koncepcí bezrizikového zajištění a podmínkou $P(S,t) \ge \max(E - S, 0)$. Uvažujme stratetii uplatněnou majitelem opce. V rámci této strategie existují dva aspekty, které je třeba brát v potaz. Prvním je zajištění portfolia prováděné na denní bázi vedoucí k Black-Scholes rovnici. Tento aspekt je obsahuje také evropská opce. Druhým aspektem je možnost předčasného uplatnění opce. Majitel opce se musí stanovit cenu podkladového aktiva, od které je optimální opci uplatnit. Hodnotícím kritériem je přitom maximalizace hodnoty opce z pohledu majitele. Protože hodnota opce splňuje parciální diferenciální rovnici s hraniční podmínkou $P(S_f(t),t) = E - S_f(t)$, ovlivní volba $S_f$ hodnotu opce $P(S,t)$ pro všechna $S > S_f$. Je zřejmé, že křivka (a) představuje příliš malou hodnotu $S_f$, kdy bezrizikového výnosu je možné dosáhnout pro hodnoty $S$ je nepatrně větší než $S_f$. Naproti tomu křivka (b) ilustruje problém příliš vysoké hodnoty $S_f$. Pro $\partial P / \partial S > -1$ může být oproti bodu $S = S_f^{(b)}$ hodnota opce zvýšena pouhým snížením hodnoty $S_f$. Zvýšení hodnoty opce se prostřednictvím particiální diferenciální rovnice promítne do všech hodnot $S$ větších než $S_f$. Funkce hodnoty americké opce tak mění tvar při každé změně $S_f$. Postupným snižováním $S_f$ bychom zkonvergovali k bodu $S_f$ mezi $S_f^{(a)}$ a $S_f^{(b)}$, pro který by platilo $\partial P / \partial S = -1$. Tento bod současně maximalizuje hodnotu opce z pohledu majitele a zároveň nevytváří prostor pro případnou arbitráž. 

\begin{center}
	\begin{pspicture}(0,0)(9.0,6.5)
        \rput(4.5,0.2){Hraniční cena (a) příliš nízká (b) příliš vysoká}

		\psline(0.5,1.0)(8.5,1.0)
        \psline(0.5,1.0)(0.5,6.0)

        \psline[linewidth=0.5mm](0.5,5.5)(5.0,1.0)(8.0,1.0)
        \pscurve(0.8,5.5)(2.2,2.5)(7.0,1.3)
        \psline[linestyle=dotted](0.9,5.1)(0.9,1.0)
        \pscurve(2.2,3.5)(4.0,2.8)(7.0,2.5)
        \psline[linestyle=dotted](2.75,3.2)(2.75,1.0)
        
        \rput(5.0,0.8){\small{$E$}}
        \rput(0.9,0.7){\small{$S_f^{(a)}$}}
        \rput(2.8,0.7){\small{$S_f^{(b)}$}}
        \rput(5.0,0.8){\small{$E$}}
        \rput(8.5,0.8){\small{$S$}}
        \rput(2.0,2.2){\small{$(a)$}}
        \rput(5.0,3.0){\small{$(b)$}}
        \rput(2.5,4.8){\small{$\max(E - S,0)$}}
	\end{pspicture}
\end{center}

Je třeba zdůraznit, že výše uvedený argument není rigorózním odvozením druhé volné hraniční podmínky. Pro účely této knihy však bude stačit předpoklad, že racionální majitel americké opce sleduje strategii, která povede k tomu, že se hodnota opce, jako hladká funkce proměnné $S$, dotkne funkce vnitřní hodnoty opce za předpokladu, že tato je také hladká.

Nyní se opět vraťme ke druhému omezení zmiňovaném při oceňování americké opce, totiž Black-Scholes nerovnici. Připomeňme, že Black-Scholes rovnice je založena na předpokladu neexistence arbitráže, kdy je riziko portfolia skládajícího z opce a odpovídajícího podkladového aktiva zcela eliminováno kontinuálním zajišťováním. Předpoklad neexistence arbitráže platí v případě Black-Scholes nerovnice pouze částečně, ačkoliv těsná vazba mezi předpokladem neexistence arbitráže a touto nerovnicí stále existuje.

Stejně jako v případě odvození Black-Scholes rovnice pro evropskou opci uvažujme delta neutrální portfolio
\begin{equation*}
\Pi = V - \Delta S
\end{equation*}
V případě americké opce nemusí být vždy možné současně držet dlouhou a krátkou pozici v dané opci. V některých situacích je totiž optimální opci předčasně uplatnit. Vypisovatel opce tak může být vyzván k plnění z opce před její splatností. Jednoduchý argument o neexistenci arbitráže použitý pro evropskou opci tak nevede k jednoznačné hodnotě výnosu portfolia. Jediné, co jsme schopni říci, je, že tento výnos nemůže být větší než bezriziková výnosová míra. Pro americkou prodejní opci tak platí
\begin{equation*}
\frac{\partial P}{\partial t} + \frac{1}{2}\sigma^2 S^2 \frac{\partial^2 P}{\partial S^2} + rS \frac{\partial P}{\partial S} - rP \le 0
\end{equation*}
Hodnotu americké prodejní opce lze popsat pomocí problému volné hraniční podmínky. Pro každé $t$ musíme rozdělit hodnoty $S$ na dvě disjunktní množiny. První množina je definovaná jako $0 \le S \le S_f(t)$ a je pro ní optimální opci předčasně uplatnit. Pro americkou projední opci na této množině platí
\begin{equation*}
P = E - S,~~~\frac{\partial P}{\partial t}+\frac{1}{2}\sigma^2S^2\frac{\partial^2 P}{\partial S^2} + rS\frac{\partial P}{\partial S} - rP < 0
\end{equation*}
V rámci druhé množiny $S_f(t) < S < \infty$ není předčasné uplatnění opce optimální a
\begin{equation*}
P > E - S,~~~\frac{\partial P}{\partial t}+\frac{1}{2}\sigma^2S^2\frac{\partial^2 P}{\partial S^2} + rS\frac{\partial P}{\partial S} - rP = 0
\end{equation*}
Hraniční podmínka v bodě $S = S_f(t)$ říká, že hodnota opce $P$ a její delta jsou spojité.
\begin{equation*}
P(S_f(t),t) = \max(E - S_f(t),0),~~~ \frac{\partial P}{\partial S}(S_f(t),t) = -1
\end{equation*}
Tyto podmínky lze považovat za jednu hraniční podmínku, přičemž druhá podmínka určuje umístění volné hranice. Je důležité si uvědomit, že podmínka
\begin{equation*}
\frac{\partial P}{\partial S}(S_f(t),t) = -1
\end{equation*}
není implikovaná skutečností $P(S_f(t),t) = E - S_f(t)$. Protože dopředu neznáme umístění bodu $S_f(t)$, potřebujeme další podmínku, která by specifikovala jeho umístění. Touto podmínkou je právě požadavek spojité delty příslušné opce.

V následujícím obrázku porovnáváme hodnoty evropské a americké prodejní opce se zbytkovou splatností $T = 0.5$, směrodatnou odchylkou $\sigma = 0.4$ a bezrizikovou úrokovou mírou $r = 0.1$.
\begin{center}
  \begin{pspicture}(0,0)(8.0,6.5)
		\rput(4.0,0.5){Porovnání hodnoty evropské (dolní křivka) a americké opce}
        \rput(4.0,0.0){(horní křivka) jako funkce $S$: $T$ = 0.5, $\sigma$ = 0.4, $r$ = 0.1}

		\psline[arrows=->](0.5,1.5)(7.5,1.5)
		\psline[arrows=->](0.5,1.5)(0.5,6.0)

        \psline(0.5,5.5)(4.5,1.5)
        \pscurve[linewidth=0.5mm](0.5,5.2)(4.0,1.9)(4.5,1.7)(5.5,1.6)(6.5,1.55)
        \psline[linewidth=0.5mm](0.5,5.5)(3.8,2.2)
        \pscurve[linewidth=0.5mm](3.8,2.2)(4.5,1.7)(5.5,1.6)(6.5,1.55)

        \rput(0.5,1.2){\small{0}}
        \rput(4.5,1.2){\small{E}}
        \rput(7.5,1.2){\small{S}}
        \rput(0.2,6.0){\small{P}}
	\end{pspicture}
\end{center}

\section{Technická poznámka: Volná hraniční podmínka}

Je třeba zdůraznit, že obě hraniční podmínky nutné pro stanovení hodnoty americké prodejní opce jsou založeny na neexistenci arbitráže. Existuje však nespočet dalších kandidátů na volné hraniční podmínky, které je možné z čistě matematického pohledu použít. Ačkoliv nevedou k řešení hodnoty americké opce, představují z matematického hlediska správně definovaný problém. Jako příklady uveďme
\begin{equation*}
\frac{\partial P}{\partial S}(S_f(t),t) = 0
\end{equation*}
je-li
\begin{equation*}
P(S_f(t),t) = E - S_f(t)
\end{equation*}
popř. podmínku
\begin{equation*}
\frac{\partial P}{\partial S}(S_f(t),t) = - \frac{d S_f}{d t}
\end{equation*}
Druhá z uvedených podmínek představuje hraniční podmínku Stefanova modelu pro tání ledu. Z pohledu americké opce je však tato podmínka zavádějící.

\section{Ostatní americké opce}

Argumenty, které jsem použili pro americkou prodejní opci, lze s přílušnými úpravami použít také pro ostatní typy plain-vanilla opcí (včetně jejich lineárních kombinací) s výplatnou definovanou jako $\Lambda(S)$ nebo dokonce $\Lambda(S,t)$. Hodnota opce musí splňovat Black-Scholes nerovnost
\begin{equation*}
\frac{\partial V}{\partial t} + \frac{1}{2} \sigma^2 S^2 \frac{\partial^2 V}{\partial S^2} + (r - D_0) S \frac{\partial V}{\partial S} - rV \le 0
\end{equation*}
Je-li předčasné uplatnění opce optimální, platí $V(S,t) = \Lambda(S)$ a nerovnost se stane ostrou.
\begin{equation*}
\frac{\partial V}{\partial t} + \frac{1}{2} \sigma^2 S^2 \frac{\partial^2 V}{\partial S^2} + (r - D_0) S \frac{\partial V}{\partial S} - rV < 0
\end{equation*}
Není-li předčasné uplatnění opce optimální, platí $V(S,t) > \Lambda(S)$ a nerovnost se změní v rovnost.
\begin{equation*}
\frac{\partial V}{\partial t} + \frac{1}{2} \sigma^2 S^2 \frac{\partial^2 V}{\partial S^2} + (r - D_0) S \frac{\partial V}{\partial S} - rV = 0
\end{equation*}
 V bodě hraniční podmínky (popř. podmínek, je-li výplatní profil opce dostatečně složitý) musí být funkce $V$ a $\partial V / \partial S$ spojité. Specifikace problému je zkompletována stanovením konečné podmínky
\begin{equation*}
V(S,T) = \Lambda(S)
\end{equation*}
a stanovením vhodných podmínek v nekonečnu. Příklady těchto diferenciálních nerovnic uvedeme v následujících kapitolách zabývajících se exotickými opcemi.

\section{Technická poznámka: Opce s nespojitou výplatní funkcí}

Vnitřní hodnota opce může být tečnou k funkci hodnoty opce pouze, je-li v bodě dotyku definována. Jako příklad uvažujme americkou cash-or-nothing kupní opci s výplatou definovanou jako
\begin{equation*}
V(S,T) = 0,~~~S < E
\end{equation*}
\begin{equation*}
V(S,T) = B,~~~S \ge E
\end{equation*}
Vnitřní hodnota tedy není spojitou funkcí. Hodnota opce je však s vyjímkou splatnosti spojitá a její delta je nespojitá v bodě $S = E$. Je zřejmé, že hraniční cena, od které je opci vždy optimální předčasně uplatnit, je $S_f = E$. Majitel opce totiž nezíská žádný dodatečný prospěch z držení opce po té, co cena podkladového aktiva dosáhne realizační ceny. Naopak ztrácí případný úrok získaný z předčasné výplaty ve výši $B$. Není tedy důvod zajišťovat se proti situaci $S > E$. Je zřejmé, že $\Delta = 0$ pro $S > E$ a $\Delta > 0$ pro $S < E$. 

Z analýzy tohoto typu americké opce vyplývá existence výplatní podmínky $V(S,T) = 0$ pro $0 \le S \le E$ a dvou hraničních podmínek $V(0,t) = 0$ a $V(E,t) = B$. Situací $S > E$ se není třeba zaobírat, protože opce je předčasně uplatněna. Narozdíl od obvyklých amerických opcí, kde jsou hraniční podmínky aplikované na neznámou hodnotu $S$, což vede k nutnosti dodatečné podmínky, jsou v případě americké cash-or-nothing opce hraniční podmínky specifikovány pro známou hodnotu $S$. Tyto tři podmínky tak vedou k jednoznačnému řešení Black-Scholes rovnice.

Americká cash-or-nothing opce dobře ilustruje myšlenku, že realizační strategie by měla maximalizovat její hodnotu pro majitele opce. Je zřejmé, že volba $S_f(t) = E$ dává největší hodnoty pro $V(S,t)$ pro $S < E$, jak ilustruje následující obrázek.
\begin{center}
  \begin{pspicture}(0,0)(8.0,7.0)
		\rput(4.0,1.0){Porovnání hodnoty evropské (dolní křivka) a americké}
        \rput(4.0,0.5){(horní křivka) cash-or-nothing opce jako funkce $S$}
        \rput(4.0,0.0){$T$ = 1.0, $\sigma$ = 0.4, $r$ = 0.1, $E$ = 10, $B$ = 5, $D$ = 0.02}

		\psline[arrows=->](0.5,2.0)(7.5,2.0)
		\psline[arrows=->](0.5,2.0)(0.5,6.7)

        \pscurve[linewidth=0.5mm](0.5,2.0)(1.5,2.2)(3.9,5.7)(7.0,6.1)

        \pscurve[curvature=2 0.1 0, linewidth=0.5mm](0.5,2.0)(1.18,2.17)(1.5,2.52)
        \psline[linewidth=0.5mm](1.5,2.52)(3.25,6.5)(7.0,6.5)

        \psline[linestyle=dotted](3.25,1.9)(3.25,6.5)

        \rput(0.5,1.7){\small{0}}
        \rput(3.3,1.7){\small{E}}
        \rput(7.5,1.7){\small{S}}
        \rput(0.2,6.5){\small{P}}
	\end{pspicture}
\end{center}

\section{Lineární komplementarita}

Z výše řečeného je patrné, že matematická analýza americké opce je komplikovanější než analýza evropské opce. Až na vyjímky je poměrně složité najít explicitní řešení problému volné hraniční podmínky. Primární snahou je tak nalézt efektivní a robustní numerické metody pro stanovení těchto řešení. To s sebou přináší potřebu teoretického rámce, který umožní analyzovat problematiku volné hraniční podmínky v obecné rovině.

Prvním krokem bude snaha o přeformulování problému s cílem zbavit se explicitní závislosti na volných hranicích. Volná hranice tak přímo nevstupuje do procesu hledání řešení a může z něj být po jeho nalezení zpětně získána. V našich úvahách začneme zabývat výše diskutovaným problémem překážky s jednoduchým případem takovéto reformulace, tzv. lineární komplementaritou. Následně použijeme takto získané poznatky při oceňování amerických opcí.

\subsection{Problém překážky - lineární komplementarita}

Vraťme se k problému překážky diskutovanému v předchozí kapitole. Předpokládejme, že konce struny se nachází v bodech $x = \pm 1$. Nechť funkce $u(x)$ popisuje pozici struny a funce $f(x)$ výšku překážky, obě na definičním oboru $x = \pm 1$.
Předpokládejme, že $f(\pm1) < 0$ a že existuje alespoň jedno $-1 \le x \le 1$, pro které platí $f(x) > 0$. To znamená, že existuje neprázdná množina bodů, ve kterých se struna dotýká překážky. Dále předpokládejme, že $f'' < 0$, kde $' = d/dx$, což implikuje existenci pouze jedné takovéto množiny. Volná hranice je pak množinou bodů označovaných jako $P(x = x_p)$ a $Q(x = x_q)$ v obrázku (7.1). Jedná se o body, ve kterých se struna prvně dotýká překážky. Body $P$ a $Q$ tak vymezují oblast dotyku. Pozice těchto bodů není dopředu známá a její určení je předmětem řešení.

V oblasti dotyku platí $u(x) = f(x)$. Pro všechna $x$, ve kterých se struna překážky nedotýká, platí $u'' = 0$. Tato podmínka znamená, že struna je pro tyto hodnoty $x$ napnutá. Standardně jsou pro určení napnuté části struny zapotřebí pouze dvě hraniční podmínky a hodnoty $u(x)$ na jejích obou koncích. Tyto podmínky jsou součástí vychozích předpokladů, kde $u(-1) = 0$, $u(x_P) = f(x_p)$ a $u(1) = 0$, $u(x_Q) = f(x_Q)$. Nicméně protože body $P$ a $Q$ nejsou známy, je zapotřebí dalších dvou podmínek. Agrumentace založená na fyzikální rovnováze sil definuje tyto podmínky jako nutnost spojitosti funkcí $u(x)$ a $u'(x)$ v bodech $P$ a $Q$. Problém překážky tak lze zformulovat jako problém nalezení funkce $u(x)$ a bodů $P$ a $Q$ při splnění následujících podmínek
\begin{equation}
u(-1) = 0
\end{equation}
\begin{equation}
u''(x) = 0,~~~ -1 < x < x_P
\end{equation}
\begin{equation}
u(x_P) = f(x_P),~~~u'(x_P) = f'(x_P)
\end{equation}
\begin{equation}
u(x) = f(x),~~~x_P < x < x_Q
\end{equation}
\begin{equation}
u(x_Q) = f(x_Q),~~~u'(x_Q) = f'(x_Q)
\end{equation}
\begin{equation}
u''(x) = 0,~~~ x_Q < x < 1
\end{equation}
\begin{equation}
u(1) = 0
\end{equation}
Jestliže máme funkci $f(x)$, která má stejný obecný tvar jako na obrázku (7.1), lze dokázat, že $u(x)$, $P$ a $Q$ jsou jednoznačně určené výše uvedenými podmínkami a nalézt je. Nicméně řešení je poměrně pracné a body $P$ a $Q$ musí být, až na případy triviální funkce $f(x)$, vypočteny numericky jako řešení algeraické popř. transcedentální rovnice. Řešení se ještě více zkomplikuje v případě, kdy $f''(x)$ není vždy menší nebo rovno nule a to z důvodu možné existence víreca oblastí kontaktu. Nicméně v i těchto případech je principiálně možné řešení nalézt.

Alternativním přístupem je uvědomit si, že struna se nachází buďto nad překážkou ($u(x) > f(x)$) a v tomto případě je napnutá ($u''(x) = 0$) nebo se překážky dotýká ($u(x) = f(x)$) a v tomto případě překážku kopíruje ($u''(x) = f''(x) < 0$). To znamená, že jsem schopni tento problém přeformulovat do podoby problému lineární komplementarity\footnote{Obecný problém
\begin{equation*}
\mathcal{AB} = 0,~~~\mathcal{A} \ge 0, \mathcal{B} \ge 0 
\end{equation*}
nazýváme komplementárním problémem a v našem konkrétním případě platí $\mathcal{A} = u''$ a $\mathcal{B} = u - f$, přičemž jak $\mathcal{A}$ tak $\mathcal{B}$ jsou lineární v $u$ a $f$.
}
\begin{equation*}
u'' (u - f) = 0,~~~-u'' \ge 0,~(u - f) \ge 0
\end{equation*}
za předpokladu $u(\pm 1) = 0$ a spojitosti funkcí $u$ a $u'$.

Takto přeformulovaný problém má v porovnání s původním problémem definovaným sérií podmínek (7.1) - (7.7) jednu značnou výhodu - explicitně neobsahuje volné body $P$ a $Q$. Volné body jsou v definici problému stále přítomné, avšak pouze implicitně skrze podmínku $u \ge f$. Jestliže budeme schopni vyvinout algoritmus, který by umožnil řešení problému lineární komplementarity, stačí následně pouze analyzovat hodnoty $u - f$. Volné hraniční body jsou tam, kde se hodnota této funkce mění z nulové na nenulovou. Jedním z takovýchto alogoritmů je např. SOR algoritmus, který je popsán v kapitole 9. V rámci této metody se nejprve stanoví počáteční odhad funkce $u$ takový, že zcela jistě platí $u > f$, a postupně se iteruje ke správnému řešení. Omezení je implementováno tak, že při vygenerování hodnoty $u$ větší než $f$ je hodnota $u$ nastavena na hodnotu $f$.

Důkaz, že lineární komplementarita je ekvivalentní problému volné hraniční podmínky, přesahuje rámec této knihy. Případný důkaz by vycházel z technik funkcionální analýzy a částečně také z teorie variačních nerovností. Nicméně nosnou myšlenkou důkazu je minimalizace vhodného energetického funkcionálu nad konvexní množinou všech přijetelně hladkých funkcí $v(x)$, které splňují podmínku $v \ge f$.

\subsection{Americká prodejní opce - lineární komplementarita}

V této kapitole se budeme zabývat americkou prodejní opcí, kterou přeformulujeme do podoby problému lineární komplementarity. V zásadě jediným výraznějším rozdílem mezi problémem překážky a americkou prodejní opcí je skutečnost, že americká opce má navíc časový rozměr\footnote{Pojmem ``časový rozměr'' máme na mysli, že se poloha hraničních bodů mění v čase.}. Problém americké opce tak lze rozdělit na dvě části - na prostorovou a časovou, kde se prostorová část zabývá hraničními body v určitý fixní časový okamžik, kdežto časová část se zabývá jejich vývojem v čase. To se pochopitelně musí promítnou také do postupu řešení. Prostorovou část řešíme stejně jako u klasického problému překážky pomocí SOR algoritmu; časová část slouží k nalezení řešení v dílčích časových okamžicích.

Nejprve stejně jako v případě evropské opce převedeme problém americké prodejní opce z původních proměnných $(S,t)$ na proměnné $(x, \tau)$. Jediným rozdílem je, že v případě americké opce figuruje navíc tzv. hranice optimální realizace. Ta byla v původním vyjádření tvořena body $S = S_f(t)$. Po transformaci proměnných budeme tuto hranici značit jako $x = x_f(\tau)$. Vzhledem k $S_f(x) < E$ a transformaci $S_f(t) = E e^{x_f(\tau)}$, platí $x_f(\tau) < 0$. Funkce vnitřní hodnoty opce $\max(E - S, 0)$ přejde do tvaru
\begin{equation*}
g(x, \tau) = e^{\frac{1}{2}(k + 1)^2 \tau}\max(e^{\frac{1}{2}(k - 1)x} - e^{\frac{1}{2}(k + 1)x}, 0)
\end{equation*}
Získáváme tak
\begin{equation}
\frac{\partial u}{\partial \tau} = \frac{\partial^2 u}{\partial x^2},~~~x > x_f(\tau)
\end{equation}
\begin{equation}
u(x, \tau) = g(x, \tau),~~~x \le x_f(\tau)
\end{equation}
počáteční podmínku
\begin{equation}
u(x, 0) = g(x, 0) = \max(e^{\frac{1}{2}(k + 1)x} - e^{\frac{1}{2}(k + 1)x},0)
\end{equation}
a hraniční podmínku\footnote{Pro $x \rightarrow -\infty$ se nacházíme v oblasti cen podkladového aktiva, kde je optimální opci předčasně uplatnit. Hraniční podmínku proto nepotřebujeme.}
\begin{equation}
\underset{x \rightarrow 0}{\lim} u(x, \tau) = 0
\end{equation}
Dále existuje omezení
\begin{equation}
u(x, \tau) \ge e^{\frac{1}{2}(k + 1)^2 \tau} \max(e^{\frac{1}{2}(k - 1)x} - e^{\frac{1}{2}(k + 1)x}, 0)
\end{equation}
a požadavek na spojitost funkcí $u$ a $\partial u / \partial x$ v bodě $x = x_f(t)$, který vychází z odpovídajících podmínek původní formulace problému.

Abychom se vyhnuli technickým komplikacím, omezme problém na konečný interval. Tento předpoklad je přijatelný, protože každé numerické řešení omezíme na konečnou síť. To znamená, že problém (7.8) - (7.12) budeme aplikovat pouze na interval $-x^{-} < x < x^{+}$. To implikuje hraniční podmínky
\begin{equation*}
u(x^{+}, \tau) = 0,~~~u(-x^{-}, \tau) = g(-x^{-}, \tau)
\end{equation*}
Přepokládáme tedy, že je možné nahradit hraniční podmínky aproximací, podle které pro malá $S$ platí $P = E - S$, zatímco pro velká $S$ je $P = 0$.

Skutečnost, že jak problém překážky tak americká prodejní opce splňují obdobné omezující podmínky, naznačuje možnost přeformulovat americkou prodejní opci do podoby lineární komplementarity. Americké prodejní opce je velmi podobná problému překážky, s tím rozdílem, že se tvar překážky mění v průběhu času. Připomeňme, že roli překážky zde plní výplatní funkce $g(x, \tau)$. Problém (7.8) - (7.12) tak lze ve formě lineární komplementarity vyjádřit jako
\begin{equation}
\Bigg( \frac{\partial u}{\partial r} - \frac{\partial^2 u}{\partial x^2}\Bigg) \cdot (u(x, \tau) - g(x, \tau)) = 0
\end{equation}
kde
\begin{equation*}
\Bigg( \frac{\partial u}{\partial r} - \frac{\partial^2 u}{\partial x^2}\Bigg) \ge 0,~~~(u(x, \tau) - g(x, \tau)) \ge 0
\end{equation*}
za předpokladu splnění počátečních hraničních podmínek (7.10) a (7.13)
\begin{equation*}
u(x, 0) = g(x, \tau)
\end{equation*}
\begin{equation*}
u(-x^{-}, \tau) = g(-x^{-}, \tau),~~~u(x^{+}, \tau) = g(x^{+}, \tau) = 0
\end{equation*}
a požadavku na spojitost funkcí $u(x, \tau)$ a $\frac{\partial u}{\partial x}(x, \tau)$.

Dvě možná řešení rovnice (7.13) odpovídají situaci, kdy je optimální opci uplatnit ($u = g$), a situaci, kdy naopak není optimální opci uplatnit ($u > g$). Na závěr pouze zopakujme, že velkou výhodou reformulace problému je, že není zapotřebí se explicitně zabývat volnou hraniční podmínkou resp. podmínkami.

Stejně jako v případě problému překážky, i v případě americké prodejní opce je poměrně složité dokázat, že reformulace do podoby lineární komplementarity je konzistentní s původním zadáním problému a že existuje pouze jedno jedinečné řešení, které je společné oběma problémům. Pro případný důkaz bychom opět použili funkční analýzu a parabolické variační nerovnosti.

\section{Americká kupní opce s dividendou}

Uvažujme americkou kupní opci s podkladovým aktivem, které vyplácí dividendu. Připomeňme, že hodnota $C(S,t)$ kupní opce splňuje rovnici
\begin{equation}
\frac{\partial C}{\partial t} + \frac{1}{2}\sigma^2 S^2 \frac{\partial^2 C}{\partial S^2} + (r - D_0)S\frac{\partial C}{\partial S} - rC = 0
\end{equation}
pokud není optimální opci předčasně uplatnit. Výplata generovaná opcí na konci životnosti je pak
\begin{equation}
C(S,T) = \max(S - E, 0)
\end{equation}
a protože opce může být uplatněna kdykoliv během své životnosti musí také platit
\begin{equation}
C(S,t) \ge \max(S - E, 0)
\end{equation}
Jestliže existuje hraniční cena $S = S_f(t)$, v tomto bodě platí
\begin{equation}
C(S_f(t),t) = S_f(t) - E
\end{equation}
\begin{equation}
\frac{\partial C}{\partial S}(S_f(t),t) = 1
\end{equation}
Navíc existuje-li hraniční cena, pak je (7.14) platná pouze pro $C(S, t) > \max(S - E, 0)$. Přímým výpočtem lze dokázat, že $\max(S - E, 0)$ není řešením Black-Scholes rovnice (7.14). Rovnici (7.14) lze stejně jako v případě americké prodejní opce nahradit nerovností
\begin{equation*}
\frac{\partial C}{\partial t} + \frac{1}{2}\sigma^2 S^2 \frac{\partial^2 C}{\partial S^2} + (r - D_0)S\frac{\partial C}{\partial S} - rC \le 0
\end{equation*}
kdy rovnost platí pouze pro $C(S,t) > \max(S - E, 0)$. Finančním důvodem předčasného uplatnění opce je opět maximalizace její hodnoty. Pokud by tak opce byla držena do splatnosti, je její očekávaná současná hodnota nižší než v případě předčasného uplatnění a investování výplaty z opce za bezrizikovou úrokovou míru po zbytkovou dobu splatnosti.

\subsection{Obecné výsledky}

V následujícím textu budeme předpokládát, že bezriziková úroková míra $r$ a dividendový výnos $D_0$ splňují podmínku $r > D_0 > 0$. Stejně jako v případě evropské kupní opce je vhodné přetrasformovat (7.14) - (7.18) do bezrozměrné podoby a zredukovat (7.14) na dopřednou diferenciální rovnici s konstantními koeficienty. S ohledem na navazující úpravy je také vhodné o ceny opce $C(S,t)$ odečíst výplatu $S - E$. Při použití substitucí
\begin{equation*}
S = Ee^x,~~~t = T - \tau / \frac{1}{2}\sigma^2,~~~C(S,t) = S - E + Ec(x,\tau)
\end{equation*}
tak získáváme
\begin{equation}
\frac{\partial c}{\partial \tau} = \frac{\partial^2 c}{\partial x^2} + (k' - 1)\frac{\partial c}{\partial x} - kc + f(x)
\end{equation}
pro $- \infty < x < \infty$ a $\tau > 0$, kde
\begin{equation*}
c(x, 0) = \max(1 - e^x, 0)
\end{equation*}
\begin{equation}
f(x) = (k' - k)e^x + k
\end{equation}
\begin{equation*}
k = \frac{r}{\frac{1}{2}\sigma^2},~~~k' = \frac{r - D_0}{\frac{1}{2}\sigma^2}
\end{equation*}
Protože $r > D_0 > 0$, je $k > k' > 0$.
\begin{center}
  \begin{pspicture}(0,0)(8.0,4.5)
		\rput(4.0,0.0){Průběh funkce $c(x,0)$}
       
		\psline(0.5,1.0)(7.5,1.0)

        \pscurve[linewidth=0.5mm](0.5,3.5)(2.6,2.5)(4.0,1.0)
        \psline[linewidth=0.5mm](4.0,1.0)(7.5,1.0)

        \rput(7.5,1.2){\small{$x$}}
        \rput(1.0,3.9){\small{$c(x,0)$}}
        \rput(4.0,0.8){\small{0}}
	\end{pspicture}
\end{center}
Pro tento okamžik předpokládejme, že existuje volná hranice $x = x_f(\tau)$ ($S = S_f(t)$ dle původního značení). Vzhledem k $C(S,t) = S - E + Ec(x, \tau)$ pro tuto hranici platí
\begin{equation*}
c(x_f(t), \tau) = \frac{\partial c}{\partial x}(x_f(\tau), \tau) = 0
\end{equation*}
Vzhledem k této substituci dále přejde podmínka $C(S,t) \ge \max(S - E,0)$ do tvaru
\begin{equation*}
c \ge \max(1 - e^x, 0)
\end{equation*}
Tvar funkce $f(x)$ má klíčový vliv na chování volné hranice. Je-li funkce $f(x)$ definovaná dle (7.20), pak existence tohoto členu implikuje existenci volné hranice. Následující obrázek zachycuje průběh této funkce.
\begin{center}
  \begin{pspicture}(0,0)(8.0,7.5)
		\rput(4.0,0.0){Typický průběh funkce $f(x)$ definované dle (7.20)}
       
		\psline(0.5,4.0)(7.5,4.0)
		\psline(4.0,1.0)(4.0,7.0)

        \pscurve[linewidth=0.5mm](0.5,6.5)(4.0,5.0)(6.5,2.0)

        \rput(7.5,4.2){\small{$x$}}
        \rput(4.5,7.0){\small{$f(x)$}}
        \rput(5.3,4.2){\small{$x_0$}}
	\end{pspicture}
\end{center}
Je zřejmé, že funkce $f(x)$ je kladná pro $x < x_0$ a záporná pro $x \ge x_0$ kde
\begin{equation*}
x_0 = \ln \Bigg(\frac{k}{k -k'} \Bigg) = \ln \Bigg( \frac{r}{D_0} \Bigg) > 0
\end{equation*}
Nyní se zaměřme na to, co se stane, nexistují-li žádná omezení a tím pádem ani volná hranice. Uvažujme počáteční data $c(x, 0)$ pro kladná $x$. Dle definice $c(x, 0)$ je hodnota této funkce nulová. Pro $x > 0$ tak platí $c(x, 0) = \partial c(x, 0)/ \partial x = \partial^2 c(x, 0)/\partial x^2 = 0$. Z (7.19) vyplývá, že době splatnosti\footnote{Připomeňme, že pro $\tau = 0$ je $t = T$.} platí
\begin{equation*}
\frac{\partial c}{\partial \tau} = f(x)
\end{equation*}
Pro $0 < x < x_0$, $f(x) > 0$ a hodnota opce $c$ se po uplynutí infinityzimálního časového okamžiku stane kladnou. Naopak pro $x > x_0$, $f(x) < 0$ se hodnota opce $c$ ihned stane zápornou. Druhý případ však nesplňuje podmínku vyžadující $c > 0$ pro všechna $x > 0$. Jestliže bychom opci drželi v bodě $x > x_0$, je tato podmínka porušena a hodnota opce klesne pod její vnitřní hodnotu. To v případě americké opce není možné, a proto musí existovat hraniční cena.

Z výše uvedeného argumentu je také zřejmé, od kterého bodu musí hraniční cena $x_f(t)$ začínat. Tímto bodem je $x_f(0^{+}) = x_0$, protože je to jediný bod, který splňuje podmínku $c(x_f(0^{+}), 0^{+})$. Ve finančním vyjádření odpovídá tento bod ceně podkladového aktiva
\begin{equation*}
S_f(T) = \frac{rE}{D_0}
\end{equation*}
a je nezávislý na $\sigma$. Proto těsně před splatností by měla být americká kupní opce uplatněna pro takové hodnoty pokladového aktiva, kdy $D_0 S > rE$. V době splatnosti bude opce pochopitelně uplatněna pro $S > E$. Hraniční cena $S_f(T)$ je tak v bodě $t = T$ nespojitá. Je-li $D_0 = 0$, pak $x_f(0) = \infty$ (a $S_f(T) = \infty$) a proto neexistuje volná hranice. V případě neexistence dividendového výnosu je tak vždy optimální držet americkou kupní opci do splatnosti.

Dále je třeba zdůraznit, že $S = rE/D$ je hodnotou $S$, pro kterou platí
\begin{equation*}
\mathcal{L}_{BS}(\max(S - E,0)) = 0
\end{equation*}

\subsubsection{Technická poznámka: Fyzikální interpretace}

Rovnice (7.19) obsahuje v porovnání s obyčejnou difúzní rovnicí další tři členy: $(k' - 1) \frac{\partial c}{\partial x}$, $-kc$ a $f(x)$. První z těchto členů může být interpretován jako konvekční, druhý jako reakční a třetí, funkce $f(x)$, jako spotřební parametr pro $f(x) < 0$ resp. jako doplňovací parametr pro $f(x) > 0$. Abychom ilustrovali dopad těchto členů na $c(x,t)$, uvažujme jejich kombinace s ostatními členy rovnice (7.19).

Nejprve uvažujme člen $(k' - 1) \frac{\partial c}{\partial x}$. Jak již bylo zmíněno, tento člen představuje konvekci, neboli tzv. drift ve finančním pojmosloví. To je patrné, vypustíme-li pro okamžik z rovnice (7.19) zbývající členy. Výsledkem je hyperbolická rovnice prvního řádu
\begin{equation*}
\frac{\partial c}{\partial \tau} = (k' - 1)\frac{\partial c}{\partial x}
\end{equation*}
Na tuto rovnici lze aplikovat metodu charakteristik s řešením $c(x, /tau) = F(x + (k' - 1)\tau)$ a obecnou funkcí $F$. ProtFunkce $c(x, \tau)$ je konstantní podél charakteristik $x + (k' - 1)\tau = \mathcal{K}$, kde $\mathcal{K}$ představuje konstantu, a představuje tak ``vlnu'', která se pohybuje konstantní rychlostí $1 - k'$. Je zřejmé, že se změnou reference na $\xi = x + (k' - 1)\tau$ lze člen $(k' - 1) \frac{\partial c}{\partial x}$ vypustit. Rovnice (7.19) se tak změní do podoby
\begin{equation*}
\frac{\partial c}{\partial \tau} = \frac{\partial^2 c}{\partial \xi^2} - kc -f(\xi - (k' - 1)\tau)
\end{equation*}

Druhý člen rovnice (7.19) $-kc$ představuje reakci (diskontování ve finanční terminologii), která je proporcionální k $c(x, \tau)$. Jestliže opět zredukujeme rovnici (7.19) na tvar
\begin{equation*}
\frac{\partial c}{\partial \tau} = -kc
\end{equation*}
lze řešení této rovnice získat pomocí metody separovaných proměnných. Řešení má tvar $c = c_0 e^{-kt}$. Tento člen lze tedy eliminovat, vyjádříme-li řešení ve tvaru $c(x, \tau) = e^{-kr}w(x, \tau)$, čímž zhodledňujeme exponenciální pokles v čase způsobený tímto členem. Po této úpravě se rovnice dále zjednodušší na tvar
\begin{equation*}
\frac{\partial w}{\partial \tau} = \frac{\partial^2 w}{\partial \xi^2} - e^{kr}f(\xi - (k' - 1)\tau)
\end{equation*}

Konečně poslední člen $f(x)$, představuje spotřební parametr pro $f(x) < 0$ resp. doplňovací parametr pro $f(x) > 0$. O tom se lze opět přesvědčit, pokud z rovnice (7.19) vypustíme všechny ostatní členy.
\begin{equation*}
\frac{\partial c}{\partial \tau} = f(x)
\end{equation*}
Jestliže $f(x) < 0$, pak $\frac{\partial c}{\partial \tau} < 0$ a $c$ je klesající funkcí $\tau$. Naopak, je-li $f(x) > 0$, je $c$ rostoucí funkcí $\tau$.

\subsection{Lokální analýza volné hranice}

Nyní se zaměřme na to, jak se volná hranice $x = x_f(\tau)$ chová v okolí bodu $x_f(0) = x_0$. Nalezení explicitního řešení problému volné hranice není možné, je však možné nalézt asymptotické řešení, které je platné v blízkosti expirace opce, tj. pro $\tau \rightarrow 0$.

Abychom mohli provést tuto analýzu, která je lokální v čase i ceně podkladového aktiva, zaměřme se rovnici (7.19) v okolí bodu $x = x_0$ a pro $\tau$ blížící se nule. Funkci $f(x)$ aproximujeme Taylorovou řadou v bodě $x_0$\footnote{Připomeňme, že bod $x_0$ odpovídá konečné hraniční ceně.}.
\begin{equation*}
f(x) = f(x_0) + f'(x_0)(x - x_0) + \mathcal{O}((x - x_0)^2) \sim (x - x_0)f'(x_0) = -k(x - x_0)
\end{equation*}
Na pravé straně rovnice (7.19) pak kromě $f(x)$ stačí ponechat pouze člen $\frac{\partial^2 c}{\partial x^2}$. Tento člen totiž v oblastech, kde se hodnota funkce $c$ rychle mění, dominuje nad $c$ a $\frac{\partial c}{\partial x}$, a je tak hlavní ``hybnou'' silou rovnice (7.19). Uvažované zjednodušení je lokálním problémem funkce $c$. Řešení $\hat{c}(x, \tau)$ tohoto problému splňuje rovnici
\begin{equation*}
\frac{\partial \hat{c}}{\partial \tau} = \frac{\partial^2 \hat{c}}{\partial x^2} - k(x - x_0)
\end{equation*}
a hraniční podmínku
\begin{equation*}
\hat{c} = \frac{\partial \hat{c}}{\partial x} = 0
\end{equation*}
pro
\begin{equation*}
x = x_f(\tau),~x_f(0) = x_0
\end{equation*}
Pro takto zformulovaný problém lze pomocí metody podobnostního řešení nalézt explicitní řešení založeného na proměnné
\begin{equation*}
\xi = \frac{x - x_0}{\sqrt{\tau}}
\end{equation*}
Toto řešení má tvar
\begin{equation*}
\hat{c} = \tau^{\frac{3}{2}}c^*(\xi)
\end{equation*}
kde funkce $c^*$ splňuje momentálně blíže nespecifikovanou diferenciální rovnici. Současně hledáme hraniční cenu ve tvaru
\begin{equation*}
x_f(\tau) = x_0 + \xi_0 \sqrt{\tau}
\end{equation*}
Ačkoliv hraniční cena stále není známa, je nyní třeba nalézt pouze konstatnu $\xi_0$. Tím se původní problém nalezení $x_f(\tau)$ jako obecné funkce proměnné $\tau$ značně zjednodušil. 

Dosazením výrazu $\hat{c} = \tau^{\frac{3}{2}}c^*(\xi)$ do diferenciální rovnice pro $\hat{c}$ a následnými úpravami získáváme
\begin{equation*}
\sqrt{\tau}\Bigg( \frac{3}{2}c^* - \frac{1}{2} \xi \frac{d c^*}{d \xi} \Bigg) = \sqrt{\tau} \frac{d^2 c^*}{d \xi^2} - k(x - x_0)
\end{equation*}
Jestliže obě strany rovnice vydělíme $\sqrt{\tau}$, dostáváme obyčejnou diferenciální rovnici
\begin{equation}
\frac{d^2 c^*}{d \xi^2} + \frac{1}{2}\xi\frac{d c^*}{d \xi} - \frac{3}{2}c^* = k \xi
\end{equation}
Hraniční podmínka
\begin{equation*}
\hat{c} = \frac{\partial \hat{c}}{\partial x} = 0,~~~x = x_f(\tau)
\end{equation*}
se zredukuje do podoby
\begin{equation*}
c^*(\xi_0) = \frac{d c^*}{d \xi}(\xi_0) = 0
\end{equation*}

Dále je třeba určit, jak se funkce $\hat{c}(x, \tau)$ chová pro $\xi \rightarrow -\infty$. Z definice $\xi$ vyplývá, že $\xi \rightarrow -\infty$ odpovídá $x \rightarrow -\infty$. Vzhledem k tomu, že $\frac{\partial^2 \hat{c}}{\partial x^2} \rightarrow 0$ pro $x \rightarrow -\infty$, přibližně platí
\begin{equation*}
\frac{\partial \hat{c}}{\partial \tau} \sim -k(x - x_0)
\end{equation*}
\begin{equation*}
\frac{\partial \hat{c}}{\partial \tau} \sim -kx
\end{equation*}
 a funkci $\hat{c}(x, \tau)$ tak lze aproximovat výrazem $-kx\tau$. Protože
\begin{equation*}
\hat{c}(x, \tau) = \tau^{\frac{3}{2}}c^*(x, \tau)
\end{equation*}
také přibližně platí
\begin{equation*}
c^*(\xi) \sim -k \xi,~\xi \rightarrow -\infty
\end{equation*}
což mimojiné implikuje $\hat{c} = \tau^{\frac{3}{2}}c^* \sim \tau^{\frac{3}{2}} \frac{x - x_0}{\sqrt{\tau}} \sim x \tau + \mathcal{O}$, kde $\mathcal{O}$ představuje ostatní zanedbatelné členy aproximace.

Prvním krokem k vyřešení problému $c^*(\xi)$ je nalezení obecného řešení homogenní diferenciální rovnice
\begin{equation*}
\frac{d^2 c^*}{d \xi^2} + \frac{1}{2}\xi\frac{d c^*}{d \xi} - \frac{3}{2}c^* = 0
\end{equation*}
První kořen této rovnice, funkci $c_1^*(\xi)$, lze snadno vypočíst pomocí metody polynomických řešení jako
\begin{equation*}
c_1^*(\xi) = \xi^3 + 6 \xi
\end{equation*} 
Druhý kořen $c_2^*(\xi)$ je pak možné nalézt pomocí metody redukce řádu a to tak, že nejprve definujeme $c_2^*(\xi)$ jako $c_2^*(\xi) = c_1^*(\xi)a(\xi)$ a následně nalezneme diferenciální rovnici prvního řádu pro $a(\xi)$. Výpočet je přímočarý avšak pracný s výsledkem
\begin{equation*}
c_2^*(\xi) = (\xi^2 + 4)e^{-\frac{1}{4}\xi^2} + \frac{1}{2}(\xi^2 + 6 \xi) \int_{-\infty}^{\xi}e^{-\frac{1}{4}s^2}ds
\end{equation*}
Obecné řešení uvažované homogenní diferenciální rovnice je tedy
\begin{equation*}
c^*(\xi) = Ac_1^*(\xi) + Bc_2^*(\xi)
\end{equation*}

Druhým krokem v řešení problému (7.21) a navazujících podmínek je zjištění, že $c_p^*(\xi) = - k \xi$ je explicitním řešením diferenciální rovnice (7.21). Obecné řešení je tak dáno součtem $c_p^*$ a obecného řešení homogenní diferenciální rovnice.
\begin{equation*}
c^*(\xi) = - k \xi + A c_1^*(\xi) + B c_2^*(\xi)
\end{equation*}
Pro $\xi$ limitně se blížící $-\infty$ platí $c_2^*(\xi) \rightarrow 0$ a $c_1^*(\xi) \rightarrow \infty$. Jednou z podmínek, které se váží k diferenciální rovnici (7.21), je také $c^*(\xi) \sim -k \xi$ pro $\xi \rightarrow -\infty$. Parametr $A$ je tedy roven nule.
\begin{equation}
c^*(\xi) = - k \xi + B c_2^*(\xi)
\end{equation}
Hraniční podmínky $c^*(\xi_0) = 0$ a $\frac{d c^*}{d \xi}(\xi_0) = 0$ umožňují výpočet zbývajících parametrů $B$ a $\xi_0$. Jejich kombinací s (7.22) získáváme
\begin{equation*}
B c_2^*(\xi) = k \xi 
\end{equation*}
a
\begin{equation*}
B \frac{d c_2^*}{d \xi}(\xi_0) = k
\end{equation*}
Spojením těchto dvou rovnic lze odvodit rovnici
\begin{equation*}
\xi_0 \frac{d c_2^*}{d \xi}(\xi_0) = c_2^*(\xi_0)
\end{equation*}
která po sérii úprav vede k transcedentální rovnici
\begin{equation}
\xi_0^3e^{\frac{1}{4}\xi_0^2} \int_{-\infty}^{\xi_0} e^{-\frac{s^2}{4}}ds = 2(2 - \xi_0^2)
\end{equation}
Konstanta $B$ je pak definována jako $B = \frac{\xi_0}{c_2^*(\xi_0)}$.

Transcedentální rovnice ve tvaru (7.22) jsou charakteristické pro podobnostní řešení problému volné hraniční ceny. Lze dokázat (např. pomocí grafické metody), že řešení této rovnice má pouze jeden kořen, který může být nalezen pomocí numerických metod. Tento kořen je roven
\begin{equation*}
\xi_0 = 0.9034...
\end{equation*}
Tímto jsme tedy nalezli řešení $\hat{c}(x, \tau)$, které je aproximací problému americké kupní opce pro $\tau$ blížící se nule a $x$ v okolí $x_0$. Již dříve jsme prokázali, že v době splatnosti je optimální hraniční cena americké kupní opce na podkladové aktivum s dividendovým výnosem $D_0$, rovna $\frac{rE}{D_0}$. Dále z výše provedené lokální analýzy víme, že pro $t \rightarrow T$ platí
\begin{equation*}
S_f(t) \sim \frac{rE}{D_0}\Bigg( 1 + \xi_0 \sqrt{\frac{1}{2}\sigma^2(T - t)} + ... \Bigg)
\end{equation*}
kde $\xi_0$ je univerzální konstanta platná pro všechny americké kupní opce. Kromě tohoto zajímavého zjištění je lokální analýza důležitá také pro ranná stádia numerických výpočtů hodnoty americké opce, která jsou charakteristická rychlou změnou optimální realizační ceny $S_f(t)$. Efekt této změny se pak promítne do celého oboru řešení a nikoliv pouze pro okolí bodu $S = S_f(T)$.
\begin{center}
  \begin{pspicture}(0,0)(8.0,5.5)
		\rput(4.0,0.0){Funkce $c(x, \tau)$ pro čtyři rozdílné hodnoty $\tau$ včetně $\tau = 0$}

		\psline(0.5,1.0)(7.5,1.0)
		\psline(4.5,1.0)(4.5,5.0)

        \pscurve(0.5,4.5)(3.0,3.70)(4.5,1.0)
        \pscurve(0.5,4.5)(3.5,3.60)(6.2,1.10)(7.0,1.0)
		\pscurve(0.5,4.5)(3.5,3.65)(6.2,1.27)(7.0,1.0)
		\pscurve(0.5,4.5)(3.5,3.70)(6.2,1.45)(7.1,1.0)

        \rput(4.5,0.8){\small{0}}
        \rput(7.5,0.8){\small{$x$}}
		\rput(5.2,4.8){\small{$c(x, \tau)$}}

	\end{pspicture}
\end{center}
\begin{center}
	\begin{pspicture}(0,0)(8.0,6.5)
		\rput(4.0,0.5){Hodnota opce $C(S,t)$ pro shodné hodnoty $\tau$ jako}
                \rput(4.0,0.0){v předchozím obrázku}

		\psline(0.5,1.5)(7.5,1.5)
		\psline(0.5,1.5)(0.5,6.0)

        \psline(3.0,1.5)(7.5,6.0)
        \pscurve(1.5,1.5)(2.5,1.60)(5.0,3.60)(7.3,5.80)
        \pscurve(1.2,1.5)(2.5,1.75)(5.0,3.70)(7.3,5.80)
        \pscurve(1.0,1.5)(2.5,1.90)(5.0,3.80)(7.3,5.80)


        \rput(0.5,1.2){\small{0}}
        \rput(3.0,1.2){\small{$E$}}
        \rput(7.5,1.2){\small{$S$}}
        \rput(1.2,5.8){\small{$C(S,t)$}}

	\end{pspicture}
\end{center}

\part{Numerické metody}

\chapter{Diferenční metoda}

\section{Úvod}

Diferenční metoda umožňuje získat numerické řešení parciálních diferenčních rovnic a problémů lineární komplementarity. Tato metoda představuje účinnou a flexibilní techniku, která je schopná generovat přesná řešení pro všechny oceňovací modely odvozené v této knize.

V páté kapitole jsme ukázali, že po zredukování Black-Scholes rovnice na difúzní rovnici je relativně snadné nalézt přesné řešení. Difúzní rovnice je totiž v porovnání s původní Black-Scholes rovnicí mnohem jednodušší. Proto, než přímo řešit Black-Scholes rovnici, je snazší nalézt numerické řešení pro difúzní rovnici a toto řešení následně převést na původní finanční veličiny. V této kapitole se tak zaměříme na řešení difúzní rovnice pomocí diferenční metody.

Výše řečené však neznamená, že by nebylo možné řešit Black-Scholes rovnici pomocí diferenční metody. Existují situace, kdy není možné převést problém do tvaru difúzní rovnice s konstatními koeficienty. Čtenář, který pochopí základní principy diferenční metody, by s její aplikací na Black-Scholes rovnici neměl mít problém.

Jak jsme ukázali v páté kapitole, lze vhodnou transformací proměnných zredukovat Black-Scholes rovnici (3.5) na difúzní rovnici
\begin{equation*}
\frac{\partial u}{\partial \tau} = \frac{\partial^2 u}{\partial x^2}
\end{equation*}
Výplatní funkce opce představuje počáteční podmínku pro $u(x, \tau)$ a hraniční podmínky jsou dány podmínkami pro $u(x, \tau)$ v nekonečnu (tj. pro $x \rightarrow \pm \infty$). V případě evropské kupní opce se jedná o rovnice (3.6), (3.7) a (3.8); v případě evropské prodejní opce o rovnice (3.9), (3.10) a (3.11).

Výstupem níže popsaných diferenčních metod je bezrozměná veličnina $u(x, \tau)$. Hodnotu opce $V(S,t)$ lze pomocí této veličiny vyjádřit jako
\begin{equation*}
v(x, \tau) = e^{-\frac{1}{2}(k-1)x - \frac{1}{4}(k+1)^2\tau}u(x, \tau)
\end{equation*}
\begin{equation*}
Ev(x, \tau) = Ee^{-\frac{1}{2}(k-1)x} e^{-\frac{1}{4}(k+1)^2\tau}u(x, \tau)
\end{equation*}
\begin{equation*}
V(S,t) = E \Big( e^{\ln \frac{S}{E}} \Big)^{-\frac{1}{2}(k-1)}e^{-\frac{1}{4}(k+1)^2(T-t)\frac{1}{2}\sigma^2}u(x, \tau)
\end{equation*}
\begin{equation*}
V(S,t) = E^{\frac{1}{2}(1 + k)}S^{\frac{1}{2}(1 - k)}e^{-\frac{1}{8}(1+k)^2 \sigma^2(T - t)}u \Big(\ln \frac{S}{E}, \frac{1}{2} \sigma^2 (T - t) \Big)
\end{equation*}

\section{Diferenční aproximace}

Nosnou myšlenkou diferenční metody je nahrazení parciální derivace aproximací založené na Taylorově rozvoji funkce v okolí bodu popř. bodů aproximace. 

Parciální derivaci $\frac{\partial u}{\partial \tau}$ lze definovat pomocí limitního rozdílu
\begin{equation*}
\frac{\partial u}{\partial \tau}(x, \tau) = \underset{\delta \tau \rightarrow 0}{\lim} \frac{u(x, \tau + \delta \tau) - u(x, \tau)}{\delta \tau}
\end{equation*}
Jestliže namísto $\delta \tau \rightarrow 0$ budeme uvažovat dostatečně malé nicméně nenulové $\delta \tau$, získáme aproximaci
\begin{equation}
\frac{\partial u}{\partial \tau}(x, \tau) \approx \frac{u(x, \tau + \delta \tau) - u(x, \tau)}{\delta \tau} + \mathcal{O}(\delta \tau)
\end{equation}
Tato aproximace je tzv. diferenční aproximací $\frac{\partial u}{\partial \tau}$. Výše uvedený příklad je tzv. dopřednou diferenční aproximací, protože posun veličiny $\tau$ je definován v dopředném směru a jsou tak použity hodnoty funkce $u$ v bodech $\tau$ a $\tau + \delta \tau$. Jak naznačuje člen $\mathcal{O}(\delta \tau)$, je aproximace tím přesnější, čím menší je $\delta \tau$.

Pariciální derivaci $\frac{\partial u}{\partial \tau}$ lze však také definovat jako
\begin{equation*}
\frac{\partial u}{\partial \tau} = \underset{\delta \tau \rightarrow 0}{\lim}\frac{u(x, \tau) - u(x, \tau - \delta \tau)}{\delta \tau}
\end{equation*}
Tímto se dostáváme k tzv. zpětné diferenční aproximaci, která je definována jako 
\begin{equation}
\frac{\partial u}{\partial \tau} \approx \frac{u(x, \tau) - u(x, \tau - \delta \tau)}{\delta \tau} + \mathcal{O}(\delta \tau)
\end{equation}

Vedle dopředné a zpětné diferenční aproximace existuje také tzv. centrální diferenční aproximace. Derivaci $\frac{\partial u}{\partial \tau}$ v tomto případě definujeme jako
\begin{equation*}
\frac{\partial u}{\partial \tau} = \underset{\delta \tau \rightarrow 0}{\lim}\frac{u(x, \tau + \delta \tau) - u(x, \tau - \delta \tau)}{2 \delta \tau}
\end{equation*}
Příslušná aproximace má pak tvar
\begin{equation}
\frac{\partial u}{\partial \tau} \approx \frac{u(x, \tau + \delta \tau) - u(x, \tau - \delta \tau)}{2 \delta \tau} + \mathcal{O}\Big( (\delta \tau)^2 \Big)
\end{equation}
Centrální diferenční aproximace je v porovnání s dopřednou a zpětnou diferenční aproximací přesnější, což ilustruje následující obrázek.
\begin{center}
  \begin{pspicture}(0,0)(8.0,8.0)
	\rput(4.0,0.5){Dopředná, zpětná a centrální diferenční aproximace - směrnice}
        \rput(4.0,0.0){úseček představují aproximaci tangenty v bodě ($x$, $\tau$)}

	\psline[arrows=->](0.5,2.0)(7.5,2.0)
	\psline[arrows=->](0.5,2.0)(0.5,7.5)

        \pscurve[linewidth=0.5mm](0.5,4.0)(3.5,3.0)(7.0,7.0)
        \psdots[dotstyle=*,dotscale=2](1.5,3.45)
        \psdots[dotstyle=*,dotscale=2](4.0,3.22)
        \psdots[dotstyle=*,dotscale=2](6.5,6.1)
        
        \psline(1.5,3.45)(4.0,3.22)(6.5,6.1)(1.5,3.45)

        \psline[linestyle=dotted](1.5,3.45)(1.5,1.8)
        \psline[linestyle=dotted]](4.0,3.22)(4.0,1.8)
        \psline[linestyle=dotted](6.5,6.1)(6.5,1.8)

        \rput(1.5,1.5){\small{$\tau - \delta \tau$}}
        \rput(4.0,1.5){\small{$\tau$}}
        \rput(6.5,1.5){\small{$\tau + \delta \tau$}}
        \rput(0.7,7.5){\small{$u$}}

        \rput(2.75,2.5){\tiny{zpětná}}
	\rput(5.25,3.0){\tiny{dopředná}}
	\rput(4.00,6.0){\tiny{centrální}}

	\psline[arrows=->, linewidth=0.1mm](2.75,2.7)(2.6,3.3)
	\psline[arrows=->, linewidth=0.1mm](5.25,3.2)(5.0,4.3)
	\psline[arrows=->, linewidth=0.1mm](4.00,5.8)(4.2,5.0)

	\end{pspicture}
\end{center}

Při aplikaci dopředné diferenční aproximace na difúzní rovnici získáváme explicitní a v případě zpětné diferenční aproximace implicitní schéma. Centrální diferenční aproximace není ve tvaru (8.3) v praxi používána, protože vede k problematickým numerickým schématům\footnote{Jedná se o schémata, která jsou vnitřně nestabilní.}. Centrální aproximace ve tvaru
\begin{equation}
\frac{\partial u}{\partial \tau} \approx \frac{u(x, \tau + \delta \tau / 2) - u(x, \tau - \delta \tau / 2)}{\delta \tau} + \mathcal{O}\Big( (\delta \tau)^2 \Big)
\end{equation}
je však použita v Crank-Nicolson metodě, které spadá do rodiny implicitních diferenčních schémat.

Stejným způsobem, jakým jsme aproximovali parciální derivaci funkce $u$ pro $\tau$, je možné definovat aproximaci parciální derivace této funkce pro $x$. Např. centrální diferenční aproximace má tvar\footnote{Ačkoliv se centrální diferenční aproximace (8.3) v praxi nepoužívá pro derivace podle $\tau$ popř. $t$, je aproximace (8.4) běžně aplikována pro derivace podle $x$ popř. $S$.}
\begin{equation*}
\frac{\partial u}{\partial x} \approx \frac{u(x + \delta x, \tau) - u(x - \delta x, \tau)}{2 \delta x} + \mathcal{O} \Big( (\delta x)^2 \Big)
\end{equation*}

 Parciální derivaci druhého řádu, např. $\frac{\partial^2 u}{\partial x^2}$, lze definovat jako dopřednou diferenční aproximaci zpětné diferenční aproximace derivace prvního řádu popř. obráceně. V obou případech získáme symetrickou centrální diferenční aproximaci
\begin{equation*}
\frac{\partial^2 u}{\partial x^2}(x, \tau) \approx \frac{\frac{u(x + \delta x, \tau) - u(x, \tau)}{\delta x} - \frac{u(x, \tau) - u(x - \delta x, \tau)}{\delta x}}{\delta x} + \mathcal{O} \Big( (\delta x)^2 \Big)
\end{equation*}
\begin{equation}
\frac{\partial^2 u}{\partial x^2}(x, \tau) \approx \frac{u(x + \delta x, \tau) - 2u(x, \tau) + u(x - \delta x, \tau)}{(\delta x)^2} + \mathcal{O} \Big( (\delta x)^2 \Big)
\end{equation}
Ačkoliv existují také jiné aproximace derivace druhého řádu, je tato aproximace z důvodu její vyšší přesnosti preferovaná.

\section{Diferenční síť}

Dalším krokem v numerickém řešení difúzní rovnice je rozdělení osy $x$ a $\tau$ na tzv. uzly. Vzdálenost mezi uzly ve směru osy $x$ je rovna $\delta x$ a ve směru osy $\tau$ rovna $\delta \tau$. Tímto jsme definovali tzv. diferenční síť.
\begin{center}
  \begin{pspicture}(0,0)(8.0,8.0)
        \rput(4.0,0.3){Diferenční síť}

	\psline[arrows=->](0.5,1.0)(7.5,1.0)
	\psline[arrows=->](0.5,1.0)(0.5,7.5)

        \rput(7.3,1.3){$\tau$}
        \rput(0.3,7.3){$x$}
        
        \psline[linestyle=dotted](1.0,1.0)(1.0,6.5)
        \psline[linestyle=dotted](1.5,1.0)(1.5,6.5)
        \psline[linestyle=dotted](2.0,1.0)(2.0,6.5)
        \psline[linestyle=dotted](2.5,1.0)(2.5,6.5)
        \psline[linestyle=dotted](3.0,1.0)(3.0,6.5)
        \psline[linestyle=dotted](3.5,1.0)(3.5,6.5)
        \psline[linestyle=dotted](4.0,1.0)(4.0,6.5)
        \psline[linestyle=dotted](4.5,1.0)(4.5,6.5)
        \psline[linestyle=dotted](5.0,1.0)(5.0,6.5)
        \psline[linestyle=dotted](5.5,1.0)(5.5,6.5)
        \psline[linestyle=dotted](6.0,1.0)(6.0,6.5)
        \psline[linestyle=dotted](6.5,1.0)(6.5,6.5)

        \psline[linestyle=dotted](0.5,1.5)(6.5,1.5)
        \psline[linestyle=dotted](0.5,2.0)(6.5,2.0)
        \psline[linestyle=dotted](0.5,2.5)(6.5,2.5)
        \psline[linestyle=dotted](0.5,3.0)(6.5,3.0)
        \psline[linestyle=dotted](0.5,3.5)(6.5,3.5)
        \psline[linestyle=dotted](0.5,4.0)(6.5,4.0)
        \psline[linestyle=dotted](0.5,4.5)(6.5,4.5)
        \psline[linestyle=dotted](0.5,5.0)(6.5,5.0)
        \psline[linestyle=dotted](0.5,5.5)(6.5,5.5)
        \psline[linestyle=dotted](0.5,6.0)(6.5,6.0)
        \psline[linestyle=dotted](0.5,6.5)(6.5,6.5)

        \psdots[dotstyle=*,dotscale=1](4.0,4.0)

        \rput(4.0,0.7){\small{$m \delta \tau$}}
        \rput(0.1,4.0){\small{$n \delta x$}}

  \end{pspicture}
\end{center}
Jednotlivé uzly mají souřadnice $(n \delta x, m \delta \tau)$. V rámci numerického řešení se zabýváme hodnotou funkce $u(x, \tau)$ v jednotlivých uzlech diferenční sítě. Hodnotu funkce $u$ v uzlu $(n \delta x, m \delta \tau)$ pak zapisujeme jako
\begin{equation*}
u_n^m = u(n \delta x, m \delta \tau)
\end{equation*}

\section{Explicitní diferenční metoda}

Uvažujme obecnou formu transformovaného Black-Scholes modelu pro hodnotu evropské opce
\begin{equation*}
\frac{\partial u}{\partial \tau} = \frac{\partial^2 u}{\partial x^2}
\end{equation*}
s počáteční podmínkou
\begin{equation}
u(x, 0) = u_0(x)
\end{equation}
a hraničními podmínkami
\begin{equation}
u(x, \tau) \sim u_{- \infty}(x, \tau),~u(x, \tau) \sim u_{\infty}(x, \tau)
\end{equation}
pro $x \rightarrow \pm \infty$.

Jestliže se omezíme na uzly diferenční sítě, parciální derivaci $\frac{\partial u}{\partial \tau}$ aproximujeme pomocí dopředné diferenční aproximace (8.1) a parciální derivaci $\frac{\partial^2 u}{\partial x^2}$ pomocí symetrické centrální diferenční aproximace (8.5), přejde difúzní rovnice do tvaru
\begin{equation*}
\frac{u_n^{m+1} - u_n^m}{\delta \tau} + \mathcal{O}(\delta \tau) = \frac{u_{n+1}^m - 2u_n^m + u_{n-1}^m}{(\delta x)^2} + \mathcal{O} \Big( (\delta x)^2 \Big)
\end{equation*}
Budeme-li ignorovat členy $\mathcal{O}(\delta \tau)$ a $\mathcal{O} \Big( (\delta x)^2 \Big)$, je možné výše uvedený vztah upravit na diferenční rovnici
\begin{equation}
u_n^{m+1} = \alpha u_{n+1}^m + (1 - 2 \alpha)u_n^m + \alpha_{n-1}^m
\end{equation}
kde
\begin{equation*}
\alpha = \frac{\delta \tau}{(\delta x)^2}
\end{equation*}
Připomeňme, že vzhledem k zanedbání zbytkových členů je rovnice (8.6) pouze aproximací.

Jestliže v časovém kroce $m$ známe $u_n^m$ pro všechna $n$, můžeme dle rovnice (8.6) dopočíst $u_n^{m+1}$, protože to je explicitně vyjádřeno pomocí $u_{n+1}^m$, $u_n^m$ a $u_{n-1}^m$. Hodnotu $u_n^m$ je tak možné přímo získat v jednom kroku výpočtu - odtud plyne název explicitní diferenční metoda.

Zvolíme-li konstatní $\delta x$, není možné řešit výše definovaný problém pro všechna $ -\infty < x < \infty$ bez toho, abychom uvažovali nekonečný počet kroků ve směru osy $x$. Tento problém lze obejít tak, že budeme uvažovat konečný, avšak dostatečně velký počet kroků. Omezíme se tak na interval $N^{-} \delta x \le x \le N^{+} \delta x$, kde $N^{+}$ a $N^{-}$ přestavují přijatelně velká celá čísla. Dále je třeba rozdělit bezrozměnou dobu do splatnosti opce $\frac{1}{2} \sigma^2 T$ na $M$ intervalů délky $\frac{1}{2} \sigma^2 \frac{T}{M}$. Dalším krokem je stanovení $u_{N^{+}}^m$ a $u_{N^{-}}^m$ pomocí hraničních podmínek (8.7).
\begin{equation*}
u_{N^{+}}^m = u_{-\infty}(N^{-} \delta x, m \delta \tau),~~~ 0 < m \le M
\end{equation*}
\begin{equation}
u_{N^{-}}^m = u_{\infty}(N^{+} \delta x, m \delta \tau),~~~ 0 < m \le M
\end{equation}
Iterační výpočet pomocí explicitní diferenční metody pak zahájíme výpočtem $u_n^0$ dle (8.6).
\begin{equation}
u_n^0 = u_0(n \delta x),~~~N^{-} \le n \le N^{+}
\end{equation}
V tabulce (8.1) porovnáváme výsledky explicitní diferenční metody s přesnými výsledky podle Black-Scholes rovnice. Přepokládejme, že výpočet byl proveden pomocí počítače. Pro účely porovnání vyberme $\alpha$ a $\delta \tau$ jako proměnné narozdíl od na první pohled intuitivnějších $\delta x$ a $\delta \tau$. Cílem tohoto výběru je ilustrovat tzv. problém stability explicitní diferenční metody. Jak je patrné z tabulky (8.1), je shoda mezi explicitní diferenční metodou a Black-Scholes rovnicí dobrá pro $\alpha = 0.25$ a $\alpha = 0.50$, avšak pro $\alpha = 0.52$ v řadě případů nedávají výsledky vůbec smysl.
\begin{table}
\begin{center}
\begin{tabular}{r r r r r}
\multicolumn {1}{c}{$S$} &
\multicolumn {1}{c}{$\alpha = 0.25$} &
\multicolumn {1}{c}{$\alpha = 0.50$} &
\multicolumn {1}{c}{$\alpha = 0.52$} &
\multicolumn {1}{c}{BS rovnice} \\
\hline
 0.00 & 9.7531 & 9.7531 &    9.7531 & 9.7531 \\
 2.00 & 7.7531 & 7.7531 &    7.7531 & 7.7531 \\
 4.00 & 5.7531 & 5.7531 &    5.7531 & 5.7531 \\
 6.00 & 3.7531 & 3.7531 &    2.9498 & 3.7532 \\
 8.00 & 1.7986 & 1.7985 &   95.3210 & 1.7987 \\
10.00 & 0.4418 & 0.4419 &  625.0347 & 0.4420 \\
12.00 & 0.0483 & 0.0483 & -208.9135 & 0.0483 \\
14.00 & 0.0028 & 0.0027 &  -15.2150 & 0.0028 \\
16.00 & 0.0001 & 0.0001 &    0.7365 & 0.0001 \\
\hline
\end{tabular}
\end{center}
\caption{Porovnání výsledků Black-Scholes rovnice a výsledků explicitní diferenční metody pro evropskou prodejní opci s parametry $E=10$, $r=0.05$, $\sigma = 0.20$ a $T - t = 0.5$}
\end{table}
Problém stability explicitní diferenční metody souvisí se zaokrouhlovací chybou, která dána způsobem uchovávání číselných hodnot v paměti počítače. Tato zaokrouhlovací chyba tak zatěžuje výpočet pomocí rovnice (8.8). Systém (8.8) označujeme jako stabilní, pokud se zaokrouhlovací chyby nezvyšují s každým následným krokem výpočtu. V opačném případě označujeme systém (8.8) jako nestabilní. Lze dokázat, že systém (8.8) je stabilní pro $0 < \alpha \le \frac{1}{2}$ a nestabilní pro $\alpha > \frac{1}{2}$. Podmínka stability tak představuje také omezení pro relaci mezi $\delta x$ a $\delta \tau$. Aby byl systém stabilní, musí platit
\begin{equation*}
0 < \frac{\delta \tau}{(\delta x)^2} \le \frac{1}{2}
\end{equation*}
Lze také dokázat, že numerické řešení diferenční rovnice konverguje k přesnému řešení pro $\delta x \rightarrow 0$ a $\delta \tau \rightarrow 0$ ve smyslu
\begin{equation*}
u_n^m \rightarrow u(n \delta x, m \delta \tau)
\end{equation*}
pouze tehdy a jen tehdy, jsou-li splněny podmínky stability. Důkaz tohoto tvrzení však překračuje záběr této knihy.

\section{Implicitní diferenční metody}

K obejití problému stability explicitní metody lze použít některou z tzv. implicitních diferenčních metod. Implicitní metody nám umožňují použít velký počet uzlů ve směru osy $x$, aniž bychom museli současně mít nesmyslně velký počet uzlů ve směru osy $\tau$\footnote{Připomeňme, že podmínka stability explicitní diferenční metody vyžaduje zečtyřnásobení počtu uzlů ve směru osy $\tau$ při zdvojnásobení počtu uzlů ve směru osy $x$.}.

Základním rozdílem mezi explicitní a implicitní metodou je ten, že v případě implicitní metody lze vyjádřit hodnotu veličiny $u_n^m$ přímo pomocí jedné rovnice. Implicitní metody jsou založeny na řešení soustavy rovnic. Pro řešení těchto systémů lze použít LU dekompozici popř. metodu SOR. Použitím těchto technik se implicitní metody blíží explicitní metodě efektivitou výpočtu na jeden výpočtový krok. Protože však implicitní metody nevyžadují rozdělení časové osy na tak velký počet časových kroků jako explicitní metoda, jsou z celkového pohledu výpočtově efektivnější.

V následujícím textu se kromě výše zmiňované LU dekompozice a metody SOR budeme zabývat také Crank-Nicolson metodou jako zástupci implicitních diferenčních metod.

\subsection{Úvod do implicitní diferenční metody}

Implicitní diferenční metoda používá pro člen $\frac{\partial u}{\partial \tau}$ zpětnou diferenční aproximaci (8.2) a pro člen $\frac{\partial^2 u}{\partial x^2}$ symetrickou centrální diferenční aproximaci (8.5). To vede k rovnici
\begin{equation*}
\frac{u_n^m - u_n^{m-1}}{\delta \tau} + \mathcal{O}(\delta \tau) = \frac{u_{n+1}^m - 2u_n^m + u_{n-1}^m}{(\delta x)^2} + \mathcal{O}\Big( (\delta x)^2 \Big)
\end{equation*}
Jestliže zanedbáme členy $\mathcal{O}(\delta \tau)$ a $\mathcal{O}\Big( (\delta x)^2 \Big)$, lze tuto rovnici upravit do tvaru
\begin{equation}
- \alpha u_{n - 1}^m + (1 + \alpha)u_n^m - \alpha u_{n + 1}^m = u_n^{m-1}
\end{equation}
kde opět
\begin{equation*}
\alpha = \frac{\delta \tau}{(\delta x)^2}
\end{equation*}
V rovnici (8.11) závisí $u_n^m$, $u_{n-1}^m$ a $u_{n+1}^m$ na $u_n^{m-1}$ v implicitním slova smyslu. To znamená, že nové hodnoty nemohou být ihned vypočteny na základě starých hodnot jako je tomu v případě explicitní metody.

Uvažujme stejně jako v případě explicitní diferenční rovnice problém evropské opce. Předpokládejme, že je možné  omezit osu $x$ diferenční sítě takovými krajními body $x = N^{-}\delta x$ a $x = N^{+}\delta x$, kde $N^{-}$ a $N^{+}$ jsou dostatečně velká čísla. Stejně jako u explicitní metody nejprve vypočteme $u_n^0$ pomocí (8.10) a $u_{N^{-}}^{m}$ a $u_{N^{+}}^{m}$ pomocí (8.9). Dále je pak třeba vypočíst $u_n^m$ pro $m \ge 1$ a $N^{-} < n < N^{+}$ na základě (8.11).

Rovnici (8.11) můžeme vyjádřit pomocí matic.
\begin{equation*}
	\begin{pmatrix}
		1 + 2 \alpha & - \alpha & 0 & \dots & 0 \\
		-\alpha & 1 + 2 \alpha & - \alpha & & 0\\
		0 & -\alpha & \ddots & \ddots & \\
		\vdots & & \ddots & \ddots & -\alpha \\
		0 & 0 & & -\alpha & 1 + 2 \alpha \\
		\end{pmatrix}
	\begin{pmatrix}
		u_{N^{-}+1}^m \\
		\vdots \\
		u_0^m \\
		\vdots \\
		u_{N^{+}-1}^m \\
	\end{pmatrix}
	=
	\begin{pmatrix}
		u_{N^{-}}^{m-1} \\
		\vdots \\
		u_0^{m-1} \\
		\vdots \\
		u_{N^{+}-1}^{m-1} \\
	\end{pmatrix}
	+
	\alpha
	\begin{pmatrix}
		u_{N^{-}}^m \\
		0 \\
		\vdots \\
		0 \\
		u_{N^{+}}^m
	\end{pmatrix}
\end{equation*}
\begin{equation*}
	\begin{pmatrix}
		1 + 2 \alpha & - \alpha & 0 & \dots & 0 \\
		-\alpha & 1 + 2 \alpha & - \alpha & & 0\\
		0 & -\alpha & \ddots & \ddots & \\
		\vdots & & \ddots & \ddots & -\alpha \\
		0 & 0 & & -\alpha & 1 + 2 \alpha \\
		\end{pmatrix}
	\begin{pmatrix}
		u_{N^{-}+1}^m \\
		\vdots \\
		u_0^m \\
		\vdots \\
		u_{N^{+}-1}^m \\
	\end{pmatrix}
	=
	\begin{pmatrix}
		b_{N^{-} + 1}^m \\
		\vdots \\
		b_0^m \\
		\vdots \\
		b_{N^{+} - 1}^m
	\end{pmatrix}
\end{equation*}
Výše uvedenou rovnici můžeme zjednodušeně vyjádřit ve zkrácené formě jako
\begin{equation}
Mu^m = b^m
\end{equation}
kde $u^m$ a $b^m$ označují ($N^{+} - N^{-} - 1$)-rozměrné vektory $u^m = (u_{N^{-} + 1}^m, \dots, u_{N^{+} - 1}^m)$ a $b^m = u^{m-1} + \alpha(u_{N^{-}}^m, 0, 0, \dots, 0, u_{N^{+}}^m)$. $M$ představuje ($N^{+} - N^{-} - 1$)-rozměrnou symetrickou čtvercovou matici definovanou první z matic rovnice (8.12). Lze dokázat, že k matici $M$ existuje inverzní matice $M^{-1}$. Proto lze $u^m$ vyjádřit jako
\begin{equation*}
u^m = M^{-1} b^m
\end{equation*}
Matici $u^m$ lze tedy vypočíst na základě znalosti matice $b^m$, která je zase dána maticí $u^{m-1}$ a hraničními podmínkami. Protože počáteční podmínka specifikuje $u^0$, je možné vypočíst $u^m$ sekvenčně.

\subsection{LU dekompozice}

Ačkoliv je možné řešit problém (8.12) inverzí matic a jejich následným roznásobováním, existuje výpočetně efektivnější metoda. Touto metodu je LU dekompozice.

V rámici LU dekompozice je matice $M$ rozložena na součin dolní trojúhelníkové matice $L$ a horní trojúhelníkové matice $U$. Platí tak
\begin{equation*}
	\begin{pmatrix}
		1 + 2 \alpha & - \alpha & 0 & \dots & 0 \\
		-\alpha & 1 + 2 \alpha & - \alpha & & 0\\
		0 & -\alpha & \ddots & \ddots & \\
		\vdots & & \ddots & \ddots & -\alpha \\
		0 & 0 & & -\alpha & 1 + 2 \alpha \\
	\end{pmatrix}
\end{equation*}
\begin{equation}
	=
	\begin{pmatrix}
		1 & 0 & 0 & \dots & 0 \\
		\ell_{N^{-} + 1} & 1 & \ddots & & \vdots \\
		0 & \ddots & \ddots & \ddots & 0 \\
		\vdots & & \ddots & \ddots & 0 \\
		0 & \dots & 0 & \ell_{N^{+}-2} & 1 \\
	\end{pmatrix}
	\begin{pmatrix}
		y_{N^{-}+1} & z_{N^{-}+1} & 0 & \dots & 0 \\
		0 & y_{N^{-} + 2} & \ddots & & \vdots \\
		0 & \ddots & \ddots & \ddots & 0 \\
		\vdots & & \ddots & \ddots & z_{n^{+} - 2} \\
		0 & & \dots & 0 & 0 & y_{N^{+}-1}
	\end{pmatrix}
\end{equation}
Abychom určili hodnoty $\ell_n$, $y_n$ a $z_n$, stačí vynásobit matice na pravé straně výše uvedené rovnice a výsledek dát do rovnosti s maticí na levé straně rovnice. Po několika následných úpravách získáme
\begin{equation*}
y_{N^{-}+1} = 1 + 2 \alpha
\end{equation*}
\begin{equation*}
y_n = (1 + 2 \alpha) -\frac{\alpha^2}{y_{n-1}},~~~n = N^{-} + 2, \dots, N^{+} - 1
\end{equation*}
\begin{equation}
z_n = -\alpha,~~ \ell_n = -\frac{\alpha}{y_n},~~~ n = N^{-} + 1, \dots, N^{+} - 2
\end{equation}
Z výše uvedeného vyplývá, že jediné, co je třeba vypočíst, je hodnota $y_n$ pro $n = N^{-} + 1, \dots, N^{+} - 1$.

Původní problém $Mu^m = b^m$ tak může být přeformulován do tvaru $L(Uu^m) = b^m$, což lze dále rozdělit na dvojici jednošších podproblémů
\begin{equation*}
Lq^m = b^m
\end{equation*}
a
\begin{equation*}
Uu^m = q^m
\end{equation*}
kde vektor $q^m$ plní roli prostředníka. Pomocí (8.14) lze eliminovat $\ell_n$ a $z_n$ z problému (8.13). Řešení (8.13) se tak nyní přetransformuje do řešení dvou podproblémů
\begin{equation}
	\begin{pmatrix}
		1 & 0 & 0 & \dots & 0 \\
		-\frac{\alpha}{y_{N^{-}+1}} & 1 & 0 & & \vdots \\
		0 & -\frac{\alpha}{y_{N^{-}+2}} & \ddots & \ddots & 0 \\
		\vdots & & \ddots & \ddots & 0 \\
		0 & & \dots & 0 & -\frac{\alpha}{y_{N^{+}-2}} & 1 \\
	\end{pmatrix}
	\begin{pmatrix}
		q_{N^{-} + 1}^m \\
		q_{N^{-} + 2}^m \\
		\vdots \\
		q_{N^{+} - 2}^m \\
		q_{N^{+} - 1}^m \\
	\end{pmatrix}
	=
	\begin{pmatrix}
		b_{N^{-} + 1}^m \\
		b_{N^{-} + 2}^m \\
		\vdots \\
		b_{N^{+} - 2}^m \\
		b_{N^{+} - 1}^m \\
	\end{pmatrix}
\end{equation}
a
\begin{equation}
	\begin{pmatrix}
		y_{N^{-} + 1} &  -\alpha & 0 & \dots & 0 \\
		0 & y_{N^{-} + 2} & -\alpha & & \vdots \\
		0 & 0 & \ddots & \ddots & 0 \\
		\vdots & & \ddots & y_{N^{+} -2} & -\alpha \\
		0 & & \dots & 0 & 0 & y_{N^{+} - 1} \\
	\end{pmatrix}
	\begin{pmatrix}
		u_{N^{-}+1}^m \\
		u_{N^{-}+2}^m \\
		\vdots \\
		u_{N^{+}-2}^m \\
		u_{N^{+}-1}^m \\
	\end{pmatrix}
	=
	\begin{pmatrix}
		q_{N^{-} + 1}^m \\
		q_{N^{-} + 2}^m \\
		\vdots \\
		q_{N^{+} - 2}^m \\
		q_{N^{+} - 1}^m \\
	\end{pmatrix}
\end{equation}
Hodnoty vektoru $q_n^m$ mohou být snadno vypočteny pomocí dopředné substituce. Hodnotu $q_{N^{-}+1}$ lze z (8.15) získat okamžitě. V ostatních případech je třeba pro výpočet $q_n^m$ dle (8.15) znát hodnotu $q_{n-1}^m$. Jestliže řešíme problém (8.15) směrem od první k poslední řádce, máme k dispozici $q_{n-1}^m$, kdykoliv potřebujeme vypočíst hodnotu $q_{n}^m$.
\begin{equation*}
q_{N^{-}+1}^m = b_{N^{-}+1}^m,~~~q_n^m = b_n^m + \frac{\alpha q_{n-1}^m}{y_{n-1}}, ~ n = N^{-}+2, \dots, N^{+}-1
\end{equation*}
Známe-li $q_n^m$, lze pomocí zpětné substituce z (8.16) dopočíst $u_n^m$. Tentokráte lze z (8.16) zíkat přímo $u_{N^{+}-1^m}$ a řešením problému (8.16) od poslední k první řádce postupně dopočítat ostatní hodnoty $u_n^m$.
\begin{equation*}
u_{N^{+}-1}^m = \frac{q_{N^{+}-1}}{y_{N^{+}-1}},~~~u_n^m = \frac{q_n^m + \alpha u_{n+1}^m}{y_n}, ~ n = N^{+}-2, \dots, N^{-} + 1 
\end{equation*}
Postup výpočtu při LU dekompozici je tedy následující:
\begin{itemize}
\item stanovení hodnot $u_n^m$ pro počáteční a hraniční podmínky
\item výpočet vektoru $b^m$
\item výpočet vektoru $q^m$
\item výpočet vektoru $u^m$
\end{itemize}

\subsection{Metoda SOR}

LU dekompozice představuje tzv. přímou metodu řešení problému (8.12). Pomocí této metody lze vypočíst přesné řešení v jednom kroku. Alternativou k přímé metodě je postup založený na počátečním odhadu, který je postupnými iteracemi zpřesňován, až zkonverguje k přesnému řešení (popř. dostatečně blízko k přesnému řešení). Výhodou iterační metody v porovnání s přímou metodou je snadná aplikace na problém americké opce a nelineární modely zahrnující transakční náklady. V případě evropských opcí jsou však nepřímé metody numericky zdlouhavější.

Zkratka metody SOR, která je příkladem iterační metody, je odvozena z anglického názvu ``Successive Over-Relaxation''. Metoda SOR je vylepšením tzv. Gauss-Seidel iterační metody, která je samotná odvozena od Jacobi metody. Výklad metody proto zahájíme vysvětlením těchto dvou jednodušších metod.

Všechny tyto tři iterační metody jsou založené na skutečnosti, že systém (8.11) může být vyjádřen ve tvaru
\begin{equation}
u_n^m = \frac{1}{1 + 2 \alpha}\Big( b_n^m + \alpha(u_{n-1}^m + u_{n+1}^m) \Big)
\end{equation} 

\subsubsection{Jacobi metoda}

Hlavní myšlenkou Jacobi metody je dosadit počáteční odhady $u_n^m$ pro $N^{-} + 1 \le n \le N^{+} - 1$ do pravé strany rovnice (8.17)\footnote{Vhodným odhadem je např. hodnota $u$ z předešlého kroku, tj. $u_{n}^{m-1}$.} s cílem získat nový odhad pro $u_n^m$ na levé straně této rovnice. Tento proces je pak opakován, dokud tyto odhady nezkonvergují k přesné hodnotě popř. nedosáhnou požadované přesnosti.

Formálně lze Jacobi metodu definovat následovně. Nechť $u_n^{m,k}$ je $k$-tá iterace $u_n^m$. Počáteční odhad označujeme $u_n^{m,0}$ a očekáváme, že $u_n^{m,k} \rightarrow u_n^m$ pro $k \rightarrow \infty$. Jestliže známe $u_n^{m,k}$, vypočteme nový odhad $u_n^{m, k +1}$ jako
\begin{equation}
u_n^{m,k+1} = \frac{1}{1 + 2 \alpha} \Big( b_n^m + \alpha(u_{n-1}^{m,k} + u_{n+1}^{m,k}) \Big),~~~ N^{-} < n < N^{+}
\end{equation}
Celý tento proces je opakován dokud chyba, měřená např. pomocí
\begin{equation*}
\| u^{m, k+1} - u^{m,k} \| = \underset{n}{\sum} (u_n^{m,k+1} - u_n^{m,k})^2
\end{equation*}
není dostatečně malá. V tomto případě prohlásíme $u_n^{m,k}$ za $u_n^m$.

Jacobi metoda konverguje k správným hodnotám pro libovolné $\alpha > 0$. Rigorózní důkaz tohoto tvrzení však překračuje záběr této knihy.

\subsubsection{Gauss-Seidel metoda}

Gauss-Seidel metoda je rozvinutím Jacobi metody. Základní myšlenkou této metody je, že při výpočtu $u_n^{m,k+1}$ pomocí (8.18) již známe $u_{n-1}^{m,k+1}$, které je použito namísto $u_{n-1}^{m,k}$. Rovnice (8.18) se tak změní na
\begin{equation*}
u_n^{m,k+1} = \frac{1}{1 + 2 \alpha} \Big( b_n^m + \alpha(u_{n-1}^{m,k+1} + u_{n+1}^{m,k}) \Big),~~~ N^{-} < n < N^{+}
\end{equation*}
Rozdíl mezi Jacobi a Gauss-Seidel metodou je tedy ten, že Gaus-Seidel metoda používá aktuální odhad v okamžiku, kdy je dostupný, zatímco Jacobi metoda až v okamžiku, kdy jsou dostupné všechny odhady pro daný iterační krok. Gauss-Seidel metoda tak konverguje ke správným hodnotám rychleji než Jacobi metoda.

Stejně jako v případě Jacobi metody i Gauss-Seidel metoda konverguje k správnému řešení pro $\alpha > 0$.

\subsubsection{Metoda SOR}

Metoda SOR je zlepšením Gauss-Seidel metody. Začněme zdánlivě triviálním konstatováním
\begin{equation*}
u_n^{m,k+1} = u_n^{m,k} + (u_n^{m,k+1} - u_n^{m,k})
\end{equation*}
S tím, jak iterační řada $u_n^{m,k}$ konverguje k $u_n^m$ pro $k \rightarrow \infty$, můžeme $(u_n^{m,k+1} - u_n^{m,k})$ chápat jako korekci, kterou je třeba přičíst k $u_n^{m,k}$ abychom se tak přiblížili správné hodnotě $u_n^m$. Nosnou myšlenkou metody je, že tuto konvergenci můžeme tím víc urychlit, čím větší bude tato ``korekce''. Uvažovaný přepoklad je správný, pokud má iterační řada $u_n^{m,k} \rightarrow u_n^m$ monotónní a nikoliv oscilační charakter, což platí pro Gauss-Seidel i SOR metodu. To znamená, že
\begin{equation*}
y_n^{m,k+1} = \frac{1}{1 + 2 \alpha} \Big( b_n^m + \alpha(u_{n-1}^{m,k+1} + u_{n+1}^{m,k}) \Big)
\end{equation*}
\begin{equation*}
u_n^{m, k+1} = u_n^{m,k} + \omega(y_n^{m,k+1}-u_n^{m,k})
\end{equation*}
kde $\omega > 1$ představuje parametr korekce. Lze dokázat, že metoda SOR konverguje k správným výsledkům pro $\alpha > 0$ a $1 < \omega < 2$\footnote{Pozamenejme, že pro $0 < \omega < 1$ by se namísto metody ``Succesive Over-Relaxation'' jednalo o metodu ``Succesive Under-Relaxation''. V případě $\omega = 1$ by se jednalo o Gauss-Seidel metodu.}. Dále lze dokázat, že v intervalu $1 < \omega < 2$ existuje právě jedno takové $\omega$, pro které je konvergence rychlejší než pro ostatní hodnoty z tohoto intervalu. Optimální hodnota $\omega$ závisí na charakteristikách matice. Ačkoliv existují způsoby výpočtu této optimální hodnoty, je zpravidla efektivnější $\omega$ změnit po každém časovém kroku, dokud není nalezena hodnota minimalizující počet iterací v jednom kroku.

\subsubsection{Aplikace implicitní diferenční metody}

K řešení (8.12) pro jednotlivé časové kroky lze použít LU dekompozici nebo metodu SOR. To umožňuje vypočíst hodnotu opce na konci těchto časových kroků.
\begin{table}
\begin{center}
\begin{tabular}{r r r r r r}
\multicolumn {1}{c}{$S$} &
\multicolumn {1}{c}{$\alpha = 0.50$} &
\multicolumn {1}{c}{$\alpha = 1.00$} &
\multicolumn {1}{c}{$\alpha = 5.00$} &
\multicolumn {1}{c}{BS rovnice} \\
\hline
 0.00 & 9.7531 & 9.7531 & 9.7531 & 9.7531 \\
 2.00 & 7.7531 & 7.7531 & 7.7531 & 7.7531 \\
 4.00 & 5.7531 & 5.7531 & 5.7530 & 5.7531 \\
 6.00 & 3.7569 & 3.7531 & 3.7573 & 3.7569 \\
 8.00 & 1.9025 & 1.9025 & 1.9030 & 1.9024 \\
10.00 & 0.6690 & 0.6689 & 0.6675 & 0.6694 \\
12.00 & 0.1674 & 0.1674 & 0.1670 & 0.1675 \\
14.00 & 0.0327 & 0.0328 & 0.0332 & 0.0326 \\
16.00 & 0.0054 & 0.0055 & 0.0058 & 0.0054 \\
\hline
\end{tabular}
\end{center}
\caption{Porovnání výsledků Black-Scholes rovnice a výsledků implicitní diferenční metody pro evropskou prodejní opci s parametry $E=10$, $r=0.1$, $\sigma = 0.4$ a $T - t = 0.25$}
\end{table}

Tabulka (8.2) porovnává výsledky Black-Scholes rovnice a výsledky získané pomocí implicitní diferenční metody. Logika výpočtu je stejná jako v případě explicitní diferenční metody, tj. nejprve je stanovena velikost $\delta x$ a následně je zvoleno $\delta \tau$ takové, aby se dosáhlo požadované hodnoty $\alpha$. Je patrné, že implicitní diferenční metoda není zatížená problémem stability pro $\alpha > 0.5$. Lze dokázat, že tato metoda je stabilní pro libovolné $\alpha > 0$. Je tedy možné pracovat s větším $\delta \tau$, což vede k vyšší efektivitě výpočtu. Dále lze dokázat, že implicitní diferenční metoda konverguje k správným výsledkům za předpokladu splnění podmínky stability, tj. $\alpha > 0$. Zmiňované důkazy však přesahují záběr této knihy, a proto je od nich opuštěno.

\subsubsection{Crank-Nicolason metoda}

Crank-Nicolson metoda umožňuje k obejití problému stability explicitní diferenční metody při míře konvergence $\mathcal{O}\Big( (\delta \tau) \Big)^2$ k přesnému řešení\footnote{Míra konvergence explicitní a klasické implicitní metody je $\mathcal{O}(\delta \tau)$.}.

Samotná Crank-Nicolson metoda je průměrem implicitní a explicitní metody. Při aplikaci dopředné diferenční aproximace na difúzní rovnici získáme explicitní schéma
\begin{equation*}
\frac{u_n^{m + 1} - u_n^m}{\delta \tau} + \mathcal{O}(\delta \tau) = \frac{u_{m+1}^m - 2u_n^m + u_{n-1}^m}{(\delta x)^2} + \mathcal{O}\Big( (\delta x)^2\Big)
\end{equation*} 
Použitím zpětné diferenční aproximace pak dostáváme implicitní schéma
 \begin{equation*}
\frac{u_n^{m + 1} - u_n^m}{\delta \tau} + \mathcal{O}(\delta \tau) = \frac{u_{m+1}^{m+1} - 2u_n^{m+1} + u_{n-1}^{m+1}}{(\delta x)^2} + \mathcal{O}\Big( (\delta x)^2 \Big)
\end{equation*}
Průměrem těchto dvou rovnic je
\begin{equation}
\frac{u_n^{m+1} - u_n^m}{\delta \tau} = \frac{1}{2}\Bigg( \frac{u_{m+1}^m - 2u_n^m + u_{n-1}^m}{(\delta x)^2} + \frac{u_{m+1}^{m+1} - 2u_n^{m+1} + u_{n-1}^{m+1}}{(\delta x)^2} \Bigg) + \mathcal{O}\Big( (\delta x)^2 \Big)
\end{equation}
Lze dokázat, že míra přesnosti (8.19) je spíše než $\mathcal{O}(\delta \tau)$ rovna $\mathcal{O}\Big( (\delta \tau) \Big)^2$. Zanedbáním chybových členů získáme Crank-Nicolson schéma
\begin{equation*}
u_n^{m+1} - \frac{1}{2}\alpha \Big( u_{n-1}^{m+1} - 2u_n^{m+1} + u_{n+1}^{m+1} \Big) = u_n^m + \frac{1}{2}\alpha \Big( u_{n-1}^m - 2u_n^m + u_{n+1}^m \Big)
\end{equation*}
kde stejně jako dříve
\begin{equation}
\alpha = \frac{\delta \tau}{(\delta x)^2}
\end{equation}
Všimněme si, že $u_n^{m+1}$, $u_{n-1}^{m+1}$ a $u_{n+1}^{m+1}$ jsou v (8.20) implicitně určeny členy $u_n^m$, $u_{n+1}^m$ a $u_{n-1}^m$. Řešení této rovnice je v zásadě shodné s řešením rovnice (8.11). To je dáno tím, že všechny členy na pravé straně rovnice (8.20) mohou být vypočteny, jsou-li $u_n^m$ známa. Problém se tak přeformuluje do podoby, kdy je nejprve třeba vypočíst
\begin{equation*}
Z_n^m = (1 - \alpha)u_n^m + \frac{1}{2}\alpha \Big( u_{n-1}^m + u_{n+1}^m \Big)
\end{equation*}
a následně vyřešit
\begin{equation}
(1 + \alpha)u_n^{m+1} - \frac{1}{2}\alpha \Big( u_{n-1}^{m+1} + u_{n+1}^{m+1} \Big) = Z_n^m
\end{equation}
Tento druhý problém v shodný s (8.11).

Opět předpokládejme, že je možné osu $x$ diferenční sítě ohraničit body $x = N^{-}\delta x$ a $x = N^{+} \delta x$, kde $N^{-}$ a $N^{+}$ jsou dostatečně velká. Dále vypočtěme $u_n^0$ pomocí (8.10) a $u_{N^{-}}^m$ a $u_{N^{+}}^m$ pomocí (8.9).

Dalším krokem je výpočet $u_n^m$ pro $m \ge 1$ a $N^{-} < n < N^{+}$ na základě (8.21). Tento problém může být zapsán jako lineární systém
\begin{equation}
Cu^{m+1} = b^m
\end{equation}
kde
\begin{equation*}
	C =
	\begin{pmatrix}
		1 + \alpha & -\frac{1}{2}\alpha & 0 & \dots & 0 \\
		-\frac{1}{2}\alpha & 1 + \alpha & -\frac{1}{2}\alpha & & \vdots \\
		0 & -\frac{1}{2}\alpha & \ddots & \ddots & 0 \\
		\vdots & & \ddots & \ddots & -\frac{1}{2}\alpha \\
		0 & 0 & & -\frac{1}{2}\alpha & 1 + \alpha
	\end{pmatrix}
\end{equation*}
\begin{equation*}
	u^{m+1} =
	\begin{pmatrix}
		u_{N^{-}+1}^{m+1} \\
		\vdots \\
		u_0^{m+1} \\
		\vdots \\
		u_{N^{+}-1}^{m+1}
	\end{pmatrix}
,~~~
	b^m =
	\begin{pmatrix}
		Z_{N^{-}+1}^{m} \\
		\vdots \\
		Z_0^{m} \\
		\vdots \\
		Z_{N^{+}-1}^{m}
	\end{pmatrix}
	+ \frac{1}{2}\alpha
	\begin{pmatrix}
		u_{N^{-}}^{m+1} \\
		0 \\
		\vdots \\
		0 \\
		u_{N^{+}}^{m+1}
	\end{pmatrix}
\end{equation*}
Při implementaci Crank-Nicolson metody je nejprve vypočten vektor $b^m$ na základě známých veličin. Následně řešíme problém (8.22) pomocí LU popř. SOR metody. To nám umožňuje se přes jednotlivé časové kroky dopracovat k výslednému řešení. Jediným rozdílem mezi Crank-Nicolson metodou a klasickými implicitními metodami LU a SOR  je nahrazení $\alpha$ členem $\frac{1}{2}\alpha$.

Tabulka (8.3) porovnává Crank-Nicolson metodu s přesnými výsledky pro Black-Scholes parciální diferenciální rovnici. Crank-Nicolson metoda je narozdíl od explicitní metody stabilní pro $\alpha < 0.5$ a navíc přesnější v porovnání s implicitními metodami LU a SOR. Lze dokázat, že Crank-Nicolson metoda je stabilní a kovergentní pro libovolné $\alpha > 0$.
\begin{table}
\begin{center}
\begin{tabular}{r r r r r r}
\multicolumn {1}{c}{$S$} &
\multicolumn {1}{c}{$\alpha = 0.50$} &
\multicolumn {1}{c}{$\alpha = 1.00$} &
\multicolumn {1}{c}{$\alpha = 10.00$} &
\multicolumn {1}{c}{BS rovnice} \\
\hline
 0.00 & 9.6722 & 9.6722 & 9.6722 & 9.6722 \\
 2.00 & 7.6721 & 7.6721 & 7.6721 & 7.6722 \\
 4.00 & 5.6722 & 5.6722 & 5.6723 & 5.6723 \\
 6.00 & 3.6976 & 3.6976 & 3.6975 & 3.6977 \\
 8.00 & 1.9804 & 1.9804 & 1.9804 & 1.9806 \\
10.00 & 0.8605 & 0.8605 & 0.8566 & 0.8610 \\
12.00 & 0.3174 & 0.3174 & 0.3174 & 0.3174 \\
14.00 & 0.1047 & 0.1047 & 0.1046 & 0.1046 \\
16.00 & 0.0322 & 0.0322 & 0.0321 & 0.0322 \\
\hline
\end{tabular}
\end{center}
\caption{Porovnání výsledků Black-Scholes rovnice a výsledků Crank-Nicolson metody pro evropskou prodejní opci s parametry $E=10$, $r=0.1$, $\sigma = 0.45$ a $T - t = \frac{1}{3}$}
\end{table}

\chapter{Metody výpočtu hodnoty americké opce}

\section{Úvod}

V případě evropské opce je aplikace diferenčních metod poměrně jednoduchá. Jak bylo ukázáno v jedné z předchozích kapitol má možnost předčasného uplatnění za následek tzv. volné hraniční podmínky. Hlavním problémem volných hraničních podmínek z pohledu numerické metody je, že neznáme dopředu jejich polohu. To znemožňuje přímou aplikaci volných hraničních podmínek, protože jejich poloha je výstupem výpočtu.

Existují dvě strategie pro aplikaci volných hraničních podmínek v rámci numerické metody. Jednou z nich je postupná specifikace polohy podmínky v průběhu výpočtu. V kontextu amerických opcí však tento postup není ideální, protože obě volné hraniční podmínky jsou implicitní. To znamená, že neexistuje přímé vyjádření volné hraniční podmínky. Další strategií je nalezení transformace, která zredukuje problém do podoby fixní hraniční podmínky, ze které lze původní volnou hraniční podmínku následně dopočítat. Existuje řada těchto transformací. V této knize se však zaměříme pouze na jednu z nich.

Připomeňme, že problém stanovení hodnoty americké opce lze s využitím lineární komplementarity vyjádřit jako
\begin{equation*}
\Bigg( \frac{\partial u}{\partial \tau} - \frac{\partial^2 u}{\partial x^2} \Bigg) \ge 0,~~~(u(x, \tau) - g(x,\tau)) \ge 0
\end{equation*}
\begin{equation}
\Bigg( \frac{\partial u}{\partial \tau} - \frac{\partial^2 u}{\partial x^2} \Bigg) \cdot (u(x, \tau) - g(x,\tau)) = 0
\end{equation}
Transformovaná výplatní funkce $g(x, \tau)$ je dána rovnicí
\begin{equation*}
g(x, \tau) = e^{\frac{1}{4}(k+1)^2 \tau}\max\Big( e^{\frac{1}{2}(k-1)x} - e^{\frac{1}{2}(k+1)x}, 0 \Big)
\end{equation*}
pro prodejní opce, rovnicí
\begin{equation*}
g(x, \tau) = e^{\frac{1}{4}(k+1)^2 \tau}\max\Big( e^{\frac{1}{2}(k+1)x} - e^{\frac{1}{2}(k-1)x}, 0 \Big)
\end{equation*}
pro kupní opci a rovnicí
\begin{equation*}
g(x, \tau) = 0, ~~~ x < 0
\end{equation*}
\begin{equation*}
g(x, \tau) = e^{\frac{1}{4}(k+1)^2 \tau}b e^{\frac{1}{2}(k-1)x}, ~~~ x \ge 0
\end{equation*}
pro cash-or-nothing kupní opci, kde $k = \frac{r}{\frac{1}{2}\sigma^2}$. Počáteční a fixní hraniční podmínky mají následující podobu\footnote{Tyto podmínky platí kromě tří výše zmiňovaných také pro ostatní výplatní funkce.}
\begin{equation*}
u(x, 0) = g(x, 0)
\end{equation*}
\begin{center}
$u(x, \tau)$ je spojité
\end{center}
\begin{center}
$\frac{\partial u}{\partial x}(x, \tau)$ je stejně spojité jako $g(x, \tau)$
\end{center}
\begin{equation}
\underset{x \rightarrow \pm \infty} {\lim} u(x, \tau) = \underset{x \rightarrow \pm \infty} {\lim} g(x, \tau)
\end{equation}
Jednou z hlavních výhod (9.1) je, že neobsahuje žádnou explicitní zmínku o volné hraniční podmínce. Po vyřešení tohoto problému je možné nalézt hraniční podmínku $x = x_f(\tau)$ pomocí podmínek, které ji definují. Konkrétně se jedná podmínku
\begin{equation*}
u(x_f(\tau), \tau) = g(x_f(\tau), \tau) ~~~\text{ale}~ u(x, \tau) > g(x, \tau) ~\text{pro}~ x > x_f{\tau}
\end{equation*}
pro prodejní opci a podmínku
\begin{equation*}
u(x_f(\tau), \tau) = g(x_f(\tau), \tau) ~~~\text{ale}~ u(x, \tau) > g(x, \tau) ~\text{pro}~ x < x_f{\tau}
\end{equation*}
pro kupní opci. Tyto podmínky zůstávají v platnosti i v případě, že existuje vícero volných hranic nebo naopak neexistuje žádná volná hranice; volné hranice jsou definovány jako body, ve kterých se $u(x, \tau)$ prvně dotkne $g(x, \tau)$.

\section{Diferenční metoda výpočtu}

Vraťme se k (9.1). Pro účely ilustrace difereční metody výpočtu budeme uvažovat pouze Crank-Nicolson metodu. Proto použijeme
\begin{equation*}
\frac{\partial u}{\partial \tau} \Big(x, \tau + \frac{1}{2} \delta \tau \Big) = \frac{u_n^{m+1} - u_n^m}{\delta \tau} + \mathcal{O}\Big( (\delta \tau)^2 \Big)
\end{equation*}
a
\begin{equation*}
\frac{\partial^2 u}{\partial x^2}\Big(x, \tau + \frac{1}{2} \delta \tau \Big) = \frac{1}{2}\frac{u_{n+1}^{m+1} - 2u_n^{m+1}+u_{n-1}^{m+1}}{(\delta x)^2} + \frac{1}{2}\frac{u_{n+1}^m - 2u_n^m + u_{n-1}^m}{(\delta x)^2} + \mathcal{O}\Big( (\delta x)^2 \Big)
\end{equation*}
Po zanedbání členů $\mathcal{O}\Big( (\delta \tau)^2 \Big)$ a $\mathcal{O}\Big( (\delta x)^2 \Big)$ lze nerovnost
\begin{equation*}
\frac{\partial u}{\partial \tau} - \frac{\partial^2 u}{\partial x^2} \ge 0
\end{equation*}
aproximovat pomocí diferenční nerovnosti
\begin{equation}
u_n^{m+1} - \frac{1}{2} \alpha (u_{n+1}^{m+1} - 2u_{n}^{m+1} + u_{n-1}^{m+1}) \ge u_n^m + \frac{1}{2} \alpha (u_{n+1}^m - 2u_{n}^m + u_{n-1}^m)
\end{equation}
kde $\alpha = \frac{\delta \tau}{(\delta x)^2}$. Výplatní funkci $g(x, \tau)$ budeme v kontextu diferenční sítě vyjadřovat jako
\begin{equation*}
g(x, \tau) = g(n \delta x, m \delta \tau)
\end{equation*}
Podmínka $u(x, tau) \ge g(x, \tau)$ je pak aproximována nerovností
\begin{equation}
u_n^m \ge g_n^m, ~~~ m \ge 1
\end{equation}
Počáteční a hraniční podmínky (9.2) implikují
\begin{equation}
u_n^0 = g_n^0
\end{equation}
\begin{equation}
u_{N^{-}}^m = g_{N^{-}}^m, ~~~ u_{N^{+}}^m = g_{N^{+}}^m
\end{equation}
Jestliže definujeme $Z_n^m$ jako
\begin{equation*}
Z_n^m = (1 - \alpha)u_n^m + \frac{1}{2} \alpha (u_{n+1}^m + u_{n-1}^m)
\end{equation*}
lze (9.3) vyjádřit jako
\begin{equation}
(1 - \alpha)u_n^{m+1} + \frac{1}{2} \alpha (u_{n+1}^{m+1} + u_{n-1}^{m+1}) \ge Z_n^m 
\end{equation}
V $m$-tém časové kroce lze $Z_n^m$ vypočíst explicitně, protože známe všechny hodnoty $u_n^m$. Podmínka lineární komplementarity
\begin{equation*}
\Bigg( \frac{\partial u}{\partial \tau} - \frac{\partial^2 u}{\partial x^2} \Bigg) \cdot \Big( u(x, \tau) - g(x, \tau) \Big) = 0
\end{equation*}
je aproximována pomocí
\begin{equation}
\Big((1 + \alpha)u_n^{m+1}-\frac{1}{2}\alpha(u_{n+1}^{m+1} + u_{n-1}^{m+1}) - Z_n^m \Big) \cdot (u_n^{m+1} - g_n^{m+1}) = 0
\end{equation}

\section{Maticový problém s podmínkou}

Diferenční aproximace (9.4) - (9.8) lze přeformulovat do podoby maticového problému s omezující podmínkou.

Nechť $u^m$ značí vektor přibližných hodnot v časovém kroce $m$ a $g^m$ značí vektor představující omezení v tomtéž časovém kroce.
\begin{equation*}
	u^m =
	\begin{pmatrix}
		u_{N^{-}+1}^m \\
		\vdots \\
		u_{N^{+}-1}^m
	\end{pmatrix}
,~~~
	g^m =
	\begin{pmatrix}
		g_{N^{-}+1}^m \\
		\vdots \\
		g_{N^{+}-1}^m
	\end{pmatrix}
\end{equation*}
Předmětem výpočtu nejsou hodnoty $u_{N^{-}}^m$ a $u_{N^{+}}^m$, protože ty jsou explicitně dány hraničními podmínkami (9.6). Definujme vektor $b^m$ jako
\begin{equation*}
b^m =
	\begin{pmatrix}
		Z_{N^{-}+1}^m \\
		\vdots \\
		Z_0^m\\
		\vdots \\
		Z_{N^{+}-1}^m
	\end{pmatrix}
	+ \frac{1}{2} \alpha
		\begin{pmatrix}
		g_{N^{-}+1}^{m+1} \\
		0 \\
		\vdots\\
		0 \\
		g_{N^{+}-1}^{m+1}
	\end{pmatrix}
\end{equation*}
Jestliže použijeme matici
\begin{equation*}
	C =
	\begin{pmatrix}
		1 + \alpha & -\frac{1}{2}\alpha & 0 & \dots & 0 \\
		-\frac{1}{2}\alpha & 1 + \alpha & -\frac{1}{2}\alpha & & \vdots \\
		0 & -\frac{1}{2}\alpha & \ddots & \ddots & 0 \\
		\vdots & & \ddots & \ddots & -\frac{1}{2}\alpha \\
		0 & 0 & & -\frac{1}{2}\alpha & 1 + \alpha
	\end{pmatrix}
\end{equation*}
kterou jsme představili v rámci Crank-Nicolson metody v předešlé kapitole, lze vyjádřit diskrétní aproximaci (9.4) - (9.8) problému (9.1) - (9.2) v maticovém tvaru jako
\begin{equation*}
Cu^{m+1} \ge b^m, ~~~ u^{m+1} \ge g^{m+1}
\end{equation*}
\begin{equation}
(u^{m+1} - g^{m+1}) \cdot (Cu^{m+1} - b^m) = 0
\end{equation}
Postupné řešení v jednotlivých časových krocích je tomuto problému vlastní. Vektor $b^m$ obsahuje informaci z časového kroku $m$, který určuje hodnotu $u^{m+1}$ v časovém kroce $m+1$. V každém časovém kroce můžeme vypočíst $b^m$ z již známých hodnot $u^m$. Hodnotu výplatní funkce $g^m$ jsme schopni vypočíst pro libovolný časový krok. Proto, abychom postupovali po jednotlivých časových krocích, stačí pouze postupně řešit problém (9.9). K řešení tohoto problému je možné použít modifikovanou metodu SOR označovanou jako řízená metoda SOR.

\section{Řízená metoda SOR}

Řízená metoda SOR je modifikací standardní metody SOR představené v předšlé kapitole. Jestliže bychom standardní metodu SOR aplikovali na Crank-Nicolson metodu, získáme rovnice
\begin{equation*}
y_n^{m+1, k+1} = \frac{1}{1 + \alpha} \Big( b_n^m + \frac{1}{2} \alpha (u_{n-1}^{m+1, k+1} + u_{n+1}^{m+1, k}) \Big)
\end{equation*}
\begin{equation*}
u_n^{m+1, k+1} = u_n^{m+1, k} + \omega(y_n^{m+1, k+1} - u_n^{m+1, k})
\end{equation*}
Jestliže bychom tyto rovnice iterovali, dokud $u_n^{m+1,k}$ nezkonverguje k $u_n^{m+1}$, získali bychom řešení rovnice $Cu^{m+1} = b^m$. Pro splnění podmínky $u^{m+1} \ge g^{m+1}$ stačí modifikovat druhou z podmínek do tvaru
\begin{equation*}
u_n^{m+1, k+1} = \max (u_n^{m+1, k} + \omega(y_n^{m+1, k+1} - u_n^{m+1, k}), g_n^{m+1})
\end{equation*}
Tato podmínka je uplatněna v okamžiku, kdy je vypočteno $u_n^{m+1, k+1}$, a promítne se tak do výpočtu $u_{n+1}^{m+1, k+1}$, $u_{n+1}^{m+1, k+1}$ atd. Cílem řízené metody SOR je tak iterovat rovnice
\begin{equation*}
y_n^{m+1, k+1} = \frac{1}{1 + \alpha} \Big( b_n^m + \frac{1}{2} \alpha (u_{n-1}^{m+1, k+1} + u_{n+1}^{m+1, k}) \Big)
\end{equation*}
\begin{equation}
u_n^{m+1, k+1} = \max (u_n^{m+1, k} + \omega(y_n^{m+1, k+1} - u_n^{m+1, k}), g_n^{m+1})
\end{equation}
dokud není rozdíl $\| u^{m+1, k+1} - u^{m+1,k} \|$ dostatečně malý.

Vzhledem k tomu, že je řízená metoda SOR iterační, je každé jednotlivé řešení samo o sobě konzistentní a neporušuje tedy žádnou z podmínek. Navíc takovéto řešení má tu vlastnost, že buď $u_n^{m+1} = g_n^{m+1}$ nebo že $n$-tý člen $Cu^{m+1}-b^m$ má nulovou hodnotu. Proto algoritmus řízené metody SOR garantuje, že platí $u^{m+1} \ge g^{m+1}$ a $(Cu^{m+1}-b^m)(u^{m+1}-b^m)=0$. Podmínka $Cu^{m+1} \ge b^m$ vyplývá ze struktury matice $C$ (zejména ze skutečnosti, že je tato matice positivně definitní).

\subsection{Technická poznámka: Vnitřní konzistence řešení}

Ze soustavy (9.9) vyplývá, že pro každý člen $u_n^{m+1}$ vektoru $u^{m+1}$ existují pouze dvě možnosti
\begin{itemize}
\item $u_n^{m+1} > g_n^{m+1}$
\item $u_n^{m+1} = g_n^{m+1}$
\end{itemize}
První případ odpovídá situaci, kdy je optimální opci držet; druhý případ pak odpovídá situaci, kdy je naopak optimální opci uplatnit. Navíc z podmínky lineární komplementarity v soustavě (9.9) je zřejmé, že výše uvedené dvě situace nezbytně implikují
\begin{itemize}
\item $(Cu^{m+1})_n = b_n^m$
\item $(Cu^{m+1})_n > b_n^m$
\end{itemize}
Všimněme si, že je zde zakomponována podmínka vnitřní konzistence řešení. Nelze vyřešit $Cu^{m+1} = b^m$ a následně uplatnit podmínku, že je-li $u_n^{m+1}$ větší než $g_n^{m+1}$, platí $u_n^{m+1}=g_n^{m+1}$. Tento postup je možné aplikovat na explicitní metodu. Nicméně v případě implicitní metody jsou spolu jednotlivé členy vektoru $u^{m+1}$ provázány a nelze proto izolovaně měnit hodnotu jednoho ze členů. V opačném by neexistovala záruka splnění podmínek $Cu^{m+1} \ge b^m$ a $(Cu^{m+1}-b^m)(u^{m+1}-g^{m+1})=0$. Výsledkem by byla ``řešení'', která buďto nesplňují volnou hraniční podmínku (tj. představují menší než optimální hodnotu opce) nebo nesplňují Black-Scholes nerovnost (což má za důsledek existenci arbitráže). Ačkoliv zde nepředkládáme důkaz, platí, že vnitřně konzistentní řešení je také jedinečným řešením problému (9.9).

\section{Algoritmus}

Jak již bylo řečeno, je algoritmus přechodu $u^m$ na $u^{m+1}$ jednoduchou modifikací metody SOR, kterou jsme uplatnili na evropskou opci. Nechť stejně jako dříve vektor $u^{m+1,k} = u_{N^{-}+1}^{m+1,k},..., u_{N^{+}-1}^{m+1,k}$ označuje $k$-tou iteraci pro $m+1$-ní časový krok. Postup algoritmu pro výpočet $u^{m+1}$ je následující.
\begin{enumerate}
\item Na základě znalosti $u^m$ nejprve vypočteme vektor $b^m$ a vektor hraniční podmínky $g^{m+1}$.
\begin{equation*}
g^{m+1} =
	\begin{pmatrix}
		g_{N^{-}+1}^{m+1} \\
		\vdots \\
		g_{N^{+}-1}^{m+1}
	\end{pmatrix}
,~~~ b^{m} =
	\begin{pmatrix}
		Z_{N^{-}+1}^{m} \\
		\vdots \\
		Z_0^{m+1}\\
		\vdots \\
		Z_{N^{+}-1}^{m}
	\end{pmatrix}
	+ \frac{1}{2} \alpha
		\begin{pmatrix}
		g_{N^{-}+1}^{m+1} \\
		0 \\
		\vdots\\
		0 \\
		g_{N^{+}-1}^{m+1}
	\end{pmatrix}
\end{equation*}
\item Iteraci zahájíme s počátečním odhadem $u^{m+1,0} = \max(u_n^m, g_n^{m+1})$.
\item V rostoucím směru proměnné $n$ nejprve vypočteme $y_n^{k+1}$
\begin{equation*}
y_n^{m+1, k+1} = \frac{1}{1 + \alpha} \Big( b_n^m + \frac{1}{2} \alpha (u_{n-1}^{m+1, k+1} + u_{n+1}^{m+1, k}) \Big)
\end{equation*}
a následně $u_n^{m+1, k+1}$
\begin{equation*}
u_n^{m+1, k+1} = \max (u_n^{m+1, k} + \omega(y_n^{m+1, k+1} - u_n^{m+1, k}), g_n^{m+1})
\end{equation*}
kde $1 < \omega < 2$.
\item Dále je třeba otestovat, zda-li je $\| u^{m+1, k+1}-u^{m+1,k}\|$ menší než zvolená odchylka $\epsilon$.
\begin{equation*}
\sum_n (u_n^{m+1,k+1} - u_n^{m+1,k})^2 \le \epsilon^2
\end{equation*}
\item Jestliže vektor $u^{m+1,k}$ zkonvergoval k požadované toleranci, položíme $u^{m+1} = u^{m+1,k+1}$.
\item Pokud nebyly vypočtěny hodnoty vektoru $u$ pro všechna $m$, vrátíme se k bodu jedna a pokračujeme dalším časovým krokem.
\end{enumerate}

\subsection{Technická poznámka: Bermudská opce}

Řízená metoda SOR je zobecněním standardní metody SOR. Obě metody jsou identické s tím rozdílem, že
\begin{equation*}
u_n^{m+1, k+1} = u_n^{m+1, k} + \omega(y_n^{m+1, k+1} - u_n^{m+1, k})
\end{equation*}
je v řízené metodě SOR nahrazeno
\begin{equation*}
u_n^{m+1, k+1} = \max (u_n^{m+1, k} + \omega(y_n^{m+1, k+1} - u_n^{m+1, k}), g_n^{m+1})
\end{equation*}
Hlavní výhodou metody SOR oproti LU dekompozici je, že se programový kód pro ocenění americké opce liší od kódu pro ocenění evropské opce pouze jedním řádkem. Tato vlastnost je klíčová pro stanovení hodnoty bermudské opce, která může být uplatněna pouze v předem stanovených časech. V časech, kdy je možné opci předčasně uplatnit, použijeme řízenou metodu SOR. Naopak v časech, kdy opci není možné předčasně uplatnit, použijeme standardní metodu SOR. Rozdíl v kódu pro jednotlivé časové kroky tak spočívá pouze v tom, zda-li použijeme funkci $\max$ či nikoliv.

\chapter{Binomické metody}

\section{Úvod}

Binomické metody pro oceňování opcí a ostatních finančních derivátů jsou založeny na diskrétním modelu náhodné procházky. Na Black-Scholes analýze závisí pouze nepřímo skrze předpoklad rizikové neutrality a z pohledu diferenčních rovnic je možné na binomické metody pohlížet jako na speciální případ explicitní diferenční rovnice.

Binomické metody jsou založeny na dvou základních myšlenkách. První z nich je, že náhodnou procházku (2.1) je možné přeformulovat do podoby nespojitého modelu s následujícími vlastnostmi.
\begin{itemize}
\item Cena podkladového aktiva $S$ se mění pouze v diskrétních časech $\delta t$, $2 \delta t$, $3 \delta t$, ..., $M \delta t$, kde $M \delta t = T$ je splatnost uvažovaného derivátu. V následujících textu budeme používat $\delta t$ namísto $d t$ k označení malého nicnéně nikoliv infinityzimálního časového intervalu.
\item Jestliže je cena podkladového aktiva v čase $m \delta t$ rovna $S^m$, může být v čase $(m+1) \delta t$ rovna pouze $u S^m > S^m$ nebo $d S^m < S^m$. To je ekvivalentní předpokladu, že na konci každého časového kroku jsou možné pouze dvě výnosové míry $\frac{\delta S}{S}$ a to $u - 1 > 0$ resp. $d - 1 < 0$.
\item Pravděpodobnost $p$ růstu ceny z $S^m$ na $u S^m$ je známa, stejně jako je známa pravděpodobnost $1 - p$ poklesu ceny na $d S^m$.
\end{itemize}
\begin{center}
  \begin{pspicture}(0,0)(6.0,5.0)
        \rput(3.0,0.7){Změna ceny podkladového aktiva}
	\rput(3.0,0.3){v rámci binomického modelu}

	\psline[arrows=->](0.7,3.0)(5.0,4.5)
	\psline[arrows=->](0.7,3.0)(5.0,1.5)

	\rput(0.4,3.0){\small{$S$}}
	\rput(5.4,4.5){\small{$uS$}}
	\rput(5.4,1.5){\small{$dS$}}

	\rput(2.5,4.0){\small{$p$}}
	\rput(2.5,2.0){\small{$1-p$}}
  \end{pspicture}
\end{center}

Druhým předpokladem je předpoklad rizikově neutrálního světa, kde se hodnota finančních instrumentů neodvíjí od rizikových preferencí investorů. Tento předpoklad je možné přijmout kdykoliv, kdy je možné pomocí zajištění vytvořit bezrizikové portfolio. V takovémto případě jsou investoři rizikově neutrální a výnos z bezrizikového portfolia pak odpovídá bezrizikové výnosové míře. Parametr $\mu$ v diferenciální rovnici $dS = \sigma S dX + \mu S dt$ je měřítkem očekávaného růstu pokladového aktiva a jak jsme již ukázali, není vstupem pro Black-Scholes rovnici. Připomeňme si diskuzi v kapitole 5.7, kde jsme došli k závěru, že v rizikově neutrálním svět nahradíme stochastickou diferenciální rovnici (2.1) rovnicí
\begin{equation}
\frac{dS}{S} = \sigma dX + rdt
\end{equation}
Volbou vhodných hodnot pro $u$, $d$ a $p$ dosáhneme toho, aby (10.1) měla požadované statistické vlastnosti, tj. míru růstu $r$ namísto $\mu$.

V rizikově neutrálním světě je hodnota finančního derivátu $V^m$ v čase $m \delta t$ rovna očekávané hodnotě derivátu v čase $(m+1) \delta t$ diskontované bezrizikovou úrokovou mírou $r$.
\begin{equation}
V^m = \varepsilon[e^{-r \delta t}V^{m+1}]
\end{equation}

V rámci binomické metody nejprve zkonstruujeme tzv. binomický strom možných hodnot podkladového aktiva a jim odpovídajících pravděpodobností pro $\delta t$, $2 \delta t$, $3 \delta t$, ..., $M \delta t$. Tento strom využijeme k ocenění finančního derivátu. Nejprve se určí hodnota derivátu v čase splatnosti $T = M \delta t$. Následně pomocí (10.2) vypočteme postupně hodnotu derivátu v časech $(M-1) \delta t$, $(M-2) \delta t$, ..., $\delta t$, kde hodnota derivátu v čase $\delta t$ představuje jehou současnou hodnotu. Výhodou tohoto postupu je, že umožňuje snadnou implementaci možnosti předčasného uplatnění opce a divendového výnosu.

\section{Diskrétní náhodná procházka}

Pravděpodobnost $p$, koeficient růstu $u$ a koeficient poklesu $d$ jsou vybrány tak, diskrétní verze náhodná procházky představované binomickým stromem a spojité verze náhodné procházky (10.1) mají stejnou střední hodnotu a rozptyl.

Je-li hodnota pokladového aktiva v čase $m \delta$ rovna $S^m$, je očekávaná hodnota $S^{m+1}$ v čase $(m+1) \delta t$ na základě (10.1) rovna
\begin{equation*}
\varepsilon_c[S^{m+1}|S^m] = \int_0^{\infty} S' p \Big( S^m, m \delta t; S', (m+1)\delta t \Big) dS' = e^{r\delta t}S^m
\end{equation*}
kde $p(S,t; S', t')$ je hustota pravděpodobnosti
\begin{equation*}
p(S, t; S', t') = \frac{1}{\sigma S' \sqrt{2 \pi(t' - t)}}e^{- \Big( \ln \frac{S'}{S} - (r - \frac{1}{2}\sigma^2)(t'-t)\Big)}
\end{equation*}
pro rizikově neutrální náhodnou procházku (10.1). Očekávaná hodnota náhodné veličiny $S^{m+1}$ pro dané $S^m$ je pro diskrétní náhodnou procházku rovno
\begin{equation*}
\varepsilon_b[S^{m+1}|S^m] = \Big( pu + (1 - p)d \Big) S^m
\end{equation*}
Dáme-li oba mezivýsledky do rovnosti, získáme
\begin{equation}
pu + (1 - p)d = e^{r \delta t}
\end{equation}

Rozptyl náhodné veličiny $S^{m+1}$ pro dané $S^m$ je definován jako
\begin{equation*}
D[S^{m+1}|S^m] = \varepsilon[(S^{m+1})^2|S^m] - \varepsilon[S^{m+1}|S^m]^2
\end{equation*}
Pro spojitý model náhodné procházky (10.1) máme
\begin{equation*}
\varepsilon_c[(S^{m+1})^2|S^m] = \int_0^{\infty} (S')^2  p \Big( S^m, m \delta t; S', (m+1)\delta t \Big) dS' = e^{(2r + \sigma^2)\delta t}(S^m)^2
\end{equation*}
Rozptyl náhodné procházky (10.1) je tak roven
\begin{equation*}
D_c[S^{m+1}|S^m] = e^{2r \delta t} \Big( e^{\sigma^2 \delta t} - 1\Big)(S^m)^2
\end{equation*}
V případě diskrétního modelu máme
\begin{equation*}
\varepsilon_b[(S^{m+1})^2|S^m] = (pu^2 + (1-p)d^2)(S^m)^2
\end{equation*}
a rozptyl je tedy roven
\begin{equation*}
D_b[S^{m+1}|S^m] = \Big( pu^2 + (1 - p)d^2 - e^{2r\delta t}\Big)(S^m)^2
\end{equation*}
Dáme-li oba mezivýsledky do rovnosti, získáme
\begin{equation}
pu^2 + (1 - p)d^2 = e^{(2r + \sigma^2)\delta t}
\end{equation}
Rovnice (10.3) a (10.4) obsahují tři neznámé $u$, $d$ a $p$. Pro jednoznačné řešení tak potřebujeme ještě třetí rovnici. Protože rovnice (10.3) a (10.4) definují všechny důležité statistické vlastnosti diskrétní náhodné procházky, je podoba třetí rovnice v zásadě na rozhodnutí toho, kdo ji aplikuje. Oblíbené rovnice jsou
\begin{equation}
u = \frac{1}{d}
\end{equation}
popř.
\begin{equation}
p = \frac{1}{2}
\end{equation}  

\subsection{Rovnice u = 1/d}

V tomto případě je binomická metoda definovaná rovnicemi (10.3), (10.4) a (10.5). Pomocí rovnic (10.3) a (10.4) lze odvodit
\begin{equation*}
p = \frac{e^{r \delta t} - d}{u - d} = \frac{e^{(2r + \sigma^2)\delta t} - d^2}{u^2 - d^2}
\end{equation*}
\begin{equation*}
u + d = \frac{e^{(2r + \sigma^2)\delta t} - d^2}{e^{r \delta t} - d}
\end{equation*}
Pomocí substituce (10.5) získáváme kvadratickou rovnici
\begin{equation*}
d^2 - 2 A d + 1 = 0
\end{equation*}
kde
\begin{equation*}
A = \frac{1}{2}\Big( e^{-r \delta t} + e^{r + \sigma^2}\delta t \Big)
\end{equation*}
Výsledným řešením je
\begin{equation}
d = A - \sqrt{A^2 - 1},~~~u = A + \sqrt{A^2 - 1},~~~p = \frac{e^{r \delta t} - d}{u - d}
\end{equation}
Jestliže je použit příliš velký časový krok, může $p$ popř. $1-p$ záporné. V tomto případě nelze binomickou metodu použít.

Volba rovnice (10.5) vede k binomickému stromu, ve kterém se počáteční cena pokladového aktiva opakuje každý sudý časový krok. Trend ceny podkladového aktiva z titulu členu $rdt$ v rovnici (10.1) je zohledněn skutečností, že pravděpodobnost $p$ růstu ceny podkladového aktiva je různá od pravděpodobnosti jejího poklesu $1 - p$.

\subsection{Rovnice p = 1/2}

V tomto případě jsou konstanty $u$ a $d$ dány rovnicemi (10.3) a (10.4) a pravděpobobnost $p$ je dána rovnicí (10.6). Proto platí
\begin{equation*}
u + d = 2e^{r \delta t}, ~~~ u^2 + d^2 = 2e^{(2r + \sigma^2) \delta t}
\end{equation*}
Tyto mezivýsledky jsou při výměně $u$ a $d$ invariantní, a proto budeme hledat takové řešení, kde $u = B + C$ a $d = B - C$. Výsledkem těchto předpokladů jsou rovnice
\begin{equation}
d = e^{r \delta t}\Big( 1 - \sqrt{e^{\sigma^2 \delta t} - 1} \Big), ~~~ u = e^{r \delta t}\Big( 1 + \sqrt{e^{\sigma^2 \delta t} - 1} \Big), ~~~ p = \frac{1}{2}
\end{equation}
V případě příliš velikého časového kroku $\delta t$ se stane $d$ záporným, což má za následek selhání binomické metody.

Pravděpodobnost růstu a poklesu ceny aktiva je v případě $p = \frac{1}{2}$ stejná. Za předpokladu $r > 0$ a přiměřeně velkého časového kroku $\delta t$ platí $ud > 1$. Binomický strom je pak orientován ve směru trendu ceny pokladového aktiva.

\begin{center}
  \begin{pspicture}(0,0)(12,7.0)
     \rput(6.0,0.3){Binomický strom}
	\rput(2.5,1.0){\small{(a) $u = 1/d$}}

     \psline(0.0,4.0)(1.0,4.5)
     \psline(0.0,4.0)(1.0,3.5)
     
     \psline(1.0,4.5)(2.0,5.0)
     \psline(1.0,4.5)(2.0,4.0)
     \psline(1.0,3.5)(2.0,4.0)
     \psline(1.0,3.5)(2.0,3.0)
     
     \psline(2.0,5.0)(3.0,5.5)
     \psline(2.0,5.0)(3.0,4.5)
     \psline(2.0,4.0)(3.0,4.5)
     \psline(2.0,4.0)(3.0,3.5)
     \psline(2.0,3.0)(3.0,3.5)
     \psline(2.0,3.0)(3.0,2.5)
     
     \psline(3.0,5.5)(4.0,6.0)
     \psline(3.0,5.5)(4.0,5.0)
     \psline(3.0,4.5)(4.0,5.0)
     \psline(3.0,4.5)(4.0,4.0)
     \psline(3.0,3.5)(4.0,4.0)
     \psline(3.0,3.5)(4.0,3.0)
     \psline(3.0,2.5)(4.0,3.0)
     \psline(3.0,2.5)(4.0,2.0)
     
     \psline(4.0,6.0)(5.0,6.5)
     \psline(4.0,6.0)(5.0,5.5)
     \psline(4.0,5.0)(5.0,5.5)
     \psline(4.0,5.0)(5.0,4.5)
     \psline(4.0,4.0)(5.0,4.5)
     \psline(4.0,4.0)(5.0,3.5)
     \psline(4.0,3.0)(5.0,3.5)
     \psline(4.0,3.0)(5.0,2.5)
     \psline(4.0,2.0)(5.0,2.5)
     \psline(4.0,2.0)(5.0,1.5)
     
     \psline[linestyle=dashed, arrows=->](0.0,4.0)(5.5,4.0)
     
     \rput(8.5,1.0){\small{(b) $p = 1/2$}}
     
     \psline(6.0,3.0)(7.0,3.7)
     \psline(6.0,3.0)(7.0,2.7)
     
     \psline(7.0,3.7)(8.0,4.4)
     \psline(7.0,3.7)(8.0,3.4)
     \psline(7.0,2.7)(8.0,3.4)
     \psline(7.0,2.7)(8.0,2.4)
     
     \psline(8.0,4.4)(9.0,5.1)
     \psline(8.0,4.4)(9.0,4.1)
     \psline(8.0,3.4)(9.0,4.1)
     \psline(8.0,3.4)(9.0,3.1)
     \psline(8.0,2.4)(9.0,3.1)
     \psline(8.0,2.4)(9.0,2.1)
     
     \psline(9.0,5.1)(10.0,5.8)
     \psline(9.0,5.1)(10.0,4.8)
     \psline(9.0,4.1)(10.0,4.8)
     \psline(9.0,4.1)(10.0,3.8)
     \psline(9.0,3.1)(10.0,3.8)
     \psline(9.0,3.1)(10.0,2.8)
     \psline(9.0,2.1)(10.0,2.8)
     \psline(9.0,2.1)(10.0,1.8)
     
     \psline(10.0,5.8)(11.0,6.5)
     \psline(10.0,5.8)(11.0,5.5)
     \psline(10.0,4.8)(11.0,5.5)
     \psline(10.0,4.8)(11.0,4.5)
     \psline(10.0,3.8)(11.0,4.5)
     \psline(10.0,3.8)(11.0,3.5)
     \psline(10.0,2.8)(11.0,3.5)
     \psline(10.0,2.8)(11.0,2.5)
     \psline(10.0,1.8)(11.0,2.5)
     \psline(10.0,1.8)(11.0,1.5)
     
     \psline[linestyle=dashed, arrows=->](6.0,3.0)(11.5,4.13)
     
  \end{pspicture}
\end{center}

\subsection{Binomický strom}

S využitím (10.7) popř. (10.8) lze zkonstruovat binomický strom, který zobrazuje možné ceny podkladového aktiva. Konstrukci stromu zahájíme v čase $t=0$, pro který známe spotovou cenu $S_0^0$. Na konci časového kroku $\delta t$ existují dvě modelové ceny podkladového aktiva a to $S_1^1 = uS_0^0$ a $S_0^1 = dS_0^0$. Pro časový krok $2 \delta t$ existují tři ceny podkladového aktiva, konkrétně pak $S_2^2 = u^2S_0^0$, $S_1^2 = udS_0^0$ a $S_0^2 = d^2S_0^0$. Na konci třetího časového kroku $3 \delta t$ již existují čtyři možné ceny a tak dále. Obecně tedy platí, že na konci $m$-tého časového kroku existuje $m+1$ modelových cen podkladového aktiva.
\begin{equation*}
S_n^m = d^{m-n}u^nS_0^0, ~~~ n = 0, 1, 2,..., m 
\end{equation*}
To má za následek dvě zajímavé implikace. První z nich je irelevance historických cen podkladového aktiva, protože ke každé z modelových cen existuje v rámci binomického stromu vícero možných ``cest''. Opce, jejichž hodnota se odvíjí od historického vývoje ceny podkladového aktiva, tak nemohou být oceňovány pomocí binomických stromů. Druhou zajímavou implikací je, že počet modelových cen roste pouze s kvadrátem počtu časových kroků binomického stromu. Proto můžeme v případě binomického stromu použít relativně velký počet časových kroků. 

\section{Ocenění opce pomocí binomického stromu}

\subsection{Evropská opce}

Uvažujme evropskou opci, jejíž výplatní funkci známe. Hodnota evropské opce se odvíjí pouze od hodnoty podkladového aktiva v době splatnosti. V tomto případě jsme s pomocí binomického stromu schopni ocenit opci v době její splatnosti, tj. v čase $M \delta t$.

Jestliže bychom uvažovali evropskou prodejní opci, platí
\begin{equation*}
V_n^M = \max(E - S_n^M, 0), ~~~ n = 0, 1, ..., M
\end{equation*}
kde $E$ představuje realizační cenu a $V_n^M$ označuje $n$-tou hodnotu prodejní opce v době splatnosti pro $n$-tou modelovou cenu podkladového aktiva $S_n^M$. Obdobně pro evropskou kupní opci platí
\begin{equation*}
V_n^M = \max(S_n^M - E, 0), ~~~ n = 0, 1, ..., M
\end{equation*}
Pro evropskou cash-or-nothing opci s výplatní funkcí
\begin{equation*}
V(S,T) = 0, ~~~ S < E
\end{equation*}
\begin{equation*}
V(S,T) = B, ~~~ S \ge E
\end{equation*}
je hodnota v době splatnosti dána
\begin{equation*}
V_n^M = 0, ~~~ S_n^M < E, ~ n = 0, 1, ..., M
\end{equation*}
\begin{equation*}
V_n^M = B, ~~~ S_n^M \ge E, ~ n = 0, 1, ..., M
\end{equation*}
Protože známe hodnoty $V_n^M$, pravděpodobnost $p$, délku časového kroku $\delta t$ a bezrizikovou úrokovou míru $r$, lze pro každou modelovou cenu na konci časového kroku $M-1$ dopočítat hodnotu uvažovaného finančního derivátu $V_n^{M-1}$ jako
\begin{equation}
V_n^{M-1} = e^{-r \delta t} \Big( pV_{n+1}^M + (1-p)V_n^{M} \Big)
\end{equation}
Tímto způsobem lze postupně vypočíst hodnotu $V_0^0$, která představuje současnou hodnotu opce.

\subsection{Americká opce}

Základní rozdíl mezi americkou a evropskou opcí spočívá v tom, že americkou opci je možné předčasně uplatnit. Tuto vlastnost lze do binomického modelu poměrně snadno implementovat.

Nejprve, stejně jako v případě evropské opce, vypočteme hodnotu $V_n^M$, tj. hodnotu opce v době její splatnosti. Narozdíl od evropské opce je však nezbytné pro časové kroky $m$, kde $m < M$, porovnat hodnotu opce vypočtenou dle (10.9) s výplatní funkcí odpovídající ceně podkladového aktiva $S_n^m$. Hodnota opce dle (10.9) představuje situaci, kdy danou opci neuplatníme; výplatní funkce pak odpovídá situaci, kdy se rozhodneme opci předčasně uplatnit. Výplatní funkce jsou
\begin{equation*}
\mathcal{P}_n^m = \max(E - S_n^m, 0)
\end{equation*}
pro prodejní opci,
\begin{equation*}
\mathcal{P}_n^m = \max(S_n^m - E, 0)
\end{equation*}
pro kupní opci a
\begin{equation*}
\mathcal{P}_n^m = 0, ~~~ S_n^m < E
\end{equation*}
\begin{equation*}
\mathcal{P}_n^m = B, ~~~ S_n^M \ge E
\end{equation*}
pro cash-or-nothing opci. Hodnotu opce $V_n^m$ pro $m < M$ tak určíme jako
\begin{equation}
V_n^m = \max \Big( \mathcal{P}_n^m, e^{-r \delta t} \Big( pV_{n+1}^M + (1-p)V_n^{M} \Big) \Big)
\end{equation}
Stejně jako v případě evropské opce postupujeme směrem k vrcholu binomického stromuvy s cílem vypočíst $V_0^0$, která je současnou hodnotu opce.

\subsection{Dividendový výnos}

Do binomické metody je možné poměrně jednoduše zapracovat konstantní dividendový výnos $D_0$ generovaný podkladovým aktivem. Aby byl zachován předpoklad neexistence arbitráže, je nutné v rovnici (10.1) nahradit $r$ členem $r - D_0$.
\begin{equation*}
\frac{d S}{S} = (r - D_0)dt + \sigma dX
\end{equation*}
Ze stejného důvodu je pak třeba také nahradit $r$ v navazujících rovnicích (10.7) a (10.8). Důsledkem těchto úprav přejde rovnice (10.7) do tvaru
\begin{equation*}
d = A - \sqrt{A^2 - 1},~~~u = A + \sqrt{A^2 - 1},~~~p = \frac{e^{(r - D_0) \delta t} - d}{u - d}
\end{equation*}
a rovnice (10.8) do tvaru
\begin{equation*}
d = e^{(r - D_0) \delta t}\Big( 1 - \sqrt{e^{\sigma^2 \delta t} - 1} \Big), ~~~ u = e^{(r - D_0) \delta t}\Big( 1 + \sqrt{e^{\sigma^2 \delta t} - 1} \Big), ~~~ p = \frac{1}{2}
\end{equation*}
kde
\begin{equation*}
A = \frac{1}{2}\Big( e^{-(r - D_0) \delta t} + e^{r - D_0 + \sigma^2}\delta t \Big)
\end{equation*}
Pro ocenění evropské resp. americké opce je pak třeba použít rovnici (10.9) resp. (10.10), jejímž výstupem je $V_0^0$ představující současnou hodnotu opce.

\part{Opční teorie - pokračování}

\chapter{Exotické a trajektorové opce}

Kromě nejjednošších forem opcí, tzv. plain-vanilla opcí, kterými jsme se zabývali v předchozích kapitolách, rozlišujeme také tzv. trajektorové a exotické opce. Ačkoliv dělící linie mezi těmito dvěma kategoriemi není ostrá, mají společné to, že narozdíl od plain-vanilla opcí nejsou obchodovány na burze a jedná se tak o OTC obchody\footnote{Over-the-counter (OTC) obchody nejsou narozdíl od burzovních obchodů standardizované a jsou uzavírány mimo oficiální trh zpravidla mezi dvěma bankami popř. mezi bankou a jejím klientem.}.

\section{Rozdělení opcí}

\subsection{Trajektorové opce}

Trajektorová (path-dependend) opce je opce, jejíž hodnota je vázána na výjov ceny podkladového aktiva. Hodnota opce tak není dána pouze cenou podkladového aktiva v době splatnosti opce, ale odvíjí se od vývoje této ceny v průběhu její životnosti.

Mezi trajektorové opce z teoretického hlediska patří také americké opce\footnote{V případě americké opce existuje pravděpodobnost, že americká opce bude předčasně uplatněná a zanikne. Možnost předčasného uplatnění vstupuje do výpočtu hodnoty a opce.}. V praxi je však na standardní americké opce pohlíženo jako na plain-vanilla produkty.

Pod pojmem trajektorová opce však v praxi zpravidla rozumíme tzv. zpětnou (look-back) a asijskou (asian) opci, kterým budou věnovány samostatné kapitola.

\subsection{Exotické opce}

Exotickou opcí rozumíme opci, kterou z hlediska výplatní funkce nelze klasifikovat jako plain-vanilla kupní popř. prodejní opci. Jedná se o poměrně širokou rodinu opcí, které mohou mít poměrně složité výplatní funkce.

Nejjednošší formou exotické opce je binární opce. Výplata standardní binární opce je vázána na to, bude-li v době splatnosti opce cena podkladového aktiva nad popř. pod realizační cenou. Binárními opcemi jsme se zabývali v předchozích kapitolách.

Další relativně jednoduchou skupinou exotických opcí jsou bariérové opce. Tyto opce jsou charakteristické tím, že opční právo zaniká popř. vzniká, když cena podkladového aktiva protne určitou hranici. V prvním případě se jedná o bariérovou opci s vnější hranicí (out barrier), v druhém případě pak o bariérovou opci s vnitřní hranicí (in barrier). Dalším úhlem klasifikace je, zda-li k protnutí hranice dojde zdola popř. shora. Závislost uplatnění opčního práva na protnutí bariéry činí z těchto opcí také trajektorové opce. Bariérové opce tedy spadají do obou uvažovaných kategorií. Vzhledem k jejich významu bude bariérovým opcím věnována samostatná kapitola.

\section{Jednodušší typy opcí}

V této kapitole popíšeme složenou a výběrovou opci, které představují jednodušší typy exotických opcí. Komplikovanějším typům opcí bude věnovány samostatné kapitoly.

\subsection{Složené opce}

Složenou opci lze charakterizovat jako opci na opci. Majitel složené opce má právo v době splatnosti koupit popř. prodat kupní popř. prodejní opci. V následujícím textu budeme předpokládat, že podkladovou opcí je evropská plain-vanilla opce a že také samotná složená opce je z pohledu možnosti předčasného uplatnění evropskou opcí.

Výpočet hodnoty složené opce budeme ilustrovat na kupní opci na kupní opci. Nechť $T_1$ je časový okamžik, kdy se majitel složené opce může rozhodnout, zda-li koupí podkladovou plain-vanilla kupní opci za částku $E_1$. Tato podkladová opce může být uplatněna v čase $T_2$ a její majitel tak získá podkladové aktivum v hodnotě $S$ za realizační cenu $E_2$.

Nejprve zkonstruujme binomický strom cen podkladového aktiva pro časový interval $0 \ge t \ge T_2$. V dalším kroce je třeba nalézt hodnotu podkladové kupní opce pro jednotlivé modelové ceny binomického stromu v čase $T_1$. Nechť existuje $n + 1$ takovýchto modelových cen a $n + 1$ jim odpovídajících hodnot podkladových opcí. Jednotlivé modelové ceny podkladového aktiva v čase $T_1$ označme jako $S_i^{T_1}$ a jim odpovídající hodnoty podkladové kupní opce jako $\mathcal{C}_i^{T_1}$, kde $0 \ge i \ge n$. Dále uvažujme pravděpodobnost $p_i$, že se cena podkladového aktiva změní z výchozí spotové ceny $S_0^0$ na $S_i^{T_1}$. Pro hodnotu složené opce tak platí
\begin{equation}
V_0^0 = e^{-r T_1} \sum_{i = 0}^n p_i \cdot \max(\mathcal{C}_i^{T+1} - E_1, 0)
\end{equation}
kde $r$ představuje bezrizikovou úrokovou míru a $V_0^0$ současnou hodnotu složené opce. Je-li podkladovou opcí prodejní plain-vanilla opce, modifikuje se rovnice (11.1) do tvaru
\begin{equation}
V_0^0 = e^{-r T_1} \sum_{i = 0}^n p_i \cdot \max(E_1 - \mathcal{P}_i^{T+1}, 0)
\end{equation}
V případě, kdy podkladovou opcí není evropská plain-vanilla opce, zůstává postup výpočtu zcela shodný. Jediný rozdíl spočívá ve výpočtu hodnoty $\mathcal{C}_i^{T+1}$ resp. $\mathcal{P}_i^{T+1}$, který musí odrážet příslušný typ podkladové opce.

Protože je hodnota složené opce dána pouze náhodnou procházkou ceny podkladového aktiva $S$, musí splňovat Black-Scholes rovnici. Jediným rozdílem oproti evropské plain-vanilla opci je počáteční podmínka. Ta má např. v případě evropské kupní opce podobu $\max(S - E, 0)$, avšak u odpovídající složené opce je ji třeba nahradit podmínkou ve tvaru $\max(\mathcal{C}^{T_1} - E_1, 0)$.  

\subsection{Výběrová opce}

Výběrová opce je podobná složené opci s tím rozdílem, že její majitel má možnost volby, zda-li koupí v čase $T_1$ kupní nebo prodejní opci.

Opět uvažujme evropskou výběrovou funkci, kde podkladovými opcemi jsou plain-vanilla kupní a prodejní opce. Majitel opce se tak má v čase $T_1$ rozhodnout, zda-li za částku $E_1$ nakoupí evropskou kupní nebo prodejní opci s realizační cenou $E_2$ a splatností v čase $T_2$. Rovnice (11.1) a (11.2), které představovaly hodnotu složené opce, tak nahradí jediná rovnice 
\begin{equation}
V_0^0 = e^{-r T_1} \sum_{i = 0}^n p_i \cdot \max(\mathcal{C}_i^{T+1} - E_1, \mathcal{P}_i^{T+1} - E_1, 0)
\end{equation}

\chapter{Bariérové opce}

\section{Úvod}

U bariérové opce záleží výplata na tom, zda-li cena podkladového aktiva dosáhne popř. nedosáhne tzv. bariéry v průběhu životnosti opce.

Základní dělení bariérových opcí je na knock-out a knock-in opce. Knock-out opce je v zásadě klasickou opcí, která však přestane existovat v okamžiku, kdy podkladové aktivum dosáhne stanovené bariéry. Naopak knock-in opce se stává regulérní opcí teprve v okamžiku, kdy cena podkladového aktiva protne stanovenou bariéru. Dalším předmětem klasifikace bariérových opcí je skutečnost, zda-li byla bariéra protnuta shora nebo zdola. V souvislosti s bariérovými opcemi se tak mluví o down-and-in, down-and-out, up-and-in a up-and-out kupních popř. prodejních opcích. Označení těchto opcí souvisí s tím, zda-li je bariéra protnuta zdola resp. shora a zdali tímto kupní popř. prodejní opce vzniká popř. zaniká.

Velmi důležitým faktorem v případě bariérových opcí je stanovení frekvence, se kterou se bude sledovat protnutí bariéry.

\section{Knock-out opce}

Uvažujme evropskou down-and-out kupní opci s výplatní funkcí v době splatnosti definovanou jako $\max(S-E,0)$ za předpokladu, že $S$ po dobu životnosti opce neklesne pod bariéru $X$. Jestliže $S$ protne bariéru $X$, stane se opce bezcennou.

Uvažujme situaci, kdy $E > X$. Platí, že pokud je cena podkladového aktiva $S$ větší než bariéra $X$, splňuje hodnota opce $V(S,t)$ Black-Scholes rovnici (3.5). Stejně jako u klasické evropské kupní opce platí pro hodnotu opce v době její splatnosti
\begin{equation*}
V(S,T) = \max(S-E,0)
\end{equation*}
S tím, jak se $S$ blíží k nekonečnu, stává se pravděpodobnost protnutí bariéry zanedbatelná. Za předpokladu, že podkladové aktivum negeneruje žádné úrokové nebo dividendové výnosy, platí
\begin{equation*}
V(S, t) \sim S,~~~S \rightarrow \infty
\end{equation*}
Až dosud byl problém identický s evropskou plain-vanilla kupní opcí. Nicméně rozdíl nastavá v druhé hraniční podmínce, která je aplikovaná pro $S = X$ a nikoliv pro $S = 0$. Jestliže cena podkladového aktiva $S$ protne bariéru $X$, stane se opce bezcennou.
\begin{equation*}
V(X, t) = 0
\end{equation*}
Tímto je formulace problému kompletní a můžeme se pokusit o nalezení explicitního řešení.

Nejprve aplikujeme stejné substituce jako v kapitole 5.4, tj.
\begin{equation*}
S = Ee^x,~~~ t = T - \frac{\tau}{\frac{1}{2}\sigma^2}, ~~~ V = Ee^{\alpha x + \beta \tau}u(x, \tau)
\end{equation*}
kde $\alpha = -\frac{1}{2}(k-1)$, $\beta = -\frac{1}{4}(k+1)^2$ a $k = \frac{r}{\frac{1}{2}\sigma^2}$. Uvažovaná bariéra se tak přetransformuje do podoby
\begin{equation*}
x_0 = \ln \frac{X}{E}
\end{equation*}
a problém bariérové opce se stane diferenciální rovnicí druhého stupně
\begin{equation}
\frac{\partial u}{\partial \tau} = \frac{\partial^2 u}{\partial x^2}
\end{equation}
s počáteční podmínkou
\begin{equation}
u(x,0) = \max \Big( e^{\frac{1}{2}(k+1)x} - e^{\frac{1}{2}(k-1)x}, 0 \Big) = u_0(x), ~~~ x \ge x_0
\end{equation}
a hraničními podmínkami
\begin{equation}
u(x,t) \sim e^{(1 - \alpha)x - \beta \tau}, ~~~ x \rightarrow \infty
\end{equation}
\begin{equation}
u(x_0, t) = 0
\end{equation}

K řešení poslední hraniční podmínky použijeme tzv. metodu obrazů. V kapitole 4 jsme problematiku diferenciální rovnice vysvětlovali pomocí šíření tepla v modelové tyči. Hraniční podmínka (12.4) však není aplikovaná v nekonečnu ale pro konečnou hodnotu $x$. Namísto nekonečně dlouhé tyče tak uvažujeme semi-konečnou tyč s nulovou teplotou v bodě $x = \ln \frac{X}{E}$. Protože je šíření tepla v tyči nezávislé na zvolené soustavě souřadnic, je rovnice (12.4) invariantní pro transformaci z $x$ na $x + x_0$ a transformaci z $x$ na $-x$. Proto, je-li $u(x, \tau)$ řešením (12.1), jsou řešením také $u(x + x_0, \tau)$ a $u(-x + x_0, \tau)$, kde $x_0$ představuje libovolnou konstantu. V rámci metody obrazů přistupujeme k semi-konečnému problému tak, že řešíme problém pro nekonečnou tyč, která se však skládá ze dvou semi-konečných tyčí. Ty mají shodné avšak opačné rozložení počátečních teplot - jedna tyč je horká a druhá studená. Výsledkem je, že bodě dotyku obou tyčí se teploty vykompenzují a výsledná teplota je tak rovna nule.

Výše popsanou metodu je možné použít pro řešení problému bariérové opce. Nechť je místem dotyku uvažovaných semi-konečných tyčí bod $x_0 = \ln \frac{X}{E}$. Namísto řešení problému (12.1) - (12.4) na intervalu $x_0 < x < \infty$ budeme řešit (12.1) pro všechna $x$ za podmínky
\begin{equation*}
u(x, 0) = u_0(x) - u_0(2x_0 - x)
\end{equation*}
tj.
\begin{equation*}
u(x, 0) = \max \Big( e^{\frac{1}{2}(k+1)x} - e^{\frac{1}{2}(k-1)x}, 0 \Big), ~~~ x > x_0
\end{equation*}
\begin{equation*}
u(x, 0) = - \max \Big( e^{(k+1)(x_0 - \frac{1}{2}x)} - e^{(k-1)(x_0 - \frac{1}{2}x)}, 0 \Big), ~~~ x < x_0
\end{equation*}
Tímto způsobem je zajištěno $u_0(x_0, 0) = 0$. Předpokládejme, že
\begin{equation*}
C(S,t) = Ee^{\alpha x + \beta \tau}u_1(x, \tau)
\end{equation*}
je hodnota plain-vanilla evropská opce se shodnou realizační cenou a datem splatnosti jako v případě uvažované bariérové opce. Tato hodnota je dána Black-Scholes rovnicí a $u_1(x, \tau)$ je řešením difúzní rovnice popisující šíření tepla v modelové nekonečné tyči. Hodnotu $u_1(x, \tau)$ lze vyjádřit jako
\begin{equation*}
u_1(x, \tau) = e^{-\alpha x - \beta  \tau}\frac{C(S,t)}{E}
\end{equation*}
Hodnotu bariérové opce lze vyjádřit jako
\begin{equation*}
V(S,t) = Ee^{\alpha x + \beta \tau} \Big( u_1(x, \tau) + u_2(x, \tau) \Big)
\end{equation*}
kde $u_2(x, \tau)$ je řešením problému s asymetrickými počátečními daty. Toto řešení je možné vyjádřit pomocí $u_1(x, \tau)$ invariancí rovnice (12.1). Tímto získáme
\begin{equation*}
u_2(x, \tau) = -u_1(2x_0 - x, \tau) = -e^{\alpha(2 \ln \frac{X}{E} - \ln \frac{S}{E}) - \beta \tau} \frac{C\Big(\frac{X^2}{S}, t\Big)}{E}
\end{equation*}
kde transformace z $x$ na $2x_0 - x$ odpovídá nahrazení $S$ členem $\frac{X^2}{S}$. Výsledná rovnice pro výpočet hodnoty bariérové opce je tak
\begin{equation*}
V(S,t) = C(S,t) - \Big( \frac{S}{X} \Big)^{1-k}C\Big( \frac{X^2}{S}, t \Big)
\end{equation*}
Je zřejmé, že $V(X,t) = 0$ a lze také dokázat, že je splněna rovnice (12.1) i počáteční podmínka (12.2)\footnote{Počáteční podmínka je splněna pouze pro $S > X$; pro $S < X$ je opce bezcenná.}.

\section{Knock-in opce}

V případě, že dojde k protnutí bariéry, stává se bariérová knock-in opce plain-vanilla opcí. V praxi je běžné, že nedojde-li k protnutí bariéry, je majiteli opce vyplacena fixní částka, která má ``kompenzovat'' ztrátu opce.

V následujícím textu budeme uvažovat down-and-in evropskou kupní opci. Hodnota opce $V(S,t)$ stále splňuje Black-Scholes rovnici (3.5) a pro úplnou specifikaci problému stačí definovat počáteční a hraniční podmínky. V okamžiku protnutí bariéry se opce stává evropskou kupní opcí, jejíž hodnota je dána Black-Scholes rovnicí.

Uvažujme nyní situaci, kdy bariéra nebyla protnuta. Pro $S \rightarrow \infty$ je opce bezcenná, protože pravděpodobnost protnutí bariéry je nulová. První hraniční podmínka je tedy
\begin{equation*}
V(S,t) \rightarrow 0, ~~~ S \rightarrow \infty
\end{equation*}
Jestliže je $S$ těsně před splatností opce větší než $X$, je opce bezcenná. Konečná podmínka je tedy
\begin{equation*}
V(S, T) = 0, ~~~ S > X
\end{equation*}

Jestliže však $S = X$ před splatností opce, stává se bariérová opce plain-vanilla kupní opcí a musí mít proto stejnou hodnotu. Druhá hraniční podmínka má tedy podobu
\begin{equation*}
V(X, t) = C(X,t)
\end{equation*}

Jak již bylo několikrát řečeno, stává se protnutím bariéry opce evropskou kupní opcí. Je tedy třeba se zabývat hodnotou bariérové opce pro $S > X$. Tímto je formulace problému evropské down-and-in bariérové kupní opce kompletní.

Abychom mohli řešit down-and-in bariérovou opci explicitně, rozepišme
\begin{equation}
V(S, t) = C(S, t) - \bar{V}(S, t)
\end{equation}
Protože Black-Scholes rovnice a hraniční podmínky jsou lineární, musí také $\bar{V}$ splňovat Black-Scholes rovnici pro konečnou podmínkou
\begin{equation*}
\bar{V}(S, T) = C(S,T) - V(S, T) = C(S, T) = \max(S - E, 0)
\end{equation*}
a počáteční podmínky
\begin{equation*}
\bar{V}(S, t) = C(S,t) - V(S, t) \sim S - 0 = S, ~~~ S \rightarrow \infty
\end{equation*}
\begin{equation*}
\bar{V}(X, t) = C(X,t) - V(X, t) = C(X, t) - C(X, t) = 0
\end{equation*}
Výše uvedené podmínky představují formulaci problému bariérové down-and-out kupní opce. To znamená, že portfolio skládající se bariérové down-and-out a down-and-in kupní opce se shodnou bariérou, realizační cenou a datem splatnosti odpovídá evropské kupní opci. Pouze jedna z bariérových opcí bude totiž v době splatnosti aktivní a její hodnota bude odpovídat hodnotě evropské kupní opce. Na základě znalosti hodnoty evropské kupní opce $C(S, t)$ a bariérové down-and-out kupní opce $\bar{V}(S, t)$ lze tedy dle (12.5) dopočíst hodnotu bariérové down-and-in kupní opce $V(S, t)$. 

Vedle evropské verze bariérové opce existuje také americká verze. Pro tyto opce sice neexistuje explicitní rovnice, nicméně jejich numerické řešení je relativně snadné.

\chapter{Teorie trajektorových opcí}

\section{Úvod}

Jestliže chceme analyzovat trajektorové opce, nelze použít standardní Black-Scholes metodu.

Uvažujme opci na průměrnou realizační cenu (average strike option), která spadá do rodiny asijských opcí. Výplatní funkce této opce je shodná s plain-vanilla opcí, avšak namísto předem známé realizační ceny je použitá průměrná cena podkladového aktiva. V případě kupní opce má výplatní funkce v době splatnosti tvar
\begin{equation*}
\Lambda = \max(S - \bar{S}, 0)
\end{equation*}
kde $\bar{S}$ je průměrnou cenou podkladového aktiva. Pokud bychom pro ocenění této opce použili binomický strom, vede ke každé z modelovaných cen podkladového aktiva několik cest. Každá z těchto cest je charakterizována jinou průměrnou cenou podkladového aktiva. Nabízí se tak možnost použít vedle $S$ a $t$ třetí nezávislou veličinu odpovídající typu trajektorové opce, tj. v našem případě průměrné ceně podkladového aktiva.

\section{Spojitý model náhodné procházky}

Uvažujme obecnou evropskou opci, jejíž výplatní funkce v době splatnosti má tvar
\begin{equation*}
\int_0^T f(S(\tau), \tau) d \tau
\end{equation*}
kde $f$ představuje funkci proměnných $S$ a $t$. Například hodnota kupní opce na průměrnou realizační cenu je tak v době splatnosti dána vztahem
\begin{equation*}
\max \Big( S - \frac{1}{T} \int_0^T S(\tau) d \tau, 0 \Big)
\end{equation*}
Definujme novou proměnnou
\begin{equation*}
I = \int_0^t f(S(\tau), \tau, \tau) d \tau
\end{equation*}
Protože je historie ceny podkladového aktiva nezávislá na současné ceně, můžeme $I$, $S$ a $t$ považovat za nezávislé proměnné. Vzhledem k tomu, že hodnota trajektorové opce je funkcí $I$ i $S$, lze ji vyjádřit jako $V(S,I,t)$.

Naším dalším krokem bude aplikace It\^o lemmy na hodnotu trajektorové opce $V(S,I,t)$. Předtím je však třeba vyjádřit stochastickou diferenciální rovnici pro $I$. Tu lze poměrně snadno definovat skrze malé změny proměnných $S$ a $t$.
\begin{equation*}
I(t + dt) = I + dI = \int_0^{t + dt} f(S(\tau), \tau) d \tau = \int_0^t f(S(\tau), \tau) d \tau + f(S(t),t) dt
\end{equation*}
\begin{equation*}
dI = f(S,t)dt
\end{equation*}
Ukázalo se, že diferenciální rovnice pro $dI$ neobsahuje žádný stochastický prvek. Nyní můžeme aplikovat It\^o lemmu.
\begin{equation}
dV = \sigma S \frac{\partial V}{\partial S} dX + \Big( \frac{1}{2} \sigma^2 S^2 \frac{\partial^2 V}{\partial S^2} + \mu S \frac{\partial V}{\partial S} + \frac{\partial V}{\partial t} + f(S,t)\frac{\partial V}{\partial I} \Big) dt
\end{equation}
Rovnice (13.1) je odvozena stejným způsobem jako rovnice v kapitole 3.3. Protože člen obsahující $I$ nevnáší do rovnice  (13.1) žádný nový náhodný element, může sestavit bezrizikové portfolio stejně jako v případě plain-vanilla evropské opce pouze z dlouhé pozice v trajektorové opci a krátké pozice v $\Delta$ jednotkách podkladového aktiva. Delta trajektorové opce je opět definována jako
\begin{equation*}
\Delta = \frac{\partial V}{\partial S}
\end{equation*}
Připomeňme, že při dodržení podmínek neexistence arbitráže generuje bezrizikové portfolio výnos odpovídající bezrizikové úrokové míře. To vše vede k diferenciální rovnici
\begin{equation}
\frac{\partial V}{\partial t} + f(S,t) \frac{\partial V}{\partial I} + \frac{1}{2} \sigma^2 S^2 \frac{\partial^2 V}{\partial S^2} + r S \frac{\partial V}{\partial S} - rV = 0
\end{equation}
Rovnice (13.2) je shodná se základní Black-Scholes rovnicí s vyjímkou členu obsahujícího $\frac{\partial V}{\partial I}$.

V případě všech finančních derivátů řešíme Black-Scholes rovnici s ohledem na konečnou podmínku. V době splatnosti známe přesný tvar výplatní funkce a hodnota opce je tak funkcí $S$ a $I$. Platí tedy
\begin{equation*}
V(S,I,T) = \Lambda(S,I,T)
\end{equation*}
Například pro kupní opci na průměrnou realizační cenu je $I$ definováno jako $\int_0^tS(\tau) d \tau$ a hodnota této opce v době splatnosti je tak dána
\begin{equation*}
\Lambda(S,I,T) = \max \Big( S - \frac{I}{T}, 0 \Big)
\end{equation*}

\subsection{Technická poznámka: Předčasné uplatnění opce}

Výše uváděné argumenty lze rozšířit o případ možnosti předčasného uplatnění, jak je tomu v případě amerických opcí. Nezbytným předpokladem je znalost výnosu, který opce generuje v případě svého předčasného uplatnění. Například u kupní opce na průměrnou realizační cenu je tento výnos zpravidla definován jako $\max \Big( S - \frac{I}{t}, 0 \Big)$. Nicméně předpokládejme, že v obecném případě je výnos z titulu předčasného uplatnění roven $\Lambda(S,I,t)$.

Podobně jako u plain-vanilla opcí je také v případě trajektorových opcí jádrem řešení problému přechod od Black-Scholes rovnice k nerovnici. Definujme operátor
\begin{equation*}
\mathcal{L}_{\mathcal{EX}} = \frac{\partial}{\partial t} + f(S,t) \frac{\partial}{\partial I} + \frac{1}{2} \sigma^2 S^2 \frac{\partial^2}{\partial S^2} + r S \frac{\partial}{\partial S} - r
\end{equation*}
Tento operátor měří rozdíl mezi výnosovou mírou generovanou portfoliem, které je z pohledu delta zajištění bezrizikové, a bezrizikovou výnosovou mírou. V případě plain-vanilla americké opce platí, že výnos z bezrizikového portfolia nemůže být větší než bezriziková výnosová míra, avšak může být menší\footnote{V tomto případě je výhodné opci předčasně uplatnit.}.
\begin{equation*}
\mathcal{L}_{\mathcal{EX}}(V) \le 0 
\end{equation*}
Předpoklad neexistence arbitráže pak má za následek
\begin{equation*}
V(S,I,t) \ge \Lambda(S,I,t)
\end{equation*}
Pro $\mathcal{L}_{\mathcal{EX}}(V) = 0$ není optimální opci předčasně uplatnit, a proto platí $V > \Lambda$. V opačném případě, kdy $\mathcal{L}_{\mathcal{EX}}(V) < 0$, je naopak racionální opci předčasně uplatnit, což odpovídá $V = \Lambda$. Výsledkem jsou tedy pouze dvě situace a to
\begin{itemize}
\item $\mathcal{L}_{\mathcal{EX}}(V) = 0$ a $V - \Lambda > 0$
\item $\mathcal{L}_{\mathcal{EX}}(V) < 0$ a $V - \Lambda = 0$
\end{itemize}
V obou případech tak platí
\begin{equation*}
\mathcal{L}_{\mathcal{EX}}(V) \cdot (V - \Lambda) = 0
\end{equation*}
čímž se dostáváme k definici problému pomocí lineární komplementarity
\begin{equation}
\mathcal{L}_{\mathcal{EX}}(V) \cdot (V - \Lambda) = 0, ~~~ \mathcal{L}_{\mathcal{EX}}(V) \le 0, ~ V - \Lambda \ge 0
\end{equation}
kde $V$ a $\Delta = \frac{\partial V}{\partial S}$ jsou spojité a konečná podmínka je definována jako
\begin{equation*}
V(S,I,T) = \Lambda(S,I,T)
\end{equation*}
Podmínka, že delta opce musí být spojitá, vyplývá ze stejně jako v případě plain-vanilla americké opce z předpokladu neexistence arbitráže. To má za následek, že výplatní funkce $\Lambda(S,I,t)$ je taktéž spojitá.

\section{Diskrétní model náhodné procházky}

V případě, že je pomocná veličina $I$ modelována diskrétně, nelze aplikovat rovnici (13.1). Namísto toho je třeba řešit základní Black-Scholes rovnici doplněnou o skokové podmínky podobně, jako tomu bylo v případě diskrétně vyplácené dividendy. Stále je třeba pracovat s třemi veličinami $S$, $I$ a $t$, avšak veličinu $I$ je možné z pohledu definice problému chápat jako pouhý parametr.

Pro ilustraci problému uvažujme opci na průměrnou realizační cenu, jejíž výplatní funkce závisí na diskrétním aritmetickém průměru cen podkladového aktiva.
\begin{equation*}
\frac{1}{N} \sum_{i=1}^N S(t_i)
\end{equation*}
Hodnota opce tak závisí na pomocné veličině
\begin{equation}
I = \sum_{i=1}^{j(t)}S(t_i)
\end{equation}
kde $j(t)$ takové nejvyšší celé číslo, pro které platí $t_j(t) \le t$. Označme hodnotu opce jako $V(S,I,t)$.

V kapitole 6 jsme ukázali, jak zakomponovat diskrétně vyplácenou dividendu do spojitého Black-Scholes modelu. Pomocí jednoduché finanční argumentace jsme odvodili nutnost skokové podmínky v časovém okamžiku výplaty dividendy. Shodný přístup lze také použít v případě diskrétní pomocné veličiny $I$.

V případě opce na průměrnou realizační cenu, je $I$ aktualizováno skokově k určitému časovému okamžiku. Platí tedy
\begin{center}
\textit{nová hodnota $I$ = stará hodnota $I$ + S}
\end{center}

Položme si nyní otázku, zda-li se hodnota opce může přes jednotlivé časové okamžiky, ke kterým aktualizujeme hodnotu pomocné veličiny $I$, skokově změnit. Stejně jako v případě diskrétní dividendy, není odpověď jednoznačná. Je zřejmé, že v případě fixního $S$ a $I$ není $V(S,I,t)$ spojité. V tomto případě se hodnota $V(S,I,t)$ mění skokově a odpoveď zní ``ano''. Mění-li se však hodnoty všech tří veličin $S$, $I$ a $t$, zní odpověď ``ne''. Druhý případ je důsledkem neexistence arbitráže. Skoková změna hodnoty opce v předem známý časový okamžik totiž představuje prostor pro arbitráž. Tyto dva zdánlivě protichůdné závěry lze uvést do souladu, uvědomíme-li si, že diskrétní aritmetický průměr a tím pádem také hodnota opce se mění skokově jednoduše proto, že je hodnota $I$ měřena nespojitě. Nespojitost veličiny $I$ a spojitost $V(S(t),I(t),t)$ pro libovolnou realizaci náhodné procházky mají za následek nutnost skokové změny hodnoty $V(S,I,t)$ (jako funkce proměnné $t$ a fixními parametry $S$ a $I$) přes jednotlivé diskrétní okamžiky.

Použijme $I_i$ pro označení sumy $I$ pro časový interval $t_i < t < t_{i+1}$ a $S_i$ pro označení hodnoty $S$ k časovému okamžiku $t_i$. Konstanta $I_i$ tak představuje hodnotu $I$ od časového okamžiku $t_i$ až do okamžiku $t_{i+1}$, kdy je aktualizována na hodnotu $I_{i+1}$. Pro aktualizaci hodnoty $I$ platí pravidlo
\begin{equation}
I_i = I_{i-1} + S_i
\end{equation}
Nechť $t_i^{-}$ představuje časový okamžik těsně před časovým okamžikem $t_i$, ke kterému se aktualizuje pomocná veličina $I$, a $t_i^{+}$ časový okamžik těsně po tomto časovém okamžiku. Protože $I_i$ je konstantou pro časový interval $t_i^{+}$ až $t_{i+1}^{-}$, lze na ni nahlížet jako na parametr hodnoty opce, tj. podobně jako na dividendovou míry v případě diskrétné vyplácené dividendy. V průběhu časové periody $t_i^{+}$ až $t_{i-1}^{-}$ je tak jedinou náhodnou veličinou, která může měnit svou hodnotu, cena podkladového aktiva $S$. To znamená, že hodnota opce musí splňovat základní Black-Scholes rovnici. Z (13.5) je zřejmé, že $I$ je v $t_i$ nespojité, nicméně hodnota opce musí být spojitá. Proto platí
\begin{equation}
V(S_i, I_i, t_i^{+}) = V(S_i, I_{i-1},t_i^{-})
\end{equation}
Realizace ceny podkladového aktiva $S$ je spojitá, a proto je také shodná pro časový okamžik $t_i^{+}$ a $t_i^{-}$. S využitím (13.5) lze (13.6) vyjádřit také ve tvaru
\begin{equation}
V(S, I_{i-1} + S, t_i^{+}) = V(S, I_{i-1},t_i^{-})
\end{equation}
Protože se $I_{i-1}$ mezi $t_{i-1}^{+}$ a $t_i^{-}$ nemění, můžeme v (13.7) vypustit index $i-1$. Získáme tak skokovou podmínku
\begin{equation}
V(S, I, t_i^{-}) = V(S, I + S, t_i^{+})
\end{equation}
pro opci na průměrnou realizační cenu s diskrétním aritmetickým průměrem. Všimněme si, že zatímco v (13.6) jsou hodnoty $S$ a $I$ výsledkem realizace náhodné procházky, což má za následek, že se mění v čase, v (13.8) jsou obě veličiny fixní. Dále si všimněme, že rovnice (13.8) implikuje zpětné řešení v čase tj. od časového okamžiku $t_i^+$ k časovému okamžiku $t_i^{-}$.

Výše uvedené odvození lze aplikovat na libovolnou opci, jejíž hodnota je funkcí nespojitě aktualizovaného parametru. Je-li opce funkcí $I$, které je definováno obecnou funkcí
\begin{equation*}
I_i = w_i(S_i, I_{i-1})
\end{equation*}
má skoková podmínka tvar
\begin{equation}
V(S, I, t_i^{-}) = V(S, w_i(S,I), t_i^{+})
\end{equation}
Postup řešení trajektorové opce s diskrétní pomocnou veličinou $I$ je tedy následující.
\begin{itemize}
\item Nejprve určíme hodnotu opce v době její splatnosti na základě znalosti její výplatní funkce, popř. postupujeme zpět v čase a řešíme diferenciální rovnici
\begin{equation*}
\frac{\partial V}{\partial t} + \frac{1}{2} \sigma^2 S^2 \frac{\partial^2 V}{\partial S^2} + rS \frac{\partial V}{\partial S} - rV = 0
\end{equation*}
mezi jednotlivými časovými okamžiky, ke kterým aktualizujeme hodnotu pomocné veličiny $I$. Tímto způsobem vypočteme hodnotu opce pro časový okamžik $t_{i+1}^{-}$. Takto vypočtenou hodnotu prohlásíme za hodnotu opce k $t_{i}^{+}$.
\item V dalším kroce aplikujeme skokovou podmínku (13.9) s cílem odvodit hodnotu opce těsně před aktuálním časovým okamžikem, tj. pro $t_i^{-}$.
\item Výše uvedené kroky opakujeme, dokud nevypočteme hodnotu opce k časovému okamžiku $t_0$. Tato hodnota představuje současnou hodnotu opce.
\end{itemize}

\chapter{Asijská opce}

\section{Úvod}

Asijské opce jsou opce, jejichž hodnota se odvíjí od průměrné ceny podkladového aktiva vypočtené pro určitou časovou periodu. Typickým příkladem asijské opce je kontrakt, který umožňuje majiteli nakoupit podkladové aktivum za cenu, která odpovídá průměrné ceně od okamžiku uzavření kontraktu do jeho splatnosti.

V předešlé kapitole jsme jako příklad asijské opce uvedli kupní opci na průměrnou realizační cenu (average strike option), která měla v době splatnosti výplatní funkci
\begin{equation*}
\Lambda = \max(S - \bar{S}, 0) 
\end{equation*}
kde $\bar{S}$ představovalo průměrnou cenu podkladového aktiva. Druhý typem asijské opce je tzv. opce na průměrnou podkladovou cenu (average rate option). V případě kupní opce má výplatní funkce v době splatnosti opce tvar
\begin{equation*}
\Lambda = \max(\bar{S} - E, 0) 
\end{equation*}

Průměrná cena podkladového aktiva může být vypočtena pomocí aritmetického popř. geometrického průměru. Samotná cena podkladového aktiva může mít charakter spojité nebo diskrétní veličiny. Dále může být opce zkonstruována jako evropská popř. americká, tj. bez popř. s právem předčasného uplatnění.

\section{Spojitá cena podkladového aktiva}

\subsection{Aritmetický průměr}

Základní model oceňování asijské opce byl představen v kapitole 13 včetně základní rovnice (13.2). Pro opci, jejíž hodnota je funkcí spojitého aritmetického průměru ceny podkladového aktiva ve tvaru
\begin{equation*}
\frac{1}{t} \int_0^t S(\tau) d \tau
\end{equation*}
zavedeme pomocnou veličinu
\begin{equation*}
I = \int_0^t S(\tau) d \tau
\end{equation*}
Dosazením do (13.2) tak získáme
\begin{equation*}
\frac{\partial V}{\partial t} + S \frac{\partial V}{\partial I} + \frac{1}{2} \sigma^2 S^2 \frac{\partial^2 V}{\partial S^2} + r S \frac{\partial V}{\partial S} - rV = 0
\end{equation*}

\subsection{Geometrický průměr}

Spojitý geometrický průměr ceny podkladového aktiva je definován jako
\begin{equation*}
e^{\frac{1}{2}\int_0^t \ln S(\tau) d \tau}
\end{equation*}
což je limita v $n \rightarrow \infty$ pro diskrétní geometrický průměr
\begin{equation*}
\Big( \prod\limits_{i = 1}^{n} S(t_i)\Big)^{\frac{1}{n}}
\end{equation*}
Pro asijskou opci se spojitým geometickým průměrem definujme pomocnou veličinu
\begin{equation*}
I = \int_0^t \ln S(\tau) d \tau
\end{equation*}
Rovnice (13.2) tak přejde do tvaru
\begin{equation*}
\frac{\partial V}{\partial t} + \ln S \frac{\partial V}{\partial I} + \frac{1}{2} \sigma^2 S^2 \frac{\partial^2 V}{\partial S^2} + r S \frac{\partial V}{\partial S} - rV = 0
\end{equation*}

\subsection{Podobnostní řešení}

Hodnota asijské opce je funkcí tří veličin, konkrétně $S$, $I$ a $t$. Nicméně v některých případech je možné hodnotu opce vyjádřit jako funkci dvou proměnných. V kapitole 5 jsme problém dvou proměnných zredukovali na problém jedné proměnné, což vyplynulo z matematické struktury příslušné diferenciální rovnice a podprovodných podmínek.

Pro asijskou opci se spojitým aritmetickým průměrem je možné výše popsaný třírozměrný problém zredukovat na dvourozměrný problém. Podmínkou je, že výplatní funkce má tvar $S^{\alpha}F(I/S,t)$. V tomto případě lze odvodit, že hodnota opce má tvar
\begin{equation*}
V = S^{\alpha}H(R,t)
\end{equation*}
kde $R = I/S$, diferenciální rovnice tvar
\begin{equation}
\frac{\partial H}{\partial t} + \frac{1}{2}\sigma^2 R^2 \frac{\partial^2 H}{\partial R^2} + (1 + (\sigma^2(1 - \alpha) -r)R)\frac{\partial H}{\partial R} - (1 - \alpha)\Big(\frac{1}{2}\sigma^2 \alpha + r \Big)H = 0
\end{equation}
a výplatní funkce v době splatnosti tvar
\begin{equation*}
H(R,T) = F(R)
\end{equation*}
Tento dvourozměrný problém však není o nic méně složitý než původní Black-Scholes diferenciální rovnice. Proto se rovnice (14.1) zpravidla řeší numericky. V případě americké varianty opce, se rovnice (14.1) změní na nerovnici, tj. $=$ je nahrazeno $\le$.

\subsection{Opce na průměrnou realizační cenu}

Uvažujme kupní opci na průměrnou realizační cenu se spojitým aritmetickým průměrem. Výplata generovaná na konci životnosti opce definována jako
\begin{equation*}
\max \Big( S - \frac{1}{T} \int_0^T S(\tau) d \tau, 0 \Big)
\end{equation*}
Jestliže bychom chtěli oceňovat americkou variantu této opce, museli bychom definovat výplatu v případě předčasného uplatnění opce. Něchť je tato výplata definovaná jako
\begin{equation}
\max \Big( S - \frac{1}{t} \int_0^t S(\tau) d \tau, 0 \Big)
\end{equation}
Definujme výše uvažovanou veličinu $R$.
\begin{equation}
R = \frac{1}{S}\int_0^t S(\tau) d \tau
\end{equation}
Výplatní funkci na konci životnosti opce resp. v případě jejího předčasného uplatnění tak lze vyjádřit jako
\begin{equation*}
S \max(1 - R/T, 0)
\end{equation*}
resp.
\begin{equation*}
S \max(1 - R/t, 0)
\end{equation*}
S ohledem na argumentaci v kapitole 14.2.3 má hodnota opce tvar
\begin{equation*}
V(S,R,t) = SH(R,t), ~~~ R = \frac{I}{S}
\end{equation*}
a příslušná diferenciální rovnice tvar
\begin{equation}
\frac{\partial H}{\partial t} + \frac{1}{2}\sigma^2R^2\frac{\partial^2 H}{\partial R^2} + (1 - rR)\frac{\partial H}{\partial R} \le 0
\end{equation}
V případě evropské varianty opce je v rovnici (14.4) ostrá nerovnost. V případě americké varianty opce můžeme mít v rovnici (14.4) opět nerovnost, avšak musí být splněna podmínka
\begin{equation*}
H(R,t) \ge \Lambda(R,t) = \max(1 - R/t, 0)
\end{equation*}
Navíc dotkne-li se hodnota opce výplatní funkce (14.3), musí být hodnota opce její tečnou. To znamená, že funkce $H(R,t)$ a její první derivace podle $R$ musí být spojité na celém svém definičním oboru.

\subsection{Opce na průměrnou podkladovou cenu}

Opce na průměrnou podkladovou cenu spadá do rodiny asijských opcí. Výplatní funkce kupní opce má v době splatnosti tvar
\begin{equation*}
\max \Big( \frac{I}{T} - E, 0\Big)
\end{equation*}
Tento druh opcí je z hlediska ocenění složitější než opce na průměrnou realizační cenu. V případě aritmetického průměru nelze na opci na průměrnou podkladovou cenu aplikovat podobnostní řešení a zredukovat tak počet nezávislých proměnných. Problém ocenění opce na průměrnou podkladovou cenu tak musí být řešen numericky. V případě geometrického průměru je však možné podobnostní řešení za účelem zredukování počtu nezávislých proměnných použít. V následujícím textu odvodíme explicitní rovnice pro ocenění opce.

\subsubsection{Geometrický průměr}

Pro evropskou opci na průměrnou podkladovou cenu existují explicitní rovnice pro případ geometrického průměru. U geometrického průměru totiž podkladové aktivum sleduje náhodnou procházku charakterizovanou rozptylem, který je nezávislý na ceně aktiva.

Uvažujme evropskou opci na průměrnou podkladovou cenu, jejíž výplatní funkce je v době splatnosti dána
\begin{equation*}
V(S,I,T) = \Lambda(I)
\end{equation*}
Protože je výplatní funkce pouze funkcí $I$ a nikoliv $S$, je možné odvodit explicitní řešení. Má-li geometrický průměr spojitý charakter, lze pomocnou veličinu $I$ vyjádřit jako
\begin{equation*}
I = \int_0^t \ln S(\tau) d \tau
\end{equation*}
S ohledem na (13.2) tak řešíme diferenciální rovnici
\begin{equation}
\frac{\partial V}{\partial t} + \ln S \frac{\partial V}{\partial I} + \frac{1}{2} \sigma^2 \frac{\partial^2 V}{\partial S^2} + r S \frac{\partial V}{\partial S} - rV = 0
\end{equation}
Je-li výplatní funkce pouze funkcí $I$, má řešení tvar $F(y,t)$, kde
\begin{equation*}
y = \frac{I + (T - t) \ln S}{T}
\end{equation*}
Diferenciální rovnice (14.5) se stává parabolickou diferenciální rovnicí s koeficienty nezávislými na $y$ a eliminovým členem obsahujícím logaritmus.
\begin{equation}
\frac{\partial F}{\partial t} + \frac{1}{2}\Big( \frac{\sigma(T - t)}{T} \Big)^2 \frac{\partial^2 F}{\partial y^2} + \Big( r - \frac{1}{2}\sigma^2 \Big) \Big( \frac{T - t}{T} \Big)\frac{\partial F}{\partial y} - rF = 0
\end{equation}
Po aplikaci logaritmické transformace připomíná (14.6) standardní Black-Scholes rovnici s volatilitou, úrokovou sazbou a dividendovým výnosem, kde všechny tři parametry vystupují jako funkce času. V kapitole 6.4 jsme ukázali, že pomocí jednoduchých modifikací základního Black-Scholes modelu odvodit explicitní rovnice pro volatilitu, úrokovou sazbu a dividendový výnos, které jsou funkcí času. Pro opci na průměrnou podkladovou cenu lze tuto modifikaci provést pomocí následujících kroků.
\begin{itemize}
\item Vezmeme Black-Scholes rovnici pro plain-vanilla opci, která má shodnou výplatu jako příslušná opce na průměrnou podkladovou cenu. Výplata je však spíše než v rámci $e^{I/T}$ definovaná v rámci $S$. Použijeme tak např.
\begin{center}
$\max(S - E, 0)$ namísto $\max(e^{I/T} - E, 0)$.
\end{center}
\item Kdekoliv se v rovnici pro $V_{BS}$ objeví $\sigma^2$, nahradíme jej
\begin{equation*}
\frac{1}{T - t} \int_t^T \sigma^2 \frac{(T - \tau)^2}{T^2} d \tau
\end{equation*}
Proto, je-li $\sigma$ konstantní, je volatilita rovna $\sigma^2 \frac{(T - t)^2}{3T^2}$.
\item Kdekoliv se v rovnici pro $V_{BS}$ objeví $r$, nahradíme jej
\begin{equation*}
\frac{1}{T - t}\int_t^T \Big( r - \frac{1}{2}\sigma^2 \Big) \frac{(T - \tau)}{T} d \tau
\end{equation*}
Jsou-li $r$ a $\sigma$ konstanty, je úroková míra rovna $\Big( \frac{1}{2}\sigma^2 - r \Big) \frac{T - t}{2T}$.
\item Výslednou rovnici vynásobíme členem
\begin{equation*}
e^{- \int_t^T (r - \frac{T - \tau}{T}(r - \frac{1}{2}\sigma^2)) d \tau}
\end{equation*}
Jsou-li $r$ a $\sigma$ konstanty je tento člen roven 
\begin{equation*}
e^{-\frac{1}{2}(\sigma^2 + (r - \frac{1}{2} \sigma^2)(T + t))(T - t)}
\end{equation*}
\item $S$ nahradíme $e^{I/T}S^{(T-t)/T}$.
\end{itemize}
Kromě spojitého geometrického průměru existuje explicitní řešení také pro diskrétní geometický průměr. Odvození ponecháme jako cvičení pro čtenáře.

\section{Diskrétní cena podkladového aktiva}

V praxi je problematické počítat průměr z kompletní časové řady cen podkladového aktiva. Důvodem mohou být různé zdroje ocenění nebo možnost zcestných kotací. Zpravidla je tak dohodnut konkrétní zdroj kotací a časový okamžik, ke kterému budou zjišťovány ceny podkladového aktiva. Cena podkladového aktiva má tedy spíše charakter nespojité veličiny.

Vraťme se k diskuzi v kapitole 13, která se týkala nespojitě aktualizované pomocné veličiny $I$. V této kapitole jsme se zabývali pouze aritmetickým průměrem, avšak příslušnou argumentaci lze snadno rozšířit na geometrický průměr. Nespojitou aritmetickou sumu jsme definovali jako
\begin{equation*}
I = \sum_{i=1}^{j(t)}S(t_i)
\end{equation*}
kde $t_i$ představovalo čas, ke kterému se aktualizovala cena podkladového aktiva a $j(t)$ bylo největší celé číslo splňující podmínku $t_{j(t)} \le t$. Nespojitý aritmetický průměr tak byl definován jako
\begin{equation*}
\frac{I}{j(t)}
\end{equation*}
Přes jednotlivé časové okamžiky $t_i$ je pomocná veličina $I$ nutně nespojitá, protože změní skokově svou hodnotu z $I$ na $I + S$. Protože hodnota opce je vždy spojitá, musí být splněna skoková podmínka
\begin{equation*}
V(S, I, t_i^{-}) = V(S, I + S, t_i^{+})
\end{equation*}
Podobně v případě geometrického průměru je nespojitá suma definována jako
\begin{equation*}
I = \sum_{i = 1}^{j()t} \ln S(t_i)
\end{equation*}
a skoková podmínka jako
\begin{equation*}
V(S, I, t_i^{-}) = V(S, I + \ln S, t_i^{+})
\end{equation*}
Protože $I$ je aktualizováno nespojitě a je tak mezi časovými okamžiky $t_i$ a $t_{i + 1}$ konstantní, je diferenciální rovnice, kterou se řídí hodnota opce, mezi těmito časovými okamžiky základní Black-Scholes rovnicí s $I$ jako konstatním parametrem. Postup ocenění libovolné asijské opce je tedy následující.
\begin{itemize}
\item Řešíme diferenciální rovnici
\begin{equation*}
\frac{\partial V}{\partial t} + \frac{1}{2} \sigma^2 S^2 \frac{\partial^2 V}{\partial S^2} + r S \frac{\partial V}{\partial S} - rV = 0
\end{equation*}
mezi jednotlivými časovými okamžiky, ke kterým dochází k aktualizaci $I$, a to od splatnostni opce směrem k počátku.
\item Aplikujeme skokovou podmínku přes aktuální časový okamžik s cílem určit hodnotu opce těsně před tímto časovým okamžikem.
\item Celý výše popsaný proces opakujeme, dokud nezískáme současnou hodnotu opce.
\end{itemize}

\chapter{Zpětná opce}

\section{Úvod}

Zpětná opce je finační derivát, jehož výplata se odvíjí od maximální popř. minimální ceny podkladového aktiva po dobu splatnosti opce. Maximum popř. minimum ceny podkladového aktiva může mít charakter spojité nebo diskrétní veličiny. Vedle evropské verze zpětné opce existuje také americká verze, která zahrnuje možnost předčasného uplatnění.

Podobně jako v případě asijské opce existuje zpětná opce na realizační cenu a zpětná opce na podkladovou cenu. Představuje-li $J$ maximum ceny podkladového aktiva definované jako
\begin{equation*}
J = \underset{0 \le \tau \le t}\max S(\tau)
\end{equation*}
generuje prodejní zpětná opce na realizační cenu v době splatnosti výplatu
\begin{equation*}
V(S,J,T) = \max(J - S, 0)
\end{equation*}
Zpětná opce na realizační cenu umožňuje aplikaci podobnostního řešení a tím pádem také redukci z proměnných $S$, $J$ a $t$ na proměnné $\frac{S}{J}$ a $t$.

V případě odpovídající prodejní zpětné opce na podkladovou cenu má výplatní funkce v době splatnosti tvar
\begin{equation*}
V(S,J,T) = \max(E - J, 0)
\end{equation*}
Narozdíl od předchozího typu neumožňuje zpětná opce na podkladovou cenu použití podobnostního řešení a musí být řešena ve třech dimenzích.

V případě kupní zpětné opce je $J$ definováno jako
\begin{equation*}
J = \underset{0 \le \tau \le t}\min S(\tau)
\end{equation*}
Protože je kupní opce z pohledu postupu ocenění velice podobná prodejní opci, zaměříme se v následujícím textu pouze na prodejní zpětné opce.

\section{Spojité maximum}

Uvažujme prodejní opci, jejíž hodnota závisí na maximální ceně podkladového aktiva, kde maximum je aktualizováno spojitě. Pro cenu podkladového aktiva tedy v libovolném okamžiku musí platit
\begin{equation*}
0 \le S \le J
\end{equation*}
Hodnota této opce je funkcí proměnných $S$, $J$ a $t$ a budeme ji zapisovat jako $V(S,J,t)$. Definujme
\begin{equation*}
I_n = \int_0^t(S(\tau))^n d \tau
\end{equation*}
a
\begin{equation*}
J_n = (I_n)^{\frac{1}{n}}
\end{equation*}
Pro $n \rightarrow \infty$ získáváme\footnote{Podmínkou je, že $S(\tau)$ je spojité. V případě, že by $S(\tau)$ bylo diskrétní, nemusí níže uvedená rovnice platit.}
\begin{equation*}
J = \underset{n \rightarrow \infty}{\lim} J_n = \underset{0 \le \tau \le t} \max S(\tau)
\end{equation*}
Podobně pro $n \rightarrow -\infty$ platí
\begin{equation*}
J = \underset{n \rightarrow -\infty}{\lim} J_n = \underset{0 \le \tau \le t} \min S(\tau)
\end{equation*}
Nyní je třeba odvodit stochastickou diferenciální rovnici pro $J_n$. Mezi časem $t$ a $t + dt$ se změní $J_n$ o $d J_n$.
\begin{equation*}
J_n + d J_n = \Big( \int_0^{t + dt}(S(\tau))^n d \tau \Big)^{\frac{1}{n}}
\end{equation*}
Z (2.1) a výše uvedené rovnice vyplývá
\begin{equation}
d J_n = \frac{1}{n}\frac{S^n}{(J_n)^{n-1}}dt
\end{equation}
Protože (15.1) neobsahuje žádnou náhodnou složku, je $dJ_n$ deteministické a můžeme tak zkonstruovat bezrizikové portfolio, které se skládá z jedné opce a $-\Delta$ jednotek podkladového aktiva.
\begin{equation*}
\Pi = V - \Delta S
\end{equation*}
Mezi časem $t$ a $t + dt$ se hodnota portfolia změní o $d \Pi$.
\begin{equation*}
d \Pi = dV - \Delta S
\end{equation*}
Použijeme-li It\^o lemmu k rozvoji $dV$ a uvědomíme-li si, že $V$ je funkcí proměnných $S$, $J_n$ a $t$, získáme
\begin{equation}
d \Pi = \frac{\partial V}{\partial t}dt + \frac{1}{n}\frac{S^n}{(J_n)^{n-1}}\frac{\partial V}{\partial J_n}dt + \frac{1}{2}\sigma^2 S^2 \frac{\partial^2 V}{\partial S^2}dt
\end{equation}
Jedná-li se o americkou verzi opce, může nastat situace, kdy je optimální opci předčasně uplatnit. V tomto případě musí pro výnos generovaný portfoliem platit, že může být nanejvýš roven bezrizikové výnosové míře.
\begin{equation}
d \Pi \le r \Pi dt = r(V - \Delta S)
\end{equation}
U evropské verze by nerovnost v (15.3) byla nahrazena rovností. Kombinací (15.3) a (15.2) získáváme
\begin{equation}
\frac{\partial V}{\partial t} + \frac{1}{n}\frac{S^n}{(J_n)^{n-1}}\frac{\partial P}{\partial J_n} + \frac{1}{2}\sigma^2 S^2 \frac{\partial^2 V}{\partial S^2} + rS\frac{\partial V}{\partial S} - rV \le 0
\end{equation}
Nyní aplikujme limitu $n \rightarrow \infty$. Protože $S \le \max S = J$, koeficient $\frac{\partial V}{\partial J_n}$ konverguje k nule a (15.4) tak přejde do tvaru
\begin{equation}
\frac{\partial V}{\partial t} + \frac{1}{2}\sigma^2 S^2 \frac{\partial^2 V}{\partial S^2} + rS\frac{\partial V}{\partial S} - rV \le 0
\end{equation}
(15.5) tak představuje standardní Black-Scholes nerovnici.

Kromě (15.4) figuruje veličina $J$ také v hraničních a konečné podmínce. Konečná podmínka je představovaná výnosem generovaným opcí v době splatnosti. V případě prodejní zpětné opce na realizační cenu má tento výnos podobu
\begin{equation}
P(S,J,T) = \max(J - S, 0)
\end{equation}
a jedná se o konečnou podmínku bez ohledu na to, jde-li o evropskou nebo americkou verzi opci nebo má-li maximum charakter spojité nebo diskrétní veličiny. Pro maximum jako spojitou veličinu vždy platí $S \le J$, a proto je možné řešení problému omezit pouze na obor $0 \le S \le J$. Zajímavostí je, že s ohledem na $S \le J$ není uvažovaná opce opcí v pravém slova smyslu, protože bude vždy uplatněna. Výjimkou je hypotetická možnost, že maximum ceny podkladového aktiva nastane přesně v době splatnosti opce.

\subsection{Evropská varianta zpětné opce}

Je-li zpětná opce bez možnosti předčasného uplatnění, změní se nerovnost v (15.5) se na rovnost. Konečná podmínka je definována skrze (15.6) a hraniční podmínky jsou uplatněny pro $S = 0$ a $S = J$.

Uvažujme evropskou variantu prodejní zpětné opce na realizační cenu. Dosáhne-li kdykoliv v průběhu životnosti cena podkladového aktiva hodnoty nula, generuje opce v době své splatnosti výplatu $J$. Hraniční podmínka pro $S = 0$ má tak podobu
\begin{equation}
P(0, J, t) = e^{-r(T - t)}J
\end{equation}

Druhou hraniční podmínku lze odvodit na základě chování náhodné procházky v blízkosti $S = J$. Předpokládejme, že v určitý čas před splatností je $S$ blízké aktuální hodnotě $J$. Lze dokázat, že pravděpodobnost, že se hodnota $J$ do splatnosti opce nezmění, je nulová. Protože aktuální hodnota $J$ není konečné maximum, je senzitivita změny hodnoty opce na změnu $J$ nulová. Zbývající hraniční podmínka má tedy tvar
\begin{equation}
\frac{\partial P}{\partial J} = 0, ~~~ S = J
\end{equation}
Tímto je definice problému evropské varianty prodejní zpětné opce na realizační cenu kompletní. Analytická rovnice pro hodnotu opce je dána vztahem
\begin{equation*}
S(-1 + N(d_7)(1 + k^{-1})) + Je^{-r(T-t)}\Big( N(d_5) - k^{-1} \Big( \frac{S}{J}^{1-k} \Big) N(d_6) \Big)
\end{equation*}
kde
\begin{equation*}
d_5 = \frac{\ln (J/S) - (r - \frac{1}{2}\sigma^2)t}{\sigma \sqrt{T - t}}
\end{equation*}
\begin{equation*}
d_6 = \frac{\ln (S/J) - (r - \frac{1}{2}\sigma^2)t}{\sigma \sqrt{T - t}}
\end{equation*}
\begin{equation*}
d_7= \frac{\ln (J/S) + (r + \frac{1}{2}\sigma^2)t}{\sigma \sqrt{T - t}}
\end{equation*}
a
\begin{equation*}
k = \frac{r}{\frac{1}{2}\sigma^2}
\end{equation*}
Řešení může být odvozeno pomocí rozšířené metody obrazů.

\subsection{Americká varianta zpětné opce}

Uvažujme americkou verzi kupní zpětné opce na realizační cenu. Nechť je výplatní funkce této opce definována jako
\begin{equation*}
\Lambda(S, J, t) = \max(J - S, 0)
\end{equation*}
V případě americké verze opce existují případy, kdy je optimální opci držet a případy, kdy je naopak optimální opci předčasně uplatnit. Z titulu neexistenci arbitráže musí hodnota opce splňovat podmínku
\begin{equation}
V(S, J, t) \ge \Lambda(S, J, t)
\end{equation}
Dále platí, že $V$, $\frac{\partial V}{\partial S}$ a $\frac{\partial V}{\partial J}$ jsou spojité.

Konečná podmínka (15.6) musí být splněna pro $t = T$. Spadl-li kdy v průběhu životnosti opce bod $S = 0$ resp. $S = J$ do oboru hodnot cen podkladového aktiva, pro které je optimální opci držet, musí být splněna také hraniční podmínka (15.7) resp. (15.8).

Pro $t < T$ nemůže $S = 0$ spadat do oboru hodnot, pro které optimální opci držet. To vyplývá z (15.7) a (15.10). Kdyby totiž $S = 0$ spadalo do tohoto oboru hodnot, pak by platilo
\begin{equation*}
V(0, J, t) = Je^{-r(T-t)} < \Lambda(0, J, t) = J
\end{equation*}
což je v rozporu s (15.9). Proto musíme definovat hranici $S_f(J,t)$, od které je optimální zpětnou opci držet\footnote{Racionální investor tedy opci předčasně uplatní, je-li $S < S_f(J,t)$, a naopak bude opci držet v případě, že $S > S_f(J,t)$}. Tento problém je možné přeformulovat do podoby lineární komplementarity a zbavit se tak explicitního odkazu na hraniční cenu $S_f(J,t)$. Definujme operátor
\begin{equation*}
\mathcal{L}_{BS}(\cdot) = \frac{\partial}{\partial t} + \frac{1}{2}\sigma^2 S^2 \frac{\partial^2}{\partial S^2} + rS \frac{\partial}{\partial S} - r
\end{equation*}
Problém americké verze kupní zpětné opce na realizační cenu tak může být vyjádřen jako
\begin{center}
$\mathcal{L}_{BS}(V) \le 0$ a $(V - \Lambda(S,J,t)) \ge 0$
\end{center}
společně s
\begin{equation*}
\mathcal{L}_{BS}(V) \cdot (P - \Lambda) = 0
\end{equation*}
konečnou podmínkou
\begin{equation*}
V(S, J, T) = \Lambda(S, J, T)
\end{equation*}
a hraniční podmínkou
\begin{equation*}
\frac{\partial V}{\partial J} = 0, ~~~ S = J
\end{equation*}
Problém je řešen pouze pro $0 \le S \le J$ a za předpokladu, že $V$, $\frac{\partial V}{\partial S}$ a $\frac{\partial V}{\partial J}$ jsou spojité.

\section{Diskrétní maximum}

V praxi jsou zpětné opce konstruovány pro diskrétní maximum popř. minimum ceny podkladového aktiva. Z technického hlediska je totiž jednodušší monitorovat cenu v předem stanovených časových okamžicích. Velice častým případem je stanovení maxima popř. minima podle uzavírací ceny. Takovéto zpětné opce jsou také ``levnější'' než jejich protějšky se spojitě aktualizovaným maximem popř. minimem.

Uvažujme prodejní zpětnou opci na realizační cenu, jejíž maximum je aktualizováno diskrétně. Nechť veličina $J$ představuje nespojité maximum ceny podkladového aktiva. Předpoklad $S \le J$ tak již nemusí platit a řešení problému hodnoty zpětné opce tak již není omezeno pouze na obor hodnot $0 \le S \le J$. Ačkoliv nadále platí, že výplatní funkce má tvar $\max(J - S, 0)$, nemusí být opce vždy uplatněna. Jestliže v době splatnosti bude platit $S > J$, racionální investor opci neuplatní.

Také v případě diskrétního maxima ceny podkladového aktiva platí Black-Scholes rovnice resp. nerovnice pro evropskou resp. americkou verzi opce, kde $J$ vystupuje jako parametr. Přes časové okamžiky, ke kterým je aktualizováno maximum, je nutné aplikovat skokovou podmínku. Důvodem je, stejně jako v předchozím případech, požadavek na splnění neexistence arbitráže. Uplatnění skokové podmínky má za následek, že hodnota zpětné opce je spojitá i tyto časové okamžiky. Diskrétní maximum ceny podkladového aktiva je aktualizováno podle
\begin{equation*}
J_i = \max(J_{i-1}, S)
\end{equation*}
kde $J_i$ je maximum platné mezi časovými okamžiky $t_i$ a $t_{i+1}$. Skoková podmínka pak má podobu
\begin{equation*}
V(S_i, J_{i-1}, t_i^{-}) = V(S_i, J_i, t_i^{+})
\end{equation*}
resp.
\begin{equation*}
V(S, J, t_i^{-}) = V(S, \max(J,S), t_i^{+})
\end{equation*}
kde veličiny $S$ a $J$ jsou vzájemně nezávislé.

\section{Podobnostní řešení}

Vzhledem k tomu, že v případě zpětné opce je možné výplatní funkci zapsat ve tvaru
\begin{equation*}
\Lambda(S, J, t) = J \bar{\Lambda}(S/J, t)
\end{equation*}
a lze aplikovat podobnostní řešení a zredukovat tak počet proměnných. V případě evropské verze prodejní zpětné opce na realizační cenu má výplatní funkce podobu
\begin{equation*}
J \max(1 - \xi, 0)
\end{equation*}
a hledané řešení tvar
\begin{equation*}
V(S,J,t) = JW(\xi,t)
\end{equation*}
kde
\begin{equation*}
\xi = \frac{S}{J}
\end{equation*}
Řešení $W$ musí splňovat diferenciální rovnici
\begin{equation*}
\frac{\partial W}{\partial t} + \frac{1}{2} \sigma^2 \xi^2 \frac{\partial^2 W}{\partial \xi^2} + r \xi \frac{\partial W}{\partial \xi} - r W = 0
\end{equation*}
při hraniční podmínce v $S = 0$
\begin{equation*}
W(0, t) = e^{-r(T-t)}
\end{equation*}
a konečné podmínce
\begin{equation*}
W(\xi, T) = \max(1 - \xi, 0)
\end{equation*}
Jestliže by maximum ceny podkladového aktiva bylo aktualizováno spojitě, stala by se hraniční podmínka v $S = J$ hraniční podmínkou v $\xi = 1$.
\begin{equation*}
\frac{\partial W}{\partial \xi} = W, ~~~ \xi = 1
\end{equation*}
V případě nespojité aktualizace maxima je nutné uvažovat hraniční podmínku pro $S \rightarrow \infty$. Po aplikaci podobnostního řešení přejde hraniční podmínka z bodu $S \rightarrow \infty$ do bodu $\xi \rightarrow \infty$.
\begin{equation*}
\frac{\xi}{W}\frac{\partial W}{\partial \xi}, ~~~ \xi \rightarrow \infty
\end{equation*}
Skoková podmínka přes časové okamžiky, ke kterým je akuatizováno maximum ceny podkladového aktiva, má tvar
\begin{equation*}
W(\xi, t_i^{-}) = \max(\xi, 1) W(\min(\xi, 1), t_i^{+})
\end{equation*}
Lze dokázat, že splňuje-li $V(S,t)$ Black-Scholes rovnici, pak $U(S,t) = S^{\alpha}V(\alpha/S,t)$ splňuje pro $\alpha = 1 - r/\frac{1}{2} \sigma^2$ taktéž Black-Scholes rovnici. Toto zjištění spolu s výše uvedenými rovnicemi vedou k explicitním rovnicím pro výpočet hodnoty evropské verze prodejní zpětné opce na realizační cenu.

\section{Numerické příklady}

Následující tabulky obsahují ilustrační hodnoty americké a evropské verze prodejní zpětné opce na realizační cenu. Ve všech případech se jedná o opce se zbytkovou splatností jeden rok, s bezrizikovou výnosovou mírou $r = 0.1$, kde podkladovým aktivem je akcie s nulovým dividendovým výnosem a směrodatnou odchylkou ceny $\sigma = 0.2$. V každé tabulce jsou příklady opcí označené písmeny $A$, $B$, $C$ a $O$. Poslední příkladě přestavuje evropskou plain-vanilla prodejní opci s jednotkovou realizační cenou. Ostatní příklady představují zpětnou opci s nespojitě aktualizovaným maximem ceny podkladového aktiva. Data, ke kterým je maximum aktualizováno jsou následující.
\begin{itemize}
\item A: 0.5, 1.5, 2.5, ..., 10.5 a 11.5 měsíce
\item B: 1.5, 3.5, 5.5, 7.5, 9.5 a 11.5 měsíce
\item C: 3.5, 7.5 a 11.5 měsíce
\end{itemize}
Hodnoty zpětných opcí byly vypočteny numericky; hodnota plain-vanilla evropské opce byla vypočtena analyticky.
\begin{table}
\begin{center}
\begin{tabular}{c c c c c}
$\xi$ & A & B & C & O\\
\hline
 0.9 & 0.125 & 0.120 & 0.114 & 0.104\\
 1.0 & 0.105 & 0.095 & 0.081 & 0.048\\
 1.1 & 0.111 & 0.098 & 0.082 & 0.021\\
\hline
\end{tabular}
\end{center}
\caption{Americká verze prodejní zpětné opce na realizační cenu pro $r = 0.1$, $\sigma = 0.2$, $T = 1$ pro různé frekvence aktualizace maxima ceny podkladového aktiva. Příklad O reprezentuje plain-vanilla americkou prodejí opci.}
\end{table}
\begin{table}
\begin{center}
\begin{tabular}{c c c c c}
$\xi$ & A & B & C & O\\
\hline
 0.9 & 0.101 & 0.094 & 0.087 & 0.074\\
 1.0 & 0.089 & 0.079 & 0.067 & 0.038\\
 1.1 & 0.095 & 0.083 & 0.068 & 0.017\\
\hline
\end{tabular}
\end{center}
\caption{Evropská verze prodejní zpětné opce na realizační cenu pro $r = 0.1$, $\sigma = 0.2$, $T = 1$ pro různé frekvence aktualizace maxima ceny podkladového aktiva. Příklad O reprezentuje plain-vanilla evropskou prodejí opci.}
\end{table}
Uvažujme americkou verzi prodejní zpětné opce na maximum ceny podkladového aktiva aktualizovaného podle příkladud B. Připomeňme, že hodnotu této opce lze vyjádřit jako
\begin{equation*}
V(S,J,t) = JW(S/J,t)
\end{equation*}
Jestliže je zbytková cena do splatnosti této opce jeden rok, aktuální maximum rovno 180 USD a aktuální cena podkladového aktiva je rovna 200 USD, pak $\xi = 180 / 200 = 0.9$. Hodnota opce je tak rovna $200 \cdot 0.120 = 24$ USD.

Všimněme si, že s tím jak klesá frekvence aktualizace ceny podkladového aktiva, klesá také hodnota zpětné opce. Hodnota opce dosahuje minima v okolí $\xi = 1$. Hodnota delty zpětné opce se může stát kladnou, protože z pohledu majitele opce je výhodné, vzroste-li cena podkladového aktiva těsně před aktualizací maxima a následně opět klesne. V souladu s očekáváním také platí, že hodnota americké verze opce je vyšší než hodnota odpovídající evropské varianty.

\section{Perpetuitní opce}

Perpetuitní opce jsou specifické tím, že nemají datum splatnosti a jsou tak vlastně ``nekonečné''. V této kapitole se budeme zabývat ruskou opcí a stop-loss opcí.

\subsection{Ruská opce}

Ruská opce je perpetuitní americkou opcí, která může být kdykoliv uplatněna svým majitelem. Opce pak generuje výplatu, která odpovídá maximální ceně podkladového aktiva od okamžiku sjednání opce do okamžiku jejího uplatnění. V následujícím textu bude uvažovat pouze spojitě aktualizované maximum, protože pro případ diskrétního maxima neexistuje přijatelné analytické řešení. Dále budeme předpokládat, že podkladová akcie generuje dividendový výnos.

Protože se jedná o perpetuitní opci, můžeme předpokládat, že její hodnota je nezávislá na čase.
\begin{equation*}
V = V(S,J)
\end{equation*}
Veličina $J$ opět představuje maximum ceny podkladového aktiva. Je-li optimální opci držet, musí být splněna Black-Scholes rovnice
\begin{equation*}
\frac{1}{2} \sigma^2 S^2 \frac{\partial^2 V}{\partial S^2} + (r - D_0)S \frac{\partial V}{\partial S} - rV = 0
\end{equation*}
doplněná o hraniční podmínku
\begin{equation*}
\frac{\partial V}{\partial J} = 0, ~~~ J = S
\end{equation*}
Také musí být splněna klasická podmínka pro americkou opci $V \ge J$, kde $J$ představuje výplatu ruské opce v případě jejího uplatnění. Aby mohla nastat situace, kdy bude optimální opci uplatnit, musí existovat volná hranice, na které musí být $V$ a $\frac{\partial V}{\partial S}$ spojité. V opačném případě by opce ztratila smysl. Řešení má tvar
\begin{equation*}
V = JW(\xi)
\end{equation*}
kde $\xi = \frac{S}{J}$. Po úpravách tak získáváme
\begin{equation}
\frac{1}{2} \sigma^2 \xi^2 W'' + (r - D_0) \xi W' - rW = 0
\end{equation}
kde $'$ představuje $d/d \xi$. Předpokládejme, že volná hranice je $\xi = \xi_0$. Hraniční podmínky tak přejdou do tvaru
\begin{equation*}
W - W' = 0, ~~~ \xi = 1
\end{equation*}
a
\begin{center}
$W = 1$ ~ a ~ $W' = 0$, ~~~ $\xi = 1$
\end{center}
Obecné řešení rovnice (15.10) lze odvodit pomocí $W = k \xi^{\alpha}$, kde $k$ a $\alpha$ jsou konstanty. To vede ke kvadratické rovnici pro $\alpha$, jejíž kořeny jsou
\begin{equation}
\alpha_{\pm} = \frac{1}{\sigma^2} \Bigg( -r + D_0 + \frac{1}{2}\sigma^2 \pm \sqrt{(r - D_0 - \frac{1}{2}\sigma^2)^2 + 2\sigma^2 r} \Bigg)
\end{equation}
S pomocí (15.11) lze tak snadno odvodit analytické řešení $W$
\begin{equation*}
W = \frac{1}{\alpha_{+} - \alpha_{-}} \Bigg( \alpha_{+} \Big( \frac{\xi}{\xi_0} \Big)^{\alpha_{-}} - \alpha_{-} \Big( \frac{\xi}{\xi_0}\Big)^{\alpha_{+}} \Bigg)
\end{equation*}
a volnou hranici
\begin{equation*}
\xi_0 = \Bigg( \frac{a_{+}(1 - \alpha_{-})}{\alpha_{-}(1 - \alpha_{+})} \Bigg)^{\frac{1}{\alpha_{-} - \alpha_{+}}}
\end{equation*}
Jestliže by byl dividendový výnos roven nule, tj. $D_0 = 0$, problém ruské opce by neměl řešení. Je zřejmé, že v takovémto případě by nikdy nebylo optimální opci držet.

\subsection{Stop-loss opce}

Stop-loss opce je perpetuitní bariérová zpětná opce s refundací, která představuje fixní část maximální hodnoty ceny podkladového aktiva. Jestliže tedy $S$ dosáhne maxima $J$ a následně klesne na $\lambda J$, kde $\lambda < 1$, generuje opce výplaty $S$\footnote{Pouze připomeňme, v čase výplaty platí $S = \lambda J$.}. Tímto způsobem lze zajistit značnou část zisku a zároveň se zbavit nejistoty spojené s odhadováním maxima ceny podkladového aktiva. Připomeňme, že opce není aktivovaná do okamžiku, než dojde k poklesu $S$.

Protože opce generuje výplatu ve výši $S$ v okamžiku, kdy $S$ klesne na úroveň $\lambda J$, platí
\begin{equation}
V(\lambda J, J) = \lambda J
\end{equation}
Opět hledáme řešení ve tvaru $V = J W(\xi)$, kde $\xi = \frac{S}{J}$. Diferenciální rovnice je opět dána (15.10), rovnice (15.12) přejde na
\begin{equation*}
W(\lambda) = \lambda
\end{equation*}
a zbývající hraniční podmínka je
\begin{equation*}
W - W' = 0, ~~~ \xi = 1
\end{equation*}
Řešením problému je
\begin{equation*}
W = \lambda \frac{\xi^{\alpha_{+}}(1 - \alpha_{-}) - \xi^{\alpha_{-}} - (2 - \alpha_{+})}{\lambda^{\alpha_{+}}(1 - \alpha_{-}) - \lambda^{\alpha_{-}} - (1 - \alpha_{+})}
\end{equation*}
kde $\alpha_{\pm}$ je dáno rovnicí (15.11). Je-li $D_0 = 0$, má řešení tvar $W = \xi$, tj. $V = S$. Opce je tak z hlediska hodnoty shodná s podkladovým aktivem.

\chapter{Opce s transakčními náklady}

\section{Úvod}

Až dosud jsme neuvažovali transakční náklady. Tento předpoklad odráží také Black-Scholes rovnice (3.5), která je založená na kontiuálním zajištění jehož výsledkem je bezrizikové portfolio. V případě, že přiznáme existenci transakčních nákladů, je nutné (3.5) modifikovat. V následujícím textu se budeme zabývat plain-vanilla evropskou opcí.

\section{Nespojité zajištění}

Jedním ze základních předpokladů základní verze Black-Scholes rovnice je kontinuální zajištění. Díky kontinuálním zajištění jsme schopni zkonstruovat bezrizikové portfolio, které generuje výnosovou míru odpovídající bezrizikové výnosové míře. Zajištění tedy probíhá v čase $dt \rightarrow 0$. V předcházejícíh úvahám jsme abstrahovali od transakčních nákladů. Nyní připusťme jejich existenci. Jestliže jsou transakční náklady nezávislé na časové frekvenci zajištění, pak nekonečný počet zajištění potřebný k udržení bezrizikového portfolia vede k nekonečným transakčním nákladům. Protože různí účastníci trhu mají různé transakční náklady, mají také různé ocenění pro tutéž opci. Hodnota opce se tak stává mimojiné také funkcí transakčních nákladů.

Leland navrhl jednoduchou modifikaci Black-Scholes modelu pro plain-vanilla evropskou opci, která umožňuje zhodlednění transakčních nákladů. Tato modifikace je aplikovatelné také na portfolio opcí. Posloupnost úvah je stejná jako v kapitole 3 s následujícími výjimkami.
\begin{itemize}
\item Složení portfolia je přehodnocováno na konci intervalu délky $\delta t$, kde $\delta t$ již není nekonečně malé, tj. neplatí $\delta t \rightarrow 0$.
\item Náhodná procházka za předpokladu času jako diskrétní veličiny je dána rovnicí
\begin{equation*}
\delta S = \sigma S \phi \sqrt{\delta t} + \mu S dt
\end{equation*}
kde $\phi$ je výsledek realizace standardizovaného normálního rozdělění.
\item Transakční náklady jsou porpocionální velikosti transakce vyjádřené v peněžních jednotkách. Představuje-li $\nu$ počet prodaných ($\nu < 0$) resp. nakoupených ($\nu > 0$) akcií za cenu $S$, jsou transakční náklady rovny $\kappa |\nu| S$, kde $\kappa$ představuje konstantu specifickou pro daný typ investora.
\item Zajištěné portfolio má očekávaný výnos rovný bezrizikové výnosové míře.
\end{itemize}
Postup je ve srovnání s kapitolou 3.3 shodný až po rovnici $d \Pi = dV - \Delta dS$. V rovnici
\begin{equation*}
d\Pi = \sigma S \Bigg( \frac{\partial V}{\partial S} - \Delta \Bigg)\phi \sqrt{\delta t} + \Bigg( \mu S \frac{\partial V}{\partial S} + \frac{1}{2}\sigma^2S^2\frac{\partial^2 V}{\partial S^2}\phi^2 + \frac{\partial V}{\partial t} - \mu \Delta S \Bigg)dt
\end{equation*}
je třeba zohlednit transakční náklady. Výsledný tvar této rovnice na konci časového kroku, kdy je provedena úprava portfolia, je tak
\begin{equation}
\delta \Pi = \sigma S \Bigg( \frac{\partial V}{\partial S} - \Delta \Bigg) \phi \sqrt{\delta t} + \Bigg( \mu S \frac{\partial V}{\partial S} + \frac{1}{2}\sigma^2S^2\frac{\partial^2 V}{\partial S^2}\phi^2 + \frac{\partial V}{\partial t} - \mu \Delta S \Bigg) \delta t - \kappa S |\nu|
\end{equation}
Protože neplatí $\delta t = 0$, nemůžeme nahradit $\phi^2$ očekávanou hodnotou 1. Aplikujme shodnou zajišťovací strategii jako v kapitole 3.3 a definujme počet jednotek podkladového aktiva v ceně $S$, které držíme v čase $t$, jako
\begin{equation*}
\Delta = \frac{\partial V}{\partial S}(S,t)
\end{equation*}
Na konci časového kroku délky $\delta t$, bude tento počet roven
\begin{equation*}
\Delta = \frac{\partial V}{\partial S}(S + \delta t,t + \delta t)
\end{equation*}
Počet jednotek podkladového aktiva, které je třeba nakoupit popř. prodat je tak roven
\begin{equation*}
\nu = \frac{\partial V}{\partial S}(S + \delta S, t + \delta t) - \frac{\partial V}{\partial S}(S,t)
\end{equation*}
Jestliže použijeme Taylorův theorem k rozvoji prvního členu, získáme
\begin{equation*}
\frac{\partial V}{\partial S}(S + \delta S, t + \delta t) = \frac{\partial V}{\partial S}(S,t) + \delta S \frac{\partial^2 V}{\partial S^2}(S,t) + \delta t \frac{\partial^2 V}{\partial S \partial t}(S,t) + ...
\end{equation*}
Protože $\delta S = \sigma S \phi \sqrt{\delta t} + \mathcal{O}(\delta t)$, je dominatním člen řádu $\mathcal{O}(\sqrt{\delta t})$, kdežto člen $\mathcal{O}(\delta t)$ můžeme zanedbat. Výsledný počet prodaných popř. nakoupených jednotek podkladového aktiva je tak roven
\begin{equation*}
\nu \approx \frac{\partial^2 V}{\partial S^2}(S,t)\delta S \approx \frac{\partial^2 V}{\partial S^2}\sigma S \phi \sqrt{\delta t}
\end{equation*}
Očekávaná výše transakčních nákladů je tak rovna
\begin{equation}
\varepsilon[\kappa S |\nu|] = \sqrt{\frac{2}{\pi}} \kappa \sigma S^2 \left| \frac{\partial^2 V}{\partial S^2} \right| \sqrt{\delta t}
\end{equation}
Člen $\sqrt{\frac{2}{\pi}}$ je výsledkem výpočtu očekávané hodnoty $|\phi|$ s využitím operátoru $\varepsilon[F(\cdot)]$, který jsme definovali v kapitole 2.3. S využitím námi zvolené $\Delta$ a (16.2) jako definice transakčních nákladů, lze z (16.1) dopočíst očekávanou změnu hodnoty portfolia.
\begin{equation}
\varepsilon[\delta \Pi] = \Bigg( \frac{\partial V}{\partial t} + \frac{1}{2} \sigma^2 S^2 \frac{\partial^2 V}{\partial S^2} - \kappa \sigma S^2 \sqrt{\frac{2}{\pi \delta t}} \left| \frac{\partial^2 V}{\partial S^2} \right| \Bigg) \delta t
\end{equation}
Předpokládejme, že investor požaduje, aby očekávaná výnosová míra portfolia byla rovna bezrizikové výnosové míře. To nám umožňuje nahradit $\varepsilon[\delta \Pi]$ v (16.3) výrazem $r(V - S \frac{\partial V}{\partial S}) \delta t$. Výsledkem je rovnice
\begin{equation}
\frac{\partial V}{\partial t} + \frac{1}{2} \sigma^2 S^2 \frac{\partial^2 V}{\partial S^2} - \kappa \sigma S^2 \sqrt{\frac{2}{\pi \delta t}} \left| \frac{\partial^2 V}{\partial S^2} \right| + r S \frac{\partial V}{\partial S} - rV = 0
\end{equation}
Připomeneme-li si kapitolu 3.7, je finanční interpretace členu, který se nevyskytuje v základní Black-Scholes rovnici, zřejmá. Druhá derivace hodnoty opce vzhledem k ceně je tzv. gamma opce.
\begin{equation*}
\Gamma = \frac{\partial^2 V}{\partial S^2}
\end{equation*}
Gamma vyjadřuje zkreslení hodnoty opce v případě, že $\delta t$ není nekonečně malé. Hlavní zdroj nejistoty byl sice v (16.4) odstraněn, zůstal však po něm zbytek proporcionální ukazateli gamma. Ten představuje zajištění v následujícím časovém kroce, tj. očekávané transakční náklady.

Z důvodu existence transakčních nákladů je rozdíl mezi oceněním jednotlivých opcí a portfolia opcí. Uvažujme portfolio, které se skládá ze zdvou plain-vanilla evropských kupních opcí, které jsou identické až na skutečnost, že v jedné držíme dlouhou a v druhé krátkou pozici. Je zřejmé, že takovéto portfolio není třeba zajišťovat, protože je samo o sobě bezrizikové - bez ohledu na cenu podkladového aktiva $S$ generuje v době splatnosti nulový výnos. Kdybychom však obě opce posuzovali nezávisle na sobě, byla by výsledná hodnota portfolia záporná. Každé zajištění by totiž pro nás představovalo náklady z titulu transakčních nákladů.

Pro dlouhou pozici v plain-vanilla evropské kupní popř. prodejní opcí platí
\begin{equation*}
\frac{\partial^2 V}{\partial S^2} > 0
\end{equation*}
což lze dokázat derivováním základních rovnic pro výpočet hodnoty odpovídajících opcí, které jsou zmiňovány v kapitole 3.6. Toto tvrzení je pravdivé také v případě existence transakčních nákladů. Z tohoto důvodu je možné v (16.4) zrušit operátor absolutní hodnoty. S využitím substituce
\begin{equation}
\hat{\sigma}^2 = \sigma^2 - 2 \kappa \sigma \sqrt{\frac{2}{\pi \delta t}}
\end{equation}
bude (16.4) shodná se standardní Black-Scholes rovnicí s tím rozdílem, že namísto volatility $\sigma$ bude použita volatilita $\hat{\sigma}^2$. V případě krátké pozice je třeba změnit všechna znaménka s vyjímkou členu, který představuje transakční náklady. Modifikovaná volatilita pak má podobu
\begin{equation}
\hat{\sigma}^2 = \sigma^2 + 2 \kappa \sigma \sqrt{\frac{2}{\pi \delta t}}
\end{equation}
Výsledek (16.5) deklaruje, že modifikovaná volatilita je nižší než skutečná volatilita. V případě růstu ceny podkladového aktiva musí majitel opce v rámci zajištění portfolia prodávat podkladové aktivum. Proti růstu ceny podkladového aktiva tak jdou transakční náklady, což snižuje tento růst z pohledu majitele. Analogickou úvahu pak lze uplatnit pro (16.6).

Pro ilustraci rozdílu v hodnotě opce bez a s transakčními náklady uvažujme
\begin{equation*}
V(S,t) - \hat{V}(S,t)
\end{equation*}
Rozvojem tohoto výrazu pro malá $\kappa$ získáme
\begin{equation*}
\frac{\partial V}{\partial \sigma}(\sigma - \bar{\sigma}) + ...
\end{equation*}
Tento rozdíl je např. pro evropskou kupní opci roven
\begin{equation*}
\frac{2 \kappa S N(d_1) \sqrt{T - t}}{\sqrt{2 \pi \delta t}}
\end{equation*}
Velice důležitá veličina, která je ukrytá v uvažovaném modelu zahrnujícím transakční náklady, je
\begin{equation*}
K = \frac{\kappa}{\sigma \sqrt{\delta t}}
\end{equation*}
Je-li $K \gg 1$, pak transakční náklady převáží základní volatilitu. To znamená, že transakční náklady jsou vzhledem k frekvenci zajištění příliš vysoké\footnote{Jsou-li transakční náklady nebo frekvence zajišťování portfolia příliš vysoké, může nastat
\begin{equation*}
\kappa = 2 \sigma \frac{2 \delta t}{\pi}
\end{equation*}
V tomto případě má difúzní rovnice pro dlouhou pozici v evropské kupní opci negativní koeficient a problém je tedy špatně formulován. V případě, že dojde k růstu ceny podkladového aktiva, hodnota opce paradoxně kvůli transakčním nákladům spojených se zajištěním portfolia poklesne.}. Je-li naopak $K \ll 1$, je frekvence zajištění příliš nízká a je vhodné ji zvýšit s cílem snížit riziko.

\section{Portfolio opcí}

V případě obecného porfolia opcí se znaménko u $\frac{\partial^2 V}{\partial S^2}$ mění. Proto není možné vynechat operátor absolutní hodnoty. Protože je problém ve své obecné formě nelineární, je třeba rovnici (16.4) řešit numericky. Nejvhodnější numerickou metodou je explicitní diferenční metoda.

Následující obrázky ukazují hodnotu a deltu dlouhé pozice v býčím spreadu (dlouhá pozici v kupní opci s $E = 45$ a krátká pozice v kupní opci s $E = 55$) se zbytkovou splatností šest měsíců. Ostatní parametry v námi uvažovaném příkladě jsou $\sigma^2 = 0.4$ a $r = 0.1$. Silná křivka představuje variantu s transakčními náklady, slabší křivka pak variantu bez transakčních nákladů.
\begin{center}
	\begin{pspicture}(0,0)(8.0,7.0)
		\rput(4.0,0){Hodnota dlouhé pozice v býčím spreadu s resp. bez transakčních nákladů}

		\psline[arrows=->](0.5,0.5)(7.5,0.5)
		\psline[arrows=->](0.5,0.5)(0.5,6.5)

                \rput(7.8,0.5){\small{$S$}}
                \rput(0.5,6.8){\small{$V$}}

		\psline[linestyle=dotted](0.5,0.5)(3.5,0.5)(4.0,6.5)(7.5,6.5)
		\pscurve[linewidth=0.2mm](1.2,0.5)(2.5,0.7)(4.5,5.0)(7.5,6.4)
		\psline[linewidth=0.5mm](0.5,0.5)(1.3,0.5)
		\pscurve[linewidth=0.5mm](1.3,0.5)(2.6,0.7)(4.6,4.9)(7.5,6.4)
	\end{pspicture}
\end{center}

\part{Úrokové deriváty}

\chapter{Modelování úrokových sazeb a úrokové deriváty}

\section{Úvod}

Až dosud jsme předpokládali, že úrokové sazby jsou konstatní popř. jsou deterministickou funkcí času. V případě opcí, jejichž splatnost ve většině případů nepřesahuje jeden rok, hraje úroková sazba z pohledu ocenění opce významnou roli. V případě jiných finančních instrumentů, jejichž splatnost může být v řádu let, je však vliv úrokové sazby na jejich ocenění zásadní.

\section{Základy oceňování dluhopisů}

Na dluhopis je možné pohlížet jako na kontrakt, který generuje v době své splatnosti předem známou výplatu. Tato výplata je rovna tzv. nominální hodnotě dluhopisu. Dluhopis může  v pravidelných intervalech generovat tzv. kupónové platby. V tomto případě se jedná o tzv. kupónový dluhopis, v opačném případě o tzv. diskontní dluhopis.

Základní problém ocenění dluhopisů lze formulovat jako otázku současné hodnoty 1 peněžní jednotky vyplácené k určitému časovému okamžiku v budoucnu. Vzhledem k tomu, že splatnost dluhopisů se narodíl od opcí počítá v řádu let, je vhodné v rámci ocenění aplikovat komplexnější model vývoje úrokových sazeb. Až dosud jsme úrokovou sazbu chápali jako deterministickou veličinu. V následujím textu budeme o úrokové sazbě uvažovat jako o stochastické veličině.

\subsection{Ocenění dluhopisu - deterministické úrokové sazby}

V první fázi našich úvah budeme předpokládat, že úrokové sazby $r(t)$, případné kupónové platby $K(t)$ a tím pádem také hodnota dluhopisu jsou $V(t)$ deterministickou funkcí času $t$\footnote{Hodnota dluhopisu je také funkcí splatnosti $T$. Správně bychom tak měli hodnotu dluhopisu vyjadřovat jako $V(t,T)$. Tuto závislost však budeme uvádět, pouze bude-li to v daném kontextu důležité.}. Protože dluhopis v době své splatnosti $T$ generuje částku $Z$, platí $V(T) = Z$.

Změnu hodnoty dluhopisu v časovém kroce délky $dt$ lze vyjádřit jako
\begin{equation*}
\Bigg(\frac{dV}{dt} + K(t) \Bigg)dt 
\end{equation*}
Předpoklad neexistence arbitráže pak vede k rovnici
\begin{equation}
\frac{dV}{dt} + K(t) = r(t)V
\end{equation}
Diferenciální rovnici (17.1) lze řešit pomocí metody variace konstant. Nejprve řešíme homogenní rovnici
\begin{equation*}
\frac{dV(t)}{dt} - r(t)V(t) = 0
\end{equation*}
která odpovídá situaci, kdy nejsou vypláceny žádné kupónové platby.
\begin{equation*}
\frac{1}{V(t)}\frac{dV(t)}{dt} = r(t)
\end{equation*}
\begin{equation*}
\int_{V(t)}^{V(T)}\frac{1}{x}dx = \int_{t}^{T}r(\tau) d \tau
\end{equation*}
\begin{equation*}
\frac{V(T)}{V(t)} = ke^{\int_t^T r(\tau) d\tau}
\end{equation*}
\begin{equation*}
V(t) = kV(T)e^{-\int_t^T r(\tau) d\tau}
\end{equation*}
Protože platí $V(T) = Z$, je výsledný tvar řešení homogenní rovnice
\begin{equation*}
V(t) = Ze^{-\int_t^T r(\tau) d\tau}
\end{equation*}
Pokud partikulární řešení rovnice (17.1) vyjádříme ve tvaru $V_p(t) = c(t)V_h(t)$, kde $V_h(t)$ představuje výše uvedené řešení homogenní rovnice a $c$ je obecná funkce třídy $C^1$, lze (17.1) upravit do podoby
\begin{equation*}
\frac{\partial c(t)V_h(t)}{\partial t} - r(t)c(t)V_h(t) + K(t)= 0
\end{equation*}
\begin{equation*}
\frac{\partial c(t)}{\partial t}V_h(t) + \frac{\partial V_h(t)}{\partial t}c(t) - r(t)c(t)V_h(t) + K(t) = 0
\end{equation*}
\begin{equation*}
\frac{\partial c(t)}{\partial t}V_h(t) + c(t)\Bigg(\frac{\partial V_h(t)}{\partial t} - r(t)V_h(t)\Bigg) + K(t) = 0
\end{equation*}
\begin{equation*}
\frac{\partial c(t)}{\partial t}V_h(t) + K(t) = 0
\end{equation*}
Výsledkem úprav je tedy diferenciální rovnice $\frac{\partial c(t)}{\partial t}V_h(t) + K(t) = 0$, protože $V_h(t)$ je řešením homogenní rovnice a tudíž platí $\frac{\partial V_h(t)}{\partial t} - r(t)V_h(t) = 0$. První derivaci funkce $c$ lze vyjádřit jako
\begin{equation*}
\frac{\partial c(t)}{\partial t} = - \frac{K(t)}{V_h(t)}
\end{equation*}
\begin{equation*}
\frac{\partial c(t)}{\partial t} = - \frac{K(t)e^{\int_t^T r(\tau) d\tau}}{Z}
\end{equation*}
Podobu rovnice $c$ získáme řešením výše uvedené diferenciální rovnice.
\begin{equation*}
c(t) = - \int_T^t \frac{K(t')e^{\int_{t'}^T r(\tau) d\tau}}{Z}dt'
\end{equation*}
\begin{equation*}
c(t) = \int{1}{Z} \int_t^T K(t')e^{\int_{t'}^T r(\tau) d\tau}dt'
\end{equation*}
Partikulární řešení diferenciální rovnice (17.1) má tvar
\begin{equation*}
V_p(t) = e^{-\int_t^T r(\tau) d \tau} \int_t^T K(t')e^{\int_{t'}^T r(\tau) d \tau}dt'
\end{equation*}
a obecné řešení pak tvar
\begin{equation}
V(t) = e^{-\int_t^T r(\tau) d \tau}\Bigg(Z + \int_t^T K(t')e^{\int_{t'}^T r(\tau) d \tau}dt' \Bigg)
\end{equation}
Rovnice (17.2), která splňuje podmínku $V(T) = Z$, představuje hodnotu dluhopisu v čase $t$.

Nyní uvažujme diskotní dluhopis, jehož hodnota v čase $t$ je rovna
\begin{equation}
V(t) = Z e^{-\int_t^T r(\tau) d \tau}
\end{equation}
Opět předpokládejme, že úrokové sazby jsou deterministické. Jestliže jsou v čase $t$ k dispozici ceny diskontních dluhopisů pro všechny budoucí časy splatnosti $T$, známe levou stranu rovnice (17.3) pro všechna $T$. Rovnici (17.3) lze upravit do podoby
\begin{equation}
-\int_t^T r(\tau) d \tau = \ln \frac{V(t,T)}{Z}
\end{equation}
Je-li možné $V(t,T)$ zderivovat podle $T$, pak derivací (17.4) získáváme
\begin{equation}
r(T) = - \frac{1}{V(t,T)}\frac{\partial V}{\partial T}
\end{equation}
Jestliže tržní ceny diskontních dluhopisů odráží deterministickou úrokovou sazbu, pak je tato budoucí úroková sazba dána rovnicí (17.5). Předpokládejme, že úroková sazba je kladná. Pak musí platit
\begin{equation*}
\frac{\partial V}{\partial T} < 0
\end{equation*}
Se zbytkovou cenou diskontního dluhopisu tak klesá jeho hodnota, což je z finančního hlediska zřejmé.

\subsection{Diskrétní kupónové platby}

Rovnice (17.2) zohledňuje také možnost výplaty kupónů. Jestliže jsou kupóny vypláceny k určitému časovému okamžiku (např. každých šest měsíců), obdrží majitel dluhopisu v čase $t_c$ kupón ve výši $K_c$. Důsledkem diskrétní výplaty kupónů je skok hodnotě dluhopisu přes časové okamžiky, ke kterým je kupón vyplácen. Hodnota dluhopisu před a po výplatě kupónu se tak liší o $K_c$.
\begin{equation*}
V(t_c^-) = V(t_c^+) + K_c
\end{equation*}
Výše uvedený vztah představuje skokovou podmínku, která musí být splněna, aby hodnota dluhopisu byla spojitá v čase a nevznikl tak prostor pro arbitráž. Splnění skokové podmínky je nutné i případě, že budeme uvažovat stochastické úrokové sazby.

Z matematického pohledu by bylo vhodnější vyjádřit $K(t)$ pomocí delta funkce. Pro zjednodušení předpokládejme, že máme pouze jednu platbu $K_c$ a čase $t_c < T$. Diferenciální rovnice (17.1) tak přejde do tvaru
\begin{equation}
\frac{dV}{dt} + K_c \delta(t - t_c) = r(t)V(t)
\end{equation}
a rovnice (17.2) do tvaru
\begin{equation*}
V(t) = e^{\int_t^T}r(\tau) d \tau \Bigg(Z + K_c \mathcal{H}(t_c - t)e^{\int_{t_c}^T r(\tau) d \tau}\Bigg)
\end{equation*}

\section{Výnosová křivka}

Výnosová křivka představuje způsob vyjádření budoucích úrokových sazeb a je ji možné zkonstruovat z dostupných cen diskontních dluhopisů jako
\begin{equation}
Y(t, T) = - \frac{\ln \frac{V(t, T)}{Z}}{T-t}
\end{equation}
Výnosová křivka je grafem, ve kterém je vynášena úroková sazba $Y$ proti zbytkové splatnosti $T - t$ diskontního dluhopisu, ze kterého byla sazba odvozena. Z pohledu časového profilu rozlišujeme
\begin{itemize}
\item rostoucí křivku - Jedná se o nejběžnější profil křivky. Dlouhodobé úrokové sazby jsou vyšší než krátkodobé.
\item klesající křivka - Jedná se o situaci, kdy je krátkodobá úroková sazba vysoká, nicméně očekává se její pokles.
\item prohnutá křivka - Úroková sazba nejprve roste a následně klesá.
\end{itemize}
\begin{center}
	\begin{pspicture}(0,0)(9.0,6.0)
		\rput(4.5,0){Profily výnosových křivek: (a) rostoucí, (b) klesající a (c) prohnutá křivka}

		\psline[arrows=->](0.5,0.5)(7.5,0.5)
		\psline[arrows=->](0.5,0.5)(0.5,5.5)

                \rput(8.1,0.5){\small{$T - t$}}
                \rput(0.5,5.8){\small{$Y$}}

		\pscurve[linewidth=0.5mm](0.5,3.0)(3.0,4.0)(6.7,4.6)
		\pscurve[linewidth=0.5mm](0.5,3.0)(3.0,2.0)(6.7,1.5)
		\pscurve[linewidth=0.5mm](0.5,3.0)(2.5,4.3)(5.0,3.3)(6.7,2.8)

		\rput(7.0,5.0){(a)}
		\rput(7.0,1.5){(b)}
		\rput(7.0,2.8){(c)}
	\end{pspicture}
\end{center}
Definice $Y$ dle (17.7) má oproti (17.5) výhodu v tom, že hodnota dluhopisu $V(t,T)$ nemusí být diferenciovatelná a a že nejsou vyžadovány ceny dluhopisů pro všechny doby splatnosti. Obě vyjádření budoucích úrokových sazeb, tj. rovnice (17.5) a (17.7), jsou totožné jsou-li úrokové sazby konstantní.

\section{Stochatická úroková sazba}

V následujícím textu opustíme předpoklad, že úrokové sazby jsou deterministické. Nechť $r$ je náhodná veličina, která představuje úrokovou sazbu, za kterou je možné uložit depozitum na nejkratší možnou dobu. Tato sazba se nazývá spotovou sazbu.

Podobně jako v případě akcií budeme modelovat $r$ pomocí náhodné procházky.
\begin{equation}
dr = w(r,t)dX + u(r,t)dt
\end{equation}
Funkce $w(r,t)$ a $u(r,t)$ ovlivňují chování spotové sazby $r$. Podobně jako v případě opcí použijeme (17.8) pro odvození pariciální diferenciální rovnice pro stanovení hodnoty dluhopisu.

\section{Diferenciální rovnice pro ocenění dluhopisu}

Předpokládejme, že spotová sazba $r$ je popsána stochastickou diferenciální rovnicí (17.8). Ocenění dluhopisu je v tomto případě v porovnání s opcí složitější, protože neexistuje podkladové aktivum, které by bylo možné použít pro zajištění. Uvažujme portfolio, které se skládá z dlouhé pozice v jednom dluhopisu se splatností v čase $T_1$ a krátké pozice v $\Delta$ jednotek dluhopisů se splatností v čase $T_2$. Hodnotu prvního dluhopisu označme jako $V_1$ a hodnotu druhého jako $V_2$.
\begin{equation}
\Pi = V_1 - \Delta V_2
\end{equation}
Změna hodnoty tohoto portfolia v čase $dt$ lze s využitím It\^o lemmy a vztahu $dr^2 \approx w^2 DX^2 \approx w^2 dt$, který je platný pro $dt \rightarrow \infty$, vyjádřit jako
\begin{equation}
d \Pi = \frac{\partial V_1}{\partial t}dt + \frac{\partial V_1}{\partial r}dr + \frac{1}{2}w^2\frac{\partial^2 V_2}{\partial \tau^2}dt - \Delta \Bigg( \frac{\partial V_2}{\partial t}dt + \frac{\partial V_2}{\partial r}dr + \frac{1}{2}w^2 \frac{\partial^2 V_2}{\partial r^2} dt \Bigg)
\end{equation}
Z rovnice (17.10) je patrné, že volbou
\begin{equation*}
\Delta = \frac{\frac{\partial V_1}{\partial r}}{\frac{\partial V_2}{\partial r}}
\end{equation*}
eliminujeme náhodnou složku v diferenciální rovnici (17.10).
\begin{equation*}
d \Pi = \Bigg( \frac{\partial V_1}{\partial t} + \frac{1}{2}w^2\frac{\partial^2 V_1}{\partial r^2} - \frac{\frac{\partial V_1}{\partial r}}{\frac{\partial V_2}{\partial r}}\Bigg( \frac{\partial V_2}{\partial t} + \frac{1}{2}w^2\frac{\partial^2 V_2}{\partial r^2}\Bigg) \Bigg)dt
\end{equation*}
Za předpokladu neexistence arbitráže platí
\begin{equation*}
d \Pi = r \Pi dt
\end{equation*}
\begin{equation*}
\Bigg( \frac{\partial V_1}{\partial t} + \frac{1}{2}w^2\frac{\partial^2 V_1}{\partial r^2} - \frac{\frac{\partial V_1}{\partial r}}{\frac{\partial V_2}{\partial r}}\Bigg( \frac{\partial V_2}{\partial t} + \frac{1}{2}w^2\frac{\partial^2 V_2}{\partial r^2}\Bigg) \Bigg)dt = r (V_1 - \frac{\frac{\partial V_1}{\partial r}}{\frac{\partial V_2}{\partial r}} v_2) dt
\end{equation*}
Následnými úpravami, jejichž cílem je přesunout všechna $V_1$ na levou a všechna $V_2$ na pravou stranu, získáme rovnici o dvou neznámých.
\begin{equation*}
\frac{\frac{\partial V_1}{\partial t} + \frac{1}{2}w^2\frac{\partial^2 V_1}{\partial r^2} - r V_1}{\frac{\partial V_1}{\partial r}} = \frac{\frac{\partial V_2}{\partial t} + \frac{1}{2}w^2\frac{\partial^2 V_2}{\partial r^2} - r V_2}{\frac{\partial V_2}{\partial r}}
\end{equation*}
Dále je třeba si uvědomit, že levá strana rovnice je funkcí $T_1$ a pravá funkcí $T_2$. Pravou popř. levou stranu rovnice lze nahradit obecnou funkcí $a(r,t)$, kterou je s ohledem na následné úpravy vhodné vyjádřit ve tvaru $a(r,t) = w(r,t)\lambda(r,t) - u(r,t)$.
\begin{equation*}
\frac{\frac{\partial V}{\partial t} + \frac{1}{2}w^2\frac{\partial^2 V}{\partial r^2} - r V}{\frac{\partial V}{\partial r}} = a(r,t)
\end{equation*}
Rovnice pro ocenění diskontního dluhopisu tak má tvar
\begin{equation}
\frac{\partial V}{\partial t} + \frac{1}{2}w^2\frac{\partial^2 V}{\partial r^2} + (u - \lambda w)\frac{\partial V}{\partial r} - rV = 0
\end{equation}
Aby rovnice (17.11) měla jednoznačné řešení, je třeba ji doplnit o konečnou podmínku, která odpovídá výplatě generovanou dluhopisem v době jeho splatnosti.
\begin{equation*}
V(r,T) = Z
\end{equation*}
Hraniční podmínky se odvíjí od tvaru funkcí $u(r,t)$ a $w(r,t)$ a budou diskutovány v návaznosti na příslušné modely úrokových sazeb.

V případě kupónových dluhopisů se rovnice (17.11) modifikuje do tvaru
\begin{equation*}
\frac{\partial V}{\partial t} + \frac{1}{2}w^2\frac{\partial^2 V}{\partial r^2} + (u - \lambda w)\frac{\partial V}{\partial r} - rV + K= 0
\end{equation*}
kde $K$ představuje kupónovou platbu, která může být funkcí $r$ a $t$. Je-li kupón vyplácen diskrétně, je možné $K(t)$ vyjádřit jako sumu delta funkcí. Navíc musí hodnota dluhopisu $V(r,t)$ splňovat skokovou podmínku
\begin{equation*}
V(r, t_c^{-}) = V(r, t_c^{+}) + K_c
\end{equation*}
kde $K_c$ představuje kupón vyplacený v čase $t_c$.

\subsection{Tržní cena rizika}

Předpokládejme, že namísto dosud uvažovaného zajištěného portfolia, držíme pouze jeden dluhopis s datem splatnosti $T$. Změnu hodnoty tohoto dluhopisu v časovém intevalu délky $dt$ lze vyjádřit jako
\begin{equation*}
dV = w \frac{\partial V}{\partial r}dX + \Bigg( \frac{\partial V}{\partial t} + \frac{1}{2}w^2\frac{\partial^2 V}{\partial r^2} + u \frac{\partial V}{\partial r} \Bigg)dt
\end{equation*}
S využitím (17.11) lze tuto rovnici přepsat do tvaru
\begin{equation*}
dV = w \frac{\partial V}{\partial r}dX + \Bigg( w \lambda \frac{\partial V}{\partial r} + rV \Bigg)dt
\end{equation*}
\begin{equation}
dV - rVdt = w \frac{\partial V}{\partial r}(dX + \lambda dt)
\end{equation}
Přítomnost náhodného členu $dX$ v (17.12) naznačuje, že se nejedná o bezrizikové portfolio. Pravá strana rovnice může být chápána jako kompenzace podstoupení určité úrovně rizika, kdy portfolio generuje dodatečný výnos ve výši $\lambda dt$ na každou jednotku rizika $dX$. Z tohoto důvodu se často o funkci $\lambda$ hovoří jako o tržní ceně rizika.

\subsubsection{Technická poznámka: Tržní cena rizika podkladového aktiva}

V kapitole 3 jsme sestavili bezrizikové portfolio z dlouhé pozice je jedné opci a krátké pozice v $-\Delta$ jednotek podkladového aktiva. Nyní předpokládejme, podobně jako v případě dluhopisů, sestavení tohoto porfolia ze dvou různých opcí, které mají shodné podkladové aktivum, ale liší se zbytkovou splatnosti popř. realizační cenou.
\begin{equation*}
\Pi = V_1 - \Delta V_2
\end{equation*}
Po analogických úpravách jako v předchozím příkladě získáme
\begin{equation}
\frac{\partial V}{\partial t} + \frac{1}{2} \sigma^2 S^2 \frac{\partial^2 V}{\partial S^2} + (\mu - \lambda_s \sigma) S \frac{\partial V}{\partial S} - rV = 0
\end{equation}
Tato rovnice se shoduje s (17.11) s $S$ namísto $r$, $\mu S$ namísto $u$, $\lambda_S$ namísto $\lambda$ a $\sigma S$ namísto $w$. Připomeňme, že zajištění opcí je jednodušší než zajištění dluhopisů a to z důvodu existence podkladového aktiva. To znamená, že $V = S$ musí být řešením rovnice (17.13). Dosazením $V = S$ do (17.13) odvodíme tržní cenu rizika podkladového aktiva.
\begin{equation*}
(\mu - \lambda_S \sigma)S - rS = 0
\end{equation*}
\begin{equation*}
\lambda_S = \frac{\mu - r}{\sigma}
\end{equation*}
Zpětným dosazením $\lambda_S = \frac{\mu - r}{\sigma}$ do (17.13) odvodíme
\begin{equation*}
\frac{\partial V}{\partial t} + \frac{1}{2} \sigma^2 S^2 \frac{\partial^2 V}{\partial S^2} + rS \frac{\partial V}{\partial S} - rV = 0
\end{equation*}
což je základní Black-Scholes rovnice bez odkazu na $\mu$ a $\lambda_S$.

\section{Řešení rovnice pro oceňování dluhopisu}

Předpokládejme, že koeficienty $u$ a $w$ v rovnici (17.8) mají podobu
\begin{equation}
w(r,t) = \sqrt{\alpha(t)r - \beta(t)}
\end{equation}
a
\begin{equation}
w(r,t) = - \gamma(t)r + \eta(t) + \lambda(r,t)\sqrt{\alpha(t)r - \beta(t)}
\end{equation}
Funkce času $\alpha$, $\beta$, $\gamma$, $\eta$ a $\lambda$ slouží k tomu, aby výnosová křivka co nejlépe odpovídala historickým datům popř. zvolenému profilu křivky. Vhodnou volbou těchto funkcí lze docílit, aby výnosová křivka splňovala některé ekonomicky opodstatněné podmínky.
\begin{itemize}
\item V případě, že $\alpha(t) > 0$ a $\beta(t) \ge 0$, lze nastavit model tak, aby $\frac{\beta}{\alpha}$ představovalo dolní hranici pro spotovou úrokovu míru. Jestliže se tedy spotová úroková míra dotkne hranice $\frac{\beta}{\alpha}$, musí bezprostředně po té vzrůst. Toho lze docílit podmínkou
\begin{equation*}
\eta(t) \ge \frac{\beta(t) \gamma(t)}{\alpha(t)} + \frac{\alpha(t)}{2}
\end{equation*}
Tímto lze mimojiné docílit neexistence záporných úrokových sazeb. Úroková míra $r$ stále může pro výše uvedené podmínky, i když s nulovou pravděpodobností, konvergovat k nekonečnu.
\item Úroková sazba může mít tendenci se vracet k určité průměrné hodnotě, která může být sama o sobě funkcí času (tzv. mean-reverting).
Dále si všimněme, že vzhledem ke zvolené formě (17.14) a (17.15) nefiguruje $\lambda(r,t)$ v diferenciální rovnici (17.11), kterou se řídí hodnota dluhopisu.
\end{itemize}

Diferenciální rovnice (17.8) představuje obecnou formu, na kterou lze aplikovat řadu konkrétních modelů. Mezi nejznámnější modely patří
\begin{itemize}
\item Vašíčkův model - $\alpha = 0$; ostatní parametry mají povahu konstant, tj. nejsou funkcí času
\item Cox, Ingersoll a Ross model - $\beta = 0$; ostatní parametry mají povahu konstant, tj. nejsou funkcí času
\item Hull a White model - $\alpha = 0$ nebo $\beta = 0$, ostatní parametry jsou funkcí času
\end{itemize}

V případě rovnic (17.14) a (17.15) mají hraniční podmínky pro (17.11) podobu
\begin{equation*}
V(r,t) \rightarrow 0, ~~~ r \rightarrow \infty
\end{equation*}
a že $V$ je konečné pro $r = \frac{\beta}{\alpha}$\footnote{Je-li $r$ zdola ohraničené $\frac{\beta}{\alpha}$, je možné provést analýzu particiální diferenciální rovnice (17.11) v okolí této hranice. Dáme-li do rovnosti $\frac{1}{2}(\alpha r - \beta) \frac{\partial^2 V}{\partial r^2}$ a $(\eta - \gamma r)\frac{\partial V}{\partial r}$, zjistíme, že $V$ je konečné pouze za předpokladu $\eta \ge \frac{\beta \gamma}{\alpha} - \frac{\alpha}{2}$.}.

Podobu parametrů $u$ a $w$ jsme zvolili s ohledem na výše uvažované modely úrokových sazeb. Tyto modely předpokládají speciální funkcionální formy pro koeficienty parametrů $dt$ a $dX$ v stochastické diferenciální rovnici pro $r$. Řešení rovnice (17.11) tak má podobu
\begin{equation}
V(r,t) = Ze^{A(t,T) - rB(t,T)}
\end{equation}
Lze dokázat, že model s nenulovými funkcemi $\alpha$, $\beta$, $\gamma$ a $\eta$ je nejvíce možnou obecnou formou diferenciální rovnice pro $r$, která vede k řešení rovnice (17.11) ve tvaru (17.16). Dosazením (17.16) do (17.11) získáme
\begin{equation}
\frac{\partial A}{\partial t} - r \frac{\partial B}{\partial t} + \frac{1}{2}w^2B^2 - (u - \lambda w)B - r = 0
\end{equation}
Některé z parametrů jsou funkcemi $t$ a $T$ (např. $A$ a $B$) a některé jsou funkcemi $r$ a $t$ (např. $u$, $w$). Derivací (17.17) podle $r$ získáme
\begin{equation*}
- \frac{\partial B}{\partial t} + \frac{1}{2}B^2\frac{\partial w^2}{\partial r} - B\frac{\partial(u - \lambda w)}{\partial r} = 0
\end{equation*}
Další derivací podle $r$ a následným dělení $B$ získáme
\begin{equation*}
\frac{1}{2}B\frac{\partial^2 w^2}{\partial r^2} - \frac{\partial^2(u - \lambda w)}{\partial r} = 0
\end{equation*}
Protože $B$ je funkcí $T$, musí, aby splněna výše uvedená rovnice, platit
\begin{equation}
\frac{\partial^2 w^2}{\partial r^2} = 0
\end{equation}
a
\begin{equation}
\frac{\partial^2(u - \lambda w)}{\partial r^2} = 0
\end{equation}
Lze dokázat, že substitujeme-li (17.14) a (17.15) do (17.18) a (17.19) a dáme-li do rovnosti členy se shodným řádem $r$, získáme následující diferenciální rovnice pro $A$ a $B$.
\begin{equation}
\frac{\partial A}{\partial t} = \eta(t)B + \frac{1}{2}\beta(t)B^2
\end{equation}
\begin{equation}
\frac{\partial B}{\partial r} = \frac{1}{2} \alpha(t)B^2 + \gamma(t)B - 1
\end{equation}
Má-li být splněna konečná podmínka $V(r,T) = Z$, musí platit
\begin{equation*}
A(t,T) = 0, ~~~ B(t,T) = 0
\end{equation*}

\subsection{Analýza pro konstantní parametry}

Řešení pro $\alpha$, $\beta$, $\gamma$ a $\eta$ lze získat integrování diferenciálních rovnic (17.20) a (17.21). Obecné řešení však není možné odvodit explicitně. Uvažujme proto nejjednodušší situaci, kdy jsou $\alpha$, $\beta$, $\gamma$ a $\eta$ konstantní. Lze odvodit
\begin{equation}
\frac{2}{\alpha}A = \alpha \psi_2 \ln(\alpha - B) + \Bigg(\psi_2 - \frac{1}{2}\beta \Bigg) b \ln \Bigg(\frac{B + b}{b}\Bigg) + \frac{1}{2}B\beta - a \psi_2 \ln a 
\end{equation}
a
\begin{equation}
B(t,T) = \frac{2(e^{\psi_1(T - t)} - 1)}{(\gamma + \psi_1)(e^{\psi_1(T-t)} - 1) + 2 \psi_1}
\end{equation}
kde
\begin{equation*}
b,a = \frac{\pm \gamma + \sqrt{\gamma^2 + 2 \alpha}}{\alpha}
\end{equation*}
a
\begin{equation*}
\psi_1 = \sqrt{\gamma^2 + 2 \alpha}, ~~~ \psi_2 = \frac{\eta + \frac{\alpha \beta}{2}}{a + b}
\end{equation*}
Parametry $A$ a $B$ jsou funkcí $\tau = T - t$. Pomocí tohoto modelu lze predikovat širokou škálu profilů výnosové křivky. Pro $\tau \rightarrow \infty$ platí
\begin{equation*}
B \rightarrow \frac{2}{\gamma + \psi_1}
\end{equation*}
a chování výnosové křivky $Y$ v dlouhém časovém horizontu je dáno
\begin{equation*}
Y \rightarrow \frac{2}{(\gamma + \psi_1)^2}(\eta(\gamma + \psi_1) + \beta)
\end{equation*}
Pro fixní $\alpha$, $\beta$, $\gamma$ a $\eta$ tak model implikuje v dlouhém časovém horizontu fixní úrokovou sazbu, které je nezávislá na spotové úrokové sazby.

\subsection{Kalibrace parametrů}

V následujícím textu budeme předpokládat, že parametry $\alpha$, $\beta$ a $\gamma$ jsou konstanty a parametr $\eta$ je funkcí času. Tato volba je dostatečná pro to, abychom vytvořili model výnosové křivky, který nakalibrujeme na libovolná historická data.

\subsubsection{Dolní limit pro spotovou úrokovou sazbu}

Předokládejme, že máme představu o dolní hranici pro spotovou úrokovou sazbu $r$. Tuto hranici je možné stanovit expertně popř. podle minimální historické úrokové sazby pro vhodně zvolené časové okno\footnote{Standardně se délka časové okna volí tak, aby se shodovala se zbytkovou dobou splatnosti oceňovaného dluhopisu.}. Jak již bylo řečeno výše, odpovídá tato hranice $\frac{\beta}{\alpha}$.

\subsubsection{Volatilita spotové úrokové sazby}

Volatilita spotové úrokové sazby je dána vztahem
\begin{equation*}
\sqrt{\alpha r - \beta}
\end{equation*}
Za předpokladu konstantní volatility spotové úrokové ji lze odhadnout na základě historických dat\footnote{Podobně jako v předchozím případě, i zde se nejčastěji délka zvoleného časového okna shoduje se zbytkovou splatností uvažovaného dluhopisu.}.

Tímto způsobem jsme odvodili dvě rovnice o dvou neznámých, což je postačující k odvození hodnot $\alpha$ a $\beta$.

\subsubsection{Volatilita sklonu výnosové křivky}

Rovnice (17.20) a (17.21) pomocí Taylorovy řady pro $t$ blízké $T$. Aplikací tohoto postupu zjistíme, že výnosová křivka, která má nyní tvar
\begin{equation*}
Y = \frac{-A + rB}{T - t}
\end{equation*}
může být čase v okolí splatnosti aproximována pomocí
\begin{equation*}
Y \sim r - \frac{1}{2}(T - t)(\gamma r - \eta(0)) + ...
\end{equation*}
Z toho je patrné, že sklon krátkého konce výnosové křivky (tj. pro $T = t$) je dán
\begin{equation}
s = \frac{1}{2}(\eta(0) - \gamma r)
\end{equation}
Sklon křivky tak závisí na aktuální spotové úrokové sazbě a pro $\gamma > 0$ implikuje tendenci úrokové míry vracet se k průměrné úrokové sazbě\footnote{Růst spotové úrokové sazby má za následek pokles směrnice výnosové křivky, která se tak stává plošší.}. Vzhledem k tomu, že spotová úroková míra $r$ sleduje náhodnou procházku, má také $s$ charakter náhodné veličiny. Protože $r$ a $s$ jsou provázány skrze (17.23), jsou perfektně korelovány
\begin{equation*}
ds = -\frac{1}{2}\gamma dr
\end{equation*}
platí
\begin{equation*}
\gamma = -\frac{2 cov(dr, ds)}{\sigma_{dr}^2}
\end{equation*}
kde $cov(dr, ds)$ resp. $\sigma_{dr}^2$ představují korelaci mezi $dr$ a $ds$ resp. rozptyl $dr$ vypočtených na základě historických dat. V praxi se může stát, že hodnota $\gamma$ vyjde záporná - náhodné procházky, které sledují $dr$ a $ds$ jsou tak pozitivně korelovány. Jestliže tedy poklesne spotová úroková míra, vzroste sklon výnosové křivky, což je indikace toho, že spotová úroková míra nemá tendenci vracet se k dlouhodobému průměru.

\subsection{Kalibrace celkové výnosové křivky}

V předchozích krocích jsme vypočetli hodnoty parametrů $\alpha$, $\beta$ a $\gamma$. Nyní zbývá zvolit $\eta(t)$ tak, aby modelovaná výnosová křivka odpovídala aktuálním tržím datům. To vede k integrální rovnici pro $\eta(t)$, která musí být, až na nejjednodušší případy, řešena numericky.

Integrováním (17.20) získáme
\begin{equation}
A = - \frac{1}{2}\beta \int_t^T B^2 (T - s) ds - \int_t^T \eta(s)B(T-s)ds
\end{equation}
kde $B(T-t)$ je dáno rovnicí (17.23) a jedná se o funkci jediné proměnné $T-t$. Výraz (17.25) je tak znám s vyjímkou posledního integrálního členu obsahujícího $\eta(t)$.

Předpokládejme, že budeme chtít kalibrovat výnosovou křivku pouze jednou a to v čase $t^{*}$, ke kterému jsou nám známy spotová úroková míra $r$, výnosová křivka $Y^{*}(T)$ a konstanty $\alpha^{*}$, $\beta^{*}$ a $\gamma^{*}$. Substitucí výnosové křivky
\begin{equation*}
Y = \frac{-A + rB}{T - t}
\end{equation*}   
do rovnice (17.25) získáme integrální rovnici pro $\eta^{*}(t)$.
\begin{equation}
\int_{t^{*}}^T \eta^{*}(s)B(T-s)ds = Br^{*} - Y^{*}(T-t^{*}) - \frac{1}{2} \beta^{*} \int_{t^{*}}^T B^2(T - s)ds
\end{equation}
Tuto rovnici je třeba řešit pro $t^{*} \le T < \infty$. V okamžiku, kdy je nalezeno řešení $\eta^{*}(t)$, je možné dosadit $\alpha^{*}$, $\beta^{*}$, $\gamma^{*}$ a $\eta^{*}$ do (17.23), čímž získáme $B$. Následným dosazením $B$ do (17.25) získáme $A$. Hodnota libovolného dluhopisu je pak rovna
\begin{equation*}
Ze^{A(t,T) - rB(t,T)}
\end{equation*}
Výše odvozený model je platný pouze za předpokladu, že při přepočítání modelu výnosové křivky pro pozdější datum zůstanou parametry $\alpha$, $\beta$, $\gamma$ a $\eta(t)$ nezměněny. V praxi je však tento předpoklad velice těžce obhajitelný, což je důsledkem toho, že (17.8) bylo zvoleno pro své analytické vlastnosti a nikoliv s ohledem na následné ekonomické modelování. Tento nedostatek je slabým místem většiny dnes populárních modelů výnosových křivek.

\section{Hull a White rozšíření Vašíčkova modelu}

Hull a White rozšířili původní Vašíčkův model. V rámci jejich modelu je $\alpha(t) = 0$ a $\beta < 0$. Ačkoliv Hull a White obhajovali parametry $\beta(t)$, $\gamma(t)$ a $\eta(t)$ jako funkce času s cílem nakalibrovat model a volatilitu pro všechny splatnosti v rámci uvažované výnosové křivky\footnote{Až dosud jsme uvažovali pouze volatilitu spotové úrokové míry.}, budeme stejně jako v předchozí kapitole předpokládat, že pouze $\eta(t)$ je funkcí času.

V rámci tohoto modelu budeme předpokládat, že $\alpha = 0$ a že $\beta^{*}$ a $\gamma^{*}$ byly vypočteny v čase $t^{*}$. V tomto případě se $B(T - t)$ zjednodušší na
\begin{equation*}
B(T - t) = \frac{1}{\gamma^{*}} \Bigg( 1 - e^{-\gamma^{*}(T - t)}\Bigg)
\end{equation*}
a integrální rovnice pro $\eta^{*}$ tak přejde do tvaru
\begin{equation}
\int_{t^{*}}^T \eta^{*}(s) \Bigg( 1 - e^{-\gamma^{*}(T - s)}\Bigg)ds = \gamma^{*}F^{*}(T)
\end{equation}
kde $F^{*}$ je známá funkce $T$, která je dána pravou stranou rovnice (17.26) a která je závislá na integrálech $B$ a aktuální výnosové křivce. Aby tato integrální rovnice měla řešení, musí platit $F(0) = 0$, což je patrné z pravé strany rovnice (17.26).

Rovnici (17.27) lze řešit dvojitým derivováním podle $T$. Po první derivaci získáme
\begin{equation*}
\int_{t^{*}}^T \eta^{*}(s) e^{-\gamma^{*}(T-s)}ds = F^{'*}(T)
\end{equation*}
Po druhé derivaci získáme
\begin{equation*}
\eta^{*}(T) - \gamma^{*} \int_{t^{*}}^T \eta^{*}(s)e^{-\gamma^{*}(T-s)}ds = F^{''*}(T)
\end{equation*}
Pomocí rovnic, které jsme odvodili první a druhou derivací, se lze zbavit integrálů a získat tak diferenciální rovnici druhého řádu pro $\eta^{*}$.
\begin{equation*}
\eta^{*}(T) = F^{''*}(T) + \gamma^{*}F^{'*}(T)
\end{equation*}
Hodnota $\eta^{*}$ je tak dána vztahem
\begin{equation}
\eta^{*}(T) = -Y^{''*}(T) - \gamma^{*}Y^{'*}(T) - \beta^{*}(T - t^{*}) - \frac{\beta^{*}}{2 \gamma^{*}} \Bigg( 1 - e^{-2 \gamma^{*}(T - t^{*})}\Bigg)
\end{equation}
Pomocí (17.28) a (17.20) je možné vypočíst $A(t,T)$.

\section{Dluhopisová opce}

Dluhopisová opce představuje analogii akciové opce s tím rozdílem, že podkladovým aktivem je dluhopis. Uvažujme evropskou verzi kupní dluhopisové opce s realizační cenou $E$ a datem splatnosti $T$ na diskontní dluhopis se splatností $T_B \ge T$. Před oceněním opce je nutné nejprve ocenit dluhopis. Nechť $V_B(r,t,T_B)$ představuje hodnotu uvažovaného dluhopisu. Platí
\begin{equation}
\frac{\partial V_B}{\partial t} + \frac{1}{2}w^2 \frac{\partial^2 V_B}{\partial r^2} + (u - \lambda w) \frac{\partial V_B}{\partial r} - r V_B = 0
\end{equation}
při splnění vhodných hraničních podmínek a konečné podmínky
\begin{equation*}
V_B(r, T_B, T_B) = Z
\end{equation*}
Dále označme hodnotu uvažované kupní opce jako $C_B(r,t)$. Protože $C_B$ je také funkcí náhodné veličiny $r$, musí také splňovat rovnici (17.29) s tím rozdílem, že konečná podmínka má podobu
\begin{equation*}
C_B(r,T) = \max(V_B(r,T,T_B) - E, 0)
\end{equation*}

\section{Ostatní úrokové deriváty}

V této kapitole načrtneme možnost ocenění základních typů tzv. úrokových derivátů. V případě ostatních úrokových derivátů je postup analogický.

\subsection{Úrokové swapy}

Předpokládejme, že v rámci sjednaného úrokového swapu má strana A zaplatit straně B fixní úrok ve výši $r^{*}$ z částky $Z$. Strana B pak z částky $Z$ hradí straně A plovoucí úrokovou sazbu $r$. Platy periodicky probíhají až do času $T$. Hodnotu tohoto úrokového swapu z pohledu A označme jako $ZV(r,t)$.

Pro ocenění produktu je důležité si uvědomit, že strana A obdrží v časovém kroce $dt$ částku $(r - r^{*})Z dt$. Jestliže budeme na tuto skutečnost nahlížet jako na kupónovou platbu generovanou dluhopisem, získáme diferenciální rovnici
\begin{equation*}
\frac{\partial V}{\partial t} + \frac{1}{2}w^2 \frac{\partial^2 V}{\partial r^2} + (u - \lambda w) \frac{\partial V}{\partial r} - rV + (r - r^{*}) = 0
\end{equation*}
a konečnou podmínku
\begin{equation*}
V(r,T) = 0
\end{equation*}
Vzhledem k tomu, že může platit $r > r^{*}$, může být také hodnota $V(r,t)$ záporná. Úrokový swap je tak z pohledu zúčastněné strany závazkem.

\subsection{Cap a floor}

Cap lze chápat jako půjčku za plovoucí úrokovou sazbu s garancí, že tato sazba nepřesáhne stanovenou hranici $r^{*}$. Nominál půjčky $Z$ je splacen v čase $T$. Hodnota cap je tak $ZV(r,t)$, přičemž $V(r,t)$ musí splňovat diferenciální rovnici
\begin{equation}
\frac{\partial V}{\partial t} + \frac{1}{2}w^2\frac{\partial^2 V}{\partial r^2} + (u - \lambda w) \frac{\partial V}{\partial r} - rV + \min(r, r^{*}) = 0
\end{equation}
a konečnou podmínku
\begin{equation*}
V(r,T) = 1
\end{equation*}
Floor je podobný jako cap s tím rozdílem, že se jedná o vklad úročený pohyblivou úrokovou mírou s garancí, že tato úroková míra neklesne pod $r^{*}$. V rovnici (17.30) tak stačí nahradit $\min(r, r^{*})$ členem $\max(r, r^{*})$.

\subsection{Swapce}

Předpokládejme, že úrokový swap se splatností $T_S$ má v čase $t \le T_S$ hodnotu $V_S(r,t)$. Hodnota opce, která umožňuje koupi tohoto úrokového swapu v čase $T$ za realizační cenu $E$, musí splňovat diferenční rovnici
\begin{equation*}
\frac{\partial V}{\partial t} + \frac{1}{2}w^2 \frac{\partial^2 V}{\partial r^2} + (u - \lambda w) \frac{\partial V}{\partial r} - rV = 0
\end{equation*}
a konečnou podmínku
\begin{equation*}
V(r,T) = \max(V_S(r,T) - E, 0)
\end{equation*}
Nejprve je tedy třeba vypočíst hodnotu podkladového úrokové swapu a následně použít při formulaci konečné podmínky při ocenění swapce.

\chapter{Konvertibilní dluhopisy}

\section{Úvod}

Konvertibilní dluhopis je dluhopis, který může jeho matitel v průběhu životnosti vyměnit za akcie jeho emitenta. Dluhopis tak v době své splatnosti generuje částku odpovídající nominální hodnotě, pokud však nebyl v průběhu své životnosti vyměněn za akcie. Dluhopis může svému majiteli vyplácet kupón a akcie může generovat dividendový výnos. V následujícím textu budeme předpokládat, že počet takto získaných akcií je malý a neovlivní tak hodnotu emitující společnosti.

\section{Deterministické úrokové sazby}

Uvažujme konvertibilní dluhopis, který svému majiteli v době splatnosti $T$ generuje částku $Z$ za předpokladu, že nebyl konvertován na $n$ akcií emitenta. Dále předpokládejme, že úrokové sazby jsou deterministické a že dluhopis generuje kupónové platby. Protože cena konvertibilního dluhopisu se odvíjí od ceny akcií emitenta, platí
\begin{equation*}
V = V(S,t)
\end{equation*}
Hodnota dluhopisu je také funkcí jeho splatnosti - od této skutečnosti však prozatím odhlédneme. Použijeme-li Black-Scholes analýzu na portfolio sestávající se z jednoho konvertibilního dluhopisu a $-\Delta$ akcií, zjistíme, že se změna jeho hodnoty řídí diferenciální rovnicí
\begin{equation*}
d \Pi = \frac{\partial V}{\partial t}dt + \frac{\partial V}{\partial S}dS + \frac{1}{2} \sigma^2 S^2 \frac{\partial^2 V}{\partial S^2}dt - \Delta dS + K(S,t)dt
\end{equation*}
kde $K(S,t)$ představuje kupónovou platbu. Stejně jako v přechozích případech zvolíme
\begin{equation*}
\Delta = \frac{\partial V}{\partial S}
\end{equation*}
s cílem eliminovat náhodnou složku. Výnosová míra bezrizikového portfolia by pak neměla být vyšší než bezriziková výnosová míra.
\begin{equation*}
\frac{\partial V}{\partial t}dt + \frac{1}{2} \sigma^2 S^2 \frac{\partial^2 V}{\partial S^2}dt + (rS - D(S,t))\frac{\partial V}{\partial S} - rV + K(S,t)dt \le 0
\end{equation*}
Jedná se o standardní Black-Scholes nerovnost rozšířenou o kupónové platby. Konečná podmínka výše pro výše uvedenou diferenciální rovnici je
\begin{equation*}
V(S,T) = Z
\end{equation*}
S ohledem na to, že dluhopis může být v průběhu své životnosti vyměněn za $n$ akcií emitenta, musí být splněna také podmínka
\begin{equation*}
V \ge nS
\end{equation*}
Dále je třeba, aby $V$ a $\frac{\partial V}{\partial S}$ byly spojité. Problém konvertibilního dluhopisu je tak podobný problému plain-vanilla americké opce. Je nutné si uvědomit, že samotná konečná data nesplňují podmínky ocenění. Ačkoliv je totiž hodnota konvertibilního dluhopisu v době jeho splatnosti rovna $Z$, je jeho hodnota těsně před tímto časovým okamžikem rovna
\begin{equation*}
\max(nS, Z)
\end{equation*}
Hraniční podmínky jsou
\begin{equation*}
V(S,t) \sim nS, ~~~ S \rightarrow \infty
\end{equation*}
a
\begin{equation*}
V(0,t) = Ze^{-r(T-t)}
\end{equation*}
Výše formulovaný problém lze řešit numericky podobně jako problém americké opce.

Lze také dokázat, že růst $D$ popř. $K$ má za následek, že konverze dluhopisu je více popř. méně pravděpodobná. V případě $D = K = 0$ je podmínka $V \le nS$ aplikována pouze v době splatnosti a konvertibilní dluhopis je tak možné oceněnit jako kombinaci hotovosti a evropské kupní opce na akcii emitenta. Na následujících obrázcích uvažujeme konvertibilní dluhopis s $Z = 1$, $n = 1$, $r = 0.1$ a $\sigma = 0.25$ a $T = 1$. V obou případech se jedná o diskontní dluhopis (tj. bez kupónových plateb). V prvním případě je dividendový výnos z podkladové akcie nulový, v druhém případě je $D_0 = 0.05$.
\begin{center}
	\begin{pspicture}(0,0)(12,7)
		\rput(6.0,0.4){Hodnota konvertibilního dluhopisu za předpokladu konstantních úrokových sazeb}
          \rput(6.0,0.0){$Z = 1$, $n = 1$, $r = 0.1$, $\sigma = 0.25$, $T = 1$, $K = 0$ a (a) $D_0 = 0$ resp. (b) $D_0 = 0.05$}

		\psline[arrows=->](0.4,1.5)(5.5,1.5)
		\psline[arrows=->](0.5,1.4)(0.5,6.5)
		
		\rput(0.3,6.5){\small{$V$}}
		\rput(5.5,1.2){\small{$S$}}
		
		\rput(3.0,0.9){\small{(a)}}
		
		\rput(0.5,1.2){\tiny{0}}
		
		\psline(2.5,1.5)(2.5,1.4)
		\rput(2.5,1.2){\tiny{1}}
		
		\psline(4.5,1.5)(4.5,1.4)
		\rput(4.5,1.2){\tiny{2}}
		
		\rput(0.3,1.5){\tiny{0}}
		
		\psline(0.5,3.5)(0.4,3.5)
          \rput(0.3,3.5){\tiny{1}}
          
          \psline(0.5,5.5)(0.4,5.5)
          \rput(0.3,5.5){\tiny{2}}
		
		\psline(0.5,1.5)(5.5,6.5)
		
		\pscurve[linewidth=0.5mm](0.5,3.3)(2.3,3.5)(3.5,4.5)(5.5,6.5)
		
		\psline[arrows=->](6.4,1.5)(11.5,1.5)
		\psline[arrows=->](6.5,1.4)(6.5,6.5)
		
		\rput(6.3,6.5){\small{$V$}}
		\rput(11.5,1.2){\small{$S$}}
		
		\rput(9.0,0.9){\small{(b)}}
		
		\rput(6.5,1.2){\tiny{0}}
          
		\psline(8.5,1.5)(8.5,1.4)
		\rput(8.5,1.2){\tiny{1}}
          
          \psline(10.5,1.5)(10.5,1.4)
		\rput(10.5,1.2){\tiny{2}}
          
		\rput(6.3,1.5){\tiny{0}}
          
		\psline(6.5,3.5)(6.4,3.5)
          \rput(6.3,3.5){\tiny{1}}
          
          \psline(6.5,5.5)(6.4,5.5)
          \rput(6.3,5.5){\tiny{2}}                             	
		
		\pscurve[linewidth=0.5mm](6.5,3.3)(8.3,3.5)(9.5,4.5)(11.5,6.5)
		
		\psline(6.5,1.5)(11.5, 6.5)
		
		\rput(10.5,3.5){\small{volná hranice}}
		\psline[arrows=->](10.5,3.8)(10.0,4.7)
		
	\end{pspicture}
\end{center}
Narozdíl od prvního případu figuruje v problematice druhého případu volná hranice. To znamená, že je-li $S$ dostatečně vysoké, je výhodné zkonvertovat dluhopis na akcie.

V některých případech mohou být dluhopisy zkonvertovány pouze ve vybraných časových okamžicích. V tomto případě je omezení $V \le nS$ aplikováno pouze v časech, ke kterým lze konverzi provést. V ostatních případech má konvertibilní dluhopis charakter evropské opce.

\subsubsection{Svolatelný a vratný konvertibilní dluhopis}

Svolatelný a vratný konvertibilní dluhopis je modifikací standardního konvertibilního dluhopisu, který umožňuje emitentovi dluhohopis odkoupit za předem dohodnutou částku. Existence opčního práva na straně emitenta snižuje z pohledu vlastníka hodnotu dluhopisu. Jestliže má emitent kdykoliv právo dluhopis odkoupit za částku $M_1$, musí být vedle podmínky $V(S,t) \ge nS$ splněna také podmínka $V(S,t) \le M_1$. Stejně jako v případě standardního konvertibilního dluhopisu musí být $V$ a $\frac{\partial V}{\partial S}$ spojité.

Vratný dluhopis obsahuje opční právo, které opravňuje jeho majitele k prodeji dluhopisu emitentovi za předem stanovenou částku. Existence opčního práva ve prospěch majitele dluhopisu zvyšuje jeho hodnotu. Je-li částka odkupu rovna $M_2$, musí být splněna podmínka $V(S,t) \ge M_2$. Protože však musí být současně splněna podmínka $V(S,t) \ge nS$, má výsledná podmínka podobu $V(S,t) \ge \max(nS, M_2)$.

\section{Stochastické úrokové sazby}

Jestliže má úroková sazba charakter náhodné veličiny, má hodnota konvertibilního dluhopisu podobu
\begin{equation*}
V = V(S,r,t)
\end{equation*}
Dalším parametrem, který vstupuje do ocenění je doba splatnosti $T$, od které v tomto textu odhlédneme. Hodnota konvertibilního dluhopisu je tak funkcí dvou náhodných veličin $S$ a $r$. Přepokládejme, že hodnota podkladového aktiva se řídí standardním modelem
\begin{equation}
dS = \sigma S dX_1 + \mu S dt
\end{equation}
a úroková míra modelem
\begin{equation}
dr = w(r,t)dX_2 + u(r,t)dt
\end{equation}
Oba procesy obsahují dvě náhodné veličiny $dX_1$ a $dX_2$. Ty, ačkoliv jsou obě výsledkem realizace normovaného normálního rozdělení, nejsou jednou a toutéž náhodnou veličinou. Tyto náhodné veličiny však mohou být vzájemně korelované. Problém ocenění konvertibilního dluhopisu za předpokladu stochastických úrokových sazeb je tedy dvourozměrný.
\begin{equation*}
\varepsilon[dX_1, dX_2] = \rho dt, ~~~ -1 \le \rho(r, S, t) \le 1
\end{equation*}
Pro řešení tohoto problému je opět možné použít It\^o lemmu, kde pro náhodné veličiny $dX_1$ a $dX_2$ platí, že $dX_1^2 = dt$, $dX_2^2 = dt$ a $dX_1 dX_2 = \rho dt$. Aplikací Taylorova teorému na $V(S + dS, r + dr, dt + t)$ získáme
\begin{equation*}
dV = \frac{\partial V}{\partial t} dt + \frac{\partial V}{\partial S}dS + \frac{\partial V}{\partial r}dr + \frac{1}{2}\Bigg( \frac{\partial^2 V}{\partial S^2}dS^2 + 2\frac{\partial^2 V}{\partial S \partial r}dS dr \Bigg) + ...
\end{equation*}
Platí
\begin{equation*}
dS^2 = \sigma^2 S^2 dX_1^2 = \sigma^2 S^2 dt
\end{equation*}
\begin{equation*}
dr^2 = w^2 dX^2 = w^2 dt
\end{equation*}
\begin{equation*}
dSdr = \sigma S w d X_1 dX_2 = \rho \sigma S w dt
\end{equation*}
S pomocí těchto vztahů lze výše uvedenou rovnici upravit do podoby
\begin{equation}
dV = \frac{\partial V}{\partial t}dt + \frac{\partial V}{\partial S}dS + \frac{\partial V}{\partial r}dr + \frac{1}{2}\Bigg( \sigma^2 S^2 \frac{\partial^2 V}{\partial S^2} + 2 \rho \sigma S w \frac{\partial^2 V}{\partial S \partial r} + w^2 \frac{\partial^2 V}{\partial r^2} \Bigg) dt
\end{equation}

Nyní přistupme k ocenění konvertibilního dluhopisu. Uvažujme portfolio, které se skládá z jednoho konvertibilního dluhopisu se splatností $T_1$, $-\Delta_2$ dluhopisů se splatností $T_2$ a $-\Delta_1$ akcií. Platí tedy
\begin{equation*}
\Pi = V_1 - \Delta_2 V_2 - \Delta_1 S
\end{equation*}
Aby uvažované portfolio bylo bezrizikové, zvolíme
\begin{equation*}
\Delta_2 = \frac{\frac{\partial V_1}{\partial r}}{\frac{\partial V_2}{\partial r}}
\end{equation*}
a
\begin{equation*}
\Delta_1 = \frac{\partial V_1}{\partial S} - \Delta_2 \frac{\partial V_2}{\partial S}
\end{equation*}
Po seskupení členů obsahujících $T_1$ a $T_2$ a po odstanění dolních indexů získáme rovnici
\begin{equation*}
\frac{\partial V}{\partial t} + \frac{1}{2}\Bigg( \sigma^2 S^2 \frac{\partial^2 V}{\partial S^2} + 2 \rho \sigma S w \frac{\partial^2 V}{\partial S \partial r} + w^2 \frac{\partial^2 V}{\partial r^2} \Bigg) + rS \frac{\partial V}{\partial S} + (u - w \lambda)\frac{\partial V}{\partial r} - rV = 0
\end{equation*}
kde $\lambda(r,S,t)$ představuje tržní cenu rizika. Výše odvozená rovnice je rovnicí pro ocenění konvertibilního dluhopisu a zahrnuje v sobě standardní Black-Scholes problém, tj. $u = w = 0$, a jednoduchou formu problému, kdy $\frac{\partial}{\partial S} = 0$. Jestliže jsou z dluhopisu vypláceny kupónové platby a z akcie dividendy, změní se tato rovnice do podoby 
\begin{multline}
\frac{\partial V}{\partial t} + \frac{1}{2}\Bigg( \sigma^2 S^2 \frac{\partial^2 V}{\partial S^2} + 2 \rho \sigma S w \frac{\partial^2 V}{\partial S \partial r} + w^2 \frac{\partial^2 V}{\partial r^2} \Bigg) + \\
+ (rS - D) \frac{\partial V}{\partial S} + (u - w \lambda)\frac{\partial V}{\partial r} - rV + K = 0
\end{multline}

Hraniční podmínky z titulu americké povahy vnořené opce a konečná podmínka jsou shodné jako v předchozím případě. Protože je  problém definován jako dvourozměrný, musíme definovat hraniční podmínky v $(S,r)$ prostoru. Konkrétně se jedná o podmínky pro $V(0,r,t)$ a $V(\infty, r,t)$ definované pro všechna t, podmínku pro $V(S, \infty, t)$ definovanou pro všechna $S$ a $t$ a o podmínku pro dolní hranici úrokové sazby $r$ definovanou opět pro všechna $S$ a $t$. Některé z těchto podmínek jsou zřejmé, jiné jsou důsledkem požadavku na konečnost $V$.

Pro ilustranci uveďme hraniční podmínky pro standardní konvertibilní dluhopis. Tyto podmínky jsou pro stanoveny mezní hodnoty $S$ a $r$. V případě $S \rightarrow \infty$ a $r \rightarrow \infty$ je tvar hraničních podmínek zřejmý.
\begin{equation*}
V(S,r,t) \sim nS, ~~~ S \rightarrow \infty
\end{equation*}
\begin{equation*}
V(S,r,t) \rightarrow 0, ~~~ r \rightarrow \infty
\end{equation*}
Pro $V(0,r,t)$ je konkrétní podoba podmínky dána řešením diferenciální rovnice (18.4) v situaci s nulovou pravděpodobností konverze. Hraniční podmínka, která je aplikována pro dolní hranici úrokové sazby $r$, je pak totožná s omezením pro dolní hodnotu konvertibilního dluhopisu $V$.

\subsection{Technická poznámka: Emise nových akcií}

Až dosud jsme uvažovali, že emise konvertibilních dluhopisů neovlivní tržní hodnotu akcií společnosti. V praxi má však konverze dluhopisu na akcie za následek emisi nových akcií, což je v rozporu s dosavadním předpokladem, že se počet akcií nezmění. Jestliže je $S$ celkovou hodnotou aktiv společnosti bez závazků z emitovaných konvertibilních dluhopisů a $N$ počet akcií před konverzí, mají omezijící podmínky z titulu konverze dluhopisu podobu
\begin{equation}
V \ge \frac{nS}{n + N}
\end{equation}
\begin{equation}
V \le S
\end{equation}
Podmínka (18.5) stanovuje dolní hranici pro hodnotu dluhopisu v případě konverze. Podmínka (18.6) zase umožňuje společnosti vyhlásit bankrot v případě, že by se hodnota konvertibilního dluhopisu stala příliš vysokou. O $\frac{N}{n + N}$ hovoříme jako o tzv. faktoru ``zředění''.

Následující obrázek zachycuje hodnotu typického konvertibilního dluhopisu pro $Z = 1$, $r = 0.1$, $\sigma = 0.25$, $D_0 = 0.05$, $T = 1$ a $\frac{N}{n + N} = 0.5$.
\begin{center}
	\begin{pspicture}(0,0)(8,8)
		\rput(4.0,0.4){Hodnota konvertibilního dluhopisu versus aktiva společnosti}
          \rput(4.0,0.0){včetně efektu ``zředění'' z titulu nově emitovaných akcií}

		\psline[arrows=->](0.4,1.5)(7.5,1.5)
		\psline[arrows=->](0.5,1.4)(0.5,7.5)
		
		\rput(0.3,7.5){\small{$V$}}
		\rput(7.5,1.2){\small{$S$}}
		
		\psline(0.5,1.5)(5.5,6.5)
		\psline(0.5,1.5)(6.5,5.0)		
		\pscurve[linewidth=0.5mm](0.5,1.5)(2.7,3.5)(5.0,4.2)(6.5,5.0)
		

		\psline(2.7,1.5)(2.7,1.4)
		\psline(4.9,1.5)(4.9,1.4)
		\rput(0.5,1.2){\tiny{0}}
		\rput(2.7,1.2){\tiny{1}}
		\rput(4.9,1.2){\tiny{2}}
		
		\psline(0.5,3.7)(0.4,3.7)
		\psline(0.5,5.9)(0.4,5.9)
		\rput(0.3,1.5){\tiny{0}}
		\rput(0.3,3.7){\tiny{1}}
		\rput(0.3,5.9){\tiny{2}}
		
		\rput(6.0,3.7){\small{volná hranice}}
		\psline[arrows=->](6.0,4.0)(6.3,4.85)
	\end{pspicture}
\end{center}

\end{document}



