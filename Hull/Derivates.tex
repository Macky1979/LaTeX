\documentclass[a4paper]{book}
\usepackage[czech]{babel}
\usepackage[utf8]{inputenc}
\usepackage{pstricks}
\usepackage{amsmath}

\setlength{\unitlength}{1.0mm}

\begin{document}

\tableofcontents

\chapter{Přehled základních derivátových obchodů}

Derivátem rozumíme finanční instrument, jehož cena je odvozena od ceny tzv. podkladového aktiva. Podkladovým aktivem je velice často jiné obchodovatelné aktivum (např. akcie) nebo finanční veličina (např. úrokové sazby nebo měnové kurzy).\\

\noindent Mezi nejvýznamnější burzy obchodujícími s deriváty patří Chicago Board of Trade, Chigago Mercantile Exchange a Chicago Board Options Exchange. Mimo to jsou deriváty obchodovány na tzv. OTC (Over-the-Counter). Objem obchodů na OTC značně převyšuje objem obchodů na burzách. Zásadní rozdíl mezi OTC a burzou je ten, že obchody na burze jsou organizováné, standardizováné a z právního hlediska je protistranou každého účastníka burza. To vede na jedné straně k vyšší likviditě a nižšímu kreditnímu riziku, na druhé straně k nižší flexibilitě při sjednávání a vypořádávání obchodů.

\section{Forwardové obchody}

Forward je dohodou o prodeji / koupi dohodnutého množství určitého aktiva za dohodnutou cenu k předem sjednanému datu. To znamená, že všechny náležitosti obchodu jsou dohodnuty v čase $T_0$, avšak obchod samotný proběhne v čase $T_1$. Majitel dlouhé pozice má povinnost koupit, majitel krátké pozice pak povinnost prodat dané aktivum v souladu s dohodnutými podmínkami. Forwardové obchody jsou obchodovány na OTC.
\begin{center}
	\begin{pspicture}(0,0)(11,8)
		\rput(5.5,0.5){Výplata z forwardového kontraktu: (a) dlouhá pozice, (b)
		krátká pozice}
		\rput(5.5,0){\small $K$ - realizační cena, $S_T$ - spotová cena v době
		splatnosti}
		\rput(3,1.5){(a)}
		\rput(8.5,1.5){(b)}

		\psline[arrows=->](0.5,5)(5.5,5)
		\psline[arrows=->](0.5,2)(0.5,8)
		\psline[linewidth=0.5mm](0.5,3)(4.5,7)
		\rput(0.3,4.7){$0$}
		\rput(2.7,4.7){$K$}
		\rput(5,4.7){$S_T$}

		\psline[arrows=->](6.5,5)(11,5)
		\psline[arrows=->](6.5,2)(6.5,8)
		\psline[linewidth=0.5mm](6.5,7)(10.5,3)
		\rput(6.3,4.7){$0$}
		\rput(8.3,4.7){$K$}
		\rput(10.5,4.7){$S_T$}
	\end{pspicture}
\end{center}
Forwardové obchody je možné ocenit za předpokladu neexistence arbitráže. Uvažujme forwardový obchod, který nám umožňuje v čase $T_n$ nakoupit jednotku pokladového aktiva za $F_0$. Spotová cena pokladového aktiva je $S_0$, bezriziková úroková sazba je rovna $j$ (roční frekvence úročení) a je konstatní po celou dobu životnosti uvažovaného forwardového obchodu. Dále předpokládejme, že pokladové aktivum po dobu životnosti forwardového obchodu negeneruje žádnou výplatu a že náklady spojené s vlastnictvím tohoto aktiva jsou nulové.
Investor může sledovat dvě investiční strategie. První strategií je nakoupit spotově pokladové aktivum za $S_0$ a zároveň uzavřít forwardový obchod na prodej tohoto aktiva splatný v čase $T_n$. Druhou investiční strategií je v čase $T_0$ uložit částku $S_0$ a inkasovat úroky odpovídající bezrizikové sazbě $j$. První strategie tak v čase $T_n$ generuje výplatu $F_0 - S_0$ a druhá strategie výplatu $S_0[(1 + j)^{T_n - T_0} - 1]$. Je zřejmé, že v případě neexistence arbitráže musí být obě investiční strategie identické a že rovnovážná cena je tak rovna
\begin{equation*}
F_0 = S_0(1 + j)^{T_n - T_0}
\end{equation*}

V případě, že by smluvená forwardová cena byla nižší než rovnovážná forwardová cena $F_0$, mohl by majitel daného pokladového aktiva toto aktivum spotově prodat za $S_0$, částku uložit na depozitní účet za bezrizikovou úrokovou míru $i$ a zároveň podkladové aktivum forwardově nakoupit. Jestliže by naopak forwardová cena byla vyšší než rovnovážná forwardová cena $F_0$, mohl by si investor půjčit $S_0$ za bezrizikovou sazbu $i$, spotově nakoupit dané pokladové aktivum a to pak forwardově prodat.\\

\noindent \textbf{Příklad:} Předpokládejme, že spotová cena zlata je 300 USD za trojskou unci a bezriziková úroková sazba je 5\% p.a. Jaká by měla být dohodnuta cena pro jednoroční forwardový obchod?
\begin{equation*}
F_0 = 300 \cdot (1 + 0.05) = 315
\end{equation*}
V případě, že by dohodnutá cena pro jednoroční forwardový obchod nebyla rovna 315 USD za trojskou unci, existoval by prostor pro arbitráž.

\section{Futures}

Podobně jako forwardové obchody je také futures dohoda o prodeji / koupi předem dohodnutého aktiva k dohodnutému budoucímu datu za dohodnutou cenu. Na rozdíl od forwardových obchodů jsou však futures obchodovány na burze. To s sebou přináší menší flexibilitu, která je však kompenzována vyšší likviditou a nižším kreditním rizikem.

\section{Opce}

V případě opce má jeden z účastníků (tzv. držitel opce) právo nakoupit / prodat předem dohodnuté aktivum za předem dohodnutou cenu k předem smluvenému termínu. Druhý z účastníků (tzv. vypisovatel opce) má povinnost prodat / nakoupit, je-li k tomu držitelem opce vyzván. Na rozdíl od forwardových obchodů a futures je tedy pro opci typický asymetrický vztah mezi účastníky dohody. Proto musí držitel opce zaplatit vypisovateli tzv. opční prémii.

Podle toho, umožňuje-li opce svému držiteli nakoupit nebo prodat, rozlišujeme tzv. kupní (call) nebo prodejní (put) opci. Jestliže může držitel uplatnit své právo kdykoliv do splatnosti opce, mluvíme o tzv. americké opci. Může-li toto právo uplatnit pouze v den splatnosti opce, hovoříme o tzv. evropské opci.\\

\begin{center}
	\begin{pspicture}(0,0)(11,8)
		\rput(5.5,0.5){Výplata z kupní opce: (a) majitel opce, (b)
		vypisovatel opce}
		\rput(5.5,0){\small $K$ - realizační cena, $c$ - opční prémie, $S_T$ -
		spotová cena v době
		splatnosti}
		\rput(3,1.5){(a)}
		\rput(8.5,1.5){(b)}

		\psline[arrows=->](0.5,5)(5.5,5)
		\psline[arrows=->](0.5,2)(0.5,8)
		\psline[linewidth=0.5mm](0.5,4)(1.5,4)
		\psline[linewidth=0.5mm](1.5,4)(5.5,7)
		\psline[linewidth=0.1mm, linestyle=dashed](1.5,4)(1.5,5)
		\rput(0.3,5){$0$}
		\rput(0,4){$-c$}
		\rput(1.5,5.2){$K-c$}
		\rput(2.9,4.7){$K$}
		\rput(5,4.7){$S_T$}

		\psline[arrows=->](6.5,5)(11,5)
		\psline[arrows=->](6.5,2)(6.5,8)
		\psline[linewidth=0.5mm](6.5,6)(7.5,6)
		\psline[linewidth=0.5mm](7.5,6)(11,3)
		\psline[linewidth=0.1mm, linestyle=dashed](7.5,6)(7.5,5)
		\rput(6.3,5){$0$}
		\rput(6.3,6){$c$}
		\rput(7.5,4.7){$K-c$}
		\rput(9,5.2){$K$}
		\rput(11,4.7){$S_T$}
			
	\end{pspicture}
\end{center}

\section{Typy obchodníků}

Účastníky obchodů s deriváty můžeme v zásadě rozdělit do tří skupin: (a) zajišťovatele, (b) spekulanty a (c) arbitražéry.\\

\noindent Zajišťovatel se obchodováním s deriváty snaží zajistit proti rizikům plynoucím z pohybů finančních veličin. Spekulanti spekulují na budoucí vývoj trhů - snaží se predikovat vývoj a dosáhnout tak zisku. Arbitražéři se snaží využít nedokonalosti trhu k dosažení bezrizikového zisku (např. formou cenových rozdílů na dvou různých trzích).

\chapter{Mechanismy trhů futures} 

Futures je dohoda o nákupu / prodeji stanoveného aktiva za stanovenou cenu ve stanovenou dobu. Jedna strana má povinnost prodat, druhá pak koupit předem dohodnuté podkladové aktivum. Futures jsou obchodovány na burze, což s sebou přináší standardizaci:
\begin{itemize}
\item aktiva
\item jednotkového množství
\item datumu popř. místa vypořádání obchodu
\end{itemize}

\noindent Platí, že jestliže je některá z položek specifikována alternativně (např. aktivum nebo místo dodání), náleží právo volby straně, která má povinnost aktivum prodat (tzv. majitel krátké pozice).\\

Cena futures se řídí nabídkou a poptávkou. Pro většinu kontraktů jsou specifikovány denní limity cen (a to oběma směry). V některých případech jsou také limitovány pozice, které je možné držet. Tento mechanismus má být ochranou před spekulanty. Může však být překážkou v situaci, kdy cena podkladového aktiva rapidně vzroste / poklesne z jiných než spekulativních důvodů.

S tím, jak se blíží doba splatnosti, blíží se cena futures spotové ceně. V okamžiku splatnosti se obě ceny rovnají\footnote{Kdyby tomu tak nebylo, existoval by prostor pro arbitráž.}. Většina obchodů je však uzavřena před datem jejich vypořádání.\\

\begin{center}
	\begin{pspicture}(0,0)(12,6)
		\rput(6,0.5){Konvergence ceny futures ke spotové ceně}
		\rput(6,0){\small $T$ - čas, $M$ - maturita}
		\rput(3,1.2){(a)}
		\rput(9,1.2){(b)}

		\psline[arrows=->](0.5,2)(5.5,2)
		\psline[arrows=->](0.5,2)(0.5,5)
		\pscurve[linewidth=0.5mm](0.5,4)(1.5,4.5)(3,4.1)(4.5,4.3)
		\pscurve[linewidth=0.5mm](0.5,2.5)(1.5,3.5)(3,3.7)(4.5,4.3)
		\psline[linewidth=0.1mm, linestyle=dashed](4.5,4.3)(4.5,2)

		\rput(4.5,1.7){$M$}
		\rput(5.5,1.7){$T$}
		\rput(3,4.7){cena futures}
		\rput(3,3.3){spotová cena}

		\psline[arrows=->](6.5,2)(11.5,2)
		\psline[arrows=->](6.5,2)(6.5,5)
		\pscurve[linewidth=0.5mm](6.5,4)(7.5,4.5)(9,4.1)(10.5,4.3)
		\pscurve[linewidth=0.5mm](6.5,2.5)(7.5,3.5)(9,3.7)(10.5,4.3)
		\psline[linewidth=0.1mm, linestyle=dashed](10.5,4.3)(10.5,2)
		\rput(10.5,1.7){$M$}
		\rput(11.5,1.7){$T$}
		\rput(9,4.7){spotová cena}
		\rput(9,3.3){cena futures}
	\end{pspicture}
\end{center}

\section{Systém marží}

Pro každého účastníka obchodu je protistranou burza, která tak přejímá kreditní riziko. Aby burza minimalizovala možný negativní dopad tohoto rizika, používá systém marží. Marže představuje částku, kterou musí každý účastník složit na určený účet, a která má kompenzovat případné nepříznivé výkyvy v ceně futures.
Každý účastník obchodu má otevřený tzv. maržový účet. Aby mu bylo povoleno uzavřít kontrakt, musí na tento účet složit tzv. počáteční marži. Na maržovém účtu jsou pak následně zúčtovávány každodenní zisky / ztráty plynoucí z dané pozice.
Jestliže zůstatek na účtu přesáhne počáteční marži, může účastník obchodu odpovídající rozdíl z účtu "stáhnout". Jestliže však zůstatek účtu klesne pod tzv. udržovací marži\footnote{Udržovací marže je o něco nižší než počáteční marže - standardně se jedná o 75\% počáteční marže.}, musí účastník obchodu dorovnat částku až do výše počáteční marže. Tato částka se nazývá pohyblivá marže. Jestliže zůstatek účtu není doplněn, je pozice burzou uzavřena.\\
Platí, že obchodovat na burze mohou pouze její členové. Ostatní subjekty pak musí obchodovat pouze prostřednictvím členů burzy. Jednotliví členové mohou pro své klienty požadovat vyšší marže než nařizuje burza, ale nemohou nastavit marže na nižší úroveň. V řadě případů jsou účastníci obchodů kompenzováni za částky držené na maržových účtech formou úroku. Při otevření maržového účtu je možné k dosažení požadované výše marže použít také cenné papíry. U státních dluhopisů se zpravidla uznává 90\% jejich nominální hodnoty, u akcií pak pouze 50\%.
Stejně jako jednotliví účastníci obchodů, musí také členové burzy udržovat tzv. zúčtovací marži. Výše této marže není ve většině případů součtem absolutní výše pozic jejich jednotlivých klientů, ale je dána celkovou výslednou pozicí (tzv. netting přístup).

\section{Typy příkazů}

Členové burzy provádí obchody na svůj účet popř. na účet klientů. Klientské obchody provádí na základě příkazů. Základní typy příkazů jsou:
\begin{itemize}
\item \textbf{Market order} - Příkaz je proveden okamžitě za nejlepší cenu dosažitelnou na trhu.
\item \textbf{Limit order} - Příkaz je proveden pouze v případě, že cena na trhu dosáhne stanovené limitní ceny nebo je z pohledu klienta výhodnější než tato limitní cena.
\item \textbf{Stop order / Stop-loss order} - Příkaz je proveden za nejlepší možnou cenu po té, co cena dosáhne stanovené limitní ceny nebo je z pohledu klienta horší než tato limitní cena.
\item \textbf{Stop-limit order} - Stop-limit order je kombinací limit order a stop order. Příkaz se stává limit order po té, co cena na trhu dosáhne určité limitní ceny nebo ceny, která je z pohledu investora horší než tato limitní cena. V rámci tohoto typu příkazu musí klient specifikovat dvě ceny - cenu, při které je aktivován limit order (tzv. spot price) a cenu, za kterou je proveden případný limit order (tzv. limit price).
\end{itemize}

\chapter{Ceny futures a forwardů}

\section{Krátký prodej}

Krátký prodej (short selling) je typ obchodu, ve kterém je prodáváno aktivum, které "prodávající" nevlastní. Tento typ obchodování je možný pouze pro některé druhy investičních instrumentů.
Typickým instrumentem, který bývá předmětem krátkého prodeje, je akcie. V praxi je možné si příslušný akciový titul za smluvenou cenu vypůjčit, prodat a v době maturity obchodu akcii zpětně nakoupit na trhu a vrátit původnímu majiteli. Z logiky obchodu vyplývá, že se jedná o spekulaci na pokles ceny. Krátký prodej je totiž úspěšný pouze tehdy, jestliže po započtení odměny za vypůjčení je akcie prodána dráž, než je později zpětně nakoupena.\\

\noindent \textbf{Poznámka:} Ve Spojených státech je možné provádět krátký prodej akcií pouze v případě, že poslední pohyb akcií byl vzestup (s výjimkou obchodů na akciový index popř. obchodů, které akciový index replikují).

\section{Měření úrokových sazeb}

Pro přesnou specifikaci úrokové sazby není důležitá pouze informace o její výši, ale také informace o frekvenci úročení.
Mějme kapitál ve výši 100 EUR, který budeme úročit sazbou ve výši 5\% po dobu jednoho roku. Uvažujme následující dva případy. V prvním případě bude frekvence úročení roční, v druhém případě čtvrtletní. Pojmem "frekvence úročení" rozumíme, jak často jsou připisovány úroky.
V prvním případě budeme mít na konci roku částku
\begin{equation*}
100 \cdot (1 + 0.05) = 105.00
\end{equation*}
a druhém případě částku
\begin{equation*}
100 \cdot \left( 1+\frac{0.05}{4} \right)^4 = 105.09
\end{equation*}
Jaká by měla být výše úrokové sazby pro čtvrtletní úročení, aby částka na konci roku byla stejná?
\begin{equation*}
4 \cdot \left( \sqrt[4]{1+0.05} - 1 \right) = 0.04909
\end{equation*}
Ekvivaletní úroková sazba by tedy byla 4.909\%. Obecně lze tento vztah vyjádřit jako
\begin{equation*}
j = m \cdot \left( \sqrt[m]{1+i} - 1 \right),
\end{equation*}
kde $j$ je hledaná ekvivalentní úroková sazba, $m$ je frekvence úročení a $i$ je úroková sazba s roční frekvencí připisování úroků.
Ekvivalentní úroková sazba bude klesat s rostoucí frekvencí úročení. Limitní $j$ je přitom dáno vztahem
\begin{equation*}
j = \lim_{m \to \infty} m \cdot \left( \sqrt[m]{1+i} - 1 \right) = \ln(1+i)
\end{equation*}
V souvislosti s ekvivalentní úrokovou sazbou pro $m \to \infty$ hovoříme o tzv. kontinuálním úročení / odúročení.
\begin{equation*}
\lim_{m \to \infty} A \cdot \left( 1 + \frac{j}{m} \right)^{m \cdot n} = A \cdot e^{n \cdot j}
\end{equation*}
\begin{equation*}
\lim_{m \to \infty} A \cdot \left( 1 + \frac{j}{m} \right)^{- m \cdot n} = A \cdot e^{-n \cdot j}
\end{equation*}

\section{Forwardová cena insvestičního aktiva}

Ocenění forwardového obchodu je založeno na myšlence neexistence arbitráže. V rámci ocenění se poměřují dvě investiční strategie, které jsou z pohledu případného investora ekvivalentní, a proto proto by měly generovat stejný výnos. Jestliže by tomu tak nebylo, mohl by investor uzavřít krátkou pozici v jedné a dlouhou pozici v druhé strategii a tak získat dodatečný výnos.

Uvažujme investiční aktivum, jehož spotová cena v čase $T_0$ je $S_0$, a forwardový obchod, kterým lze v čase $T_n$ zafixovat cenu tohoto aktiva na $F_0$. Definujme $n = T_n - T_0$. Předpokládejme, že bezriziková úroková sazba je rovna $j$ (kontinuální úročení) a je konstatní po celou dobu životnosti forwardového obchodu. Dále předpokládejme, že uvažované aktivum negeruje svému majiteli po dobu životnosti žádné cash-flow a že náklady spojené s jeho držením jsou nulové. Rovnovážnou forwardovou cenu definujeme jako cenu, při které neexistuje možnost arbitráže. Tato rovnovážná cena je tedy dána vztahem
\begin{equation*}
F_0 = S_0e^{j \cdot n}
\end{equation*}
Jestliže by forwardová cena byla nižší než $F_0$, mohl by majitel příslušného investičního aktiva toto aktivum spotově prodat, prostředky získané prodejem uložit za bezrizikovou úrokovou míru a aktivum forwardově nakoupit. V případě, že by forwardová cena byla vyšší než $F_0$, mohl by si investor půjčit $S_0$, spotově nakoupit pokladové aktivum, které by v čase $T_n$ forwardově prodal.\\

\noindent \textbf{Příklad:} Uvažujme forwardový obchod na nákup akcie, která nepřináší žádné dividendy. Obchod bude realizován za tři měsíce. Aktuální cena akcie je 40 USD a bezriziková úroková sazba je 5\% p.a. (kontinuální úročení). Jaká je rovnovážná forwardová cena (tj. cena, která neumožňuje arbitráž)?
\begin{equation*}
F_0 = S_0 \cdot e^{j \cdot n} = 40 \cdot e^{0.05 \cdot 0.25} = 40.50
\end{equation*}
Rovnovážná cena je 40.50 USD. V případě, že je forwardová cena různá od 40.50 USD, je možné provést arbitráž.\\

Jestliže pokladové investiční aktivum generuje svému majiteli určité cash-flow, ať už kladné (např. kupónové platby, dividendy) nebo záporné (např. náklady na skladování), je nutné spotovou cenu upravit o současnou hodnotu tohoto cash-flow. Rovnice pro výpočet forwardové ceny se tak přejde do tvaru
\begin{equation*}
F_0 = (S_0 - I)e^{j \cdot n}
\end{equation*}
kde $I$ je současná hodnota cash-flow, které obdrží majitel aktiva po dobu životnosti forwardového obchodu.\\

\noindent \textbf{Příklad:} Uvažujme forwardový obchod na nákup kupónového dluhopisu, jehož aktuální cena je 900 USD. Dále předpokládejme, že dluhopis bude splatný za 5 roků a že forwardový obchod je splatný za 1 rok. Dluhopis přinese svému držiteli kupón 40 USD po 6 a 12 měsících, bezriziková úroková sazba je 9\% p.a. pro 6 měsíců a 10\% p.a. pro 12 měsíců (kontinuální úročení). Jaká je rovnovážná forwardová cena?
\begin{equation*}
F_0 = (S_0 - I)e^{n \cdot j}
\end{equation*}
\begin{equation*}
I = 40 \cdot e^{-0.5 \cdot 0.09} + 40 \cdot e^{-0.1} = 74.43
\end{equation*}
\begin{equation*}
F_0 = (900 - 74.43) \cdot e^{0.1} = 912.39
\end{equation*}
Rovnovážná forwardová cena je 912.39 USD.\\

Jestliže má cash-flow generované pokladovým investičním aktivem charakter výnosové míry spíše než absolutních částek, musí v případě neexistence arbitráže platit
\begin{equation*}
S_0 \int^n_0 e^{st} e^{qt} dt = S_0 \int^n_0 e^{rt}dt 
\end{equation*}
kde $s$ představuje míru růstu ceny investičního aktiva a $q$ výnosovou míru z titulu cash-flow generovaného aktivem. Protože $r = s + q$, lze výše uvedenou rovnici dále upravit do tvaru
\begin{equation*}
S_0e^{(s + q)n} = S_0e^{rn} 
\end{equation*}
Vzhledem k tomu, že $F_0 = S_0e^{sn}$, lze rovnovážnou forwardovou cenu vyjádřit jako
\begin{equation*}
F_0 = S_0e^{(r - q)t} 
\end{equation*}
\noindent \textbf{Příklad:} Uvažujme aktivum, které přináší výnos 2\% p.a. (kontinuální úročení). Bezriková úroková míra je 10\% p.a. (kontinuální úročení). Současná cena aktiva je 25 USD. Forwardový kontrakt je splatný za 6 měsíců. Jaká je rovnovážná cena?
\begin{equation*}
F_0 = 25e^{0.5 \cdot (0.1 - 0.02)} = 26.02
\end{equation*}
Rovnovážná forwardová cena je 26.02 USD.

\section{Ocenění forwardového kontraktu}

Dlouhou pozici forwardového kontraktu lze v daný časový okamžik ocenit pomocí vztahu
\begin{equation}
f_L = (F_0 - K)e^{-n \cdot j},
\end{equation}
kde $K$ je dodací cena dle kontraktu.
V okamžiku sjednání obchodu platí $F_0 = K$, tj. hodnota kontraktu je nulová. S tím, jak se však v průběhu času mění cena podkladové veličiny, mění se také rovnovážná forwardová cena. Dodací cena však zůstává nezměněná.
V době splatnosti forwardového kontraktu platí $F_0 = S_T$, kde $S_T$ je spotovou cenou.
Logika vztahu (3.1) vyplývá z toho, že $F_0$ představuje cenu, za kterou jsme schopni pozici uzavřít a zafixovat tak zisk popř. ztrátu.\\
Vzhledem k tomu, že forwardové obchody jsou symetrické finanční deriváty, platí pro krátkou pozici vztah podobný (3.1).
\begin{equation}
f_S = -f_L = (K - F_0)e^{-n \cdot j},
\end{equation}
Jestliže budeme uvažovat, že dané aktivum generuje cash-flow se současnou hodnotou $I$, modifikují se výše uvedené rovnice do tvaru
\begin{equation*}
f_L =(F_0 - K)e^{-n \cdot j} - I
\end{equation*}
resp.
\begin{equation*}
f_S =(K - F_0)e^{-n \cdot j} + I
\end{equation*}\\
Použijeme-li vztah
\begin{equation*}
F_0 = S_0e^{n \cdot j}
\end{equation*}
můžeme rovnice (3.1) a (3.2) dále upravit na
\begin{equation*}
f_L = S_0 - Ke^{-n \cdot j} - I
\end{equation*}
resp.
\begin{equation*}
f_S = Ke^{-n \cdot j} - S_0 + I
\end{equation*}

\section{Rozdíl mezi cenami futures a forwards}

V případě obchodů, jejichž splatnost je jen několik málo měsíců, je rozdíl mezi futures a forwards zanedbatelný. S tím, jak se doba splatnosti zvětšuje, zvětšuje se také rozdíl mezi cenou futures a ekvivalentním forwardovým obchodem. Rozdíly je možné vysvětlit daňovými a transakčními důvody, systémem vypořádání marží a rozdílným kreditní riziko protistran.

\section{Spotřební komodity}

Komodity mohou vedle funkce investičního instrumentu plnit také funkci výrobního vstupu (např. měď). Proto pro komodity nemusí nezbytně platit
\begin{equation*}
F_0 = (S_0 + I)e^{n \cdot j},
\end{equation*}
kde parametr $I$ představuje současnou hodnotu nákladů na držení komodit\footnote{Na komodity lze pohlížet tak, že přináší záporný výnos. Tento záporný výnos je představován především náklady na skladování.}.
Důvodem je to, že se na trhu mohou vyskytovat subjekty, které preferují držbu dané komodity např. jako "pohotovostní" výrobní zásoby. Z pohledu majitele je tak výhodnější držet přímo komoditu než např. dlouhou forwardovou pozici. Proto může platit
\begin{equation*}
F_0 < (S_0 + I)e^{n \cdot j}.
\end{equation*} 
Výhody vyplývající z fyzické držby dané komodity lez vyjádřit pomocí sazby $y$.
\begin{equation*}
F_0e^{n \cdot y} = (S_0 + I)e^{n \cdot j}
\end{equation*}
Jsou-li náklady na držbu komodity funkcí času (např. nájemné za skladovací prostory), můžeme výše uvedený vztah rozepsat do podoby
\begin{equation*}
F_0e^{n \cdot y} = S_0 \cdot e^{n \cdot (j + u)}
\end{equation*}

\section{Cena forwardových obchodů / futures a očekávaná budoucí cena}

Uvažujme spekulanta, který vstoupil do dlouhé pozice v rámci futures kontraktu. Očekává tedy, že spotová cena daného aktiva bude v době splatnosti kontraktu vyšší než dohodnutá cena. Dále uvažujme, že spekulant uloží současnou hodnotu této dohodnuté ceny do bezrizikového instrumentu s výnosovou mírou $j$\% p.a. (kontinuální úročení). Očekávaná současná hodnota tohoto obchodu je tedy dána vztahem
\begin{equation*}
E[f_L] = -F_0e^{-n \cdot j} + E[S_T]e^{-n \cdot k},
\end{equation*}
kde parametr $k$ představuje výnosovou míru adekvátní danému aktivu. Pro účely následujícího odvození budeme uvažovat, že mezi futures a forwards není rozdíl. Předpokládáme-li neexistenci arbitáže\footnote{Zde je třeba si uvědomit rozdíl mezi pojmy arbitráž a spekulace. Arbitráž umožňuje dosažení zisku bez rizika - arbitražér si tedy může vždy dopředu vypočítat svůj zisk. Naproti tomu spekulant podstupuje riziko s výhledem na možný zisk. Jeho skutečný zisk popř. ztráta však nejsou dopředu známy.}, musí v době uzavření kontraktu platit
\begin{equation*}
F_0e^{-n \cdot j} = E[S_T]e^{-n \cdot k}
\end{equation*}
nebo-li
\begin{equation*}
F_0 = E[S_T]e^{-n \cdot (k-j)}
\end{equation*}

Hodnota parametru $k$ závisí na systematickém riziku investice. Jestliže je $S_T$ nekorelované s trhem, má nulové systematické riziko a platí $k = j$. To implikuje $F_0 = E[S_T]$. Jestliže je $S_T$ pozitivně korelováno s trhem, má pozitivní systematické riziko a platí $k > j$. To znamená, že $F_0 < E[S_T]$. V případě záporné korelace platí analogicky $k < j$ a z toho vyplývající $F_0 > E[S_T]$.
Jestliže platí $F_0 = E[S_T]$, pak lze očekávat, že forwardové ceny budou stejně často růst jako klesat. Výsledný průměrný zisk z držení této pozice je proto nulový. Jestliže $F_0 < E[S_T]$, pak lze očekávat, že forwardové ceny budou v průměru růst. Důvodem je to, že v době splatnosti by mělo platit, že forwardová cena se rovná ceně spotové. Očekávaný průměrný zisk z držení této pozice by proto měl být kladný.
Analogická tvrzení platí také pro situaci, kdy $F_0 > E[S_T]$.\\

Výsledky z praktického pozorování trhu jsou smíšené. Některé výzkumy výše uvedené scénáře potvrdily, některé nikoliv. To, že je futures cena indikátorem budoucí ceny, se tedy nepodařilo potvrdit ani vyvrátit.

\chapter{Zajištění pomocí futures}

Cílem zajištění (hedging) je minimalizovat negativní dopady vyplývající z pohybů veličin jako jsou
ceny finančních instrumentů, úrokové sazby, měnové kurzy. Co se týče hospodářského výsledku společnosti, může zajištění paradoxně vést v některých případech k destabilizaci vykazovaných finančních výsledků. To nastane v situaci, kdy se cena zboží / služeb produkovaných danou společností přizpůsobuje cenám výrobních vstupů. To má za následek relativně stabilní výnosovou marži. Tato rovnováha však může být porušena právě zajištěním.\\

\noindent \textbf{Poznámka:} V následující kapitole budeme s futures zacházet jako s forwardovými obchody.

\section{Bazické riziko}

Báze představuje rozdíl mezi spotovou cenou zajišťovaného aktiva a futures cenou daného kontraktu. V případě, že jsou zajišťované aktivum a podkladové aktivum pro danou futures stejné, platí, že v době splatnosti je báze nulová.
\begin{center}
	\begin{pspicture}(0,0)(6,5.5)
		\rput(3,0.5){Vývoj báze v čase}
		\rput(3,0){\small $T$ - čas, $M$ - maturita}

		\psline[arrows=->](0.5,1.5)(5.5,1.5)
		\psline[arrows=->](0.5,1.5)(0.5,5)
		\pscurve[linewidth=0.5mm](0.5,4)(1.5,4.5)(3,4.1)(4.5,4.3)
		\pscurve[linewidth=0.5mm](0.5,2.5)(1.5,3.5)(3,3.7)(4.5,4.3)
		\psline[linewidth=0.1mm, linestyle=dashed](4.5,4.3)(4.5,1.5)
		\rput(4.5,1.2){$M$}
		\rput(5.5,1.2){$T$}
		\rput(3,4.7){cena futures}
		\rput(3,3.3){spotová cena}
	\end{pspicture}
\end{center}
Uvažujme zajišťovatele, který v čase $t_1$ zajistil svou krátkou pozici a v čase $t_2$ toto zajištění zrealizoval. V době sjednání zajištění byla báze
\begin{equation*}
b_1 = S_1 - F_1
\end{equation*}
a v době realizace zajištění pak
\begin{equation*}
b_2 = S_2 - F_2
\end{equation*}
Efektivní cena, kterou zajišťovatel takto získal je
\begin{equation*}
F_1 + (S_2 - F_2) = F_1 + b_2
\end{equation*}

V okamžiku sjednávání hedgingu, tj. v čase $t_1$, je již známo $F_1$, avšak není známo $b_2$. Riziko spojené s $b_2$ se nazývá bazické riziko. Z důvodů, které byly vysvětleny dříve, je bazické riziko větší pro komodity než např. pro akciové trhy.
Velice často také platí, že podkladové aktivum pro danou futures je odlišné od aktiva, proti jehož cenovým výkyvům se chtěl zajišťovatel zajistit\footnote{I když by pochopitelně mělo platit, že se jsou jejich ceny pokud možno co nejvíce korelovány.}.

Definujme $S_2^*$ jako cenu pokladového aktiva v čase $t_2$ a $S_2$ je cenu hedgovaného aktiva v čase $t_2$. Vztah
\begin{equation*}
F_1 + (S_2 - F_2)
\end{equation*}
může pak modifikovat do následující podoby
\begin{equation*}
F_1 + (S_2^* - F_2) + (S_2 - S_2^*)
\end{equation*}
Jestliže by zajišťované a podkladové aktiva byla totožná, byla by hodnota $S_2 - S_2^*$ nulová a vztah by se zredukoval na původní tvar.

\section{Zajišťovací poměr}

Zajišťovací poměr (hedging ratio) je poměr velikosti pozice ve futures k velikosti obchodu, který chceme zajistit. Až dosud jsme uvažovali poměr 1.00.

Uvažujme, že zajišťovatel má dlouhou pozici v akciích a krátkou ve futures. Změna v hodnotě celkové pozice je tak dána vztahem
\begin{equation*}
f = \partial S - h \partial F, 
\end{equation*}
kde $\partial S$ je změna spotové ceny, $\partial F$ změna v cenách futures a $h$ je zajišťovací poměr.
\noindent Rozptyl náhodné veličiny $f$ je
\begin{equation*}
\sigma_f^2 = E[f^2] - E[f]^2
\end{equation*}
Cílem zajištění je minimální změna hodnoty $f$, tj. dosáhnout toho, aby $\sigma_f^2$ bylo pokud možno nulové.

\begin{equation*}
\sigma_f^2 = \sigma_S^2 + h^2 \sigma_F^2 - 2h\rho_{S,F} \sigma_S \sigma_F = 0
\end{equation*}
\begin{equation*}
h = \rho_{S,F} \frac {\sigma_S} {\sigma_F}
\end{equation*}

Efektivitu zajištění lze posuzovat jako část rozptylu, která je jím eliminována. Ta je vyjádřená pomocí $\rho_{S,F}^2$ neboli $h^2 \frac{\sigma_F^2}{\sigma_S^2}$. Parametry $\rho_{S,F}$, $\sigma_F$ a $\sigma_S$ jsou zpravidla vypočteny na základě historických hodnot $\partial S$ a $\partial F$.\\

Jestliže $N_A$ je výše zajišťované pozice a $Q_F$ velikost lotu (v kusech), lze počet futures kontraktů potřebných k zajištění pozice vyjádřit jako
\begin{equation*}
N = \frac{h \cdot N_A}{Q_F}
\end{equation*}\\

\noindent \textbf{Příklad:} Letecká společnost plánuje nákup jednoho miliónu galonů leteckého paliva v průběhu příštího měsíce a chce se zajistit proti změně cen pomocí futures. Tabulka udává změny cen futures a spotových cen. Jako nejbližší komoditou obchodovanou na burze futures byl vybrán topný olej.\\

\begin{center}
\begin{tabular}{c r r}
\textbf{Měsíc} &
\multicolumn {1}{c}{\textbf{Změna ceny}} & 
\multicolumn {1}{c}{\textbf{Změna spotové}} \\
\textbf{} &
\multicolumn {1}{c}{\textbf{futures}} & 
\multicolumn {1}{c}{\textbf{ceny}}\\
\hline
 1 &  0.021 &  0.029 \\
 2 &  0.035 &  0.020 \\
 3 & -0.046 & -0.044 \\
 4 &  0.001 &  0.008 \\
 5 &  0.044 &  0.026 \\
 6 & -0.029 & -0.019 \\
 7 & -0.026 & -0.010 \\
 8 & -0.029 & -0.007 \\
 9 &  0.048 &  0.043 \\
10 & -0.006 &  0.011 \\
11 & -0.036 & -0.036 \\
12 & -0.011 & -0.018 \\
13 &  0.019 &  0.009 \\
14 &  0.027 & -0.032 \\
15 &  0.029 &  0.023 \\
\hline 
\end{tabular}
\end{center}

\noindent Jeden lot obsahuje 42 000 galonů topného oleje. Velikost pozice je 2 000 000 USD. Jaký je optimální počet kontraktů potřebných k zajištění obchodu?\\

\noindent Z údajů uvedených v zadaní a v tabulce lze vypočíst $\sigma_S = 0.0263$, $\sigma_F = 0.0313$, $\rho_{S,F} = 0.9280$ a $h = 0.7800$. Optimální počet lotů pak určíme ze vztahu

\begin{equation*}
N = \frac{0.78 \cdot 2 000 000}{42 000} = 37.14
\end{equation*}

\section{Futures na akciové indexy}

Futures na akciové indexy je možné použít k zajištění akciového portfolia. Důležitou roli zde hraje parametr $\beta$ z oceňovacího modelu kapitálových aktiv (CAPM). $\beta$ můžeme chápat jako směrnici přímky, která aproximuje vztah mezi výnosovou mírou portfolia a výnosovou mírou trhu měřenou akciovým indexem.
\begin{center}
	\begin{pspicture}(0,0)(10,6.5)
		\rput(5,0.5){Závislost mezi výnosovou mírou portfolia a výnosovou mírou trhu}
		\rput(5,0){\small{(aproximováno přímkou)}}

		\psline[arrows=->](0.5,3.5)(9.5,3.5)
		\psline[arrows=->](5,1)(5,6)
		\rput(8.5,3.7){\tiny{výnosová míra trhu}}
		\rput(3.5,6){\tiny{výnosová míra portfolia}}

		\psline[linewidth=0.5mm](0.5,1.5)(9.5,6)

		\psline[linewidth=0.1mm, linestyle=dashed, dash=1pt 1pt](5.5,3.5)(5.5,4)
		\rput(5.7,3.75){\small{$x$}}
		\rput(5.2,3.3){\small{$y$}}
		\rput(9.5,4.5){$\beta = \frac{x}{y}$}

		\psdot(1,1.4)
		\psdot(1.2,1.3)
		\psdot(1.5,2)
		\psdot(1.5,2.3)
		\psdot(1.8,2.5)
		\psdot(2,2.2)
		\psdot(2.5,2.7)
		\psdot(2.7,2.5)
		\psdot(3,2.6)
		\psdot(3.2,3)
		\psdot(3.5,3.2)
		\psdot(3.8,2.8)
		\psdot(4,3)
		\psdot(4.3,3.3)
		\psdot(4.5,3.6)
		\psdot(4.9,4)
		\psdot(5.2,4.1)
		\psdot(5.5,4.4)
		\psdot(5.7,4.5)
		\psdot(6,4.3)
		\psdot(6.5,4.7)
		\psdot(6.7,4.5)
		\psdot(7,4.4)
		\psdot(7.2,4.7)
		\psdot(7.5,4.6)
		\psdot(7.7,5)

     \end{pspicture}
\end{center}
\noindent Počet lotů potřebných pro zajištění pozice je dán vztahem
\begin{equation*}
N = \beta \cdot \frac{P}{A}, 
\end{equation*}
kde $P$ vyjadřuje objem portfolia a $A$ představuje velikost lotu. Výše uvedený vzorec předpokládá, že splatnost futures kontraktu je přibližně stejná jako období, ke kterému chceme zajistit portfolio a ignoruje denní kalkulaci zisků a ztrát.

Futures kontrakty lze použít také ke změně $\beta$ faktoru daného portfolia na hodnotu různou od nuly. Jestliže chcemem změnit hodnotu $\beta$ daného portfolia na hodnotu $\beta^*$ (pro $\beta~>~\beta^*$) lze vyjádřit počet lotů, které je třeba nakoupit pomocí vztahu
\begin{equation*}
N = (\beta - \beta^*) \frac{P}{A}
\end{equation*}\\
Vedle akciového portfolia lze zajistit také jednotlivé akcie. Účinnost tohoto zajištění je však horší a to z následujících důvodů:
\begin{itemize}
\item Zajištění pomocí futures kontraktů chrání pouze před nepříznivými pohyby trhu. Toto riziko však vysvětluje relativně malou část cenových změn dané akcie.
\item Parametr $\beta$ je pro jednu akcii méně stabilní než v případě portfolia.
\end{itemize}

\chapter{Úrokové sazby}
Následující seznam poskytuje přehled základních úrokových sazeb, se kterými je možné se setkat na finančních trzích.
\begin{itemize}
\item \textbf{Pokladniční sazba (treasury rate)} - Úroková sazba, za kterou si půjčuje vláda ve své vlastní měně. Velice často je pokladniční sazba používána jako bezriziková sazba.
\item \textbf{LIBOR} - Tato sazba má charakter průměrné úrokové sazby na mezibankovním trhu\footnote{LIBOR (London Interbank Offer Rate) značí průměrnou úrokovou sazbu na mezibankovním trhu v Londýně. V případě ostatních trhů se používají obdobné názvy jako např. PRIBOR pro Prahu. V praxi je je však pojem LIBOR často používán jako obecné označení průměrné mezibankovní sazby na daném trhu.}. LIBOR vyjadřuje průměrnou sazbu, za kterou je možné si na mezibankovním trhu vypůjčit peníze. Vedle sazby LIBOR existuje také sazba LIBID, která představuje průměrnou úrokovou sazbu, kterou jsou úročena depozita na mezibankovním trhu. Tyto sazby jsou velice často používány jako referenční a zpravidla se pohybují nad pokladniční sazbou.
\item \textbf{Repo sazba} - Repo sazba se vztahuje k tzv. repo obchodům. Repo obchody spočívají v prodeji a zpětném nákupu určitého cenného papíru, přičemž nákupní cena je o něco vyšší než cena prodejní. Na tento cenový rozdíl je možné pohlížet jako na úrok a související úrokovou sazbu pak označujeme jako repo sazbu. Repo obchody jsou např. využívány bankami ve vztahu k centrální bance za účelem doplnění likvidity.
\end{itemize}

\section{Zero sazba}

$n$-roční zero sazba je úroková míra generovaná investicí, která začíná dnes, trvá $n$ roků a jejíž veškeré cash-flow je realizována v době splatnosti. Klasickým příkladem finančního instrumentu, pro který lze triviálních způsobem určit zero sazbu, je diskontní dluhopis.

\section{Oceňování dluhopisů}

Teoretickou cenu dluhopisu lze vypočítat jako současnou hodnotu veškerého cash-flow, které obdrží majitel dluhopisu. Pro diskontování těchto cash-flow je třeba používat odpovídající zero sazbu.
Následující vzorec platí pro klasické dluhopisy, které v pravidelných intervalech přinášejí svému majiteli kupón a s posledním kupónem také nominální hodnotu.
\begin{equation*}
P_T = \sum_{i=1}^n \frac{c}{m} \cdot N \cdot e^{-t_i \cdot j_i} + N \cdot e^{-t_n \cdot j_n}
\end{equation*}\\
Parametr $n$ představuje počet období, ve kterých je vyplácen kupón ve výši $\frac{c}{m} \cdot N$. Parametr $c$ představuje kupónovou sazbu v ročním vyjádření, $m$ počet období v roce, kdy je vyplácen kupón, a $N$ představuje nominální hodnotu dluhopisu, ze které je vypočtena výše kupónu. Úroková sazba $j_i$ je zero sazby pro jednotlivá "kupónová" období, jejichž délka je $t_i$.\\

\noindent \textbf{Příklad:} Uvažujme dvouroční dvouletý státní dluhopis, ze kterého je vyplácen kupón každého půl roku. Nominální hodnota dluhopisu je 100 USD, kupónová sazba je 6\% p.a. (pololetní úročení). Zero sazby pro jednotlivá půlroční období jsou 5.0\% p.a., 5.8\% p.a., 6.4\% p.a. a 6.8\% p.a. (kontinuální úročení). Jaká je teoretická cena tohoto dluhopisu?
\begin{equation*}
P_T = 3e^{-0.05 \cdot 0.5} + 3e^{-0.058 \cdot 1.0} + 3e^{-0.064 \cdot 1.5} + 103e^{-0.068 \cdot 2.0} = 98.39
\end{equation*}
Teoretická cena výše zmiňovaného státního dluhopisu je 98.39 USD.

\section{Výpočet pokladniční zero sazby}

Uvažujme následující dluhopisy
\begin{center}
\begin{tabular}{c c c c}
\textbf{Nominále (USD)} &
\multicolumn {1}{c}{\textbf{Splatnost (roky)}} & 
\multicolumn {1}{c}{\textbf{Kupónová sazba (\%)}} &
\multicolumn {1}{c}{\textbf{Cena (USD)}}\\
\hline
 100 & 0.25 & 0  & 97.5\\
 100 & 0.50 & 0  & 94.9\\
 100 & 1.00 & 0  & 90.0\\
 100 & 1.50 & 8  & 96.0\\
 100 & 2.00 & 12 & 101.6\\
\hline 
\end{tabular}
\end{center}
Výpočet zero sazby pro první tři dluhopisy je triviální. Vzhledem k tomu, že se jedná
o tzv. diskontní dluhopisy, lze jejich teoretickou cenu vyjádřit jako
\begin{equation*}
P_T = N \cdot e^{-n \cdot j}
\end{equation*}\\
Zero sazba je pak dána vztahem
\begin{equation}
j = - ln \frac {P_T}{N} \cdot \frac {1}{n}
\end{equation}
Zero sazby pro první tři dluhopisy jsou tedy
\begin{equation*}
j_{0.25} = 0.10127
\end{equation*}
\begin{equation*}
j_{0.50} = 0.10469
\end{equation*}
\begin{equation*}
j_{1.00} = 0.10536
\end{equation*}
V případě zbývajících dvou dluhopisů je postup o něco složitější. Předpokládejme, že kupón je vyplácen pololetně. Cena dluhopisu je definována vztahem
\begin{equation*}
P_T = \sum_{i = 0}{n} \frac {c}{m} \cdot N \cdot e^{-t_i j_i} + N \cdot e^{-t_n}{j_n}
\end{equation*}
Zero sazba je pak dána rovnicí
\begin{equation*}
j_n = -\frac{ln(\frac {P_T}{N}-\frac{c}{m} \sum_{i = 0}^{n-1}e^{-t_i j_i}) - ln (1 + \frac{c}{m})}{t_n}
\end{equation*}
Tento vztah je zobecněním vztahu (5.1). Abychom mohli vypočítat zero sazbu pro dané období, jsou zapotřebí zero sazby pro předcházející období. O této metodě hovoříme jako o tzv. "bootstrapingu".
\begin{equation*}
j_{1.50} = -\frac{ln(\frac {96}{100}-\frac{0.08}{2}(e^{-0.5 \cdot 0.10469} + e^{-1.0 \cdot 0.10536})) - ln (1 + \frac{0.08}{2})}{1.5} = 0.106809
\end{equation*}
a analogicky
\begin{equation*}
j_{2.00} = 0.10808
\end{equation*}
\begin{center}
	\begin{pspicture}(0,0)(8,6)
		\rput(4,0){Zero sazby vypočtené pomocí metody "bootstraping"}

		\psline[arrows=->](0.5,1)(7.5,1)
		\psline[arrows=->](0.5,1)(0.5,5.5)
		\rput(7,1.2){\tiny{splatnost (roky)}}
		\rput(1.5,5.5){\tiny{úroková sazba}}

		\psline(0.5,2.5)(0.6,2.5)
		\psline(0.5,3)(0.55,3)
		\psline(0.5,3.5)(0.6,3.5)
		\psline(0.5,4)(0.55,4)
		\psline(0.5,4.5)(0.6,4.5)
		\rput(0,2.5){\tiny{10\%}}
		\rput(0,4.5){\tiny{11\%}}

		\psline(1,1)(1,1.05)
		\psline(1.5,1)(1.5,1.05)
		\psline(2,1)(2,1.05)
		\psline(2.5,1)(2.5,1.1)
		\rput(2.5,0.8){\tiny{1}}
		\psline(3,1)(3,1.05)
		\psline(3.5,1)(3.5,1.05)
		\psline(4,1)(4,1.05)
		\psline(4.5,1)(4.5,1.1)
		\rput(4.5,0.8){\tiny{2}}
		\psline(5,1)(5,1.05)
		\psline(5.5,1)(5.5,1.05)
		\psline(6,1)(6,1.05)
		\psline(6.5,1)(6.5,1.1)
		\rput(6.5,0.8){\tiny{3}}

		\psline[linewidth=0.5mm](0.5,2.754)(1,2.754)
		\psline[linewidth=0.5mm](1,2.754)(1.5,3.438)		
		\psline[linewidth=0.5mm](1.5,3.438)(2.5,3.572)
		\psline[linewidth=0.5mm](2.5,3.572)(3.5,3.8618)
		\psline[linewidth=0.5mm](3.5,3.8618)(4.5,4.116)
		\psline[linewidth=0.5mm](4.5,4.116)(6.5,4.116)
	\end{pspicture}
\end{center}

\section{Forwardové sazby}

S pomocí zero sazeb lze vypočítat také forwardovou sazbou. V podmínkách nemožnosti arbitráže musí totiž platit
\begin{equation*}
\prod_{i = 1}^{n-1}e^{j_{i-1, i} \cdot t_{i-1, i}} \cdot e^{j_{n-1, n} \cdot t_i} = e^{j_{0,n} \cdot t_{0,n}}
\end{equation*}
Forwardová sazba pro období $n-1$ až $n$ je tedy dána vztahem 
\begin{equation*}
j_{n-1,n} = \frac{j_{0,n} \cdot t_{0,n} - \sum_{i=1}^{n-1}j_{i-1,i} \cdot t_{i-1, i}}{t_{n-1,n}}
\end{equation*}

\begin{center}
\begin{tabular}{c c c}
\textbf{Rok} &
\multicolumn {1}{c}{\textbf{Zero rate (\%)}} & 
\multicolumn {1}{c}{\textbf{Forwardová sazba (\%)}}\\
\hline
 1 & 10.0 &  -  \\
 2 & 10.5 & 11.0\\
 3 & 10.8 & 11.4\\
 4 & 11.0 & 11.6\\
 5 & 11.1 & 11.5\\
\hline 
\end{tabular}
\end{center}

Jestliže nemáme k dispozici úrokové sazby v požadované časové struktuře, je možné potřebné sazby odvodit pomocí interpolace.

\section{Forward rate agreement}

Forward rate agreement (FRA) je OTC obchod, kterým je stanovena úroková sazba aplikovaná v dohodnutém budoucím období na předem sjednanou jistinu.

Uvažujme FRA, ve kterém získá dlouhá strana výnos $j_{FRA}$ v období mezi $T_1$ a $T_2$ z jistiny $L$. Dále definujme $j_F$ jako forwardovou sazbu LIBOR pro období $T_1$ až $T_2$. Cash-flow v čase $T_1$ je tedy $-L$ a v čase $T_2$ pak $Le^{j_{FRA}(T_2 - T_1)}$. Je zřejmé, že současná hodnota (tj. hodnota v čase $T_0$) je nulová, jestliže platí $j_{FRA} = j_F$\footnote{Tato podmínka je splněna vždy v době uzavírání obchodu. V opačném případě by existovala možnost arbitráže.}. Z pohledu dlouhé strany je současná hodnota FRA rovna
\begin{equation}
PV_{FRA} = L(e^{j_{FRA}(T_2 - T_1)} - e^{j_F(T_2 - T_1)})e^{-j_{L_{T_2}} \cdot T_2}
\end{equation}
kde $j_{L_{T_2}}$ je zero sazba pro období $T_0$ až $T_1$.\\

\noindent \textbf{Příklad:} Uvažujme, že tří měsíční LIBOR je 5\% p.a. a šesti měsíční LIBOR je 5.5\% p.a. (kontinuální úročení). Předpokládejme, že v rámci FRA obdržíme 7\% p.a. (čtvrtletní úročení) z jistiny 1 000 000 USD za období mezi třetím a šestým měsícem. Forwardová sazba je 6.0452\% p.a. (čtvrletní úročení).\\

Hodnota FRA je tedy
\begin{equation*}
PV_{FRA} = 1 000 000 \cdot (e^{0.07 \cdot 0.25} - e^{0.060452 \cdot 0.25}) \cdot e^{-0.055 \cdot 0.5} = 2 360
\end{equation*}

Alternativně lze FRA ocenit následovně. Opět uvažujme FRA, které generuje výnos $j_{FRA}$ v období $T_1$ až $T_2$. V čase $T_1$ je možné půjčit si za úrokovou sazbu $j$ jistinu $L$ a splatit ji v čase $T_2$. Jestliže tuto transakci zkombinujeme s FRA, získáme v čase $T_1$ cash-flow 0 a v čase $T_2$ bude cash-flow $Le^{j_FRA(T_2 - T_1)}$ a $-Le{j(T_2 - T_1)}$. Výsledná cash-flow v čase $T_2$ tedy bude
\begin{equation}
L(e^{j_{FRA}(T_2 - T_1)} - e^{j(T_2 - T_1)})e^{-j_{L_{T_2}} \cdot T_2}
\end{equation}
kde $j_{L_{T_2}}$ je opět zero sazba pro období $T_0$ až $T_1$. Můžeme-li předpokládat, že $j = j_F$, je (5.3) totožné s (5.2).
Výše uvedeným jsme tedy prokázali, že:
\begin{itemize}
  \item FRA je ekvivalentní dohodě, kdy je předem známá sazba $j_{FRA}$ vyměněna za tržní sazbu $j$,
  \item FRA může být oceněno za předpokladu, že je garantována forwardová sazba
\end{itemize}
  
\section {Tvar úrokových křivek}

Existují tři základní teorie, které vysvětlují tvar úrokových křivek.
\begin{itemize}
  \item \textbf{Teorie očekávání:} Podle této teorie by dlouhodobé úrokové sazby měly reflektovat očekáváné krátkodobé úrokové sazby v budoucnosti. V souladu s touto teorií by forwardové sazby pro příslušné období v budoucnosti měly být odhadem zero sazeb pro toto období.
  \item \textbf{Segmentační teorie:} Podle této teorie neexistuje vztah mezi krátkodobými, střednědobými a dlouhodobými úrokovými sazbami. Každý z těchto trhů se řídí vlastní nabídkou a poptávkou a finanční instrumenty na nich obchodované nejsou vzájemné substituty. Tato teorie tedy neříká nic o "typické" úrokové křivce.
  \item \textbf{Teorie preference likvidity:} Podle této teorie by mělo platit, že forwardové sazby jsou vyšší než očekávané budoucí zero sazby. Základním předpokladem této teorie je, že investoři preferují likviditu a investují do krátkodobých aktiv. Dlužníci naproti tomu preferují dlouhodobější půjčky. Z tohoto důvodu jsou dlouhodobé úrokové sazby vyšší než odpovídající kombinace očekávaných krátkodobých úrokových sazeb. To vede k tomu, že úroková křivka má rostoucí charakter.
\end{itemize}

\section{Konvence počtu dní}

Standardně by se pro určení počtu dní při výpočtu úroků mělo vycházet z kalendářního počtu dní. Nicméně na finančních trzích se používají také jiné konvence. Jednotlivé konvence jsou označovány ve tvaru X/Y. Nejpoužívanější konvence jsou Actual/Actual, 30/360 a Actual/360, kde Actual značí skutečný kalendářní počet dní. Pro jednotlivé trhy a instrumenty jsou pak dohodnuty konvence, podle kterých jsou počítány úroky. Výsledky se pak mohou mírně lišit podle použité konvence.

\section{Futures na státní dluhopisy}

Nejpopulárnějším úrokovým futures kontraktem v USA je futures na státní dluhopisy\footnote{Státními dluhopisy se rozumí dluhové cenné papíry emitované vládou Spojených států amerických. Konkrétně se jedná o střednědobé státní dluhopisy (treasury note), které mají v době emise splatnost 2, 5 a 10 let a dlouhodobé státní dluhopisy (treasury bonds) se splatností 30 let v době emise.}, který je obchodovaný na Chicago Board of Trade (CBOT). V době splatnosti může krátká strana dodat libovolný státní dluhopis, který má, s ohledem na typ futures kontraktu, stanovenou zbytkovou splatnost\footnote{Například v případě desetiletého futures kontraktu se musí jednat o střednědobý státní dluhopis, jehož zbytková splatnost nesmí být kratší než šest a půl roku a zároveň delší než deset roků.}. Jako protiplnění pak obdrží vypořádací cenu modifikovanou v závislosti na dodaném státním dluhopisu. Na burze jsou obchodovány dvou, pěti a desetileté futures kontrakty na střednědobé dluhopisy a futures kontrakty na dlouhodobé státní dluhopisy.

Jak již bylo zmíněno výše, vypořádací cena závisí na dodaném státním dluhopisu. Korekce vypořádací ceny je určena tzv. konverzním faktorem. Každý státní dluhopis, který může být v době splatnosti dodán krátkou stranou, má přiřazen konverzní faktor. Konverzní faktor představuje cenu dluhopisu k okamžiku splatnosti futures kontraktu za předpokladu konstatní diskontní sazby. V současné době je aplikována diskotní sazba 6\% p.a. (půlroční úročení). Výsledná vypořádací cena je pak rovna součinu konverzního faktoru a kotované ceny futures navýšeného o případný naběhlý úrok.
\begin{center}
 \textit{(kotovaná futures cena $\cdot$ konverzní faktor) + naběhlý úrok}
\end{center}
Každý kontrakt je na dodání 100 USD nominální hodnoty dluhopisu. Jestliže je kotovaná futures cena 90 USD, konverzní faktor 1.38 a naběhlý úrok 3 USD, pak platí, že krátká strana dodá dluhopis s nominální hodnotou 100 USD a obdrží
\begin{equation*}
  (1.38 \cdot 90.00) - 3.00 = 121.20
\end{equation*}

\subsection{Výpočet konverzního faktoru}

Konverzní faktor je roven hodnotě dluhopisu na 1 USD jeho nominální hodnoty k prvnímu možnému datu dodání. Pro účely diskontování cash-flow se předpokládá, že úroková sazba konstatní a rovna 6\% p.a. (pololetní úročení) po celou zbytkovou dobu splatnosti dluhopisu. Splatnost dluhopisu je navíc pro účely výpočtu zaokrouhlena dolů na nejbližší celé měsíce (dvou, tříleté a pětileté futures kontrakty) resp. celá čtvrtletí (desetileté futures kontrakty a kontrakty na dluhodobé státní dluhopisy). Jako počáteční datum pro výpočet zbytkové splatnosti dluhopisu je stanoven první den dodacího měsíce pro příslušný futures kontrakt. Kupón ze státních dluhopisů je vyplácen pololetně. V rámci výpočtu konverzního faktoru se předpokládá, že poslední kupón je vyplácen společně s nominální hodnotou v poslední den zaokrouhlené splatnosti uvažovaného státního dluhopisu. Jestliže zbytková splatnost dluhopisu po zaokrouhlení není násobkem šesti měsíců, je třeba vypočítat naběhlý úrok.\\

\noindent \textbf{Příklad:} Uvažujme 10\% kupónový dluhopis se splatností 20 let a 2 měsíce, který může krátká strana dodat v rámci futures kontraktu na dlouhodobé státní dluhopisy. Pro účely výpočtu konverzního faktoru je splanost dluhopisu 20 let a první kupón bude vyplacen po 6 měsících. Kupóny budou vypláceny pololetně po dobu 20 let, kdy je společně s posledním kupónem uhrazena také jistina. Nechť je nominální hodnota dluhopisu 100 USD a diskontní sazba je 6\% p.a. (pololetní úročení). Hodnota dluhopisu je

\begin{equation*}
  P_T = \sum_{i=1}^{40} \frac{5}{1.03^i} + \frac{100}{1.03^{40}} = 146.23
\end{equation*}

Jestliže hodnotu dluhopisu vydělíme jeho niminální hodnotou, získáme konverzní faktor 1.4623.\\

\noindent \textbf{Příklad:} Uvažujme 3.375\% kupónový dluhopis se splatností 4 roky a 2 měsíce, který může krátká strana dodat v rámci pětiletého futures kontraktu. Pro účely výpočtu konverzního faktoru je splatnost dluhopisu 4 roky a 2 měsíce. Hodnota dluhopisu je s použitím diskontní sazby 6\% p.a. (pololetní úročení) vztažená k 2. měsíci od prvního možného data jeho dodání krátkou stranou je
\begin{equation*}
P_{T_{2M}} = \sum_{i=0}^{8} \frac{1.6875}{1.03^i} + \frac{100}{1.03^{8}} = 92.47
\end{equation*}
Po diskontování a očištění o kumulovaný úrok je hodnota dluhopisu
\begin{equation*}
  P_T = \frac {P_{T_{2M}}} \cdot {1.03^{1/3}}-1.6875 \cdot {\frac{2}{3}}  = 90.44
\end{equation*}
Konverzní faktor je tedy 90.44.\\

\subsection{Cheapest to deliver}

V průběhu dodacího měsíce může existovat více dluhopisů, jejichž dodáním splní krátká strana svůj závazek. Právo volby konkrétního dluhopisu náleží právě krátké straně. Ta v rámci kontraktu obdrží
\begin{center}
\textit{(kotovaná futures cena $\cdot$ naběhlý faktor) + naběhlý úrok}
\end{center}
a dodá dluhopis, jehož cena je
\begin{center}
\textit{kotovaná cena + naběhlý úrok}
\end{center}
Proto se krátká strana bude snažit dodat dluhopis, pro nějž dosahuje výraz
\begin{center}
\textit{kotovaná cena - (kotovaná futures cena $\cdot$ konverzní faktor)}
\end{center}
minimální hodnoty.\\

\subsection{Wild card play}

V souvislosti s výše popsanými obchody se můžeme setkat s pojmem "Wild Card Play". Obchodování na Chicago Board of Trade končí ve 14:00. Nicméně státní dluhopisy jsou obchodovány na spotovém trhu do 16:00. Krátká strana může do 20:00 zadat příkaz k uzavření pozice za dodací ceny z 14:00. Jestliže cena dluhopisu v rozmezí 14:00 až 16:00 klesne, může krátká strana nakoupit a uzavřít své pozice. V opačném případě může krátká strana počkat do následujícího dne. Tuto strategii pak může uplatňovat po celé dodací období. "Wild Card Play" představuje výhodu pro krátkou stranu. Tato výhoda však není zadarmo - ceny futures jsou o něco nižší, než by byly bez této možnosti.

\subsection{Výpočet ceny futures}

Stanovit cenu futures pro obchodování se státními dluhopisy je poměrně složité. Výpočet totiž komplikují dva faktory - "Wild Card Play" a možnost volby krátné strany dodat libovolný z předem specifikovaných dluhopisů. Jestliže však odhlédneme od "Wild Card Play" a budeme uvažovat, že dluhopis, který bude dodán krátkou stranou, je dopředu znám, je forwardová cena dána klasickým vztahem
\begin{equation*}
 F_0 = (S_0 - I) \cdot e^{rT},
\end{equation*}
kde $I$ přestavuje současnou hodnotu kupónů generovaných dluhopisem v průběhu života futures kontraktu, $T$ čas do splatnosti kontraktu a $r$ bezrizikovou úrokovou sazbu aplikovatelnou pro období $T$.\\

\noindent \textbf{Příklad:} Předpokládejme, že předem známe dluhopis, který bude výhodné dodat krátkou stranou. Nechť se jedná o 6\% kupónový dluhopis s konverzním faktorem 1.4. Dále předpokládejme, že dodání tohoto dluhopisu proběhne za 270 dní.\\
\begin{center}
	\begin{pspicture}(0,0)(11,2.5)

		\psline(0.5,1)(10.5,1)

		\psline(0.5,0.95)(0.5,1.05)
		\psline(2,0.95)(2,1.05)
		\psline(5,0.95)(5,1.05)
		\psline(9,0.95)(9,1.05)
		\psline(10.5,0.95)(10.5,1.05)
		\rput(0.5,1.3){\psframebox*{\tiny{kupón}}}
		\rput(1.25,0.7){\psframebox*{\tiny{60 dní}}}
		\rput(2,1.3){\psframebox*{\tiny{$T_0$}}}
		\rput(3.5,0.7){\psframebox*{\tiny{122 dní}}}
		\rput(5,1.3){\psframebox*{\tiny{kupón}}}
		\rput(7,0.7){\psframebox*{\tiny{148 dní}}}
		\rput(9,1.9){\psframebox*{\tiny{splatnost}}}
		\rput(9,1.6){\psframebox*{\tiny{futures}}}
		\rput(9,1.3){\psframebox*{\tiny{kontraktu}}}
		\rput(9.75,0.7){\psframebox*{\tiny{35 dní}}}
		\rput(10.5,1.3){\psframebox*{\tiny{kupón}}}
	\end{pspicture}
\end{center}
\noindent Z výše uvedeného obrázku je patrné, že k poslední výplatě kupónu došlo před 60 dny a do další výplaty zbývá 122 dní. V době splatnosti kontraktu by pak měla proběhnout kompenzace za naběhlý úrok, který nabíhal po 148 dní. Uvažujme, že diskontní sazba  je 10\% p.a. (kontinuální úročení) a že současná cena dluhopisu je 120 USD. Hodnota dluhopisu $S_0$ v čase $T_0$ je tedy
\begin{equation*}
S_0 = 120 + \frac{60}{60 + 122} \cdot 6 = 121.978
\end{equation*}
Kupón ve výši 6 USD bude vyplacen za 122 dní, a proto je jeho současná hodnota $I$ rovna
\begin{equation*}
I = 6 \cdot e^{-0.1 \frac{122}{365}} = 5.803
\end{equation*}
Kontrakt trvá 270 dní, a proto je $F_0$ dáno vztahem
\begin{equation*}
F_0 = (121.978 - 5.803)e^{-0.1 \frac{270}{365}}=125.094
\end{equation*}
Dále je třeba vzít v úvahu naběhlý úrok, který nabíhá 148 dní. Po této úpravě by futures cena daného dluhopisu měla být
\begin{equation*}
F_0 = 125.094 - \frac{148}{148 + 35} \cdot 6 = 120.242
\end{equation*}
Z definice konverzního faktoru vyplývá, že výsledná futures cena kotovaná na burze by byla
\begin{equation*}
\frac{120.242}{1.4} = 85.887
\end{equation*}

\section {Lema 5A - Durace}

Durace v případě dluhopisu znamená, jak dlouho musí držitel dluhopisu v průměru čekat, než obdrží cash-flow. Diskontní dluhopis, který je splatný za $n$ roků, má tak duraci $n$ roků. V případě dluhopisů, které přináší kupón, je durace menší než jejich splatnost.\\
Uvažujme dluhopis, který svému držiteli v čase $t_i$ generuje cash-flow $c_i$, kde $1 \le i \le n$. Vztah mezi cenou dluhopisu $B$ a výnosem $y$ je dán vztahem
\begin{equation*}
B = \sum_{i=1}^{n} c_i e^{-y t_i}
\end{equation*}
Durace dluhopisu je pak
\begin{equation*}
D = \frac{\sum_{i=1}^{n} c_i t_i e^{-y t_i}}{B} = \sum_{i = 1}{n} \frac{c_i e^{-y t_i}}{B}
\end{equation*}
Dále platí
\begin{equation*}
\frac {\partial B}{\partial y} = - \sum_{i = 1}^{n} c_i t_i e^{-y t_i} = -B \cdot D
\end{equation*}
Tento vzorec se dá interpretovat tak, že malý posun výnosové křivky o $\partial y$ způsobí změnu ceny dluhopisu o $\partial B$ peněžních jednotek. Výše uvedený vztah tak lze přepsat do tvaru
\begin{equation*}
\frac{\partial B}{B} = -D \cdot \partial y
\end{equation*}

Malé změny úrokové sazby jsou často měřeny pomocí tzv. bazických bodů. Jeden bazický bod odpovídá 0.01\% p.a.\\

Jestliže je $y$ vyjádřeno v ročním úročení namísto kontinuálního, lze dokázat, že
\begin{equation*}
\frac {\partial B}{B} = -D \cdot \frac {\partial y}{1 + y}
\end{equation*}
Modifikovaná durace $MD$ je pak definována jako
\begin{equation*}
MD = \frac {D}{1 + y}
\end{equation*}
\begin{equation*}
\frac{\partial B}{B} = -DM \cdot \partial y
\end{equation*}

Durace portfolia může být stanovena jako vážený průměr durací jednotlivých titulů, kde váha odpovídá podílů jednotlivých titulů na portfoliu. Výše uvedené vzorce v případě portfolia předpokládají, že pohyb křivek pro jednotlivé dluhopisy bude stejný.

\subsection{Konvexita}

Duraci lze aplikovat pouze na malé změny výnosové míry $y$. Důvodem je to, že v rámci durace je vztah mezi změnou výnosové míry a ceny aproximován přímkou, ačkoliv ve skutečnosti lineární není. Z tohoto důvodu se někdy odhad změny ceny zlepšuje pomocí tzv. konvexity, která bere částečně v potaz nelineární závislost ceny dluhopisu na výnosové míře. Konvexita $C$ je dána vztahem

\begin{equation*}
C = \frac {1}{B} \frac {\partial^2 B} {\partial^2 y} = \frac{\sum_{i=1}^{n}c_i t_i^2 e^{-y t_i}}{B}
\end{equation*}

Duraci lze také využít při hedgování portfolia. Platí totiž, že portfolio je imunní proti změnám úrokových sazeb za předpokladu, že je jeho durace nulová. Hovoříme o tzv. imunizaci portfolia. Je však vhodné si uvědomit, že tento přístup zabezpečí portfolio pouze proti paralelnímu pohybu výnosových křivek. Nulové durace portfolia lze také dosáhnout nákupem / prodejem úrokových futures.

\chapter {Swapy}

Swap je dohodou dvou stran o vzájemné výměně cash-flow v budoucnosti. Součástí dohody jsou datumy realizace těchto cash-flow, způsob, jakým budou stanoveny, a nominály, ke kterým se vztahují.

V této kapitole se budeme zabývat dvěma základními typy swapů - úrokovým a měnovým swapem.

\section {Úrokové swapy}

Úrokový swap je obchodem, kdy dochází k výměně peněžních toků mezi zúčastněnými stranami. Klasický úrokový swap, kdy dochází ke směně floatové úrokové sazby za fixní, lze modelovat pomocí imaginárního floatového a fixního dluhopisu. Vzhledem k tomu, že úrokový swap lze rozložit na jednodušší instrumenty, jedná se o syntetický instrument.

Úrokové swapy tedy použít k transformaci aktiv / pasiv. Aktiva / pasiva spojená s fixním úročením lze tak převést na floatové úročení a naopak.

\noindent \textbf{Příklad:} Uvažujme tříletý úrokový swap s podkladovým kapitálem 100 miliónů USD. Jedna ze smluvních stran platí dvakrát ročně úroky odpovídající aktuální sazbě LIBOR a zároveň obdrží od druhé strany jednou ročně fixní sazbu 4\% p.a. Úrokové platby jsou vztaženy k podkladovému kapitálu 100 miliónů USD.
\begin{center}
	\begin{pspicture}(0,0)(8,7.5)
		\rput(4,0.0){Rozklad úrokového swapu: (a) floatový dluhopis, (b) fixní dluhopis}
		\psline[arrows=->](0.5,4)(7.5,4)
		\rput(7.3,3.7){$t$}
		\rput(0.2,3.7){\tiny 0}
		\psline[arrows=>, linewidth=0.1mm, linestyle=dashed](0.5,7)(0.5,4)
		\rput(0.8,6.8){(a)}
		\psline[arrows=>, linewidth=0.1mm](0.5,1)(0.5,4)
		\rput(0.8,1.2){(b)}
		\psline[arrows=>, linewidth=0.1mm, linestyle=dashed](1.5,3)(1.5,4)
		\rput(2.2,3.7){\tiny 1Y}
		\psline[arrows=>, linewidth=0.1mm, linestyle=dashed](2.5,3)(2.5,4)
		\psline[arrows=>, linewidth=0.1mm](2.5,6)(2.5,4)
		\psline[arrows=>, linewidth=0.1mm, linestyle=dashed](3.5,3)(3.5,4)
		\rput(4.2,3.7){\tiny 2Y}
		\psline[arrows=>, linewidth=0.1mm, linestyle=dashed](4.5,3)(4.5,4)
		\psline[arrows=>, linewidth=0.1mm](4.5,6)(4.5,4)
		\psline[arrows=>, linewidth=0.1mm, linestyle=dashed](5.5,3)(5.5,4)
		\rput(6.1,3.7){\tiny 3Y}
		\psline[arrows=>, linewidth=0.1mm](6.5,7)(6.5,4)
		\psline[arrows=>, linewidth=0.1mm, linestyle=dashed](6.5,1)(6.5,4)
		\psline[arrows=>, linewidth=0.1mm, linestyle=dashed](6.4,3)(6.4,4)
		\psline[arrows=>, linewidth=0.1mm](6.4,6)(6.4,4)
	\end{pspicture}
\end{center}
Tento swap lze tedy "rozložit" na dva imaginární dluhopisy. První z nich je fixní s kupónovou sazbou 4\% p.a. a roční výplatou kupónů. Druhý dluhopis má kupónovou sazbu definovanou sazbou LIBOR a půlroční výplatu kupónů. Tu část úrokového swapu, kterou představuje cash-flow generované fixním dluhopisem, nazýváme fixní nohou. Cash-flow generované imaginárním floatovým dluhopisem pak nazýváme floatovou nohou. Jednotlivé platby, je-li to možné, se vzájemně kompenzují a vyplácí se pouze rozdíl mezi oběma peněžními toky. Z toho principu mimojiné vyplývá, že podkladový kapitál se na začátku a na konci kontraktu nesměňuje. Z hlediska oceňování není kompenzace cash-flow důležitá.

\subsection{Hodnota úrokového swapu v čase $t_0 = 0$}

Uvažujme jednoroční úrokový swap s podkladovým kapitálem $A$, v rámci kterého dochází k výměně floatové sazby $F$ za fixní sazbu $R$. Předpokládejme, že rok má 360 dní a definujme časový interval mezi $t_2$ a $t_1$ v ročním vyjádření jako 
\begin{equation*}
\alpha_{t_1, t_2} = \frac{t_2 - t_1}{360}
\end{equation*}
Dále definujme diskontní faktor $DF_{t_1, t_2}$ pro časové období $t_1$ až $t_2$ jako
\begin{equation*}
DF_{t_1, t_2} = \frac{1}{1 + F_{t_1, t_2} \alpha_{t_1, t_2}}
\end{equation*}
kde $F_{t_1, t_2}$ je forwardová úroková míra v čase $t_0$ platná pro časový interval $t_1$ až $t_2$. Předpokládejme, že pro forwardovou sazbu platí
\begin{equation*}
(1 + F_{t_0, t_2} \alpha_{t_0, t_2})  = (1 + F_{t_0, t_1} \alpha_{t_0, t_1})(1 + F_{t_1, t_2} \alpha_{t_1, t_2})
\end{equation*}

\subsubsection{Fixní noha}

Výplata $L_{fix}^{1Y}$ z fixní nohy na konci prvního roku je definována jako
\begin{equation*}
L_{fix}^{1Y} = A R_{1Y} \alpha_{0,1Y}
\end{equation*}
kde $R_{1Y}$ představuje výši fixního kupónu jednoročního swapu. Současná hodnota fixní nohy je tedy definována jako
\begin{equation*}
PV(L_{fix}^{1Y}) = DF_{0, 1Y} L_{fix}^{1Y} =  DF_{0, 1Y} A R_{1Y} \alpha_{0,1Y}
\end{equation*}

Obecný vzorec pro současnou hodnotu fixní nohy $n$-letého úrokového swapu v době uzavření kontraktu je
\begin{equation}
AR_{nY}(DF_{0,t_1} \alpha_{0,t_1} + DF_{0, t_2} \alpha_{t_1,t_2} + ... + DF_{0, t_n} \alpha_{t_{n-1},t_n})
\end{equation}

\subsubsection{Floatová noha}

Floatová noha se skládá ze dvou cash-flow. První cash-flow $L_{float}^{6M}$ generované po půl roce trvání kontraktu je rovno
\begin{equation*}
L_{float}^{6M} = A F_{0,6M} \alpha_{0,6M}
\end{equation*}
Floatovou sazbu platnou za půl roku tedy odhadujeme pomocí forwardové sazby $F_{0,6M}$. Současná hodnota tohoto cash-flow je
\begin{equation*}
PV(L_{float}^{6M}) = DF_{0,6M} L_{float}^{6M} = DF_{0,6M} A F_{0,6M} \alpha_{0,6M} =
\end{equation*}
\begin{equation*}
= DF_{0,6M} A \bigg( \frac{DF_{0,0}}{DF_{0,6M}} - 1 \bigg) = DF_{0,6M} A \bigg( \frac{1}{DF_{0,6M}} - 1 \bigg) = A (1 - DF_{0,6M})
\end{equation*}
Obdobně současná hodnota druhého cash-flow $L_{float}^{1Y}$ generovaného na konci životnosti kontraktu je rovna
\begin{equation*}
PV(L_{float}^{1Y}) = DF_{0,1Y} A \bigg( \frac{DF_{0,6M}}{DF_{0,1Y}} - 1 \bigg) = A (DF_{0,6M} - DF_{0,1Y})
\end{equation*}
Současná hodnota celé floatové nohy je tedy
\begin{equation*}
PV(L_{float}^{6M}) + PV(L_{float}^{1Y}) = A (1 - DF_{0,6M}) + A (DF_{0,6M} - DF_{0,1Y}) = A (1 - DF_{0,1Y})
\end{equation*}

Obecný vzorec pro současnou hodnotu floatové nohy $n$-letého úrokového swapu v době uzavření kontraktu je
\begin{equation}
A(1 - DF_{0,nY})
\end{equation}

\subsubsection{Hodnota úrokového swapu}

Vzhledem k tomu, že při neexistenci arbitráže by měla být hodnota úrokového swapu v okamžiku jeho sjednání rovna nule, musí se současná hodnota fixní nohy rovnat současné hodnotě floatové nohy. Pro námi uvažovaný jednoroční úrokový swap tedy musí platit
\begin{equation*}
PV(L_{fix}^{1Y}) = PV(L_{float}^{6M}) + PV(L_{float}^{1Y}) 
\end{equation*}
\begin{equation}
DF_{0, 1Y} A R_{1Y} \alpha_{0,1Y} = A (1 - DF_{0,1Y})
\end{equation}

Dle (6.1) a (6.2) je obecná rovnice pro $n$-roční úrokový swap definována jako
\begin{equation*}
AR_{nY}(DF_{0,t_1} \alpha_{0,t_1} + DF_{t_1, t_2} \alpha_{0,t_2} + ... + DF_{0, t_n} \alpha_{t_{n-1},t_2}) = A(1 - DF_{0,nY})
\end{equation*}
Jestliže budeme uvažovat směnu podkladového kapitálu na začátku a na konci kontraktu, změní se nám tento obecný vzorec do podoby
\begin{equation*}
-A + AR_{nY}(DF_{0,t_1} \alpha_{0,t_1} + DF_{0, t_2} \alpha_{t_1,t_2} + ... + DF_{0, t_n} \alpha_{t_{n-1},t_n}) + A DF_{0, t_n} =
\end{equation*}
\begin{equation*}
= -A + A(1 - DF_{0,nY}) + A DF_{0, t_n}
\end{equation*}
\begin{equation*}
-A + AR_{nY}(DF_{0,t_1} \alpha_{0,t_1} + DF_{0, t_2} \alpha_{t_1,t_2} + ... + DF_{0, t_n} \alpha_{t_{n-1},t_n}) + A DF_{0, t_n} = 0
\end{equation*}
\begin{equation*}
R_{nY}(DF_{0,t_1} \alpha_{0,t_1} + DF_{t_1, t_2} \alpha_{0,t_2} + ... + DF_{0, t_n} \alpha_{t_{n-1},t_2}) + DF_{0, t_n} = 1
\end{equation*}
Tento vzorec v podstatě neříká nic jiného, než že v případě rovnovážné swapové sazby $R_{nY}$ je současná hodnota imaginárního fixního dluhopisu rovna jeho nominální hodnotě. Z toho vyplývá, že také hodnota floatové nohy je rovna nominální hodnotě podkladového aktiva\footnote{V opačném případě by nebyla zachována rovnost.}. 

\subsection{Hodnota úrokového swapu v čase $t$}

Výše uvedený postup lze aplikovat v případě, že chceme vypočítat hodnotu úrokového swapu v čase $t_0 = 0$. Hodnotu úrokového swapu v obecném čase $t$ lze opět vyjádřit jako rozdíl hodnoty fixní a floatové nohy diskontované k času $t$.

Současná hodnota fixní nohy v čase $t$ je za předpokladu roční výplaty rovna
\begin{equation*}
PV_t(L_{fix}^{t_1}) + PV_t(L_{fix}^{t_2}) + ... + PV_t(L_{fix}^{t_n}) = A R_{nY}(DF_{t, t_1} + DF_{t, t_2} + ... + DF_{t, t_n})
\end{equation*}
Současná hodnota floatové nohy v čase $t$ je pak rovna
\begin{equation*}
PV_t(L_{float}^{t_1}) + PV_t(L_{float}^{t_2}) + ... + PV_t(L_{float}^{t_n}) = A \bigg(\frac{DF_{t,t_1}}{DF_{t_1 - 1, t_1}} - DF_{t, nY} \bigg)
\end{equation*}
Hodnota úrokového swapu v čase $t$ je tedy rovna
\begin{equation*}
A R_{nY}(DF_{t, t_1} + DF_{t, t_2} + ... + DF_{t, t_n}) - A \bigg(\frac{DF_{t,t_1}}{DF_{t_1 - 1, t_1}} - DF_{t, nY} \bigg)
\end{equation*}
Ačkoliv by hodnota úrokového swapu v okamžiku jeho uzavření měla být nulová, neplatí tento předpoklad po dobu životnosti kontraktu. Hodnota swapu se mění s tím, jak se mění forwardové úrokové sazby.

\subsection{Výpočet diskontního faktoru}

Z (6.3) lze vypočíst diskontní faktor $DF_{0, 1Y}$. Po elemetnárních úpravách získáváme
\begin{equation*}
DF_{0, 1Y} = \frac{1}{1 + R_{1Y} \alpha_{0,1Y}}
\end{equation*}
Vzhledem k tomu, že platí $R_{1Y} = F_{0,1Y}$, je tento vzorec v souladu s výše uvedenou definicí diskontního faktoru.

Analogickým postupem lze z dvouletého swapu odvodit diskontní faktor $DF_{0,2Y}$ jako
\begin{equation*}
DF_{0,2Y} = \frac{1 - R_{2Y} \alpha_{0,1Y}DF_{0,1Y}}{1 + R_{2Y} \alpha_{1Y, 2Y}}
\end{equation*}
a diskontní faktor $DF_{0,3Y}$ z tříletého swapu jako
\begin{equation*}
DF_{0,3Y} = \frac{1 - R_{3Y}(\alpha_{0,1Y}DF_{0,1Y} + \alpha_{1Y,2Y}DF_{0,2Y})}{1 + R_{3Y} \alpha_{2Y, 3Y}}
\end{equation*}
Pro výpočet diskontního faktoru je tedy zapotřebí znalost diskotních faktorů pro předchozí období.

Obecný vzorec pro výpočet diskontního faktoru $DF_{0,nY}$ je
\begin{equation}
DF_{0,nY} = \frac{1 - R_{nY} \bigg( \sum_{t = 1Y}^{(n-1)Y} \alpha_{(t-1)Y,tY DF_{0,t}} \bigg)}{1 + R_{nY} \alpha_{(n-1)Y,nY}}
\end{equation}

\section{Swapová křivka}

Swapová křivka je často používána jako bezriziková křivka pro oceňování investičních intrumentů. Tato křivka je definována tzv. zero  sazbami\footnote{$n$-roční zero sazba je výnosovou mírou investice, která začíná dnes, je splatná za $n$ roků a jejíž veškeré cash-flow je realizováno v době splatnosti.}. Zero sazby se splatností do jednoho roku jsou zpravidla konstruovány na základě sazeb LIBOR. Zero sazby se splatností od jednoho roku výše jsou odvozeny z kotovaných sazeb úrokových swapů. Z těchto sazeb jsou nejprve vypočteny diskontní faktory, ze kterých jsou následně vypočteny jednotlivé zero sazby.

V případě zero sazeb konstruovaných na základě sazeb LIBOR se potřebné diskontní faktory vypočtou podle
\begin{equation*}
DF_{0,t} = \frac{1}{1 + i_{0,t}\alpha_{t, t_0}}
\end{equation*}
kde $i_{0,t}$ představuje referenční sazbu LIBOR. Hodnoty zbývajících diskotních faktorů získáme dle (6.4). Vzorec pro výpočet diskotního faktoru, ze kterého se určí zero sazby $Y_{0,t}$ této křivky, je definován jako
\begin{equation*}
DF_{0,t} = \frac{1}{(1 + Y_{0,t})^{\alpha_{t, t_0}}}
\end{equation*}
Elementární úpravou pak získáváme
\begin{equation*}
Y_{0,t} = \sqrt[\alpha_{t, t_0}]{\frac{1}{DF_{0,t}}} - 1
\end{equation*}

\section {Měnový swap}

Ve své nejjednodušší podobě představuje měnový swap výměnu jistiny a úroků v jedné měně za jistinu a úroky v druhé měně. Nominální hodnoty v obou měnách jsou vyměněny na začátku a na konci kontraktu.

Podobně jako v případě úrokového swapu je měnový swap syntetickým instrumentem, který je možné rozložit na dvojici fiktivních dluhopisů. U měnového swapu jsou na rozdíl od úrokového swapu možné následující kombinace
\begin{itemize}
\item \textbf{fix-fix měnový swap} - směna pevné sazby za pevnou sazbu (dva fixní dluhopisy)
\item \textbf{float-fix měnový swap} - směna floatové sazby za pevnou sazbu (jeden floatový a jeden fixní dluhopis)
\item \textbf{float-float} - směna floatové sazby za floatovou sazbu (dva floatové dluhopisy)
\end{itemize}
Měnový swap umožňuje změnit úrokovou expozici v jedné měně na úrokovou expozici v jiné měně.\\

\noindent \textbf{Příklad:} Uvažujme tříletý float-fix měnový swap, v rámci kterého dojde v čase $t_0$ ke směně $A_{eur}$ za $A_{usd}$. Po dobu životnosti kontraktu budou na konci roku mezi zúčastněnými stranami vyměněny úrokové platby. Strana, která v čase $t_0$ směňovala EUR za USD ekvivalent, bude hradit úrok z nominální hodnoty $A_{usd}$ odpovídající sazbě LIBOR aktuální k datu výměny úrokových plateb. Druhá strana pak bude platit fixní úrok 4\% p.a. z nominální hodnoty $A_{eur}$. V čase $t_3$, tj. v době splatnosti uvažovaného měnového swapu, dojde ke zpětné směně nominálních hodnot $A_{eur}$ a $A_{usd}$.
\begin{center}
	\begin{pspicture}(0,0)(8,7.5)
		\rput(4,0.0){Rozklad měnového swapu: (a) fixní EUR dluhopis, (b) floatový USD dluhopis}
		\psline[arrows=->](0.5,4)(7.5,4)
		\rput(7.3,3.7){$t$}
		\rput(0.2,3.7){\tiny 0}
		\psline[arrows=>, linewidth=0.1mm, linestyle=dashed](0.5,7)(0.5,4)
		\rput(0.8,6.8){(a)}
		\psline[arrows=>, linewidth=0.1mm](0.5,1)(0.5,4)
		\rput(0.8,1.2){(b)}
		\rput(2.2,3.7){\tiny 1Y}
		\psline[arrows=>, linewidth=0.1mm, linestyle=dashed](2.5,3)(2.5,4)
		\psline[arrows=>, linewidth=0.1mm](2.5,6)(2.5,4)
		\rput(4.2,3.7){\tiny 2Y}
		\psline[arrows=>, linewidth=0.1mm, linestyle=dashed](4.5,3)(4.5,4)
		\psline[arrows=>, linewidth=0.1mm](4.5,6)(4.5,4)
		\rput(6.1,3.7){\tiny 3Y}
		\psline[arrows=>, linewidth=0.1mm](6.5,7)(6.5,4)
		\psline[arrows=>, linewidth=0.1mm, linestyle=dashed](6.5,1)(6.5,4)
		\psline[arrows=>, linewidth=0.1mm, linestyle=dashed](6.4,3)(6.4,4)
		\psline[arrows=>, linewidth=0.1mm](6.4,6)(6.4,4)
	\end{pspicture}
\end{center}

\subsection{Ocenění měnového swapu}

Měnové swapy se oceňují podobným způsobem jako úrokové swapy. Nejprve se konktrakt rozloží na dvojici fiktivních dluhopisů, kde každý dluhopis představuje jednu nohu měnového swapu. Dále je třeba stanovit cash-flow jednotlivých nohou. Posledním krokem je určení současné hodnoty těchto cash-flow v měně příslušné nohy. Pomocí spotového kurzu se pak obě nohy vyjádří ve společné měně a hodnota měnového swapu je pak dána jejich součtem.

V případě fixních noh je určení cash-flow triviální. Úrokové platby jsou dány fixní sazbou, která je neměnná po celou dobu životnosti měnového swapu.

Floatové nohy se oceňují stejným způsobem jako floatové dluhopisy s tím rozdílem, že se neuvažuje naběhlý úrok. Nejprve je třeba vypočítat forwardové sazby z aktuální swapové křivky. Tyto forwardové sazby, které jsou odhadem budoucích úrokovoých sazeb, jsou pak použity pro stanovení očekávaného cash-flow.

\subsection{Basis swap spread}

Teoreticky je možné pro výpočet forwardových sazeb a diskontních faktorů použít standardní swapovou křivku. V praxi se u měnového swapu se často provádí úprava swapové křivky o tzv. basis swap spread. Basis swap spread odráží skutečnost, že trh nepovažuje směňované měny za "ekvivalentní". Důvody pro vysvětlení spreadu mohou být makroekonomického charakteru nebo mohou spočívat v rozdílném kreditním riziku.

Hodnoty basis swap spreadu pro jednotlivé měny a splatnosti jsou kotovány na trhu. Referenční měnou, tj. měnou s nulovým spreadem, je nejčastěji USD. Standardní swapové sazby $R_{nY}$ je třeba navýšit basis swap spread a dle (6.4) vypočíst diskontní faktory. Z diskontních faktorů jsou pak dle (6.5) vypočteny zero sazby tzv. basis swapové křivky. Z této křivky jsou určeny diskotní faktory a forwardové sazby pro odhad cash-flow floatových nohou.

Z výše uvedeného je tedy patrné, že kdybychom měnový swap v praxi oceňovali pomocí standardní swapové namísto basis swapové křivky, měl by tento swap již v okamžiku svého sjednání nenulovou hodnotu.

\chapter{Základy opčních trhů}

\section{Pokladová aktiva}
Opce obchodované na burze mají jako pokladová aktiva akcie, akciové indexy, měny a futures kontrakty. V následujícím textu se bude zabývat především akciovými opcemi.

\section{Akciové opce}
Akciové opce obchodované na burze v USA jsou americké opce na nákup/prodej 100 kusů akcií. Detaily kontraktu - datum splatnosti, dodací cena, podmínky v případě výplaty dividend, maximální objem pozice atd. - jsou stanoveny burzou.\\

"In-the-money" ("at-the-money") opce je opce, která by svému majiteli přinesla kladné cash-flow, jestliže by byla v daný  okamžik uplatněna. Opakem "in-the-money" opce je pak "out-of-the-money" opce. Jestliže $S$ je cena akcie a $K$ tzv. realizační cena, platí, že kupní opce je "in-the-money" pro $S<K$, "at-the-money" pro $S=K$ a "out-of-the-money" pro $S>K$.\\

Hodnota opce se skládá ze dvou částí - tzv. vnitřní hodnoty a časové hodnoty. Vnitřní hodnota akcie je definována jako $\max(S-K,0)$ pro kupní opci a jako $\max(K-S,0)$ pro prodejní opci, což odpovídá okamžité hodnotě akcie. Časová hodnota opce je dána možným pozitivním vývojem akciových kurzů a představuje tedy potenciální navýšení ceny opce. Tuto hodnotu můžeme definovat jako rozdíl mezi cenou opce a její vnitřní hodnotou. Pro americkou opci vždy platí, že její časová hodnota musí být větší nebo rovna nule. Dále platí, že časová hodnota opce v době její maturity je rovna nule.\\

Dříve byla realizační cena korigována o případnou výplatu dividend. V případě, že byla vyplacena dividenda, byla realizační cena snížena o výši této dividendy. V dnešní době však tato korekce prováděna není, což má dopad na způsob oceňování opcí.\\
Další možnou komplikací je tzv. štěpení akcií. V tomto případě je sice snížena realizační cena v poměru štěpení akcí, avšak je zvýšen počet akcií, které mají být prodány/nakoupeny. Dopad na pozici držitele/vypisovatele opce je tedy nulový. Podobným způsobem je ošetřena také akciová dividenda.\\

Burza velice často stanovuje limity pro velikost pozice a objem, které může investor zobchodovat v průběhu pěti po sobě jdoucích pracovních dní. Tento limit se ve většině případů rovná limitu pozice.\\

Tvůrcem trhu je entita, která kotuje nákupní/prodejní cenu opce, za kterou je ochotna zrealizovat obchod. Tento mechanismus zajišťuje existenci ceny, za kterou je možné opci okamžitě zobchodovat. Tím je také zajištěna likvidita trhu.\\

Clearingové centrum plní v případě opcí stejnou funkci jako v případě trhu futures. Garantuje účastníkům, že budou moci provést obchod za sjednaných podmínek (tj. eliminuje riziko protistrany) a vede seznam obchodů včetně pozic jednotlivých účastníků. Stejně jako v případě futures platí, že obchodovat na burze mohou pouze členové clearingového centra. Clearingové centrum také spravuje maržové účty vypisovatelů opcí.

\section{Vlastnosti akciových opcí}

Následující faktory ovlivňují cenu akciových opcí:
\begin{itemize}
\item současná cena akcie $S_0$
\item realizační cena $K$
\item zbytková splatnost opce $T$
\item volatilita ceny akcie $\sigma$
\item bezriziková úroková sazba $r$
\item očekávané dividendy v průběhu životnosti opce
\end{itemize}

\begin{center}
\begin{tabular}{l c c c c}
\textbf{} &
\textbf{Evropská} &
\textbf{Evropská} &
\textbf{Americká} &
\textbf{Americká} \\
\textbf{Faktor} &
\textbf{kupní} &
\textbf{prodejní} &
\textbf{kupní} &
\textbf{prodejní} \\
\textbf{} &
\textbf{opce} &
\textbf{opce} &
\textbf{opce} &
\textbf{opce} \\
\hline
současná hodnota akcie & + & - & + & - \\
realizační cena & - & + & - & - \\
zbytková splatnost & ? & ? & + & + \\
volatilita & + & + & + & + \\
bezriziková sazba & + & - & + & - \\
dividendy & - & + & - & + \\
\hline 
\end{tabular}
\end{center}

\noindent \textbf{Cena akcie a realizační cena:} Jestliže je kupní opce uplatněna, je cash-flow, které získá její držitel, dáno právě rozdílem mezi spotovou cenou akcie a realizační cenou. Hodnota kupní opce tedy roste s růstem spotové ceny akcie. V případě prodejní opce je tomu naopak.\\
\noindent \textbf{Čas do splatnosti:} Hodnota americké opce roste s růstem zbytkové maturity. Toto pravidlo platí ve většině případů také pro evropské opce, i když se nejedná o pravidlo (např. z důvodu očekávané výplaty dividend).\\
\noindent \textbf{Volatilita:} Hodnota opcí roste s rostoucí volatilitou.\\
\noindent \textbf{Bezriziková úroková míra:} Vliv bezrizikové úrokové míry na ceny opcí není tak přímočarý jako v předcházejících případech. Platí, že s růstem bezrizikové úrokové míry rostou také požadavky investorů na výnosnost akcií. Současně však současná hodnota budoucího cash-flow klesá. Tyto dva efekty pak mají za následek pokles ceny prodejní opce a růst ceny nákupní opce.\\
\noindent \textbf{Dividendy:} Dividendy mají za následek pokles cen akcií v tzv. "ex-dividend" den. Výplata dividend tedy způsobí růst hodnoty prodejní opce a pokles hodnoty nákupní opce.\\

\section{Horní a dolní limit pro ceny opcí}

V následujícím textu budeme používat symboly
\begin{itemize}
\item $S_0$ - současná cena akcie
\item $K$ - realizační cena
\item $T$ - zbytková maturita
\item $S_T$ - cena akcie v době maturity opce
\item $r$ - bezriziková úroková míra investice s maturIt\^ou v čase $T$ (kontinuální úročení)
\item $C$ - hodnota americké kupní opce (na 1 akcii)
\item $c$ - hodnota evropské kupní opce (na 1 akcii)
\item $P$ - hodnota americké prodejní opce (na 1 akcii)
\item $p$ - hodnota evropské prodejní opce (na 1 akcii)
\end {itemize}

\subsection{Horní limit ceny opcí}

\subsubsection{Kupní opce}

Pro kupní opce platí, že jejich hodnota nemůže být vyšší než cena podkladové akcie\footnote{Důvodem tohoto tvrzení je, že realizační cena nemůže být záporná.}.
\begin{equation*}
c \le S_0
\end{equation*}
\begin{equation*}
C \le S_0
\end{equation*}
Jestliže by výše uvedené nerovnosti neplatily, mohl by arbitražér velice snadno vydělat tím, že koupí akcii a prodá kupní opci.\\

\subsubsection{Prodejní opce}

Pro prodejní opci platí, že její hodnota nemůže být vyšší než $K$\footnote{Toto tvrzení vychází z předpokladu, že hodnota akcie nemůže být záporná.}.

\begin{equation*}
p \le K
\end{equation*}
\begin{equation*}
P \le K
\end{equation*}
Navíc u evropské opce, která může být uplatněna pouze v době své splatnosti, platí, že v jakémkoliv okamžiku nemůže být její hodnota vyšší než $K \cdot e^{-rT}$.

\subsection{Dolní limit ceny opcí}

\subsubsection{Kupní opce}

Dolní limit hodnoty evropské kupní opce pro akcii, ze které není vyplácena dividenda, je
\begin{equation*}
S_0 - K \cdot e^{-rT}
\end{equation*}

\noindent \textbf{Příklad:} Uvažujme následující dvě portfolia:
\begin{itemize}
\item Portfolio A: evropská kupní opce a hotovost ve výši $K \cdot e^{-rT}$
\item Portfolio B: jedna akcie
\end{itemize}
V případě portfolia A bude hodnota hotovosti v čase maturity opce $T$ rovna $K$ (bude-li investována do instrumentu, který přináší svému vlastníkovi bezrizikovou úrokovou míru $r$). Jestliže $S_T>K$ bude navíc uplatněna opce a portfolio A bude mít hodnotu $S_T$. Jestliže však platí $S_T \le K$, opce uplatněna nebude a hodnota portfolia A bude rovna $K$. Hodnota portfolia A v čase $T$ je tedy $\max(S_T, K)$.\\
Portfolio B má v čase $T$ vždy hodnotu $S_T$ a portfolio A má tedy vždy minimálně hodnotu portfolia B (může však mít také hodnotu vyšší). Proto platí
\begin{equation*}
c+K \cdot e^{-rT} \ge S_0
\end{equation*}
neboli
\begin{equation*}
c \ge S_0 - K \cdot e^{-rT}
\end{equation*}
Vzhledem k tomu, že hodnota opce nemůže být záporná musí platit
\begin{equation*}
c \ge \max(S_0 - K \cdot e^{-rT},0)
\end{equation*}

\subsubsection{Prodejní opce}

Dolní limit hodnoty evropské prodejní opce pro akcii, ze které není vyplácena dividenda, je
\begin{equation*}
K \cdot e^{-rT} - S_0
\end{equation*}
\textbf{Příklad:} Uvažujme následující dvě portfolia:
\begin{itemize}
\item Portfolio C: evropská prodejní opce a jedna akcie
\item Portfolio D: hotovost ve výši $K \cdot e^{-rT}$
\end{itemize}
Vztah mezi oběma portfolii je dán vztahem
\begin{equation*}
p+S_0 \ge K \cdot e^{-rT}
\end{equation*}
přičemž logika věci je obdobná jako v předchozím případě. Vzhledem k tomu, že hodnota opce nemůže být záporná, musí platit
\begin{equation*}
p \ge \max(K \cdot e^{-rT} - S_0, 0)
\end{equation*}

\section{Put-call parita}

Uvažujme dvě portfolia:
\begin{itemize}
\item Portfolio A: evropská kupní opce a hotovost ve výši $K \cdot e^{-rT}$
\item Portfolio C: evropská prodejní opce a jedna akcie
\end{itemize}
Obě portfolia mají hodnotu $\max(S_T,K)$ v době své maturity. Protože obě opce lze uplatnit až době jejich splatnosti, musí být dnešní cena obou portfolií rovna, tj.
\begin{equation*}
c+K \cdot e^{-rT} = p + S_0
\end{equation*}
Tento vztah nazýváme put-call paritou. To znamená, že hodnota evropské prodejní opce může být odvozena od hodnoty evropské kupní ceny a obráceně. Jestliže by totiž výše uvedená rovnice neplatila, vznikl by tímto prostor pro arbitráž.\\
Put-call parita platí pouze pro evropské opce. Pro americké opce platí následující vztah
\begin{equation*}
S_0 - K \le C - P \le S_0 - K \cdot e^{-rT}, 
\end{equation*}
který definuje maximální rozdíl mezi cenou americké kupní a prodejní opce.

\section{Předčasné uplatnění opce}

V případě amerických opcí má jejich majitel právo uplatnit opce před jejich splatností. U evropských opcí tato možnost neexistuje.

\subsection{Předčasné uplatnění kupní opce}

Uvažujme americkou kupní opci na akcii, ze které není vyplácena dividenda, se zbytkovou splatností 1 měsíc. Spotová cena této akcie je 50 USD a realizační cena je 40 USD. To znamená, že opce je "in-the-money". Mohlo by se zdát, že ideální je opci ihned uplatnit.

V případě, že investor plánuje držet akcii, kterou by získal uplatněním opce, nemusí být tato strategie optimální. Jestliže by totiž mezitím cena akcie klesla pod 40 USD, investor by tratil - výhodnější by totiž bylo opci neuplatnit a akcii nakoupit na trhu.

Dokonce ani v případě, kdy investor předvídá propad ceny akcie nemusí být výhodnější opci uplatnit a akcii obratem ruky prodat za 50 USD. Optimální je opci prodat jinému investorovi\footnote{Takový investor musí existovat, jinak by cena akcií nebyla 50 USD.}. Cena této prodané opce bude vyšší než vnitřní hodnota 10 USD, protože musí platit
\begin{equation*}
C \ge S_0 - K \cdot e^{-rT}
\end{equation*}
\begin{center}
	\begin{pspicture}(0,0)(10,6)
		\rput(5.5,0){Kupní opce}

		\psline[arrows=->](0.5,1)(9.5,1)
		\psline[arrows=->](0.5,1)(0.5,5.5)
		\rput(9.5,0.7){$S_0$}
		\rput(1.5,5.2){cena opce}
		
		\pscurve[linewidth=0.5mm](0.5,1)(4,2)(8,5)
		\psline[linestyle=dashed](4,1)(8,4.7)
		\rput(4,0.7){$K$}			
	\end{pspicture}
\end{center}
Výše uvedený obrázek ukazuje, že hodnota kupní opce se mění se změnou $K$ a $S_0$ a že hodnota opce je vždy nad její vnitřní hodnotou $\max(S_0 - K,0)$. S růstem $r$, $T$ a volatility se tento rozdíl zvětšuje - hovoříme o tzv. časové hodnotě opce.

\subsection{Předčasné uplatnění prodejní opce}
Narozdíl od kupní opce může být předčasné uplatnění prodejní opce optimální. Uvažujme situaci, kdy je realizační cena 10 USD a cena akcie se blíží nule. Uplatněním opce tak investor získá 10 USD. Investor nemůže nikdy získat více než 10 USD, a proto je uplatnění opce optimální. Obecně platí, že uplatnění prodejní opce je tím atraktivnější, čím více klesá $S_0$, roste $r$ a klesá volatilita. Pro cenu americké prodejní opce navíc platí $P \ge K - S_0$. Americká prodejní opce by tedy měla být uplatněna, když $S_0 \le A$. V tomto případě je totiž časová hodnota opce záporná. V případě americké opce proto, narozdíl od evropské opce, nemůže dojít k situaci, kdy je vnitřní hodnota opce vyšší než celková hodnota opce, což by implikovalo zápornou časovou hodnotu opce.\\
\begin{center}
	\begin{pspicture}(0,0)(10,6)
		\rput(5,0){Americká prodejní opce: vnitřní a časová hodnota}

		\psline[arrows=->](0.5,1)(9.5,1)
		\psline[arrows=->](0.5,1)(0.5,5.5)
		\rput(9.5,0.7){$S_0$}
		\rput(1.5,5.2){cena opce}
		
		\pscurve[linewidth=0.5mm](0.5,4)(4,2)(8,1.2)
		\psline[linestyle=dashed](3,1)(3,2.45)
		\psline[linestyle=dashed](5,1)(3,2.45)
		\rput(3,0.7){$A$}
		\rput(5,0.7){$K$}			
	\end{pspicture}
\end{center}
\begin{center}
	\begin{pspicture}(0,0)(10,6)
		\rput(5,0){Evropská prodejní opce: vnitřní a časová hodnota}

		\psline[arrows=->](0.5,1)(9.5,1)
		\psline[arrows=->](0.5,1)(0.5,5.5)
		\rput(9.5,0.7){$S_0$}
		\rput(1.5,5.2){cena opce}
		
		\pscurve[linewidth=0.5mm](0.5,4)(4,2)(8,1.2)
		\psline[linestyle=dashed](3,1)(3,2.45)
		\psline[linestyle=dashed](5,1)(0.5,4.26)
		\rput(3,0.7){$A$}
		\rput(5,0.7){$K$}			
	\end{pspicture}
\end{center}

\section{Vliv dividend}

Až dosud jsme předpokládali, že akcie svému majiteli nepřináší žádnou dividendu. V případě opcí, které mají maturitu řádově několik měsíců, je možné výši případné dividendy poměrně přesně odhadnout. Současnou hodnotu dividendy vyplácené v průběhu životnosti opce označíme jako $D$. Pro výpočet $D$ předpokládejme, že výplata dividendy proběhla k tzv. "ex-dividend" dni.

\subsection{Dolní limit pro cenu evropské opce}

\subsubsection{Kupní opce}

Uvažujme následující portfolia:
\begin{itemize}
\item Portfolio A: evropská kupní opce a hotovost $D + K \cdot e^{-rT}$
\item Portfolio B: jedna akcie
\end{itemize}
Pro cenu evropské kupní opce platí
\begin{equation*}
c \ge S_0 - D - K \cdot e^{-rT}
\end{equation*}

\subsubsection{Prodejní opce}

Uvažujme následující dvě portfolia:
\begin{itemize}
\item Portfolio C: evropská prodejní opce a jedna akcie
\item Portfolio D: hotovost ve výši $D+K \cdot e^{-rT}$
\end{itemize}
Pro cenu evropské prodejní opce platí
\begin{equation*}
p \ge D + K \cdot e^{-rT} - S_0
\end{equation*}

\subsection{Předčasné uplatnění americké kupní opce}
Jestliže je očekávána výplata dividend, nemusí vždy platit, že není optimální uplatnit americkou kupní opci předčasně. Někdy může být naopak výhodnější uplatnit opci před "ex-dividend" dnem.

\subsection{Put-call parita}
Jestliže budeme porovnávat výše definovaná portfolia A a C, bude put-call parita převedena do tvaru
\begin{equation*}
c+D+K \cdot e^{-rT} = p + S_0
\end{equation*}
pro evropskou opci popř. do tvaru
\begin{equation*}
S_0 - D -K \le C - P \le S_0 - K \cdot e^{-rT}
\end{equation*}
pro americkou opci.

\section{Opční strategie}

Jednotlivé typy opcí je možné vzájemně kombinovat a modelovat tak nejrůznější výnosové profily. Opce jsou tak ideálním nástrojem pro implementaci investičních strategií.

Všechny opce uvažované v této kapitole jsou evropské. V případě amerických opcí, se kterými je spojeno právo předčasného uplatnění, by výnosový profil jednotlivých strategií byl mírně odlišný.

\subsection{Opce a akcie}

Nejjednodušší opční strategie jsou založeny na kombinaci pokladového aktiva, v našem případě akcie, a opce. 
\begin{center}
	\begin{pspicture}(0,0)(11.5,7.0)
		\rput(5.75,1.0){Výnosové profily opční strategie}
		\rput(5.75,0.5){\small{(a) dlouhá pozice v akcii, krátká pozice v kupní opci}}
		\rput(5.75,0.0){\small{(b) krátká pozice v akcii, dlouhá pozice v kupní opci}}
		\rput(3.0,1.5){\small{(a)}}
		\rput(8.5,1.5){\small{(b)}}

          	\psline[arrows=->](0.5,4.0)(5.5,4.0)
          	\psline[arrows=->](0.5,1.5)(0.5,6.5)
          	\psline(3.0,3.9)(3.0,4.1)
          	\rput(5.5,3.7){$S_T$}
          	\rput(3.0,3.7){$K$}
          
          	\psline[arrows=->](6.0,4.0)(11.0,4.0)
          	\psline[arrows=->](6.0,1.5)(6.0,6.5)
          	\psline(8.5,3.9)(8.5,4.1)
          	\rput(11,3.7){$S_T$}
          	\rput(8.5,4.3){$K$}

		\psline[linestyle=dashed](0.5,4.5)(3.0,4.5)(5.5,2.5)
		\psline[linestyle=dashed](0.5,2.2)(5.0,6.7)
		\psline[linewidth=0.5mm](0.5,2.6)(3.0,5.1)(5.5,5.1)

		\psline[linestyle=dashed](6.0,3.5)(8.5,3.5)(11.0,6.0)
		\psline[linestyle=dashed](6.0,5.8)(10.0,1.8)
		\psline[linewidth=0.5mm](6.0,5.4)(8.5,2.8)(11.0,2.8)
	\end{pspicture}
\end{center}
\begin{center}
	\begin{pspicture}(0,0)(11.5,7.0)
		\rput(5.75,1.0){Výnosové profily opční strategie}
		\rput(5.75,0.5){\small{(c) dlouhá pozice v akcii, dlouhá pozice v prodejní opci}}
		\rput(5.75,0.0){\small{(d) krátká pozice v akcii, krátká pozice v prodejní opci}}
		\rput(3.0,1.5){\small{(c)}}
		\rput(8.5,1.5){\small{(d)}}

          	\psline[arrows=->](0.5,4.0)(5.5,4.0)
          	\psline[arrows=->](0.5,1.5)(0.5,6.5)
          	\psline(3.0,3.9)(3.0,4.1)
          	\rput(5.5,3.7){$S_T$}
          	\rput(3.0,4.3){$K$}
          
          	\psline[arrows=->](6.0,4.0)(11.0,4.0)
          	\psline[arrows=->](6.0,1.5)(6.0,6.5)
          	\psline(8.5,3.9)(8.5,4.1)
          	\rput(11,3.7){$S_T$}
          	\rput(8.7,3.7){$K$}

		\psline[linestyle=dashed](0.5,5.8)(3.0,2.8)(5.5,2.8)
		\psline[linestyle=dashed](0.5,2.2)(5.0,6.7)
		\psline[linewidth=0.5mm](0.5,3.7)(3.0,3.7)(5.5,6.2)

		\psline[linestyle=dashed](6.0,2.2)(8.5,4.7)(11.0,4.7)
		\psline[linestyle=dashed](6.0,6.1)(10.0,2.1)
		\psline[linewidth=0.5mm](6.0,4.3)(8.5,4.3)(11.0,1.8)
	\end{pspicture}
\end{center}

Výnosový profil výše uvedených příkladů (a), (b), (c) a (d) má podobný tvar jako elementární evropské opce. Ten v příkladě (a) odpovídá krátké pozici v prodejní opci, v příkladě (b) dlouhé pozici v prodejní opci, v příkladě (c) dlouhé pozici v kupní opci a v příkladě (d) krátké pozici v kupní opci. Toto zjištění je v souladu s již dříve zmiňovanou put-call paritou
\begin{equation}
p + S_0 = c + Ke^{-rT} + D
\end{equation}
kde $p$ je cenou evropské kupní opce, $S_0$ spotovou cenou akcie, $c$ cenou evropské prodejní opce, $K$ realizační cenou společnou prodejní i kupní opci, $r$ bezrizikovou úrokovou mírou, $T$ zbytkovou splatností obou opcí a $D$ současnou hodnotou očekávaných dividend vyplácených z podkladové akcie.

Rovnice (7.1) tak například říká, že dlouhá pozice v prodejní opci a dlouhá pozice v akcii je ekvivalentní dlouhé pozici v kupní opci a navýšené o objem hotovosti ve výši $Ke^{-rT}+D$. Tímto se vysvětluje proč má případ (a) podobný výnosový profil jako krátká pozice v prodejní opci. Elementárními úpravami (7.1) lze odvodit také výnosové profily pro příklady (b), (c) a (d).

\subsection{Spready}

Opční strategie založené na spreadech předpokládají, že investor má pozice ve dvou nebo více opcích stejného druhu. Nejběžnějšími druhy těchto strategií jsou býčí, medvědí, motýlí a kalendářní spread. Každý z těchto spreadů je možné vytvořit z kupních i prodejních opcí.

\subsubsection{Býčí spread}

Býčí spread (bull spread) je možné vytvořit dlouhou pozicí v evropské kupní opci s realizační cenou $K_1$ a krátkou pozicí v evropské kupní opci s realizační cenou $K_2$, kde $K_2 > K_1$. Obě uvažované opce mají stejnou zbytkovou splatnost. Protože je hodnota kupní opce nepřímo závislá na realizační ceně, vyžaduje býčí spread vytvořený z kupních opcí počáteční investici.
\begin{center}
	\begin{pspicture}(0,0)(10.0,6.0)
		\rput(5.0,0.0){Výnosový profil býčího spreadu vytvořeného z kupních opcí}

          	\psline[arrows=->](0.5,3.0)(9.5,3.0)
          	\psline[arrows=->](0.5,0.5)(0.5,5.5)
          	\psline(4.0,2.9)(4.0,3.1)
          	\psline(6.0,2.9)(6.0,3.1)
          	\rput(9.5,2.7){$S_T$}
          	\rput(4.0,2.7){$K_1$}
          	\rput(6.0,2.7){$K_2$}

		\psline[linestyle=dashed](0.5,3.5)(6.0,3.5)(9.5,0.5)
		\psline[linestyle=dashed](0.5,1.5)(4.0,1.5)(8.0,5.5)
		\psline[linewidth=0.5mm](0.5,2.0)(4.0,2.0)(6.0,4.0)(9.5,4.0)

	\end{pspicture}
\end{center}
Z výše uvedeného obrázku je patrné, že býčí spread limituje riziko investora shora i zdola.

Kromě kupních opcí je možné býčí spread zkonstruovat také z prodejních opcí. Narozdíl od předchozího případu generuje vytvoření býčího spreadu z prodejních opcí pozitivní cash-flow z titulu opčních prémií.
\begin{center}
	\begin{pspicture}(0,0)(10.0,6.0)
		\rput(5.0,0.0){Výnosový profil býčího spreadu vytvořeného z prodejních opcí}

          	\psline[arrows=->](0.5,3.0)(9.5,3.0)
          	\psline[arrows=->](0.5,0.5)(0.5,5.5)
          	\psline(3.0,2.9)(3.0,3.1)
          	\psline(6.0,2.9)(6.0,3.1)
          	\rput(9.5,2.7){$S_T$}
          	\rput(3.0,2.7){$K_1$}
          	\rput(6.0,2.7){$K_2$}

		\psline[linestyle=dashed](0.5,4.0)(3.0,2.2)(9.5,2.2)
		\psline[linestyle=dashed](1.5,0.5)(6.0,5.0)(9.5,5.0)
		\psline[linewidth=0.5mm](0.5,1.0)(3.0,1.0)(6.0,4.0)(9.5,4.0)

	\end{pspicture}
\end{center}

\subsubsection{Medvědí spread}

V případě býčího spreadu spekuluje investor na vzestup ceny podkladové akcie. U medvědího spreadu (bear spread) je tomu naopak. Medvědí spread se konstruuje podobným způsobem jako býčí spread. K vytvoření medvědího spreadu je třeba prodat evropskou kupní opci s realizační cenou $K_1$ a koupit evropskou kupní opci s realizační cenou $K_2$, kde $K_1 < K_2$. Obě kupní opce mají stejnou zbytkovou splatnost. Medvědí spread je tedy zrcadlový k býčímu spreadu a jeho vytvoření z kupních opcí generuje kladné cash-flow.
\begin{center}
	\begin{pspicture}(0,0)(10.0,6.0)
		\rput(5.0,0.0){Výnosový profil medvědího spreadu vytvořeného z kupních opcí}

          	\psline[arrows=->](0.5,3.0)(9.5,3.0)
          	\psline[arrows=->](0.5,0.5)(0.5,5.5)
          	\psline(3.0,2.9)(3.0,3.1)
          	\psline(6.0,2.9)(6.0,3.1)
          	\rput(9.5,2.7){$S_T$}
          	\rput(3.0,2.7){$K_1$}
          	\rput(6.0,2.7){$K_2$}

		\psline[linestyle=dashed](0.5,5.0)(3.0,5.0)(7.5,0.5)
		\psline[linestyle=dashed](0.5,2.0)(6.0,2.0)(9.5,5.5)
		\psline[linewidth=0.5mm](0.5,4.0)(3.0,4.0)(6.0,1.0)(9.5,1.0)

	\end{pspicture}
\end{center}
Podobně jako býčí spread i medvědí spread stanovuje limity pro maximální zisk popř. ztrátu, které může investor dosáhnout.

Vedle kupních opcí je možné medvědí spread sestavit také z prodejních opcí. V tomto případě však konstrukce medvědího spreadu vyžaduje počáteční investici titulu opční prémií.
\begin{center}
	\begin{pspicture}(0,0)(10.0,6.0)
		\rput(5.0,0.0){Výnosový profil medvědího spreadu vytvořeného z prodejních opcí}

          	\psline[arrows=->](0.5,3.0)(9.5,3.0)
          	\psline[arrows=->](0.5,0.5)(0.5,6.0)
          	\psline(3.0,2.9)(3.0,3.1)
          	\psline(6.0,2.9)(6.0,3.1)
          	\rput(9.5,2.7){$S_T$}
          	\rput(3.0,2.7){$K_1$}
          	\rput(6.0,3.3){$K_2$}

		\psline[linestyle=dashed](1.0,5.5)(6.0,0.5)(9.5,0.5)
		\psline[linestyle=dashed](0.5,1.5)(3.0,4.0)(9.5,4.0)
		\psline[linewidth=0.5mm](0.5,4.5)(3.0,4.5)(6.0,1.5)(9.5,1.5)

	\end{pspicture}
\end{center}

\subsubsection{Motýlí spread}

Motýlí spread (butterfly spread) vzniká kombinací čtyř opcí s rozdílnou realizační cenou a shodnou zbytkovou splatností. Motýlí spread je možné vytvořit krátkou pozicí ve dvou evropských kupních opcích s realizační cenou $K_2$ a dlouhou pozicí v evropských kupních opcí s realizačními cenami $K_1$ a $K_3$. Pro realizační ceny uvažovaných opcí platí $K_2 = (K_1 + K_3)/2$.
\begin{center}
	\begin{pspicture}(0,0)(10.0,6.0)
		\rput(5.0,0.0){Výnosový profil motýlího spreadu vytvořeného z kupních opcí}

          	\psline[arrows=->](0.5,3.0)(9.5,3.0)
          	\psline[arrows=->](0.5,0.5)(0.5,6.0)
          	\psline(3.0,2.9)(3.0,3.1)
          	\psline(4.5,2.9)(4.5,3.1)
          	\psline(6.0,2.9)(6.0,3.1)
          	\rput(9.5,2.7){$S_T$}
          	\rput(3.0,3.3){$K_1$}
          	\rput(4.2,2.8){$K_2$}
          	\rput(6.2,3.3){$K_3$}

		\psline[linestyle=dashed](0.5,1.0)(3.0,1.0)(7.5,5.5)
		\psline[linestyle=dashed](0.5,2.5)(6.0,2.5)(9.0,5.5)
		\psline[linestyle=dashed](0.5,4.5)(4.5,4.5)(6.5,0.5)
		\psline[linewidth=0.5mm](0.5,2.0)(3.0,2.0)(4.5,3.5)(6.0,2.0)(9.5,2.0)

	\end{pspicture}
\end{center}

Motýlí spread generuje výnos za předpokladu, že cena pokladové akcie je v době splatnosti opcí v okolí $K_2$. Sestavení motýlího spreadu pomocí kupních opcí vyžaduje počáteční investici, která je však v porovnání s nominální hodnotou spreadu relativně malá.

Podobně jako v předchozích případech je možné motýlí spread sestavit z prodejních opcí. Také v tomto případě je zapotřebí počáteční investice. Pomocí put-call parity definované pomocí (7.1) je možné dokázat, že výše této investice se shoduje s investicí, která je zapotřebí pro konstrukci motýlího spreadu z kupních opcí.
\begin{center}
	\begin{pspicture}(0,0)(10.0,6.0)
		\rput(5.0,0.0){Výnosový profil motýlího spreadu vytvořeného z prodejních opcí}

          	\psline[arrows=->](0.5,3.0)(9.5,3.0)
          	\psline[arrows=->](0.5,0.5)(0.5,6.0)
          	\psline(3.0,2.9)(3.0,3.1)
          	\psline(4.5,2.9)(4.5,3.1)
          	\psline(6.0,2.9)(6.0,3.1)
          	\rput(9.5,3.3){$S_T$}
          	\rput(3.0,3.3){$K_1$}
          	\rput(4.2,2.8){$K_2$}
          	\rput(6.0,3.3){$K_3$}

		\psline[linestyle=dashed](0.5,4.5)(3.0,2.5)(9.5,2.5)
		\psline[linestyle=dashed](1.5,0.5)(4.5,4.5)(9.5,4.5)
		\psline[linestyle=dashed](2.0,5.5)(6.0,1.5)(9.5,1.5)
		\psline[linewidth=0.5mm](0.5,2.5)(3.0,2.5)(4.5,3.5)(6.0,2.5)(9.5,2.5)

	\end{pspicture}
\end{center}

\subsubsection{Kalendářní spread}

Kalendářní spread (calendar spread) je možné rozložit na krátkou pozici v evropské kupní opci a dlouhou pozici v evropské kupní opci. Obě uvažované opce mají stejnou realizační cenu, avšak opce, ve které je investor dlouhý, má delší dobu splatnosti. Vzhledem k tomu, že hodnota opce je rostoucí funkcí splatnosti opce, vyžaduje sestavení kalendářního spreadu z kupních opcí počáteční investici. 
\begin{center}
	\begin{pspicture}(0,0)(10.0,6.0)
		\rput(5.0,0.0){Výnosový profil kalendářního spreadu vytvořeného z kupních opcí}

          	\psline[arrows=->](0.5,3.0)(9.5,3.0)
          	\psline[arrows=->](0.5,0.5)(0.5,6.0)
          	\psline(4.5,2.9)(4.5,3.1)
          	\rput(9.5,3.3){$S_T$}
          	\rput(4.5,3.3){$K$}

		\psline[linestyle=dashed](0.5,4.5)(4.5,4.5)(8.5,0.5)
		\pscurve[linestyle=dashed](0.5,1.0)(4.5,2.6)(8.0,6.0)
		\pscurve[linewidth=0.5mm](0.5,1.5)(2.8,2.6)(4.5,4.0)
		\pscurve[linewidth=0.5mm](4.5,4.0)(5.2,3.6)(7.2,3.0)(9.5,2.5)

	\end{pspicture}
\end{center}
Výše uvedeného výnosového profilu je docíleno tak, že v době splatnosti první opce je prodána opce s delší splaností. Tvar výsonového profilu připomíná motýlí spread. Kalendářní spread tak svému majiteli generuje výnos v případě, že se spotová cena pokladové akcie v době splatnosti první opce nachází v okolí realizační ceny.

Uvažujme situaci, kdy se v době splatnosti první opce cena pokladové akcie blíží nule. Splatná opce je bezcenná a hodnota druhé opce se blíží nule. Investor tedy utrpěl ztrátu, jejíž výše přibližně odpovídá počáteční investici potřebné pro vytvoření kalendářního spreadu. Dále uvažujme situaci, kdy je v době splatnosti první opce spotová cena $S_T$ pokladové akcie vzhledem k realizační ceně $K$ velmi vysoká. První opce tedy stojí investora v době své splatnosti $S_T - K$ a současně druhá opce má z pohledu investora hodnotu o něco málo vyšší než $S_T - K$. Investor opět utrpí ztrátu, která je blízká počáteční investici spojené s vytvoření kalendářního spreadu. Jestliže naopak v době splatnosti první opce bude spotová cena pokladové akcie blízká realizační ceně, je hodnota první opce nulová nebo blízká nule, kdežto hodnota druhé z opcí je kladná.\\

Kalendářní spread může být vytvořen také z prodejních opcí a to tak, že investor koupí evropskou prodejní opci s delší splatností a prodá evropskou prodejní opci s kratší splatností.
\begin{center}
	\begin{pspicture}(0,0)(10.0,6.0)
		\rput(5.0,0.0){Výnosový profil kalendářního spreadu vytvořeného z prodejních opcí}

          	\psline[arrows=->](0.5,3.0)(9.5,3.0)
          	\psline[arrows=->](0.5,0.5)(0.5,6.0)
          	\psline(4.5,2.9)(4.5,3.1)
          	\rput(9.5,3.3){$S_T$}
          	\rput(4.5,2.7){$K$}

		\psline[linestyle=dashed](1.0,1.0)(4.5,4.5)(8.5,4.5)
		\pscurve[linestyle=dashed](0.5,5.0)(4.3,2.2)(9.0,1.0)
		\pscurve[linewidth=0.5mm](0.5,2.5)(3.0,3.0)(4.5,3.7)
		\pscurve[linewidth=0.5mm](4.5,3.7)(6.5,2.9)(9.0,2.5)
	\end{pspicture}
\end{center}

\subsection{Kombinace}

Až dosud jsme v rámci opční strategie uvažovali pouze jeden typ opcí. Kombinace je naproti tomu opční strategií, která zahrnuje pozice jak v kupních tak prodejních opcích. Mezi nejznáměší kombinace patří straddle a strangle.

\subsubsection{Straddle}

Opční strategii typu straddle lze rozložit na dlouhou pozici v evropské kupní opci a evropské prodejní opc. Obě opce mají stejnou splatnost a realizační cenu.
\begin{center}
	\begin{pspicture}(0,0)(10.0,6.0)
		\rput(5.0,0.0){Výnosový profil straddle vytvořeného nákupem opcí}

          	\psline[arrows=->](0.5,3.0)(9.5,3.0)
          	\psline[arrows=->](0.5,0.5)(0.5,6.0)
          	\psline(4.5,2.9)(4.5,3.1)
          	\rput(9.5,3.3){$S_T$}
          	\rput(4.5,3.3){$K$}

		\psline[linestyle=dashed](0.5,2.5)(4.5,2.5)(7.5,5.5)
		\psline[linestyle=dashed](0.5,5.5)(4.5,1.5)(9.0,1.5)
		\psline[linewidth=0.5mm](0.5,5.0)(4.5,1.0)(8.5,5.0)
	\end{pspicture}
\end{center}
Jestliže je spotová cena podkladové akcie v době splatnosti blízká realizační ceně, generuje výše uvedený straddle ztrátu. Naopak v případě, že se spotová cena od realizační ceny výrazněji odchýlí, dosahuje investor zisku. Z výsového profilu je tedy zřejmé, že tato opční strategie se hodí v situacích, kdy investor očekává výraznější změnu ceny podkladové akcie, avšak není si jistý jejím směrem. Nicméně v situacích, kdy trh očekává změnu ceny podkladové akcie, bude toto očekávání již promítnuto v ceně opcí. Cena opcí tak bude výrazně vyšší než v případě, kdy trh žádnou změnu neočekává.

Kromě výše uvedeného příkladu je možné straddle sestavit také tak, že investor prodá evropskou kupní opci a evropskou prodejní opci se stejnou splatností a realizační cenou. V tomto případě však bude investor naopak spekulovat na to, že se spotová cena v době splanosti opcí výrazněji neodchýlí od realizační ceny. Výnosový profil této strategie tedy připomíná motýlí spread s tím rozdílem, že možná ztráta investora není limitována.
\begin{center}
	\begin{pspicture}(0,0)(10.0,6.0)
		\rput(5.0,0.0){Výnosový profil straddle vytvořeného prodejem opcí}

          	\psline[arrows=->](0.5,3.0)(9.5,3.0)
          	\psline[arrows=->](0.5,0.5)(0.5,6.0)
          	\psline(4.5,2.9)(4.5,3.1)
          	\rput(9.5,3.3){$S_T$}
          	\rput(4.5,2.7){$K$}

		\psline[linestyle=dashed](0.5,3.5)(4.5,3.5)(7.5,0.5)
		\psline[linestyle=dashed](1.0,1.0)(4.5,4.5)(8.5,4.5)
		\psline[linewidth=0.5mm](0.5,1.0)(4.5,5.0)(8.5,1.0)
	\end{pspicture}
\end{center}

\subsubsection{Strangles}

Strangles je opční strategie velice podobná straddle s tím rozdílem, že uvažované opce mají různou realizační cenu. Opční strategii stangles je možné vytvořit jak prodejem tak nákupem opcí. Jako příklad uveďme strangle vytvořeného nákupem opcí.
\begin{center}
	\begin{pspicture}(0,0)(10.0,6.0)
		\rput(5.0,0.0){Výnosový profil strangle vytvořeného nákupem opcí}

          	\psline[arrows=->](0.5,3.0)(9.5,3.0)
          	\psline[arrows=->](0.5,0.5)(0.5,6.0)
          	\psline(3.5,2.9)(3.5,3.1)
                \psline(6.5,2.9)(6.5,3.1)
          	\rput(9.5,3.3){$S_T$}
          	\rput(3.5,3.3){$K_1$}
                \rput(6.6,3.3){$K_2$}

		\psline[linestyle=dashed](0.5,2.5)(6.5,2.5)(9.0,5.0)
		\psline[linestyle=dashed](0.5,5.0)(3.5,2.0)(9.0,2.0)
		\psline[linewidth=0.5mm](0.5,4.5)(3.5,1.5)(6.5,1.5)(9.5,4.5)
	\end{pspicture}
\end{center}


\chapter {Úvod do binomických stromů}
Binomický strom je diagram, který zobrazuje možný vývoj ceny akcie po dobu životnosti akcie.
\begin{center}
	\begin{pspicture}(0,0)(7,4)
		\rput(3.5,0.5){Binomický strom: vývoj ceny akcie a kupní opce}
		\rput(3.5,0){$K$ = 21~USD, $S$ - spotová cena, $c$ - cena kupní opce}

		\psline[arrows=->](1.5,2.5)(5.5,3.5)
		\psline[arrows=->](1.5,2.5)(5.5,1.5)
		\rput(0.8,2.5){$S_0$=20}
		\rput(6.2,3.5){$S_T^{a}$=22}
		\rput(6.2,3.2){$c$=1}
		\rput(6.2,1.5){$S_T^{b}$=18}
		\rput(6.2,1.2){$c$=0}	
	\end{pspicture}
\end{center}

Uvažujme evropskou opci, která nám dává právo koupit akcii za 21 USD za tři měsíce. Spotová cena akcie je 20 USD. Zjednodušeně předpokládejme, že na konci těchto tří měsíců může být cena akcie 22 nebo 18 USD a to se stejnou pravděpodobností. V prvním případě bude hodnota opce 1 USD, druhém pak nulová.

Mějme portfolio skládající se z $\Delta$ akcií a krátké pozice jedné kupní opce. Aby na konci uvažovaného tříměsíční období neexistovala možnosti arbitráže, musí platit
\begin{equation*}
22 \Delta - 1 = 18 \Delta - 0
\end{equation*}
\begin{equation*}
\Delta = 0.25
\end{equation*}
Bez ohledu na budoucí vývoj bude hodnota portfolia v naší modelové situaci po třech měsících 4.50 USD. Dané portfolio je tedy bezrizikové. Při neexistenci arbitráže musí bezrizikové portfolio přinášet výnos ve výši bezrizikové úrokové sazby. Jestliže je bezriziková úroková sazba 12\% p.a., je současná hodnota portfolia
\begin{equation*}
4.50 \cdot e^{-0.12 \cdot 3/12} = 4.367
\end{equation*}
Spotová hodnota akcie je 20 USD. Hodnota opce $f$ je pak
\begin{equation*}
20 \cdot 0.25 - f = 4.367
\end{equation*}
\begin{equation*}
f = 0.633
\end{equation*}
Jestliže by cena opce byla více než 0.633 USD, portfolio by stálo méně než 4.367 USD a výnos by byl vyšší než bezriziková výnosová míra. Jestliže by cena opce byla méně než 0.633, představovala by krátká pozice v portfoliu způsob půjčení si peněz za méně než bezrizikovou výnosovou míru.\\

Uvažujme akcii, jejíž spotová cena je $S_0$ a opci, jejíž současná cena je $f$. Předpokládejme, že životnost opce je $T$ a že v průběhu této doby může dojít k růstu ceny akcií na $S_0u$ nebo poklesu na $S_0d$, kde $u > 1$ a $d < 1$. Procentní růst cen akcií je $u-1$; procentní pokles pak $1-d$. Jestliže se cena akcie zvýší na $S_0u$, bude cena opce $f_u$, v případě poklesu ceny akcie na $S_0d$ bude cena opce $f_d$. Portfolio se skládá z $\Delta$ akcií a krátké pozice v jedné opci.\\
Jestliže cena akcie poroste, je hodnota portfolia
\begin{equation*}
S_0u \cdot \Delta - f_u
\end{equation*}
a v případě poklesu cen akcie pak
\begin{equation*}
S_0d \cdot \Delta - f_d
\end{equation*}
Při neexistenci arbitráže musí platit
\begin{equation*}
S_0u \cdot \Delta - f_u = S_0d \cdot \Delta - f_d
\end{equation*}
\begin{equation}
\Delta = \frac{f_u - f_d}{S_0u - S_0d}
\end{equation}
Bezrizikové portfolio musí přinášet výnos odpovídající bezrizikové úrokové míře. Hodnota portfolia je tedy
\begin{equation*}
(S_0u \cdot \Delta - f_u) \cdot e^{-rT}
\end{equation*}
Náklady na pořízení portfolia jsou
\begin{equation*}
S_0 \cdot \Delta - f
\end{equation*}
z čehož vyplývá
\begin{equation*}
S_0 \cdot \Delta - f = (S_0u \cdot \Delta - f_u) \cdot e^{-rT}
\end{equation*}
\begin{equation}
f = S_0 \cdot \Delta - (S_0u \cdot \Delta - f_u) \cdot e^{-rT}
\end{equation}
Dosazením (8.1) za $\Delta$ do (8.2) a s využitím vztahu
\begin{equation}
p = \frac{e^{rT}-d}{u-d}
\end{equation}
lze cenu opce vyjádřit do tvaru
\begin{equation}
f = e^{-rT}[p \cdot f_u + (1-p) \cdot f_d]
\end{equation}

\section{Rizikově neutrální ocenění}

Proměnou $p$ ve vztahu (8.3) je možné interpretovat jako pravděpodobnost růstu ceny akcie a výraz
\begin{equation*}
p \cdot f_u + (1-p) \cdot f_d
\end{equation*}
pak jako očekávanou hodnotu opce. Současná hodnota opce je pak dána (8.4).\\

Nechť $p$ je pravděpodobnost růstu ceny akcie. Pak platí, že očekávaná cena akcie je
\begin{equation*}
E[S_T] = p \cdot S_0u + (1-p)S_0d
\end{equation*}
\begin{equation*}
E[S_T] = p \cdot S_0(u-d) + S_0d
\end{equation*}
Dosazením za $p$ dostáváme
\begin{equation}
E[S_T] = S_0 \cdot e^{rT},
\end{equation}
čímž jsem dokázali, že akcie roste v průměru tempem odpovídajícímu bezrizikové úrokové míře.
V rizikově neutrálním "světě" jsou všichni investoři rizikově neutrální a nepožadují kompenzaci za podstupované riziko. Všechny investice tak přinášejí svému majiteli v průměru výnos odpovídající bezrizikové úrokové míře. Tomuto závěru také odpovídá zjištění zformulované vztahem (8.5) - hovoříme o tzv. konceptu rizikově neutrálního oceňování.\\

\noindent \textbf{Příklad:} Dosazením do (8.3) podle zadání výše uvedeného příkladu dostáváme

\begin{equation*}
p = \frac{e^{0.12 \cdot \frac{3}{12}} - \frac{18}{20}}{\frac{22}{20} - \frac{18}{20}}
\end{equation*}
\begin{equation*}
p = 0.6523
\end{equation*}
Dosazením do (8.4) pak vypočteme hodnotu opce.
\begin{equation*}
f= e^{-0.12 \cdot 3/12}[0.6523 \cdot 1 + 0.3477 \cdot 0]
\end{equation*}
\begin{equation*}
f = 0.63302
\end{equation*}

Pravděpodobnost $p$ je pravděpodobností růstu cen akcií v rizikově neutrálním "světě". Nejedná se o pravděpodobnost růstu cen akcie v reálném světě. Investoři totiž požadují kompenzaci za podstupované riziko - namísto 12\% p.a. by tak mohli požadovat např. 16\% p.a. V tomto případě by pravděpodobnost $p$ byla
\begin{equation*}
p = \frac{e^{0.16 \cdot 3/12} - \frac{18}{20}}{\frac{22}{20} - \frac{18}{20}}
\end{equation*}
\begin{equation*}
p = 0.7041
\end{equation*}

\section{Dvoustupňový binomický strom}

Uvažujme akcii, jejíž cena je 20 USD. Tato cena se může v každém ze dvou kroků zvýšit nebo snížit o 10\%. Předpokládejme, že délka těchto kroků je jeden měsíc a že bezriziková úroková sazba je 12\% p.a. Realizační cena opce je 21 USD. Při výpočtu ceny opce lze postupovat analogicky jako ve výše popsaném případě jednostupňového binomického stromu.

Nejprve namodeluje možné ceny akcie $S_2$ na konci druhého roku a jednotlivým cenám přiřadíme pravděpodobnost, s kterou nastanou. Na základě namodelovaných cen určíme výnosy z opce na konci druhého kroku (tj. $\max(S_2 - s,0$, kde $S_2$ je cena akcie na konci druhého kroku a $s$ je realizační cena). Tyto výnosy pak násobíme odpovídajcí pravděpodobností a výsledné číslo diskontujeme bezrizikovou úrokovou mírou k $T_0$. Výsledná cena opce je v našem konkrétním případě 1.2823 USD. Obdobný postup je možné aplikovat i na vícestupňové binomické stromy.\\

\subsection{Americká opce}

Výše popsaný způsob lze uplatnit při oceňování evropské opce. Podobný mechanismus oceňování lze také použít pro americkou opci. Zásadní rozdíl oproti evropské opci spočívá v tom, že americká opce může být předčasně uplatněna. Proto nestačí zabývat se pouze cenou opce v posledním kroku binomického stromu, ale je nutné uvažovat také kroky předchozí a určit, zda-li je předčasné uplatnění opce optimální či nikoliv.

Uvažujme binomický strom o $n$ krocích. Při oceňování americké opce se postupuje od konce binomického stromu směrem k jeho počátku. V koncových uzlech $n$-tého kroku binomického stromu se americká opce oceňuje hodnotou danou (8.4). V uzlech $m$-tého kroku, kde $m < n$, je hodnota opce dána vyšší z následujících dvou hodnot:
\begin{itemize}
\item současnou hodnotou váženého průměru ceny opce v sousedících uzlech kroku $m+1$, kde váhy jsou pravděpodobostmi přesunu z uvažovaného uzlu do odpovídajícího sousedícího uzlu
\item výnosem z okamžitého uplatnění opce
\end{itemize}
Při ocenění americké opce je třeba projít všechny uzly binomického stromu. Hodnota americké opce v čase nula je rovna hodnotě ve výchozím uzlu binomického stromu. Pro diskontování hodnot mezi jednotlivými kroky binomického stromu se používá bezriziková sazba.

\section{Výpočet parametru $d$ a $u$}

Předpokládejme, že očekávaná míra růstu cen akcií je $\mu$ a jeho volatilita $\sigma$. Nechť je časová délka kroku binomického stromu $\delta t$. V každém kroku cena akcie buďto vzroste nebo poklesne. Pravděpodobnost růstu ceny akcie je $q$\footnote{Pravděpodobnost $q$ na rozdíl od $p$ značí pravděpodobnost růstu akcie v reálném světě.}. Očekávaná cena na konci prvního kroku je $S_0 e^{\mu \delta t}$. Dle logiky binomického stromu lze tuto cenu také vyjádřit jako
\begin{equation*}
q S_0u + (1-q)S_0d
\end{equation*}
Proto musí platit
\begin{equation}
q S_0u + (1-q)S_0d = S_0 e^{\mu \delta t}
\end{equation}
\begin{equation*}
q = \frac {e^{\mu \delta t} - d}{u - d}
\end{equation*}
Směrodatná odchylka výnosové míry akcie v čase $\delta t$ je s ohledem na zadání $\sigma \sqrt{\delta t}$. Volatilu výnosové míry akcie lze pomocí (8.6) vyjádřit také jako
\begin{equation*}
q u^2 + (1-q)d^2 - [qu+(1-q)d]^2
\end{equation*}
Musí tedy platit
\begin{equation*}
q u^2 + (1-q)d^2 - [qu+(1-q)d]^2 = \sigma^2 \delta t
\end{equation*}
Dosazením za parametr $q$ dostáváme po následných úpravách
\begin{equation*}
e^{\mu \delta t}(u+d)-ud-e^{2 \mu \delta t}= \sigma^2 \delta t
\end{equation*}

Vzhledem k tomu, že pro malá $\delta t$ konverguje člen řádu $\delta t^2$ k nule, a při podmínce $ud = 1$, je řešením výše uvedené rovnice
\begin{equation*}
u = e^{\sigma \sqrt{\delta t}}
\end{equation*}
\begin{equation*}
d = e^{-\sigma \sqrt{\delta t}}
\end{equation*}

\noindent \textbf{Poznámka:} Platí, že volatilita v rizikově neutrálním modelu a reálném světě je stejná, ačkoliv požadované výnosové míry jsou odlišné. Výše popsaný postup lze tedy aplikovat v obou případech.

\subsection{Delta opce}
Delta akciové opce je poměr změny ceny opce ku změně ceny akcie. Delta tak představuje počet akcií, které je třeba držet na jednu opci, abychom tak vytvořili bezrizikové portfolio. Delta kupní opce je kladná, delta prodejní opce pak záporná.

\begin{equation*}
\Delta = \frac{f_1 - f_0}{S_1 - S_0}
\end{equation*}

\noindent Hodnota parametru $\Delta$ se mění v čase. Proto je třeba pozici průběžně korigovat.

\chapter{Modelování cen akcií}
O proměnné, jejíž hodnota v čase není určena deterministicky, říkáme, že sleduje stochastický proces. Tento proces pak může být definován diskrétně nebo spojitě.

\section{Markovův proces}
Markovův proces je jeden z typů stochastických procesů, který je používán pro modelování cen finančních instrumentů. Pro predikci budoucí ceny je rozhodující pouze současná cena - minulý vývoj cen je tak irelevantní. Tato teorie tedy předpokládá alespoň slabou formu efektivity trhu.
O akciích se většinou předpokládá, že sledují Markovův proces.

\section{Spojité pojetí stochastického modelu}
Uvažujme proměnnou, která sleduje Markovův proces. Předpokládejme, že se meziroční změna ceny akcie řídí pravděpodobnostním rozdělením $\phi(0,1)$\footnote{Pravděpodobnostním rozdělením $\phi(\mu, \sigma)$ budeme označovat normální rozdělení se střední hodnotou $\mu$ a směrodatnou odchylkou $\sigma$.}. Vzhledem k nezávislosti meziročních změn se změna ceny akcie v průběhu $n$ let řídí pravděpodobnostním rozdělením $\phi(0, \sqrt{n})$.

\subsection{Wienerův proces}

Výše popsaný proces nazýváme Wienerův proces a jedná se o konkrétní případ Markovova procesu. Tento proces se používá ve fyzice k modelování pohybů částic, kde je známější pod pojmem Brownův pohyb.\\

\noindent Wienerův proces má následující dvě základní vlastnosti:
\begin{itemize}
\item Změna ceny $\delta z$ v průběhu malé časové periody $\delta t$ je
	\begin{equation*}
	\delta z = \epsilon \sqrt{\delta t},
	\end{equation*}
kde $\epsilon$ je náhodný výběr z pravděpodobnostního rozdělení $\phi(0,1)$.
\item Hodnoty $\delta z$ jsou pro libovolné dva rozdílné časové intervaly $\delta t$ vzájemně nezávislé.
\end{itemize}
Z těchto dvou vlastností pak vyplývá
\begin{itemize}
\item $E[\delta z] = 0$
\item $D[\delta z] = \delta t$
\end{itemize}

Předpokládejme, že došlo k růstu ceny akcie v relativně dlouhém časovém horizontu $T$. Tuto změnu označme $[z(T)-z(0)]$. Změna ceny akcie může být považována za sumu $N$ dílčích změn, které nastaly v časovém intervalu $\delta t$
\begin{equation*}
N = \frac{T}{\delta t}
\end{equation*}
\begin{equation*}
\epsilon_i \in \phi(0,1)
\end{equation*}
\begin{equation*}
z(T)-z(0) = \sum_{i = 1}^N \epsilon_i \sqrt{\delta t}
\end{equation*}
Pro náhodnou veličinu $[z(T)-z(0)]$ platí
\begin{itemize}
\item $E[z(T)-z(0)] = 0$
\item $D[z(T)-z(0)] = N \cdot \delta t = T$
\end{itemize}

Zobecněný Wienerův proces proměnné $x$ může být pomocí výše definované proměnné $d z$ popsán jako
\begin{equation*}
d x = a \cdot d t + b \cdot d z
\end{equation*}
Člen $a \cdot d t$ značí tzv. trend, který náhodná veličina $x$ sleduje, a člen $b \cdot d z$ představuje oscilaci kolem tohoto trendu. Pro malý časový interval $\delta t$ platí
\begin{equation*}
\delta x = a \cdot \delta t + b \cdot \epsilon \sqrt{\delta t},
\end{equation*}
kde $\epsilon \in \phi(0,1)$. To znamená, že
\begin{itemize}
\item $E[\delta x] = a \cdot \delta t$
\item $D[\delta x] = b^2 \cdot \delta t$
\end{itemize}

Investory spíše než absolutní změna ceny akcie zajímá jejich relativní změna - tzv. výnosová míra. Jestliže označíme výnosovou míru jako $\mu$, lze předcházející vztahy vyjádřit jako
\begin{equation}
\delta S = \mu S \delta t
\end{equation}
resp.
\begin{equation}
\frac{\delta S}{S} = \mu \delta t
\end{equation}
za předpokladu, že volatilita výnosové míry je nulová. Cenu akcie v čase $T$ pak lze pomocí kontinuálního úročení vyjádřit jako
\begin{equation*}
S_T = S_0 e^{\mu T}
\end{equation*}
Jestliže navíc přidáme volatilní složku, modifikují se rovnice (9.1) a (9.2) do tvaru
\begin{equation*}
\delta S = \mu S \delta t + \sigma S \delta z
\end{equation*}
resp.
\begin{equation}
\frac{\delta S}{S} = \mu \delta t + \sigma \delta z
\end{equation}
Pro $\delta t$ blížící se 0 lze (9.3) upravit do tvaru
\begin{equation*}
\frac{d S}{S}=\mu d t + \sigma d z,
\end{equation*}
který je nejpoužívanějším modelem pro ceny akcií. Proměnná $\mu$ je očekávaná výnosová míra, $\sigma$ pak představuje volatilitu cen akcií.

Výše uvedený vztah lze pro naše účely dále přeformulovat do podoby
\begin{equation}
d S = \mu S d t + \sigma S \epsilon \sqrt{d t}
\end{equation}
Člen $\delta S$ představuje změnu ceny akcie v malém časovém intervalu délky $\delta t$ a $\epsilon$ je výběr z pravděpodobnostního rozdělení $\phi (0,1)$.
Lze dokázat, že náhodnou proměnnou $\frac{\delta S}{S}$ lze popsat pomocí náhodného rozdělení
\begin{equation*}
\frac{\delta S}{S} \sim \phi(\mu \delta t, \sigma \sqrt{\delta t})
\end{equation*}
Tohoto vztahu lze např. využít při simulaci cen metodou Monte Carlo.

\subsection{It\^o proces}

Dalším z rodiny stochastických procesů je It\^o proces. Jedná se o zobecněný Wienerův proces, kde
parametry $a$ a $b$ jsou funkcí hodnoty podkladové proměnné $x$ v čase $t$. Oba parametry se tak v průběhu času mění.
\begin{equation*}
d x = a(x,t) + b(x,t)d z
\end{equation*}
\textbf{It\^o lemma:} Předpokládejme, že náhodná veličina $x$ sleduje It\^o proces.
\begin{equation*}
d x = a(x,t)+b(x,t)d z,
\end{equation*} 
kde $d z$ představuje Wienerův proces a parametry $a$ a $b$ jsou funkcí $x$ a $t$. Funkce $G$ proměnných $x$ a $t$ sleduje proces
\begin{equation}
d G = \bigg( \frac{\partial G}{\partial x}a + \frac{\partial G}{\partial t} + \frac{1}{2} \frac{\partial^2 G}{\partial^2 x^2}b^2 \bigg) d t + \frac{\partial G}{\partial x}b \cdot d z
\end{equation}
a jedná se tedy také o It\^o proces.\\

V našem případě je $G$ funkcí $S$ a $t$. Můžeme proto s využitím (9.4) převést (9.5) do tvaru
\begin{equation*}
d G = \bigg( \frac{\partial G}{\partial S}\mu S + \frac{\partial G}{\partial t} + \frac{1}{2} \frac{\partial^2 G}{\partial^2 S^2}\sigma^2 S^2 \bigg) d t + \frac{\partial G}{\partial x}\sigma S \cdot d z
\end{equation*}\\

Výše uvedený vztah lze snadno aplikovat např. na forwardové obchody. Jestliže $r$ je bezriziková úroková míra, $T$ délka kontraktu a $S_0$ spotová cena akcie, lze forwardovou cenu akcie vyjádřit jako
\begin{equation*}
F_0 = S_0 e^{rT}
\end{equation*} 
Jestliže nás zajímá možný vývoj ceny forwardového kontraktu v čase $t$, kde $t < T$, je možné použít právě It\^o proces.
\begin{equation*}
F=Se^r(T-t)
\end{equation*} 
\begin{equation*}
d F = \bigg( \frac{\partial F}{\partial S}\mu S + \frac{\partial F}{\partial t} + \frac{1}{2}\frac{\partial^2 G}{\partial^2 S^2}\sigma^2 S^2 \bigg) d t + \frac{\partial G}{\partial S}\sigma S \cdot dz
\end{equation*}
\begin{equation*}
d F = \big(e^{r(T-t)}\mu S - rSe^{r(T-t)} \big) dt + e^{e(T-t)}\sigma S \cdot dz
\end{equation*}

\section{Lema 9A - Odvození It\^o procesu}

Uvažujme funkci $G$, která je funkcí dvou proměnných $x$ a $y$. S použitím Taylorova rozvoje lze $\delta G$ vyjádřit jako
\begin{equation}
\delta G = \frac{\partial G}{\partial x}\delta x + \frac{\partial G}{\partial y}\delta y + \frac{1}{2}\frac{\partial^2 G}{\partial x^2}\delta x^2 + \frac{1}{2}\frac{\partial^2 G}{\partial x \partial y}\delta x \delta y + \frac{1}{2}\frac{\partial^2 G}{\partial y^2}\delta y^2 + ...
\end{equation}
S tím, jak se $\delta x$ a $\delta y$ limitně blíží 0, přejde výše uvedený vztah do tvaru
\begin{equation*}
d G = \frac{\partial G}{\partial x} + \frac{\partial G}{\partial y}\delta y
\end{equation*}
Předpokládejme, že náhodná veličina $x$ sleduje It\^o proces
\begin{equation*}
dx = a(x,t)dt + b(x,t)dz,
\end{equation*}
který je možné v diskrétním tvaru vyjádřit jako
\begin{equation*}
\delta x = a(x,t)\delta t + b(x,t) \epsilon \sqrt{\delta t}
\end{equation*}
Pro funkci $G(x,t)$ tedy platí
\begin{equation}
\delta G = \frac{\partial G}{\partial x}\delta x + \frac{\partial G}{\partial t}\delta t + \frac{1}{2}\frac{\partial^2 G}{\partial x^2}\delta x^2 + \frac{1}{2}\frac{\partial^2 G}{\partial x \partial t}\delta x \delta t + \frac{1}{2}\frac{\partial^2 G}{\partial t^2}\delta t^2 + ...
\end{equation}
Narozdíl od (9.6) nelze jednoduše z Taylorova rozvoje vypustit člen, který obsahuje $\delta^2 x$, protože ten obsahuje část řádu $\delta t$.
\begin{equation*}
\delta x^2 = a^2(x,t)\delta t^2 + a(x,t)b(x,t) \epsilon \delta t^\frac{3}{2} + b^2(x,t) \epsilon^2 \delta t
\end{equation*}
Pro $\delta t$ blížící se nule je možné první dva členy zanedbat, protože jsou vyššího řádu než $\delta t$. Vzhledem k tomu, že $\epsilon \in \phi(0,1)$, platí
\begin{equation*}
E[\epsilon^2]=1
\end{equation*}
Střední hodnota třetího členu je tedy rovna $b^2(x,t) \delta t$. S pomocí $D[ax]= a^2D[x]$ lze dokázat, že rozptyl členu $b^2(x,t) \epsilon^2 \delta t$ má řád $\delta t^2$. Výraz $\delta x^2$ tak lze pro $\delta t$ blížící se nule považovat za deterministický a roven
\begin{equation*}
\delta x^2 = b^2(x,t) \delta t
\end{equation*}
Rovnice (9.7) tak přechází do tvaru
\begin{equation*}
\delta G = \frac{\partial G}{\partial x}\delta x + \frac{\partial G}{\partial t}\delta t + \frac{1}{2}\frac{\partial^2 G}{\partial x^2}\delta x^2
\end{equation*}
\begin{equation*}
\delta G = \bigg( \frac{\partial G}{\partial x}a(x,t) + \frac{\partial G}{\partial t} + \frac{1}{2} \frac{\partial^2 G}{\partial x^2}b^2(x,t) \bigg) \delta t + \frac{\partial G}{\partial x} b(x,t) \epsilon \sqrt{\delta t}
\end{equation*}
Dosazením $\delta z$ za $\epsilon \sqrt{\delta t}$ pak získáváme konečnou podobu It\^o procesu v diskrétním tvaru.
\begin{equation*}
\delta G = \bigg( \frac{\partial G}{\partial x}a(x,t) + \frac{\partial G}{\partial t} + \frac{1}{2} \frac{\partial^2 G}{\partial^2 x^2}b^2(x,t) \bigg) \delta t + \frac{\partial G}{\partial x}b(x,t) \cdot \delta z
\end{equation*}

\section{Lema 9B - Lognormální rozdělení}

Uvažujme náhodnou veličinu $S$, která sleduje proces $d S = \mu S dt + \sigma S dz$. Dále definujme funkci $G$ jako $G = \ln S$. Platí tedy
\begin{equation*}
\frac{\partial G}{\partial S} = \frac{1}{S}
\end{equation*}
\begin{equation*}
\frac{\partial^2 G}{\partial^2 S} = -\frac{1}{S^2}
\end{equation*}
\begin{equation*}
\frac{\partial G}{\partial t} = 0
\end{equation*}
Dosazením do (9.5) tak získáme
\begin{equation*}
d G = \big( \mu - \frac{\sigma^2}{2} \big) dt + \sigma dz
\end{equation*}
Protože $\mu$ i $\sigma$ jsou konstanty, jedná se opět o případ Wienerova procesu. To mimojiné znamená, že
\begin{equation*}
\ln S_T - \ln S_0 \sim \phi \bigg[\big(\mu - \frac{\sigma^2}{2} \big)T, \sigma \sqrt{T} \bigg] 
\end{equation*}
\begin{equation*}
\ln S_T \sim \phi \bigg[\ln S_0 + \big(\mu - \frac{\sigma^2}{2} \big)T, \sigma \sqrt{T} \bigg]
\end{equation*}

Parametr $S_T$ představuje cenu akcie v budoucím čase $T$, $S_0$ je spotová cena akcie a $\phi(m,s)$ označuje normální rozdělení se střední hodnotou $m$ a směrodatnou odchylkou $s$. Z výše uvedeného vztahu vyplývá, že $\ln S_T$ má normální rozdělení. Samotnou náhodnou veličinu $S_T$ lze tedy popsat pomocí lognormálního rozdělení.

\chapter{Black-Scholes model}

V předchozí kapitole jsme dokázali
\begin{equation*}
\frac{\delta S}{S} \sim \phi(\mu \delta t, \sigma \sqrt{\delta t}),
\end{equation*}
tj. že relativní změna ceny akcie v čase $\delta t$ sleduje normální rozdělení se střední hodnotou $\mu \delta t$ a směrodatnou odchylkou $\sigma \sqrt{t}$. Dále jsme dokázali, že
\begin{equation}
\ln \frac{S_T}{S_0} \sim \bigg[ \bigg(\mu - \frac{\sigma^2}{2} \bigg)T, \sigma \sqrt{T} \bigg]
\end{equation}
Náhodná proměnná $\ln S_T$ tedy sleduje normální rozdělení. Z toho vyplývá, že náhodná proměnná $S_T$ sleduje lognormální rozdělení. Pro $S_T$ tedy s ohledem na (10.1) platí
\begin{equation*}
E[S_T]=S_0 e^{\mu T}
\end{equation*}
\begin{equation*}
D[S_T]={S_0}^2e^{2 \mu T} \Big(e^{\sigma^2 T}-1 \Big)
\end{equation*}

\section{Výnosová míra}

Uvažujme náhodnou veličinu $\eta$. Nechť tato náhodná veličina představuje roční výnosovou míru v kontinuálním vyjádřením, kterou přináší svému držiteli konkrétní akcie. Cenu akcie v čase $S_T$ lze pak vyjádřit jako
\begin{equation*}
S_T = S_0 e^{\eta T}
\end{equation*}
Z toho plyne
\begin{equation}
\eta = \frac{1}{T} \ln \frac{S_T}{S_0}
\end{equation}
S použitím (10.1) je možné $\eta$ vyjádřit v následujícím tvaru
\begin{equation}
\eta \sim \phi \Bigg[ \mu - \frac{\sigma^2}{2},\frac{\sigma}{\sqrt{T}} \Bigg]
\end{equation}
Parametr $\mu$ závisí na riziku akcie a bezrizikové výnosové míře - čím vyšší je riziko popř. bezriková úroková míra, tím vyšší výnost investoři požadují. Pro krátký časový interval $\delta t$ lze očekávanou výnosovou míru $E[\eta]$ aproximovat právě pomocí parametru $\mu$ jako $\mu \delta t$. Pro delší časové intervaly je však nutné s ohledem na (10.3) vyjádřit očekávanou výnosovou míru jako
\begin{equation*}
E[\eta] = \mu - \frac{\sigma^2}{2}
\end{equation*}
Vzhledem k (10.2) lze $E[\eta]$ vyjádřit jako
\begin{equation}
E[\eta] = \frac{1}{T}\bigg(E[\ln \frac{S_T}{S_0}]\bigg)
\end{equation}
Jedinou náhodnou proměnnou je ve výše uvedeném vztahu $S_T$. Pro očekávanou hodnotu této veličiny sice platí
\begin{equation*}
E[S_T] = S_0 e^{\mu}{T}
\end{equation*}
\begin{equation*}
\ln (E[S_T]) = \ln S_0 + \mu T
\end{equation*}
V (10.4) však figuruje $E[\ln(S_T)]$ a z důvodu nelinearity přirozeného logaritmu nelze tvrdit následující
\begin{equation*}
\ln (E[S_T])=E[\ln (S_T)]
\end{equation*}
Protože je přirozený logaritmus nelineární, platí
\begin{equation*}
\ln(E[S_T]) > E[\ln(S_T)]
\end{equation*}
\begin{equation*}
E[\ln \frac{S_T}{S_0}] < \mu T
\end{equation*}
Z tohoto důvodu není vhodné aproximovat očekávanou výnosovou míru parametrem $\mu$.

\section{Volatilita}

Volatilita akcie $\sigma$ definuje nejistotu ohledně výnosu, který poskytne svému držiteli akcie. Pro akcie je volatilita typicky 20\% - 50\% na roční bázi. Volatilita v absolutním vyjádření s ohledem na cenu akcie pak dána vztahem
\begin{equation*}
S_0 \sigma \sqrt{T}
\end{equation*}
Jestliže sledujeme cenu akcie v pravidelných intervalech (např. na denní bázi), lze relativní změnu ceny v $i$-tém období vyjádřit jako
\begin{equation*}
u_i = \ln \bigg( \frac{S_i}{S_{i-1}} \bigg)
\end{equation*}
Směrodatnou odchylku výnosové míry lze odhadnout pomocí $s$.
\begin{equation*}
s=\sqrt{\frac{1}{n-1} \sum_{i=1}^n (u_i - \overline{u})^2}
\end{equation*}
Jestliže $\hat{\sigma}$ je odhad roční volatility a $\tau$ délka období pro výpočet $u_i$, platí
\begin{equation*}
\hat{\sigma}=\frac{s}{\sqrt{\tau}}
\end{equation*}
Standardní chyba tohoto odhadu je $\hat{\sigma} / \sqrt{2n}$, kde $n$ je počet pozorování. Z toho tedy vyplývá, že s rostoucím $n$ se zvyšuje přesnost odhadu.\\

\noindent \textbf{Poznámka:} Empirické výzkumu prokázaly, že pro odhad volatility je vhodnější použít pouze pracovní dny. V opačném případě by došlo k zavádějícímu snížení volatity.

\section{Koncept modelu}
Předpoklady modelu:
\begin{enumerate}
\item Cena akcie sleduje proces popsaný v kapitole 9. Parametry $\mu$ a $\sigma$ jsou konstantní.
\item Krátký prodej je povolen.
\item Neexistují transakční náklady ani daně. Všechny cenné papíry jsou dokonale dělitelné (tj. je např. možné koupit 1/3 akcie).
\item V průběhu životnosti opce nedochází k výplatě dividend z podkladové akcie.
\item Neexistuje možnost bezrizikové arbitráže.
\item Obchodování s cennými papíry je nepřetržité.
\item Bezriziková úroková sazba $r$ je konstantní po celou dobu životnosti opce pro všechny splatnosti.
\end{enumerate}
Za předpokladů zmiňovaných v předchozí kapitole sleduje změna ceny akcie proces
\begin{equation*}
d S = \mu S d t + \sigma S d z
\end{equation*}
Uvažujme cenu kupní opce $f$. Je zřejmé, že proměnná $f$ je funkcí $S$ a $t$. Proto platí
\begin{equation*}
d f = \bigg( \frac{\partial f}{\partial S}\mu S + \frac{\partial f}{\partial t} + \frac{1}{2}\frac{\partial^2 f}{\partial S^2}\sigma^2 S^2 \bigg) dt + \frac{\partial f}{\partial S}\sigma S d z
\end{equation*}
Jestliže výše uvedené rovnice převedeme do diskrétního tvaru, dostáváme
\begin{equation}
\delta S = \mu S \delta t + \sigma S \delta z
\end{equation}
\begin{equation}
\delta f = \bigg(\frac{\partial f}{\partial S} \mu S + \frac{\partial f}{\partial t} + \frac{1}{2}\frac{\partial^2 f}{\partial S^2}\sigma^2 S^2 \bigg) \delta t + \frac{\partial f}{\partial S}\sigma S \delta z
\end{equation}
Uvažujme portfolio, které se skládá z $-1$ opce a $\partial f / \partial S$ akcií. Jestliže $\Pi$ je hodnota portfolia, platí
\begin{equation*}
\Pi = -f + \frac{\partial f}{\partial S}S
\end{equation*}
Změnu ceny tohoto portfolia v čase $\delta t$ pak lze vyjádřit jako
\begin{equation*}
\delta \Pi = -\delta f + \frac{\partial f}{\partial S}\delta S
\end{equation*}
Dosazením (10.5) a (10.6) a následnými úpravami lze výše uvedenou rovnici vyjádřit jako
\begin{equation*}
\delta \Pi = \bigg( -\frac{\partial f}{\partial t} - \frac{1}{2}\frac{\partial^2 f}{\partial S^2}\sigma^2 S^2 \bigg) \delta t
\end{equation*}
Protože tento vztah neobsahuje $\delta z$, musí být portfolio v časovém horizontu $\delta t$ bezrizikové. Proto musí svému vlastníkovi přinášet výnos odpovídající bezrizikové výnosové míře.
\begin{equation*}
\delta \Pi = r \Pi \delta t
\end{equation*}
\begin{equation*}
\bigg( \frac{\partial f}{\partial t}+\frac{1}{2}\frac{\partial^2 f}{\partial S^2}\sigma^2 S^2 \bigg) \delta t = r \bigg(f - \frac{\partial f}{\partial S} \bigg) \delta t
\end{equation*}
\begin{equation}
rf = \frac{\partial f}{\partial t}+rS \frac{\partial f}{\partial S} + \frac{1}{2} \sigma^2 S^2 \frac{\partial^2 f}{\partial S^2}
\end{equation}
(10.7) je Black-Sholes-Mertonova diferenciální rovnice. Řešením této rovnice s ohledem na omezující podmínku $f=\max(S-K,0)$ pro $t=T$ v případě evropské kupní opce (popř. $f=\max(K-S,0) $ pro $t=T$ v případě evropské prodejní opce) jsou pak následující rovnice
\begin{equation*}
d_1 = \frac{ln(S_0/K)+(r+\sigma^2/2)T}{\sigma \sqrt{T}}
\end{equation*}
\begin{equation*}
d_2 = \frac{ln(S_0/K)+(r-\sigma^2/2)T}{\sigma \sqrt{T}}=d_1 - \sigma \sqrt{T}
\end{equation*}
\begin{equation}
c = S_0 N(d_1)-Ke^{-rT}N(d_2)
\end{equation}
\begin{equation}
p = Ke^{-rT}N(-d_2)-S_0 N(-d_1)
\end{equation}
(10.8) resp. (10.9) představuje cenu evropské kupní opce  resp. evropské prodejní opce v čase $t$. $N(x)$ je kumulativní distribuční funkcí normovaného normálního rozdělení a $K$ je realizační cena opce.

Výše uvedené rovnice jsou založeny na principu rizikově neutrálního ocenění. V těchto rovnicích totiž nefiguruje žádný člen, který by odrážel rizikový profil investora. Jak již bylo zmíněno, je tímto členem $\mu$, který však při odvozování diferenciální rovnice vypadl.
Výše uvedené rovnice je tak  možné použít ve světě s libovolným rizikovým profilem. Uvažujme svět, kde investoři požadují vyšší očekávanou výnosovou míru jako kompenzaci za podstupované riziko\footnote{Toto pojetí odpovídá reálnému světu více než koncept rizikově neutrálního modelu}. V tomto případě dojde k změně dvou faktorů, které se vzájemně vykompenzují:
\begin{enumerate}
\item zvýší se požadovaný očekávaný výnos z akcie jako důsledek zvýšené averze k riziku
\item vzroste diskontní faktor, který je používán k určení současné hodnoty budoucího cash-flow.
\end{enumerate}

\subsection{Odvození rovnic Black-Sholes modelu}
Vedle odvození rovnic (10.8) a (10.9) prostřednictvím diferenciální rovnice (10.7) existuje ještě další možnost řešení.

Uvažujme hustotu pravděpodobnosti $g(V)$ náhodné proměnné $V$. Platí
\begin{equation}
E[\max(V-K,0)]= \int_K^\infty (V-K) g(V) dV
\end{equation}
Nechť proměnná $\ln V$ sleduje normální rozdělení se směrodatnou odchylkou $s$. Z definice střední hodnoty lognormálního rozdělení vyplývá
\begin{equation}
m=\ln(E[V])-\frac{s^2}{2},
\end{equation}
kde $m$ je střední hodnota proměnné $\ln V$. Definujme novou proměnnou
\begin{equation*}
Q=\frac{\ln V - m}{s}
\end{equation*}
Proměnná $Q$ sleduje normované normální rozdělení. Platí tedy
\begin{equation*}
h(Q)=\frac{1}{\sqrt{2 \pi}}e^{-Q^2/2}
\end{equation*}
Rovnici (10.10) tedy lze vyjádřit jako
\begin{equation*}
E[\max(V-K,0)] = \int_{(\ln K - m)/s}^\infty (e^{Qs+m}-K)h(Q)dQ
\end{equation*}
\begin{equation}
E[\max(V-K,0)] = \int_{(\ln K - m)/s}^\infty e^{Qs+m}h(Q)dQ-K \int_{(\ln K - m)/s}^\infty h(Q)dQ
\end{equation}
Úpravami výrazu $e^{Qs+m}h(Q)$ z rovnice (10.12) získáme
\begin{equation*}
e^{Qs+m}h(Q)=\frac{1}{\sqrt{2 \pi}}e^{(-Q^2+2Qs+2m)/2}=\frac{e^{m+s^2/2}}{\sqrt{2 \pi}}e^{[-(Q-s)^2]/2}=e^{m+s^2/2}h(Q-s)
\end{equation*}
Rovnici (10.12) tak lze dále upravit do tvaru
\begin{equation}
E[\max(V-K,0)]=e^{m+s^2/2} \int_{(\ln K - m)/s}^\infty h(Q-s)dQ - K \int_{(\ln K - m)/s}^\infty h(Q)dQ
\end{equation}
Definujme funkci $N(x)$ jako pravděpodobnost, že náhodná proměnná, která sleduje normované normální rozdělení, je menší než $x$. První člen pravé strany rovnice (10.13) je tak
\begin{equation*}
1-N[(\ln K - m)/s-s]
\end{equation*}
neboli
\begin{equation*}
N[(-\ln K + m)/s+s]
\end{equation*}
Dosazením za $m$ dle (10.11) pak získáme
\begin{equation*}
N \bigg( \frac{\ln(E[V]/K)+s^2/2}{s}\bigg)=N(d_1)
\end{equation*}
Podobně lze převést druhý člen pravé strany rovnice rovnice (10.13) do tvaru
\begin{equation*}
N \bigg(\frac{\ln(E[V]/K)-s^2/2}{s}\bigg)=N(d_2)
\end{equation*}
Celou rovnici (10.13) tak lze vyjádřit jako
\begin{equation*}
E[\max(V-K,0)]=e^{m+s^2/2}N(d_1)-KN(d_2)
\end{equation*}
Dosazením za $m$ a následnými úpravami pak získáme rovnici pro výpočet ceny evropské kupní opce.
\begin{equation*}
E[\max(V-K,0)]=e^{\ln(E[V])-s^2/2}N(d_1)-KN(d_2)
\end{equation*}
\begin{equation*}
E[\max(V-K,0)]=e^{E[ln(V)]}N(d_1)-KN(d_2)
\end{equation*}
\begin{equation*}
E[\max(V-K,0)]=E[V]N(d_1)-KN(d_2)
\end{equation*}
Diskontováním pak získáme konečnou formu rovnice pro výpočet ceny evropské kupní opce
\begin{equation*}
c=S_0N(d_1)-Ke^{-rT}N(d_2)
\end{equation*}

\section{Implikovaná volatilita}
Rovnice Black-Scholes modelu považují volatilitu ceny akcie za jeden ze vstupních parametrů. Volatilita se běžně počítá z relativních změn ceny akcií mezi jednotlivými pracovními dny. Pro účely výpočtu historické volatility se uvažuje 252 pracovních dní v roce.

Nicméně je možné, známe-li cenu opce, volatilitu zpětně vypočítat. Platí, že takto vypočtená volatilita se nemusí vždy rovnat volatilitě historické. V tomto případě hovoříme o tzv. implikované volatilitě, kterou je možné chápat jako volatilitu očekávanou trhem.\\

\noindent \textbf{Poznámka:} Volatilitu není možné z rovnic Black-Scholes modelu vyjádřit analyticky - je třeba ji určit iteračně. Dále není vhodné počítat implikovanou volatilitu z opcí, které jsou hluboce in-the-money nebo out-of-the-money, protože cena opce je v těchto případech velice málo citlivá na změnu volatility.

\section{Dividendy}
Až dosud jsme uvažovali pouze akcie, které nepřináší svému majiteli žádnou dividendu.

Jsou-li však držiteli akcie po dobu životnosti evropské opce vypláceny dividendy, je třeba v rovnici (10.8) a (10.9) snížit $S_0$ o současnou hodnotu očekávané dividendy. Tyto dividendy pak považujeme za "bezrizikovou" složku akcie. Spolehlivé určení očekávané výše dividend ovšem vyžaduje stabilní dividendovou politiku. Dividendy se diskontují k tzv. ex-dividend dni bezrizikovou úrokovou sazbou $r$.

Dalším problémem je zdanění dividend. Míra zdanění totiž nemusí být shodná pro všechny subjekty. Jistým vodítkem může být v tomto případě vývoj cen akcie v minulosti. Jestliže tedy např. cena akcie k ex-dividend dni v minulosti klesla v průměru o 80\% hodnoty vyplácené dividendy je vhodné pro výpočet současné hodnoty použít nikoliv 100\% ale pouze 80\% hodnoty dividendy.

\section{Modifikace Black-Scholes modelu}

\subsection{Akcie s kontstantním dividendovým výnosem}

V případě evropské opce, kde podkladovým aktivem je akcie popř. akciový index s konstantním dividendovým výnosem $q$, jsou původní rovnice Black-Scholes modelu modifikovány následujícím způsobem
\begin{equation*}
c = S_0 e^{-qT} N(d_1) - Ke^{-rT}N(d_2)
\end{equation*}
\begin{equation*}
p = K e^{-rT}N(-d_2)-S_0e^{-qT}N(-d_1)
\end{equation*}
\begin{equation*}
d_1 = \frac{\ln (S_0/K)+(r-q+\sigma^2/2)T}{\sigma \sqrt{T}}
\end{equation*}
\begin{equation*}
d_2 = \frac{\ln (S_0/K)+(r-q-\sigma^2/2)T}{\sigma \sqrt{T}}=d_1-\sigma \sqrt{T}
\end{equation*}
Důvod výše uvedených úprav spočívá v hlavní myšlence rizikově neutrálního oceňování - tj. že každé aktivum přináší svému majiteli očekávaný výnos odpovídající bezrizikové úrokové míře $r$.

Uvažujme dvě akcie, jejichž spotová cena je $S_0$. První z akcií přináší svému majiteli konstatní dividendový výnos $q$, z druhé akcie dividendy vypláceny nejsou. Jestliže odhlédneme od kapitálových výnosů, pak na konci časového období délky $T$, bude cena první akcie $S_0e^{qT}$, avšak cena druhé akcie zůstane $S_0$. Aby se ceny obou akcií na konci období délky $T$ rovnaly $S_0$, musí být cena první akcie na začátku období rovna $S_0e^{-qT}$. To vysvětluje nahrazení $S_0$ výrazem $S_0e^{-qT}$ v rovnicích pro výpočet ceny opce. Vzhledem k tomu, že dividendy generují výnos $q$, musí kapitálový růst první akcie odpovídat $r-q$. Celkový výnos tak bude odpovídat bezrizikové úrokové sazbě $r$. To vyžaduje úpravy parametrů $d_1$ a $d_2$. 

\subsection{Měnové opce}

Logika měnových opcí je podobná jako v případě opcí, jejichž podkladovým aktivem je akcie s konstatním dividendovým výnosem. Parametr $r$ představuje bezrizikovou výnosovou míru domácí měny\footnote{Domácí měna je měna, ve které je vyjádřena cena opce.}, parametr $r_f$ pak bezrizikovou výnosovou míru cizí měny. Obě bezrizikové výnosové míry musí být vztaženy k časovému období $T$. Parametr $\sigma$ je volatilitou měnového kurzu. V případě měnových opcí navíc platí, že prodejní a kupní opce jsou symetrické\footnote{To znamená, že prodej měny $A$ znamená současný prodej měny $B$ a naopak.}.
\begin{equation*}
c = S_0 e^{-r_fT} N(d_1) - Ke^{-rT}N(d_2)
\end{equation*}
\begin{equation*}
p = K e^{-rT}N(-d_2)-S_0e^{-r_fT}N(-d_1)
\end{equation*}
\begin{equation*}
d_1 = \frac{\ln (S_0/K)+(r-r_f+\sigma^2/2)T}{\sigma \sqrt{T}}
\end{equation*}
\begin{equation*}
d_2 = \frac{\ln (S_0/K)+(r-r_f-\sigma^2/2)T}{\sigma \sqrt{T}}=d_1-\sigma \sqrt{T}
\end{equation*}

\subsection{Futures opce}

Futures opce je označována maturitou podkladové futures. Splatnost opce se pak odvíjí od tzv. delivery dne pro podkladovou futures - končí první možným termínem popř. několik dní před tímto datem.

Futures kontrakt vyžaduje nulovou počáteční investici. V rizikově neutrálním světě by proto očekávaná výnosnost takovéto investice měla být nulová. Z toho logicky vyplývá, že očekávaný růst ceny futures by měl být nulový. Touto úvahou se dostáváme do situace analogické s případem akciové opce, kdy podkladová akcie generovala konstatní dividendový výnost $q$. Tato akcie rostla mírou $r-q$. V případě futures opcí pak tedy speciálně platí $r=q$. To znamená, že cena futures roste nulovou mírou.
\begin{equation}
c = e^{-rT} \bigg[F_0N(d_1) - KN(d_2) \bigg]
\end{equation}
\begin{equation}
p = e^{-rT} \bigg[ K N(-d_2)-S_0N{-d_1} \bigg]
\end{equation}
\begin{equation*}
d_1 = \frac{\ln (F_0/K)+\sigma^2T/2}{\sigma \sqrt{T}}
\end{equation*}
\begin{equation*}
d_2 = \frac{\ln (F_0/K)+\sigma^2T/2}{\sigma \sqrt{T}}=d_1-\sigma \sqrt{T}
\end{equation*}
\textbf{Poznámka:} Většina futures opcí je americká. Výše uvedené rovnice se však vztahují pouze na evropské opce.

\chapter{Řecká písmena}

Uvažujme situaci, kdy obchodník prodal evropskou prodejní opci za cenu 300~000 USD, přičemž teoretická cena této opce byla 240~000 USD\footnote{Teoretickou cenou rozumíme cenu stanovenou podle Black-Scholes modelu.}. Nechť je realizační cena za jednu podkladovou akcii 50 USD a celkový objem 100~000 kusů.
Obchodník sice realizoval zisk 60~000 USD, avšak současně si prodejem opce otevřel pozici.  Existuje několik možností, jak se obchodník může zachovat. První možností je nedělat nic. Tato strategie je optimální v případě, že cena akcie nevzroste v době maturity opce nad 50~USD. V opačném případě však může vyústit ve značné ztráty. Druhou možností je nakoupit v době prodeje opce 100~000 akcií a ty držet a do splatnosti opce. Tato strategie je optimální v případě, kdy cena akcie vzroste nad 50 USD.

Vzhledem k tomu, že budoucí vývoj ceny akcie nelze předjímat, neposkytuje ani jedna z výše uvedených možností dostatečné zajištění. Jako ideální by se mohla jevit strategie, kdy by obchodník akcie nakoupil v okamžiku, kdy by cena vzrostla nad 50~USD a prodal v případě, že cena klesne pod 50~USD. Tato kombinace výše uvedených přístupů se označuje jako "stop-loss strategy". Celkové teoretické náklady této strategie by byly nulové (obchodník by nakupoval a prodával za cenu 50~USD na akcii). Zásadním problémem je, že z důvodu časového zpoždění by obchodník nebyl schopen zrealizovat prodej popř. nákup za tuto cenu. Dalším problémem je existence transakčních nákladů. Poslední skutečností, která hovoří v neprospěch této strategie, je fakt, že v případě efektivního trhu není minulý vývoj cen akcií není indikátorem budoucího vývoje\footnote{Jestliže cena akcie na efektivním trhu protne určitou cenovou hranici, je pravděpodobnost, že se vrátí pod tuto hranici stejná jako pravděpodobnost, že bude potračovat v nastoupeném trendu.}. "Stop-loss strategy", ačkoliv na je na první pohled velice atraktivní, v praxi selhává. Pro zajištění pozice používají obchodníci následujících technik a postupů.

\section{Delta}

Delta opce je definována jako podíl změny ceny opce ku změně ceny podkladového aktiva. Jestliže je $\Pi$ cenou opce, pak je možné deltu definovat jako
\begin{equation*}
\Delta = \frac{\partial \Pi}{\partial S}
\end{equation*}
\begin{center}
	\begin{pspicture}(0,0)(7,6)
		\rput(3.5,0){Výpočet parametru delta}

		\psline[arrows=->](0.5,1)(6.5,1)
		\psline[arrows=->](0.5,1)(0.5,5.5)

		\pscurve[linewidth=0.5mm](0.5,1)(4,2)(6.5,5)
		\psline(2,1)(6,3)

		\psline[linewidth=0.1mm, linestyle=dashed](4,1)(4,2)
		\psline[linewidth=0.1mm, linestyle=dashed](4,2)(0.5,2)

		\rput(6.5,0.7){$S_T$}
		\rput(1.5,5.5){cena opce}
		\rput(4,0.8){$A$}
		\rput(0.2,2){$B$}

		\rput(6,1.8){$\Delta = \frac{B}{A}$}
	\end{pspicture}
\end{center}
Význam tohoto parametru pro zajištění je patrný na první pohled. Jestliže obchodník drží jednu jednotku pokladového aktiva, musí nakoupit $-\Delta$ jednotek opce. Pro hodnotu tohoto portfolia totiž v případě malých změn $\delta S$ platí
\begin{equation*}
- \Delta \delta S  + \delta \Pi = 0
\end{equation*}
O takovéto pozici hovoříme jako o delta neutrální. Jak již bylo zmíněno, výše uvedená rovnice platí pouze pro malé změny $\delta S$, což znamená, že obchodník má zajištěnou pozici pouze na krátký časový interval. Aby jeho pozice zůstala delta neutrální, musí průběžně korigovat objem opcí v portfoliu v závislosti na tom, jak se mění delta.

Pro evropskou akciovou opci je delta dána vztahem
\begin{equation*}
\Delta_c = e^{-qT}N(d_1)
\end{equation*}
\begin{equation*}
\Delta_p = e^{-qT}\bigg[N(d_1)-1 \bigg]
\end{equation*}
Stejné rovnice možné použít také na výpočet delty měnových a futures opcí. V případě měnových opcí se namísto dividendové výnosové míry $q$ použije bezriziková výnosová míra cizí měny $r_f$. U futures opcí se pak namísto $q$ použije bezriziková úroková sazba $r$.

Pro portfolio opcí platí, že delta portfolia je rovna váženému průměru delt jednotlivých opcí.
\begin{equation*}
\Delta = \sum_{i=1}^{n} w_i \Delta_i
\end{equation*}

\noindent \textbf{Poznámka:} Parametr delta lze určit také pro jiné deriváty. Delta je opět definována jako citlivost ceny derivátu na změnu ceny podkladového aktiva.

\section{Theta}

Theta lze definovat jako změnu ceny opci v závislosti na tom, jak plyne čas, za předpokladu, že všechny ostatní faktory zůstávají neměnné.
\begin{equation*}
\Theta = \frac{\partial \Pi}{\partial t}
\end{equation*}
\begin{center}
	\begin{pspicture}(0,0)(10,8)
		\rput(5,0.5){Parametr theta pro jednotlivé typy evropské kupní opce}

		\psline[arrows=->](0.5,6.5)(9.5,6.5)
		\psline[arrows=->](0.5,1)(0.5,7.5)

		\rput(0.2,6.5){0}
		\rput(1,7.5){\tiny{theta}}
		\rput(8.5,6.7){\tiny{Doba do splatnosti}}

		\pscurve[linewidth=0.5mm](0.5,6.5)(2,5.5)(5,6)(8.5,6.3)
		\rput(2.5,5.8){\tiny{out-of-the-money}}
		\pscurve[linewidth=0.5mm](0.5,5.5)(0.8,5.4)(2.5,4)(5,5.5)(9,6.2)
		\rput(2.7,4.7){\tiny{in-the-money}}
		\pscurve[linewidth=0.5mm](1,1.5)(5.5,5.3)(9.5,6.1)
		\rput(1.5,3){\tiny{at-the-money}}
	\end{pspicture}
\end{center}
V případě akcie s konstantním dividendovým výnosem lze theta příslušné evropské opce vyjádřit jako
\begin{equation*}
\Theta_c = - \frac{S_0 N'(d_1) \sigma e^{-qT}}{2 \sqrt{T}-qS_0N(d_1)e^{-qT}-rKe^{-rT}N(d_2)}
\end{equation*}
\begin{equation*}
\Theta_p = - \frac{S_0 N'(d_1) \sigma e^{-qT}}{2 \sqrt{T}+qS_0N(-d_1)e^{-qT}+rKe^{-rT}N(-d_2)}
\end{equation*}
kde $N'(x)$ je definováno jako
\begin{equation*}
N'(x)=\frac{1}{\sqrt{2 \pi}}e^{-x^2/2}
\end{equation*}
Obdobné vztahy je možné použít také pro výpočet theta pro měnové a futures opce. V případě měnové opce se namísto $q$ použije bezriziková výnosová míra cizí měny $r_f$, v případě futures opcí pak bezriziková úroková míra $r$.

Pro theta platí, že v čase klesá. To znamená, že cena opce se se zkracující maturitou zmenšuje. Vzhledem k tomu, že theta není měřítkem rizika\footnote{Čas je čistě deterministickým procesem - nezahrnuje žádnou náhodnou složku.} používá se jako popisná statistika.\\

\noindent \textbf{Poznámka:} Výše uvedené vzorce měří čas v rocích. V praxi je však theta často vyjádřena ve dnech. Abychom získali theta ve dnech, je zapotřebí výsledek podělit 365 (pro kaledářní dny) resp. 252 (pro pracovní dny).

\section{Gamma}

Gamma vyjadřuje citlivost parametru delta na změnu ceny pokladového aktiva. Jestliže je gamma malé, mění se delta pouze pozvolna a korekce je potřeba provádět s nižší frekvencí. Je-li gamma v absolutním vyjádření vysoké, je reaguje delta velmi silně na změnu ceny podkladového aktiva. 
\begin{equation*}
\Gamma = \frac{\partial^2 \Pi}{\partial S^2}
\end{equation*}
\begin{center}
	\begin{pspicture}(0,0)(7,6)
		\rput(3.5,0){Výpočet parametru gamma}

		\psline[arrows=->](0.5,1)(6.5,1)
		\psline[arrows=->](0.5,1)(0.5,5.5)
		\rput(6.5,0.7){$S_T$}
		\rput(1.5,5.5){cena opce}

		\pscurve[linewidth=0.5mm](0.5,1)(4,2)(6.5,5)
		\psline(2,1)(6,3)

		\psline[linewidth=0.1mm, linestyle=dashed](4,1)(4,2)
		\rput(4,0.8){$A$}
		\psline[linewidth=0.1mm, linestyle=dashed](4,2)(0.5,2)
		\rput(0.2,2){$B$}
		\psline[linewidth=0.1mm, linestyle=dashed](5,1)(5,2.9)
		\rput(5,0.8){$A'$}
		\psline[linewidth=0.1mm, linestyle=dashed](5,2.9)(0.5,2.9)
		\rput(0.2,3){$B'$}
		\psline[linewidth=0.1mm, linestyle=dashed](5,2.5)(0.5,2.5)
		\rput(0.2,2.5){$B''$}
		
		\rput(6.3,2.3){$\Delta = \frac{B}{A} = \frac{B''}{A'}$}
		\rput(6.3,1.8){$\Delta' = \frac{B'}{A'}$}
	\end{pspicture}
\end{center}
\begin{center}
	\begin{pspicture}(0,0)(10,8)
		\rput(5,0.5){Parametr gamma pro jednotlivé typy opcí}

		\psline[arrows=->](0.5,1)(9.5,1)
		\psline[arrows=->](0.5,1)(0.5,7.5)

		\rput(0.2,1){0}
		\rput(1,7.5){\tiny{gamma}}
		\rput(8.5,1.2){\tiny{Doba do splatnosti}}

		\pscurve[linewidth=0.5mm](0.5,1)(1.5,2.5)(4,2)(8.5,1.5)
		\rput(4.5,1.4){\tiny{in-the-money}}

		\pscurve[linewidth=0.5mm](0.5,1)(1.5,1.7)(3.5,5)(5,3.5)(8.5,2)
		\rput(8,2.7){\tiny{out-of-the-money}}

		\pscurve[linewidth=0.5mm](0.8,7)(4,2.7)(8.5,1.8)
		\rput(1.8,7){\tiny{at-the-money}}		
	\end{pspicture}
\end{center}
Podobně jako pro parametr delta platí, že gamma portfolia je dána váženým průměrem dílčích parametrů gamma.
\begin{equation*}
\Gamma = \sum_{i=1}^{n} w_i \Gamma_i
\end{equation*}
Parametr gamma je pro evropskou kupní a prodejní akciovou opci shodný a dán vztahem
\begin{equation*}
\Gamma = \frac{N'(d_1)e^{-qT}}{S_0 \sigma \sqrt{T}}
\end{equation*}
V případě měnové opce je parametr $q$ ve výše uvedé rovnici nahrazen parametrem $r_f$ a v případě futures opcí platí $q=r$.\\

\noindent \textbf{Poznámka:} Parametry delta a gamma jsou vzájmně provázány - změna jednoho z nich vyvolá změnu druhého a naopak.

\section{Vztah mezi delta, theta a gamma}
Nechť je hodnota portfolia $\Pi$ funkcí času $t$ a ceny podkladového aktiva $S$. Pomocí Taylorova rozvoje lze $\delta \Pi$ vyjádřit jako
\begin{equation*}
\delta \Pi = \frac{\partial \Pi}{\partial S} \delta S + \frac{\partial \Pi}{\partial t}\delta t + \frac{1}{2}\frac{\partial^2 \Pi}{\partial S^2}\delta S^2 -\frac{1}{2}\frac{\partial^2 \Pi}{\partial t^2} \delta t^2 + \frac{\partial^2 \Pi}{\partial S \partial t}\partial S \partial t + ...
\end{equation*}
S použitím řeckých písmen lze výše uvedený vztah dále upravit do tvaru
\begin{equation}
\delta \Pi = \Delta \delta S + \Theta \delta t + \frac{1}{2}\Gamma \delta S^2 - \frac{1}{2}\frac{\partial^2 \Pi}{\partial t^2} \delta t^2 + \frac{\partial^2 \Pi}{\partial S \partial t}\partial S \partial t + ...
\end{equation}
Změnu ceny portofolia tak lze přibližně vyjdářit jako
\begin{equation}
\delta \Pi \approx \Delta \delta S + \Theta \delta t + \frac{1}{2}\Gamma \delta S^2
\end{equation}
Pro další členy rozvoje (11.1) by tak bylo možné definovat další řecká písmena a aproximaci dále zpřesnit.

\section{Vega}
Až dosud jsme uvažovali, že volatilita ceny pokladového aktiva je konstantní. Tato volatilita se však v čase může měnit. Parametr vega vyjadřuje citlivost ceny opce na změnu volatility podkladového aktiva. Z logiky věci tedy vyplývá, že pozice v podkladovém aktivu má vega rovnu nule. Dále platí, že vega je vždy kladná - tj. že cena opce s rostoucí volatilitou roste.
\begin{equation*}
\nu = \frac{\partial \Pi}{\partial \sigma}
\end{equation*}
Pro portfolio instrumentů platí
\begin{equation*}
\nu = \sum_{i=1}^{n} w_i \nu_i
\end{equation*}
Zásadním problémem je, že delta neutrální portfolio není ve většině případů vega neutrální a naopak. Jestliže chce obchodník vytvořit portfolio, které bude současně delta a vega neutrální, musí použít alespoň dva deriváty, které jsou odvozeny od požadovaného pokladového aktiva.
Pro evropskou kupní a prodejní akciovou opci s konstatním dividendovým výnosem platí
\begin{equation*}
\nu = S_0 \sqrt{T}N'(d_1)e^{-qT}
\end{equation*}

\noindent \textbf{Poznámka:} Odvození parametru vega z Black-Scholes modelu je na první pohled paradoxní, protože tento model předpokládá konstantní volatilitu. Přesnost výše uvedených vzorců se však v praxi ukázala jako dostatečná.

\section{Rho}
Parametr rho vyjadřuje míru citlivosti ceny opce na změnu úrokové míry.
\begin{equation*}
\rho = \frac{\partial \Pi}{\partial  r}
\end{equation*}
Pro evropskou akciovou opci s konstatním dividendovým výnosem platí
\begin{equation*}
\rho_c = KTe^{-rT}N(d_2)
\end{equation*}
\begin{equation*}
\rho_p = -KTe^{-rT}N(-d_2)
\end{equation*}
Pro každou měnovou opci existují vždy dvě rho. Jedna vyjadřuje citlivost na změnu úrokové míry domácí a druhá citlivost na změnu úrokové míry cizí měny. Parametr rho pro domácí měnu je možné vypočíst podle výše uvedených vztahů. Pro cizí měnu je rho dáno rovnicemi
\begin{equation*}
\rho_c = -Te^{-r_fT}S_0N(d_1)
\end{equation*}
\begin{equation*}
\rho_p = Te^{-r_fT}S_0N(-d_1)
\end{equation*}
Pro futures opce je rho dáno
\begin{equation*}
\rho_c = -cT
\end{equation*}
\begin{equation*}
\rho_p = -pT
\end{equation*}

\section{Syntetická opce}

Jestliže obchodník drží akciové portfolio, může pro něj být velice zajímavé doplnit portfolio o prodejní opci na tyto akcie. Tímto způsobem může limitovat maximální ztrátu, kterou může v případě nepříznivého vývoje cen utrpět. Problém nastává, když opce není na trhu k dispozici. Požadovanou opci je však možné vytvořit synteticky. Hlavní myšlenka syntetické opce spočívá v tom, že se udržuje taková pozice v podkladovém aktivu (popř. futures na pokladové aktivum), jejíž delta je rovna deltě požadované opce. Vzhledem k tomu, že delta evropské prodejní akciové opce je
\begin{equation*}
\Delta = e^{-qT} \bigg[ N(d_1)-1 \bigg]
\end{equation*}
a vzhledem k tomu, že delta podkladového aktiva je vždy rovna jedné, platí, že syntetickou opci je možné vytvořit prodejem právě $\Delta$ jednotek podkladového aktiva. Za takto získané prostředky je pak třeba nakoupit bezrizikové aktivum, které generuje bezrizikovou výnosovou míru. S tím, jak se mění delta portfolia, je třeba postupně dokupovat / prodávat podkladové aktivum.
V praxi se velice často namísto konkrétních akciových titulů používá jako pokladového aktiva akciového indexu. V tomto případě je odpovídající delta na jeden futures kontrakt definována jako
\begin{equation*}
\Delta = e^{-qT}e^{-(r-q)T^*} \bigg[1-N(d_1)\bigg]
\end{equation*}
Počet futures kontraktů na daný akciový index je pak dán vztahem
\begin{equation*}
e^{-qT}e^{-(r-q)T^*} \bigg[1-N(d_1)\bigg] \frac{K_1}{K_2} \beta,
\end{equation*}
kde $T^*$ představuje maturitu futures kontraktu, $K_1$ podíl hodnoty portfolia a indexu, $K_2$ podíl hodnoty futures kontraktu a indexu a $\beta$ je beta koeficient portfolia.\\

\noindent \textbf{Poznámka:} Princip syntetické opce selhává v okamžiku, kdy se volatilita indexu mění skokově v krátkém období. Navíc tento mechanismus paradoxně přispívá zvýšené volatilitě trhu - obchodníci totiž prodávají akcie popř. futures na index při poklesu trhu, čimž pokles dále prohlubují. Typickým příkladem je krize z 19. října 1987.

\chapter{Volatility smile}

Black-Scholes model je založen na předpokladu konstatní volatility podkladového aktiva. Tento předpoklad však v reálném světě přestal platit po finanční krizi v roce 1987. V dnešní době je obvyklé, že implikovaná volatilita\footnote{Připomeňme, že implikovaná volatilita je volatilita zpětně vypočtená z Black-Scholes modelu na základě znalosti ceny evropské opce.} out-of-the-money prodejní opce je vyšší než implikovaná volatilita out-of-the-money kupní opce. Vzhledem k tomu, že volatilita přímoúměrně ovlivňuje cenu opce, jsou out-of-the-money prodejní opce dražší než out-of-the-money kupní opce.

Řešením tohoto rozporu je tzv. volatility smile. Volatility smile je funkcí vyjadřující závilost realizační ceny opce a implikované volatility a říká nám, pro jakou volatilitu je Black-Scholes model konzistentní s kotovanými cenami opcí. Protože tvar příslušné křivky připomínal po krizi v roce 1987 tvar písmene ``U'', začalo se této křivce říkat volatility smile. Název se udržel až do dnešní doby, ačkoli toto pravidlo již neplatí. Tvar volatility smile závisí na typu podkladového aktiva (akcie, měnový kurz apod.) a může se lišit v čase.\\

\noindent \textbf{Poznámka:} V praxi se namísto realizační ceny opce používá její delta. Jestliže v rámci Black-Scholes modelu zafixujeme všechny parametry s vyjímkou realizační ceny, lze snadno dokázat, že pro každou realizační cenu existuje právě jedna delta a naopak. Dále platí, že opce, která je hluboce in-the-money, má deltu blízkou jedné a opce, která je hluboce out-of-the-money, má deltu blízkou nule. V případě kupní opce je tedy delta klesající funkcí realizační ceny, u prodejní opce je tomu naopak.

\section{Argumenty vysvětlující volatility smile}

Možná vysvětlení, proč jsou out-of-the-money prodejní opce dražší než out-of-the-money kupní opce, jsou
\begin{itemize}
\item rozdílná poptávka a nabídka, kdy si většina investorů kupuje out-of-the-money prodejní opci jako určitou formu pojištění svého portfolia, a proto jsou ochotni akceptovat vyšší cenu
\item skutečnost, že volatilita není konstantní, ale je funkcí ceny podkladového aktiva
\end{itemize}

\subsection{Rozdílná nabídka a poptávka}

Dle Black-Schloles modelu je hodnota evropské kupní opce dána vztahem
\begin{equation*}
c = S_0N(d_1)-Ke^{-rT}N(d_2)
\end{equation*}
Tento vztah lze interpretovat tak, že pozici v evropské kupní opci je možné replikovat pomocí dlouhé pozice $N(d_1)$ kusů v podkladovém aktivu se spotovou cenou $S_0$ a krátké pozice $N(d_2)$ kusů v bezrizikovém diskontním dluhopisu, který v době splatnosti $T$ generuje cash-flow $K$. Obdobně lze interpretovat také evropskou prodejní opci.

Uvažujme evropskou prodejní a kupní opci se shodným podkladovým aktivem. Shodné pokladové aktivum tak pro obě opce implikuje shodnou spotovou cenu $S_0$, výnosovou míru $q$ a volatilitu $\sigma$. Ostatní parametry, které figurují v Black-Scholes modelu, se mohou lišit. Hodnotu uvažovaných opcí lze vyjádřit jako
\begin{equation}
c = S_0N(d_1^c)-K^ce^{-r^cT^c}N(d_2^c)
\end{equation}
\begin{equation}
p = K^pe^{-r^pT^p}N(-d_2^p)-S_0N(-d_1^p) 
\end{equation}
kde $c$ představuje hodnotu kupní a $p$ hodnotu prodejní opce. Hodnoty $d_1$ a $d_2$ jsou definovány standardními rovnicemi.
\begin{equation*}
d_1^c = \frac{\ln(S_0/K^c)+(r^c - q +\sigma^2/2)T}{\sigma \sqrt{T^c}}
\end{equation*}
\begin{equation*}
d_2^c = d_1^c - \sigma \sqrt{T^c}
\end{equation*}
\begin{equation*}
d_1^p = \frac{\ln(S_0/K^p)+(r^p - q +\sigma^2/2)T}{\sigma \sqrt{T^p}}
\end{equation*}
\begin{equation*}
d_2^p = d_1^p - \sigma \sqrt{T^p}
\end{equation*}
Z rovnice (12.2) je možné vyjádřit spotovou cenu $S_0$ jako
\begin{equation*}
S_0 = \frac{K^p e^{-r^pT^p}N(-d_2^p)-p}{N(-d_1^p)}
\end{equation*}
Dosazením do rovnice (12.1) pro výpočet hodnoty kupní opce získáme
\begin{equation}
c = K^ce^{-r^cT^c}N(-d_2^c)-K^pe^{-r^pT^p}\frac{N(-d_2^p)N(-d_1^c)}{N(-d_1^p)} + p \frac{N(-d_1^c)}{N(-d_1^p)} 
\end{equation}
Rovnici (12.3) lze tedy interpretovat tak, že odhlédneme-li od transakčních nákladů, lze v nekonečně malém časovém intervalu replikovat uvažovanou kupní opci pomocí
\begin{itemize}
\item dlouhé pozice v $N(-d_2^c)$ kusech bezrizikového diskontního dluhopisu, který v době své splatnosti $T^c$ generuje cash-flow $K^c$
\item krátké pozice v $\frac{N(-d_2^p)N(-d_1^c)}{N(-d_1^p)}$ kusech bezrizikového diskontního dluhopisu, který v době své splatnosti $T^p$ generuje cash-flow $K^p$
\item dlouhé pozice v $\frac{N(-d_1^c)}{N(-d_1^p)}$ kusech prodejní opce $p$
\end{itemize}
Argument rozdílné nabídky a poptávky tak naráží na skutečnost, že prodejní a kupní opce jsou vzájemnými substituty. Proto se jako vysvětlení rozdílné implikované volatility preferuje hypotéza, že volatilita je funkcí ceny pokladového aktiva.

\subsection{Volatilita jako funkce ceny podkladového aktiva}

Black-Scholes model je založený na předpokladu, že cena $S$ podkladového aktiva sleduje stochatický proces
\begin{equation*}
\frac{dS}{S} = \mu dt + \sigma d z
\end{equation*}
Black-Scholes model tedy předpokládá, že volatilita $\sigma$ je konstantní. Rozdílná implikovaná volatilita pro různé realizační ceny opce však naznačuje, že cena pokladového aktiva sleduje proces
\begin{equation*}
\frac{dS}{S} = \mu dt + \sigma(S) dz
\end{equation*}
V tomto pojetí je tedy volatilita funkcí ceny podkladového aktiva. Obecně by mělo platit, že v době, kdy je cena pokladového aktiva nízká (např. v době finanční krize), je volatilita vyšší a naopak. Toto tvrzení je konzistentní se skutečností, že implikovaná volatilita out-of-the-money prodejních opcí je vyšší než implikovaná volatilita out-of-the-money kupních opcí\footnote{Pro out-of-the-money prodejní opci platí, že realizační cena je v porovnání se spotovou cenou relativně nízká. V případě out-of-the-money kupní opce je tomu naopak.}.

\subsubsection{Implikovaná volatilita a put-call parita}

Volatilitu je tedy možné chápat jako funkci ceny podkladového aktiva. Lze dokázat, že i v reálném světě se volatilita kupní opce musí rovnat volatilitě odpovídající prodejní opce. Uvažujme evropskou prodejní a kupní opci, které mají stejné parametry jako je realizační cena, splatnost, podkladové aktivum apod. Put-call parita říká, že pro ceny těchto dvou opcí musí platit
\begin{equation*}
p_m +S_0e^{-qT} = c_m + Ke^{-rT}
\end{equation*}
Protože put-call parita platí také v reálném světě, musí všechny kotované opce splňovat stejnou podmínku.
\begin{equation*}
p_r +S_0e^{-qT} = c_r + Ke^{-rT}
\end{equation*}
Z těchto dvou rovnic vyplývá
\begin{equation}
c_m - c_r = p_m - p_r
\end{equation}
Rozdíl mezi oceněním trhem a podle Black-Scholes modelu musí být tedy stejný pro prodejní i kupní opci. Je-li tento rozdíl nulový, platí, že modelová volatilita je rovna implikované volatilitě.

Protože v rámci Black-Scholes modelu se uvažuje stejná volatilita pro kupní i prodejní opci, znamená rovnice (12.4), že implikovaná volatilita pro obě opce musí být taktéž shodná.

\section{Implikované pravděpodobnostní rozdělení}

Důvodem pro existenci volatility smile je, že cena pokladového aktiva ve skutečnosti neodpovídá modelovému lognormálnímu rozdělení. Pravděpodobnostní rozdělení, kterým se cena pokladového aktiva řídí v reálném světě, označujeme jako tzv. implikované pravděpodobnostní rozdělení. V případě, kdy implikované pravděpodobnostní rozdělení indikuje vyšší cenu opce než modelové lognormální rozdělení, má toto za následek vyšší implikovanou volatilitu a naopak.

Uvažujme evropskou kupní opci. Pro cenu této opce platí
\begin{equation*}
c = e^{-rT} \int_{S_T = K}^\infty (S_T - K)g(S_T)dS_T,
\end{equation*}
kde $g$ je hustota pravděpodobnosti náhodné proměnné $S_T$. Derivace ceny opce $c$ podle realizační ceny $K$ je
\begin{equation*}
\frac{\partial c}{\partial K} = -e^{-rT} \int_{S_T = K}^\infty g(S_T)dS_T
\end{equation*}
Jestliže budeme derivovat dále podle realizační ceny, dostaneme
\begin{equation*}
\frac{\partial^2 c}{\partial K^2}=e^{-rT}g(K)
\end{equation*}
Hustotu pravděpodobnosti $g$ pak lze vyjádřit jako
\begin{equation*}
g(K)=e^{rT} \frac{\partial^2 c}{\partial K^2}
\end{equation*}
Druhá derivace funkce $f(x)$ je definována jako
\begin{equation*}
\frac{\partial^2 f(x)}{\partial x^2} = \lim \limits_{\Delta \to 0} \frac{[f(x + \Delta) - f(x)] - [f(x) - f(x - \Delta)]}{\Delta^2}
\end{equation*}
Za předpokladu dostatečně malého $\delta$ lze tak $g(K)$ odhadnout jako
\begin{equation*}
g(K) = e^{rT}\frac{c_1 + c_3 - 2c_2}{\delta^2}
\end{equation*}
kde $c_1$, $c_2$ a $c_3$ jsou ceny kupní opce se zbytkovou splatností $T$ a realizačními cenami $K+\delta$, $K$ a $K-\delta$.
\begin{center}
	\begin{pspicture}(0,0)(10,8)
		\rput(5,0){Teoretické lognormální a implikované rozdělení}

		\psline[arrows=->](0.5,1)(9.5,1)
		\psline[arrows=->](0.5,1)(0.5,7.5)

		\pscurve[curvature=1 1 0](0.5,1)(2.5,1.4)(3.7,2.5)(5,7.5)(6.3,2.5)(7.5,1.4)(9.5,1)
		\rput(7.5,4){\tiny{implikované rozdělení}}
		\pscurve[linestyle=dashed, curvature=1 0.8 0](0.5,1)(2.5,1.4)(5,6.5)(7.5,1.4)(9.5,1)
		\rput(6.5,7.5){\tiny{lognormální rozdělení}}
		
		\psline(2.5,1)(2.5,1.1)
		\rput(2.5,0.7){$K_1$}
		\psline(7.5,1)(7.5,1.1)
		\rput(7.5,0.7){$K_2$}

	\end{pspicture}
\end{center}

\section{Volatility surface}

Volatilita používaná pro výpočet ceny opcí nezávisí pouze na realizační ceně ale také na zbytkové splatnosti. Obecně platí, že je-li historická volatilita nízká, je implikovaná volatilita rostoucí funkcí maturity, protože se očekává její růst v budoucnu a naopak.
V praxi jsou pak často oba přístupy kombinovány do jedné tabulky a výsledná volatilita je tak dána realizační cenou a splatností opce. Vzhledem k tomu, že výslednou tabulku lze interpretovat pomocí třírozměrného grafu, hovoříme o tzv. volatility surface.
\begin{center}
\begin{tabular}{l r r r r r}
\textbf{} &
\multicolumn {5}{c}{\textbf{Realizační cena}}\\
\hline
\textbf{} &
\multicolumn {1}{c}{0.90} &
\multicolumn {1}{c}{0.95} &
\multicolumn {1}{c}{1.00} &
\multicolumn {1}{c}{1.05} &
\multicolumn {1}{c}{1.10}\\
\hline
 1 měsíc  & 14.2 & 13.0 & 12.0 & 13.1 & 14.5 \\
 3 měsíce & 14.0 & 13.0 & 12.0 & 13.1 & 14.2 \\
 6 měsíců & 14.1 & 13.3 & 12.5 & 13.4 & 14.3 \\
 1 rok    & 14.7 & 14.0 & 13.5 & 14.0 & 14.8 \\
 2 roky   & 15.0 & 14.4 & 14.0 & 14.5 & 15.1 \\
 5 roků   & 14.8 & 14.6 & 14.4 & 14.7 & 15.0 \\
\hline 
\end{tabular}
\end{center}

\chapter{Value at risk (VaR)}

Value at risk (VaR) představuje maximální ztrátu, kterou může obchodník utrpět na zvolené hladině pravděpodobnosti pro daný časový interval. VaR lze tedy interpretovat následovně: "Jsme si na $X$ procent jisti, ze naše případná ztráta nepřesáhne částku $Y$ CZK v následujícících $N$ dnech." Jestliže tedy budeme hodnotu portfolia v následujících $N$ dnech považovat za náhodnou veličinu, pak lze VaR odvodit od $100-X$tého kvantilu pravděpodobnostního rozdělení této náhodné veličiny.
\begin{center}
	\begin{pspicture}(0,0)(10,7)
		\rput(5,0){Hodnota portfolia}

		\psline[arrows=->](0.5,0.5)(9.5,0.5)
		\psline[arrows=->](0.5,0.5)(0.5,6.5)

		\pscurve[curvature=1 0.8 0](0.5,0.5)(2.5,0.9)(5,6.0)(7.5,0.9)(9.5,0.5)
		\psline[linestyle=dashed](3.0,0.4)(3.0,1.4)
		\psline[arrows=->](1.5,2.0)(2.7,0.8)
		\rput(1.5,2.2){\small{100-X\%}}
		\rput(3.0,0.3){\small{$\pi$}}
		\rput(0.5,0.3){\small{0}}

	\end{pspicture}
\end{center}
\begin{equation*}
\int_0^\pi f(\Pi)d\Pi = 100 - X\%
\end{equation*}
Ve výše uvedeném vztahu je $\Pi$ náhodná veličina, která představuje hodnotu portfolia na konci uvažovaného časového období, $f(\Pi)$ její hustota pravděpodobnosti a $\pi$ vyjadřuje $100-X~\%$ kvantil této náhodné veličiny. VaR pak lze vypočítat jako
\begin{equation*}
VaR = \Pi_0 - \pi
\end{equation*}
kde $\Pi_0$ je hodnota portfolia v okamžiku výpočtu VaR.

VaR se vždy váže k určitému časovému horizontu. Je možné přepočítat VaR z jednoho časového horizontu na jiný podle
\begin{equation*}
VaR_N = VaR_{den} \sqrt{N}
\end{equation*}
kde $VaR_{den}$ představuje jednodenní a $VaR_N$ $N$-denní VaR. Aby tento vztah platil, musí být hodnota portfolia mezi jednotlivými časovými horizonty nezávislá a mít identické normální rozdělení s nulovou střední hodnotou. V ostatních případech se jedná pouze o aproximaci.

\section{Podmíněný VaR}

Vedle pojmu VaR je možné se setkat také s pojmem podmíněný VaR. Podmíněný VaR je definovaný jako očekávaná ztráta v případě, že ztráta přesáhne částku $Y$. Ačkoliv je z praktického hlediska vypovídací schopnost podmíněného VaR vyšší, nejrozšířenějším ukazatelem je právě VaR.

Podmíněný VaR lze vypočíst podle
\begin{equation*}
VaR_C = \Pi_0 - \int_0^\pi \Pi f(\Pi)d\Pi
\end{equation*}
Kvantil $\pi$ lze určit ze vztahu $Y = \Pi_0 - \pi$.

\section{Historický VaR}

Velice často se v případě VaR pracuje s tzv. historickými scénáři. Historické scénáře mají většinou podobu vývoje úrokových sazeb, měnových kurzů, akciových indexů apod. Minulý vývoj se tak používá jako indikátor toho, co se může stát v budoucnu. Postup výpočtu hodnoty VaR je velice jednoduchý. V prvním kroku je třeba kvantifikovat dopad jednotlivých scénářů na hodnotu portfolia. V druhém kroku se tyto scénáře seřadí podle míry ztráty, která se k nim váže. Ve třetím a posledním kroku se vypočte příslušný kvantil. Jestliže např. chceme určit 99\% VaR a máme k dispozici 250 denních pozorování\footnote{250 denních pozorování odpovídá přibližně jednomu roku.}, stačí zprůměrovat ztrátu, která by byla generována druhým a třetím nejhorším scénářem.

Někdy jsou vedle historických scénářů používány také tzv. hypotetické scénáře. Ty odrážejí názor managementu na několik z jejich pohledu kritických situací. Klasickými příklady je např. dramatický růst popř. pokles úrokových sazeb, propad akciového indexu apod. Ze sady těchto scénářů se však nepočítá VaR. Pouze se vybere nejhorší scénář a ten se uvádí jako dokreslující statistika k hodnotě VaR.

\section{Modelovácí techniky}

Vedle historických scénářů je možné pro výpočet hodnoty VaR použít také modelů. Vzhledem k tomu, že VaR se vztahuje k časovému horizontu jednoho popř. několika málo dnů, je volatilita aktiv měřena na denní bázi. Jak již bylo zmiňováno dříve, lze roční volatilitu vyjádřit z roční volatility jako
\begin{equation*}
\sigma_{den} = \frac{\sigma_{rok}}{\sqrt{252}},
\end{equation*}
kdy předpokládáme, že rok má 252 pracovních dní.
Při výpočtu hodnoty VaR se dále často uvažuje, že očekávaná hodnota změny ceny aktiva je v sledovaném časovém horizontu nulová. Tento předpoklad je pochopitelně přijatelný pouze relativně krátká časová období. Obecně platí, že čím vyšší je směrodatná odchylka v poměru k očekávané výnosové míře, tím delší toto období může být. Dalším předpokladem je, že relativní změna ceny uvažovaného aktiva sleduje normální rozdělení. Výsledkem těchto úvah tedy je, že výnosová míra tohoto aktiva je náhodná veličina, kterou lze popsat rozdělením $N[0,\sigma]$.\\

\noindent \textbf{Příklad:} Uvažujme akcii, jejíž spotová cena je 50 USD. Jaký je 99\% VaR portfolia sestávajícího se ze 10~000 kusů této akcie za předpokladu, že uvažujeme časový horizont 14 dní? V rámci jednoho dne sleduje výnosová míra této akcie normální rozdělení se střední hodnotu 0 a směrodatnou odchylkou 1\%.\\
\begin{equation*}
VaR= 10~000 \cdot 50 \cdot 0.01 \cdot \Theta(1-0.99) \sqrt{14} = -43~522
\end{equation*}
Výsledný VaR tohoto portfolia je -43~522 USD\footnote{Z teoretického hlediska je pro výpočet hodnoty VaR lhostejné, zda-li je pozice v portfoliu dlouhá nebo krátká. V případě, že by se výnosová míra neřídila normálním rozdělením $N[0,\sigma]$, byla by hodnota VaR pro dlouhou a krátkou pozici rozdílná.}.\\

Jestliže se portfolio skládá z $n$ různých aktiv, je volatilita tohoto portfolia dána vztahem
\begin{equation*}
\sigma_{\Pi}^2 = \sum_{i=1}^n \sum_{j=1}^n \rho_{ij} \alpha_i \alpha_j \sigma_i \sigma_j,
\end{equation*}
kde $\alpha_i$ představuje váhu aktiva v portfoliu\footnote{Platí tedy $\sum_{i=1}^n \alpha_i=1$} a $\rho_{ij}$ je korelační koeficient mezi výnosovými mírami $i$-tého a $j$-tého aktiva. Postup výpočtu VaR je totožný s případem, kdy je portfolio tvořeno jediným aktivem.

\subsection{Lineární model}

Uvažujme portfolio, jehož hodnota je $\Pi$ a které se skládá z $n$ aktiv. Definujme parametr $\Delta$ $i$-tého aktiva jako
\begin{equation*}
\Delta_i = \frac{\delta \Pi}{\delta S_i},
\end{equation*}
kde $S_i$ představuje cenu tohoto aktiva. Dále definujme výnosovou míru $i$-tého aktiva $\delta x$ jako
\begin{equation*}
\delta x_i = \frac{\delta S_i}{S_i}
\end{equation*}
Změnu ceny portfolia $\Pi$ pak je možné vyjádřit vztahem
\begin{equation*}
\delta \Pi = \sum_{i=1}^n S_i \Delta_i \delta x_i
\end{equation*}
Platí, že v lineárním modelu lze váhu aktiva v portfoliu $\alpha_i$ vyjádřit jako $\alpha_i = S_i \Delta_i$. V tomto případě je možné změnu ceny portfolia vyjádřit jako $\delta \Pi = \sum_{i=1}^n \alpha_i \delta x_i$.

\subsection{Kvadratický model}

Jestliže portfolio obsahuje opce, je lineární model pouhou aproximací. Tento model nebere v potaz nelinearitu vztahu mezi cenou portfolia a podkladovými tržními proměnnými. Aproximaci je možné zlepšit použitím parametru gamma - z lineárního modelu se tak stává model kvadratický. Parametr gamma $i$-tého a $j$-tého aktiva definujeme jako
\begin{equation*}
\Gamma_{ij} = \frac{\delta^2 \Pi}{\delta S_i \delta S_j}
\end{equation*}
Změna ceny portfolia je pak dána vztahem
\begin{equation*}
\delta \Pi = \sum_{i=1}^n S_i \Delta_i \delta x_i + \sum_{i=1}^n \sum_{j=1}^n \frac{1}{2}S_i S_j \Gamma_{ij} \delta x_i \delta x_j
\end{equation*}
Tento vzorec je velice často vyjadřován také ve tvaru
\begin{equation}
\delta \Pi = \sum_{i=1}^n \alpha_i \delta x_i + \sum_{i=i}^n \sum_{j=1}^n \beta_{ij} \delta x_i \delta x_j
\end{equation}
kde $\beta_{ij}=\frac{1}{2}S_i S_j \Gamma_{ij}$

\subsection{Monte Carlo}
Metoda Monte Carlo je spíše než na analytickém přístupu založena na hrubé výpočetní síle. V případě výpočtu hodnty VaR portfolia lze postupovat následovně:
\begin{enumerate}
\item určit aktuální hodnotu portfolia s ohledem na dnešní hodnoty tržních proměnných
\item vygenerovat vektor náhodných veličin $\delta x_i$
\item určit hodnotu tržních proměnných s využitím vygenerovaných proměnných $\delta x_i$
\item určit hodnotu portfolia na základě odvozených tržních proměnných
\item vypočítat $\delta \Pi$ jako rozdíl hodnot získaných v krocích 1. a 4.
\item opakovat kroky 2. až 5. s cílem získat pravděpodobnostní rozdělení náhodné veličiny $\delta \Pi$
\end{enumerate}

\noindent \textbf{Poznámka:} Jednou z možností, jak zjednodušit výše uvedený postup, je využití vztahu (13.1). To zredukuje celý výše uvedený postup pouze na kroky 2 a 5\footnote{V tomto případě bude $\delta \Pi$ vypočteno přímo na základě (13.1).}.

\section{Mapování cash-flow}
Uvažujme portfolio, které se skládá pouze z jednoho dluhopisu s nominální hodnotou 1~000~000 USD a zbytkovou splatností 0.8 roku. Předpokládejme, že máme k dispozici údaje pro půlroční a roční zero-coupon bond. Nechť je příslušná půlroční úroková sazba rovna 6\% p.a. a roční úroková sazba rovna 7\% p.a. Interpolovaná sazba je rovna 6.8\% p.a. a současná hodnota dluhopisu je tedy
\begin{equation*}
\frac{1~000~000}{1.068^{0.8}}=948~730.80
\end{equation*}
Dále je třeba také interpolovat volatilitu. Je-li volatilita denních výnosů 6 měsíčního dluhopisu 0.1\% a votatilita denních výnosů  ročního dluhopisu 0.2\%, je volatilita 0.8 ročního dluhopisu rovna 0.18\%.
\begin{center}
\begin{tabular}{l c c}
\multicolumn{3}{c}{\textbf{Korelace denních výnosů}}\\
\textbf{} &
6 měsíční dluhopis & 
1 roční dluhopis \\
\hline
6 měsíční dluhopis & 1.0 &  0.7 \\
1 roční dluhopis   &  0.7 & 1.0 \\
\end{tabular}
\end{center}
Námi uvažovaný bond je možné replikovat pomocí 6 měsíčního a ročního dluhopisu.
\begin{equation*}
0.0018^2 = 0.001^2 \alpha^2 + 0.002^2(1-\alpha)^2 + 2 \cdot 0.7 \cdot 0.001 \cdot 0.002 \alpha (1-\alpha)
\end{equation*}
\begin{equation*}
\alpha = 0.156518
\end{equation*}
To znamená, že námi uvažovaný dluhopis je možné replikovat pomocí 6 měsíčního dluhopisu, jehož současná hodnota je $0.156518 \cdot 948~730.80 = 148~493.45$ USD, a pomocí ročního dluhopisu se současnou hodnotou $(1 - 0.156518) \cdot 948~730.80 = 800~237.35$ USD.
Výhodou tohoto přistupu je, že respektuje jak současnou hodnotu původního dluhopisu, tak rozptyl jeho cash-flow.
\begin{center}
\begin{tabular}{l r}
\textbf{Pozice v} &
\multicolumn {1}{c}{\textbf{Cash-flow}}\\
\hline
6 měsíční dluhopis &  148~493.45\\
roční dluhopis & 800~237.35\\
\hline 
\end{tabular}
\end{center}
Rozptyl výnosové míry dluhopisu je dán obecným vzorcem
\begin{equation*}
\sigma^2 = \sum_{i=1}^n \sum_{j=1}^n \alpha_i \alpha_j \sigma_i \sigma_j \rho_{ij}
\end{equation*}
tj. v našem případě
\begin{equation*}
\Big( \frac{148~493.45}{948~730.80} \Big)^2 0.001^2 + \Big( \frac{800~237.35}{948~730.80} \Big)^2 0.002^2 + 2 \cdot \frac{148~493.45}{948~730.80} \cdot \frac{800~237.35}{948~730.80} \cdot 0.001 \cdot 0.002 \cdot 0.7 = 3.24 \cdot 10^{-6}
\end{equation*}
což odpovídá původně vypočetné směrodatné odchylce 0.18\%. Dvacetidenní 99\% VaR je roven 7~637.10 USD.
\begin{equation*}
\sqrt{3.24 \cdot 10^{-6}} \cdot 948~730.80 \cdot \sqrt{20} = 7~637.10
\end{equation*}

\chapter{Odhad volatilit a korelací}

\section{Odhad volatilit}

Definujme $\sigma_n$ jako volatilitu tržní proměnné pro den $n$ odhadnutou koncem dne $n-1$. Dále definujme mezidenní změnu ceny (tj. denní výnosovou míru) aktiva jako
\begin{equation*}
u_i = \ln \Big(\frac{S_i}{S_{i-1}} \Big)
\end{equation*}
Máme-li k dispozici $m$ denních pozorování $u_i$, lze $\sigma_n^2$ odhadnout pomocí
\begin{equation}
\sigma_n^2 = \frac{1}{m-1} \sum_{i=1}^m (u_{n-i}-\bar{u})^2,
\end{equation}
kde
\begin{equation*}
\bar{u} = \frac{1}{m} \sum_{i=1}^m u_{n-i}-1
\end{equation*}

Pro účely výpočtu VaR se (14.1) upravuje následovně:
\begin{itemize}
\item $u_i$ je definováno jako
\begin{equation*}
u_i = \frac{S_i - S_{i-1}}{S_{i-1}}
\end{equation*}
\item Předpokládá se, že $\bar{u}$ je rovno nule.
\item $m-1$ je nahrazeno $m$.
\end{itemize}

\subsection{Aritmetický průměr}

Výše uvedené úpravy, které představují pouze marginální změny, umožňují převést rovnici (14.1) do tvaru
\begin{equation*}
\sigma_n^2 = \frac{1}{m} \sum_{i=1}^m u_{n-i}^2
\end{equation*}
Skutečná volatilita $\sigma_n^2$ náhodné veličiny $S$ je tak odhadnuta pomocí aritmerického průměru $u_{n-i}^2$.

\section{Vážený průměr}
V případě odhadu volatility $\sigma_n^2$ pomocí aritmetického průměru je všem pozorovaným $u_{n-i}^2$ přiřazena stejná váha. Další možnou modifikací je použití váženého průměru, kdy je větší váha přisuzována aktuálnějším datům.
\begin{equation}
\sigma_n^2 = \sum_{i=1}^m \alpha_i u_{n-i}^2
\end{equation}
Pro váhy $\alpha_i$ musí platit
\begin{equation*}
\sum_{i=1}^m \alpha_i = 1
\end{equation*}

\subsection{Model ARCH}

Je-li $V_L$ dlouhodobá míra volatility a $\gamma$ jí přiřazená váha, pak lze rovnici (14.2) upravit do tvaru
\begin{equation}
\sigma_n^2 = \gamma V_L + \sum_{i=1}^m \alpha_i u_{n-i}^2,
\end{equation}
přičemž opět musí platit
\begin{equation*}
\gamma + \sum_{i=1}^{m} \alpha_i = 1
\end{equation*}
Rovnice (14.3) je rovnicí tzv. ARCH(m) modelu.

\subsection{Model EWMA}
Dalším modelem je EWMA (Exponentially Weighthted Mowing Average), ve kterém váhy exponenciálně rostou s aktuálností dat. Pro tento model se nejčastějí používají váhy $\alpha_i = \lambda \alpha_{i-1}$. Rovnice pro výpočet odhadované volatility pak dána vztahem
\begin{equation*}
\sigma_n^2= \lambda \sigma_{n-1}^2 + (1-\lambda)u_{n-1}^2
\end{equation*}
Postupnými rekurentními substitucemi za $\sigma_{n-i}^2$ získáváme
\begin{equation*}
\sigma_n^2 = (1-\lambda) \sum_{i=1}^m \lambda^{i-1}u_{n-i}^2+\lambda^m \sigma_{n-m}^2
\end{equation*}
V případě druhé rovnice je výraz $\lambda^m \sigma_{n-m}^2$ pro dostatečně velká $m$ natolik malý, že je možné jej zanedbat. Rovnice pak odpovídá (14.2) pro $\alpha_i = (1-\lambda)\lambda^{i-1}$.

\subsection{Model GARCH(1,1)}
Model GARCH(1,1) je dán rovnicí
\begin{equation*}
\sigma_n^2 = \gamma V_L + \alpha u_{n-1}^2 + \beta \sigma_{n-1}^2
\end{equation*}
$V_L$ představuje dlouhodobou rovnovážnou volatilitu. Váhy $\alpha$, $\beta$ a $\gamma$ musí splňovat podmínku
\begin{equation*}
\gamma + \alpha + \beta = 1
\end{equation*}
GARCH(1,1) model je postaven na aktuálních odhadech volatility $\sigma_{n-1}^2$ a poslední dostupné změně ceny podkladového aktiva $u_{n-1}^2$.

Jestliže definujeme $\omega$ jako $\omega = \gamma V_L$, lze rovnici upravit do tvaru
\begin{equation}
\sigma_n^2 = \omega + \alpha u_{n-1}^2 + \beta \sigma_{n-1}^2
\end{equation}

Model GARCH(1,1) má tendenci konvergovat k dlouhodobé průměrné volatilitě $V_L$. Vzhledem k tomu, že tato jeho vlastnost odpovídá také empirickým pozorováním, patří GARCH k nejpoužívanějším modelům. Tento model je ekvivalentní stochastickému procesu, ve kterém je  volatilita $V$ popsána jako
\begin{equation*}
dV = \gamma(V_L - V)dt + \xi Vdz
\end{equation*}
kde čas je měřen ve dnech, $\gamma = 1 - \alpha - \beta$ a $\xi = \gamma \sqrt{2}$. Modelovaná volatilita konverguje k dlouhodobé volatilitě $V_L$ mírou $\gamma$.

\section{Odhad parametrů}

\subsection{Metoda maximální pravděpodobnosti}
Metoda maximální pravděpodobnosti slouží k odhadu parametrů statistických modelů. Parametry jsou vybírány tak, aby byla maximalizována pravděpodobnost, že model vygeneruje stejná data, jako na základě kterých byl příslušný model vytvořen.

Uvažujme náhodnou veličinu $X$, která je dána normálním rozdělením s nulovou střední hodnotou a volatilitou $v$. Dále uvažujme pozorování $u_1$, $u_2$, ..., $u_m$, která jsou daná stejným pravděpodobnostním rozdělením jako náhodná veličina $X$. Pravděpodobnost, že hodnota náhodné veličiny $X$ bude rovna $u_i$, je funkcí hustoty pravděpodobnosti
\begin{equation*}
\frac{1}{\sqrt{2 \Pi v}}e^{-\frac{u_i^2}{2v}}
\end{equation*}
Pravděpodobnost, že náhodná veličiny $X$ nabude hodnot $u_1$, $u_2$, ..., $u_m$ je pak dána
\begin{equation}
\prod_{i=1}^m \frac{1}{\sqrt{2 \Pi v}}e^{-\frac{u_i^2}{2v}}
\end{equation}
Maximalizace výše uvedeného výrazu je ekvivalentní maximalizaci jeho logaritmu. Jestliže (14.5) zlogaritmujeme a ignorujeme konstatní členy, lze tento výraz upravit do tvaru
\begin{equation*}
-m \ln v - \sum_{i=1}^m \frac{u_i^2}{v}
\end{equation*}
Derivováním výše uvedeného výrazu podle $v$ a položením rovno nule, zjistíme, že hodnota $v$, pro kterou je maximalizována pravděpodobnost (14.5), je rovna
\begin{equation*}
\frac{1}{m} \sum_{i=1}^m u_i^2
\end{equation*}

\subsection{Odhad parametrů modelu GARCH(1,1)}
Definujme $v_i = \sigma_i^2$ jako odhad volatility pro den $i$ učiněný na konci dne $i-1$ a $\omega$ jako $\omega = \gamma V_L$, kde $V_L$ je  dlouhodobá průměrná volatilita. Předpokládejme, že $u_i$ sleduje normální rozdělení s nulovou střední hodnotou. V případě modelu GARCH(1,1) je výchozí výraz pro odhad parametrů podobný (14.5), s tím rozdílem, že namísto $v$ je použito $v_i$.
\begin{equation*}
\prod_{i=1}^m \frac{1}{\sqrt{2 \Pi v_i}}e^{-\frac{u_i^2}{2v_i}}
\end{equation*}
Zlogaritmováním tohoto vztahu a zanedbáním konstant získáme
\begin{equation}
\sum_{i=1}^m \Big( -\ln v_i - \frac{u_i^2}{v_i} \Big)
\end{equation}

\subsubsection{Ilustrace na příkladě}

Následující tabulka obsahuje historické denní údaje o kurzu $S_i$ pro měnový pár JPY/USD. Tyto údaje použijeme pro odhad parametrů modelu GARCH(1,1).
\begin{center}
\begin{tabular}{c c c c c c}
\textbf{Datum} &
\textbf{Den i} &
\textbf{$S_i$} &
\textbf{$u_i$} &
\textbf{$v_i = \sigma_i^2$} &
\textbf{$-\ln{v_i}-u_i^2/v_i$} \\
\hline
06.01.1988 & 1 & 0.007728 &           &            & \\
07.01.1988 & 2 & 0.007779 &  0.006599 &            & \\
08.01.1988 & 3 & 0.007746 & -0.004242 & 0.00004355 & 9.6283\\
11.01.1988 & 4 & 0.007816 &  0.009037 & 0.00004198 & 8.1329\\
12.01.1988 & 5 & 0.007837 &  0.002687 & 0.00004455 & 9.8568\\
13.01.1988 & 6 & 0.007924 &  0.011101 & 0.00004220 & 7.1529\\
... & ... & ... & ... & ... & ...\\
13.08.1997 & 2421 & 0.008643 &  0.003374 & 0.00007626 & 9.3321\\
14.08.1997 & 2422 & 0.008493 & -0.017309 & 0.00007092 & 5.3294\\
15.08.1997 & 2423 & 0.008495 &  0.000144 & 0.00008417 & 9.3824\\
\hline 
     &            &          &           &            &22 063.58\\
\end{tabular}
\end{center}
Připomeňme, že parametr $u_i$ je definován jako
\begin{equation*}
u_i = \frac{S_i - S_{i-1}}{S_{i-1}}
\end{equation*}
Odhad volatility $\sigma_3^2$ pro třetí den je roven $u_2^2$. Pro následující dny je volatilita vypočtena pomocí (14.4). Hodnoty parametrů $\omega$, $\alpha$ a $\beta$ modelu GARCH(1,1) jsou určeny iteračně tak, aby volatilita vypočtená podle (14.4) maximalizovala (14.6). To odpovídá maximalizaci součtu posledního sloupce tabulky. Pro námi uvažovaná historická data jsou hodnoty těchto parametrů $\gamma = 0.00000176$, $\alpha = 0.0626$ a $\beta = 0.8976$. Dlouhodobá volatilita $V_L$ je tedy rovna
\begin{equation*}
\frac{\omega}{1-\alpha-\beta}=\frac{0.00000176}{0.0398}=0.00004422
\end{equation*}
Další možností je cílovou volatilitu $V_L$ určit expertním  odhadem popř. jako průměrnou historickou hodnotu volatility a zbývají parametry $\alpha$ a $\beta$\footnote{Parametr $\gamma$ je dán vztahem $\gamma = 1 - \alpha - \beta$.} opět určit iterativně.

\subsubsection{Autokorelace}

Autokorelace je nástroj pro odhalování opakujících se sekvencí. Autokorelace tedy vyjadřuje míru korelace náhodné veličiny se sebou sama. Autokorelace diskrétní náhodné veličiny $X$ je rovna
\begin{equation*}
\eta_k = \frac{1}{(n-k)\sigma^2}\sum_{i=1}^{n-k}[X_i - \bar{X}][X_{i+k}-\bar{X}]
\end{equation*}
kde $\sigma^2$ je volatilita a $\bar{X}$ střední hodnota náhodné veličiny $X$. Parametr $k$ pak představuje délku kroku, přes který se autokorelace měří.

Pomocí autokorelace je možné posuzovat kvalitu výstupních hodnot získaných pomocí modelu GARCH(1,1). Empirickými pozorováními bylo zjištěno, že existují období relativně vysoké a relativně nízké volatility. Pro $u_i^2$ tedy mělo platit, že je-li $u_i^2$ vysoké, jsou také $u_{i+1}^2$, $u_{i-2}^2$, ... , $u_{i-n}^2$ vysoká a naopak. Parametr $u_i^2$ by tak měl vykazovat poměrně vysokou míru autokorelace pro vhodně zvolený krok $k$. V případě, že model GARCH(1,1) dává správné výstupní hodnoty pro volatilitu $\sigma_i^2$, měla by naopak proměnná $u_i^2/\sigma_i^2$ vykazovat pro stejný krok $k$ nízkou autokorelaci. Následující tabulka udává autokorelaci proměnných $u_i^2$ a $u_i^2/\sigma_i^2$ pro kroky $k = 1, 2, ..., 15$ vypočtenou na základě dříve uváděných údajů pro měnový pár JPY/USD.
\begin{center}
\begin{tabular}{c c c}
\textbf{Krok} &
\textbf{Autokorelace $u_i^2$} &
\textbf{Autokorelace $u_i^2/\sigma_i^2$}\\
\hline
 1 & 0.072 &  0.004\\
 2 & 0.041 & -0.005\\
 3 & 0.057 &  0.008\\
 4 & 0.107 &  0.003\\
 5 & 0.075 &  0.016\\
 6 & 0.066 &  0.008\\
 7 & 0.019 & -0.033\\
 8 & 0.085 &  0.012\\
 9 & 0.054 &  0.010\\
10 & 0.030 & -0.023\\
11 & 0.038 & -0.004\\
12 & 0.038 & -0.021\\
13 & 0.057 & -0.001\\
14 & 0.040 &  0.002\\
15 & 0.007 & -0.028\\
\end{tabular}
\end{center}
Autokorelace pro $u_i^2$ jsou u všech kroků pozitivní. V případě $u_i^2/\sigma_i^2$ figurují pozitivní i negativní autokorelace a jejich výše je nižší než v případě $u_i$. Zdá se, že výše nastíněné podmínky jsou splněny. Pro testování míry autokorelace je používán tzv. Ljung-Boxův test. V případě řady $m$ pozorování je příslušná statistika rovna
\begin{equation*}
s = m \sum_{k=1}^K w_k \eta^2_k
\end{equation*}
kde $\eta_k$ je autokorelace pro krok $k$ a
\begin{equation*}
w_k = \frac{m-2}{m-k}
\end{equation*}
V případě $k=15$ může být hypotéza nulové autokorelace zamítnuta s 95\% pravděpodobností je-li $s>25$. Pro $u_i^2$ je hodnota statistiky 123 a pro $u_i^2/\sigma_i^2$ pak 8.2.

\subsubsection{Použití modelu GARCH(1,1) pro modelování budoucí volatility}

Model GARCH(1,1) je možné použít také pro prognózu vývoje budoucí volatility. S využitím $\gamma = 1 - \alpha - \beta$ lze volatilitu na konci $n-1$ dne pro $n$-tý den vyjádřit jako
\begin{equation*}
\sigma_n^2 = (1 - \alpha - \beta)V_L + \alpha u_{n-1}^2 + \beta \sigma_{n-1}^2
\end{equation*}
respektive pro $n+k$-tý den jako
\begin{equation*}
\sigma_{n+k}^2 = V_L + \alpha (u_{n+k-1}^2-\alpha)+\beta(\sigma_{n+k-1}^2-V_L)
\end{equation*}
Vzhledem k tomu, že dle (14.1) je očekávaná hodnota $u_{n+k-1}^2$ za předpokladu $\bar{u}=0$ rovna $\sigma_{n+k-1}^2$, platí
\begin{equation*}
E[\sigma_{n+k}^2]=V_L + (\alpha + \beta)(E[\sigma_{n+k-1}^2]-V_L)
\end{equation*}
Rekurzivní aplikací této rovnice na sebe sama ji lze upravit do tvaru 
\begin{equation}
E[\sigma_{n+k}^2]=V_L + (\alpha + \beta)^k(\sigma_n^2-V_L)
\end{equation}
Pomocí (14.7) je možné předvídat budoucí volatilitu v den $n+k$ s využitím informací, které jsou dostupné na konci dne $n-1$. K tomu, aby predigovaná volatilita s rostoucím $k$ konvergovala k dlouhodobé volatilitě $V_L$, musí platit $\alpha + \beta < 1$.

Uvažujme opci s životností mezi dny $n$ a $n + N$. S použitím (14.7) je možné vypočítat očekávanou volatilitu opce v daném období jako
\begin{equation*}
\frac{1}{N}\sum_{k=0}^{N-1}E[\sigma_{n+k}^2]
\end{equation*}
Čím delší je životnost opce, tím blíže je její volatilita dlouhodobé rovnovážné volatilitě $V_L$.

\section{Korelace}

\subsection{Definice korelace a kovariance}

Korelace mezi dvěma náhodnými proměnnými $x$ a $y$ je definována jako
\begin{equation*}
\rho_{xy}=\frac{cov(x,y)}{\sigma_x \sigma_y}
\end{equation*}
Kovariance $cov(x,y)$ mezi náhodnými veličinami $x$ a $y$ je pak definována jako
\begin{equation*}
cov(x,y) = E[(X-\mu_x)(Y-\mu_y)]
\end{equation*}

\subsection{Odhad korelace}

Odhad korelace se tak skládá ze dvou částí a to odhadu volatilit a odhadu kovarince. Odhadem volatilit jsme se zabývali v předchozí podkapitole. Pro odhad kovariance lze použít analogické modely jako v případě volatility. Pro účely výpočtu korelace by kovariance a volatility měly být vypočteny na základě stejného modelu, aby tak byla zaručena vzájemná porovnatelnost.

Definujme $x_i$ a $y_i$ jako
\begin{equation*}
x_i = \frac{X_i - X_{i-1}}{X_{i-1}}
\end{equation*}
\begin{equation*}
y_i = \frac{Y_i - Y_{i-1}}{Y_{i-1}}
\end{equation*}
Rovnice pro výpočet kovariance pomocí váženého průměru je dána vztahem
\begin{equation*}
cov(x,y)_n = \sum_{i=1}^m \alpha_i x_{n-i} y_{n-i} 
\end{equation*}
Kovarianci podle modelu EWMA lze vypočíst dle
\begin{equation*}
cov(x,y)_n = \lambda \cdot cov(x,y)_{n-1} + (1-\lambda)x_{n-1}y_{n-1}
\end{equation*}
V případě modelu GARCH(1,1) je kovariance dána vztahem
\begin{equation*}
cov(x,y)_n = \gamma C_L + \alpha x_{n-1}y_{n-1}+\beta \cdot cov(x,y)_{n-1}
\end{equation*}
Korelace $\rho_n^{xy}$ pro $n$-tý den je definována jako
\begin{equation*}
\rho_n^{xy} = \frac{cov(x,y)_n}{\sigma_n^x \sigma_n^y}
\end{equation*}

\subsection{Kovariační a korelační matice}

Z vypočtených kovariancí resp. korelací je možné zkonstruovat kovarianční resp. korelační matici. Aby tyto matice by interně konzistentní, musí být pozitivně semidefinitivní.

Matice $\Omega$ typu $N \times N$ je pozitivně semidefinitivní tehdy a jen tehdy, je-li pro všechny reálné vektory $w$ typu $N \times 1$ splněna podmínka
\begin{equation}
w \Omega w^T \ge 0
\end{equation}

Uvažujme portfolio, které se skládá z $N$ investičních instrumentů. Z kovariancí těchto investičních instrumentů je možné zkonstruovat kovarianční matici $\Omega$. Rozptyl celého portfolia je definován jako $\sigma_P^2 = w \Omega w^T$, kde $w$ představuje vektor vah jednotlivých investičních instrumentů v rámci portfolia. Vzhledem k tomu, že rozptyl portfolia pro libovolné $w$ nemůže být záporný, musí být kovarianční matice pozitivně semidefinitivní.

Interpretace v případě korelační matice již bohužel není tak přímočará. Uvažujme korelační matici
\begin{equation*}
\Omega = \left[
{\begin{array}{c c c}
1.0 & 0.0 & 0.9\\
0.0 & 1.0 & 0.9\\
0.9 & 0.9 & 1.0\\
\end{array}}
\right]
\end{equation*}
Podle této matice je první prvek silně korelován se třetím prvkem, který je zase silně korelován s druhým prvkem. Z toho vyplývá, že první a třetí prvek by měly být také silně korelovány. Podle korelační matice však vykazují nulovou korelaci. Korelační matice tak postrádá vnitřní logiku a lze dokázat, že nesplňuje podmínku (14.8).

\chapter{Numerické metody}

\section{Binomický strom}
Uvažujme opci na akcii s nulovým dividendovým výnosem. Rozdělme zbytkovou dobu splatnosti opce na velký počet intervalů délky $\delta t$. Dále uvažujme, že na konci každého tohoto intervalu se původní cena $S$ změní na $Su$ nebo $Sd$, kde $u > 1$ a $  0 \le d < 1$. Pravděpodobnost růstu ceny akcie označme $p$; pravděpodobnost poklesu ceny pak bude rovna $1-p$. 

\begin{center}
	\begin{pspicture}(0,0)(7,4)
		\rput(3.5,0.5){Binomický strom}
		\rput(3.5,0){změna ceny v časovém intervalu $\delta t$}

		\psline[arrows=->](1.5,2.5)(5.5,3.5)
		\psline[arrows=->](1.5,2.5)(5.5,1.5)
		\rput(3.5,3.5){$p$}
		\rput(3.5,1.5){$1-p$}
		\rput(1,2.5){$S$}
		\rput(6,3.5){$Su$}
		\rput(6,1.5){$Sd$}
	\end{pspicture}
\end{center}

Předpokládejme, že se nacházíme v rizikově neutrálním světě. Očekávaný výnos ze všech investičních instrumentů tak odpovídá bezrizikové úrokové míře. Očekávaná cena akcie na konci časového období je tedy $Se^{r \delta t}$. Proto platí
\begin{equation*}
Se^{r \delta t} = pSu + (1-p)Sd
\end{equation*}
\begin{equation}
e^{rt}=pu + (1-p)d
\end{equation}
Protože volatilita relativní změny ceny akcie je v časovém intervalu $\delta t$ rovna $\sigma^2 \delta t$ a rozptyl náhodné veličiny $X$ lze vyjádřit jako $D[X]= E[X^2]-(E[X])^2$, platí
\begin{equation*}
pu^2 + (1+p)d^2 - \Big[pu + (1-p)d\Big]^2 = \sigma^2 \delta t
\end{equation*}
Substitucí za $p$ dle (15.1) se výše uvedený vztah zredukuje na
\begin{equation*}
e^{r \delta t}(u+d)-ud-e^{2r \delta t}=\sigma^2 \delta t
\end{equation*}
Přidáme-li předpoklad
\begin{equation*}
d = \frac{1}{u}
\end{equation*}
získáme po následných úpravách, kdy ignorujeme členy vyššího řádu než $\delta t$, následující rovnice.
\begin{equation}
p = \frac{e^{r\delta t}-d}{u-d}
\end{equation}
\begin{equation}
u = e^{\sigma \sqrt{\delta t}}
\end{equation}
\begin{equation}
d = e^{- \sigma \sqrt{\delta t}}
\end{equation}

Jestliže určitý časový horizont rozdělíme do $i$ intervalů délky $\delta t$, kdy na konci každého intervalu cena akcie vzroste o $1-u$ popř. poklesne o $1-d$ procent, existuje na konci tohoto období $i+1$ cen. Jednotlivé ceny jsou rovny
\begin{equation*}
Su^id^{i-j},~j = 0, 1, ..., i
\end{equation*}
Použijeme-li vztah $d=\frac{1}{u}$, lze všechny možné ceny na konci $i$-tého intervalu vyjádřit jako
\begin{equation*}
Su^{2j-i},~j = 0, 1, ..., i
\end{equation*}

\subsection{Ocenění opcí pomocí binomického stromu}

\subsubsection{Ocenění evropské opce}

S pomocí binomického stromu je možné ocenit akciové opce. V případě evropské opce je její cena dána vztahy
\begin{equation*}
c = e^{-r n \delta t} \sum_{i=0}^n \max (Su^id^{n-i}-K,0)  \binom{n}{i} p^j(1-p)^{n-i}
\end{equation*}
\begin{equation*}
p = e^{-r n \delta t} \sum_{i=0}^n \max (K-Su^id^{n-i},0) \binom{n}{i} p^i(1-p)^{n-i}
\end{equation*}
kde $c$ představuje cenu kupní opce, $p$ cenu prodejní opce, $n$ je počet kroků binomického stromu, $K$ je realizační cena, výraz $\max(Su^id^{n-i}-K,0)$ resp. $\max(K-Su^id^{n-i},0)$ představuje cenu opce pro jednotlivé koncové uzly binomického stromu a $\binom{n}{n-i}p^i(1-p)^{n-i}$ odpovídá pravděpodobnosti výskytu daného uzlu. Pro účely výpočtu pravděpodobnosti $p$ připomeňme, že $p = \frac{e^{r\delta t}-d}{u-d}$.

\subsubsection{Ocenění americké opce}

Pomocí binomických stromů lze ocenit také americké opce. Narozdíl od výše uvedeného příkladu se americké opce oceňují od konce binomického stromu směrem k jeho počátku. Důvodem je možnost předčasného uplatnění opce.

Uvažujme americkou prodejní opci na akcii s nulovým dividendovým výnosem. Rozdělme životnost této opce na $N$ intervalů délky $\delta t$. Definujme $f_{i,j}$ jako hodnotu opce v uzlu ($i$,$j$).
\begin{center}
	\begin{pspicture}(0,0)(10,10.5)
		\rput(4.5,0.5){Binomický strom}
                \rput(4.5,0.0){intervaly a uzly}
                
		\psline(0.5,6.0)(2.5,7.0)
                \psline(0.5,6.0)(2.5,5.0)

                \psline(2.5,7.0)(4.5,8.0)
                \psline(2.5,7.0)(4.5,6.0)
                \psline(2.5,5.0)(4.5,6.0)
                \psline(2.5,5.0)(4.5,4.0)

                \psline[linestyle=dashed](4.5,8.0)(6.5,9.0)
                \psline[linestyle=dashed](4.5,8.0)(6.5,7.0)
                \psline[linestyle=dashed](4.5,6.0)(6.5,7.0)
                \psline[linestyle=dashed](4.5,6.0)(6.5,5.0)
                \psline[linestyle=dashed](4.5,4.0)(6.5,5.0)
                \psline[linestyle=dashed](4.5,4.0)(6.5,3.0)

                \psline(6.5,9.0)(8.5,10.0)
                \psline(6.5,9.0)(8.5,8.0)
                \psline(6.5,7.0)(8.5,8.0)
                \psline(6.5,7.0)(8.5,6.0)
                \psline(6.5,5.0)(8.5,6.0)
                \psline(6.5,5.0)(8.5,4.0)
                \psline(6.5,3.0)(8.5,4.0)
                \psline(6.5,3.0)(8.5,2.0)
                
                \psline[linestyle=dotted](0.5,6.3)(0.5,1.5)
                \psline[linestyle=dotted](2.5,7.3)(2.5,1.5)
                \psline[linestyle=dotted](4.5,8.3)(4.5,1.5)
                \psline[linestyle=dotted](6.5,9.3)(6.5,1.5)
                \psline[linestyle=dotted](8.5,10.3)(8.5,1.5)

                \rput(0.5,1.2){\small{$0$}}
                \rput(2.5,1.2){\small{$1$}}
                \rput(4.5,1.2){\small{$2$}}
                \rput(6.5,1.2){\small{$N-1$}}
                \rput(8.5,1.2){\small{$N$}}

                \rput(0.0,5.8){\tiny{($0$,$0$)}}
                \rput(2.0,7.2){\tiny{($1$,$1$)}}
                \rput(2.0,4.8){\tiny{($1$,$0$)}}
                \rput(4.0,8.2){\tiny{($2$,$2$)}}
                \rput(3.8,6.0){\tiny{($2$,$1$)}}
                \rput(4.0,3.8){\tiny{($2$,$0$)}}
                \rput(5.6,9.2){\tiny{($N-1$,$N-1$)}}
                \rput(5.4,7.0){\tiny{($N-1$,$N-2$)}}
                \rput(5.4,5.0){\tiny{($N-1$,$N-j$)}}
                \rput(5.8,2.8){\tiny{($N-1$,$0$)}}

                \rput(9.0,10.0){\tiny{($N$,$N$)}}
                \rput(9.2,8.0){\tiny{($N$,$N-1$)}}
                \rput(9.2,6.0){\tiny{($N$,$N-j$)}}
                \rput(9.0,4.0){\tiny{($N$,$1$)}}
                \rput(9.0,2.0){\tiny{($N$,$0$)}}
                
	\end{pspicture}
\end{center}
Hodnota opce pro jednotlivé koncové uzly ($N$,$j$) v době splatnosti je
\begin{equation*}
f_{N,j} = \max \big( K - Su^jd^{N-j},0 \big),~j=0, 1, ...,N 
\end{equation*}
V koncových uzlech opce zanikne, a proto neuvažujeme možnost jejího předčasného uplatnění. V uzlech, které předcházejí koncovým uzlům, je však třeba brát možnost předčasného uplatnění opce v potaz. Jestliže by možnost předčasného uplatnění opce neexistovala, bylo by možné její hodnotu v uzlu ($i$,$j$) vyjádřit jako
\begin{equation*}
f_{i,j}=e^{-r \delta t} \big[pf_{i+1,j+1} + (1-p)f_{i+1,j} \big]
\end{equation*}
V případě, že budeme uvažovat možnost předčasného uplatnění opce, modifikuje se výše uvedený vztah do podoby
\begin{equation*}
f_{i,j} = \max \big\{ K - Su^jd^{i-j},e^{-r \delta t} \big[pf_{i+1,j+1}+(1-p)f_{i+1,j} \big]\big\}
\end{equation*}
kde $ K - Su^jd^{i-j}$ představuje výnos z okamžitého uplatnění opce v uzlu ($i$,$j$) a $e^{-r \delta t} \big[pf_{i+1,j+1}+(1-p)f_{i+1,j} \big]$ očekávanou hodnotu opce v uzlu ($i$,$j$) za podmínky, že nebude v tomto uzlu předčasně uplatněna. Vzhledem k tomu, že výpočet začíná v čase splatnosti opce $T$ a postupuje směrem k počátečnímu uzlu binomického stromu, odráží hodnota opce v čase $i \delta t$ nejen efekt předčasného uplatnění v čase $i \delta t$, ale také efekt předčasného uplatnění v následujících intervalech. Výsledná hodnota opce je rovna hodnotě opce v počátečním uzlu ($0$,$0$) binomického stromu.

\subsection{Řecká písmena}

V následujícím textu představuje $f_{ij}$ hodnotu opce v uzlu ($i$,$j$). Definice jednotlivých řeckých písmen odpovídá definici z kapitoly 11.

\subsubsection{Delta}

Hodnoty získané v rámci modelování binomických stromů lze použít také pro výpočet řeckých písmen. Obecná definice řeckého písmene delta je
\begin{equation*}
\Delta = \frac{\delta \Pi}{\delta S},
\end{equation*}
kde $\delta \Pi$ reps. a $\delta S$ představuje změnu ceny opce resp. akcie v časovém intervalu délky $\delta t$. Pro časový interval $1 \cdot \delta t$ lze tak deltu vyjádřit jako
\begin{equation*}
\Delta = \frac{f_{11}-f_{10}}{S_0u - S_0d}
\end{equation*}

\subsubsection{Gamma}

Pro výpočet řeckého písmene gamma je třeba vypočítat dvě delty v čase $2 \cdot \delta t$. Připomeňme, že
\begin{equation*}
\Gamma = \frac{\delta^2 \Pi}{\delta S^2}
\end{equation*}
Vzorec pro numerický výpočet druhé derivace funkce $f(x)$ je
\begin{equation*}
\frac{\frac{f(x+h_1)-f(x)}{h_1}-\frac{f(x)-f(x-h_2)}{h_2}}{\frac{1}{2}(h_1 + h_2)}
\end{equation*}
V našem případě je $x = S_0$, $x+h_1 = Su^2$, $x-h_2 = Sd^2$, $h_1 = S_0(u^2-1)$ a $h_2=S_0(1-d^2)$. Delty, které jsou zapotřebí pro výpočet řeckého písmene gamma, lze v případě klasického binomického stromu vypočíst jako
\begin{equation*}
\Delta_1 = \frac{f_{22}-f_{21}}{Su^2 - S_0}
\end{equation*}
\begin{equation*}
\Delta_2 = \frac{f_{21}-f_{20}}{S_0 - S_0d^2}
\end{equation*}
Řecké písmeno gamma lze pak definovat jako
\begin{equation*}
\Gamma = \frac{\frac{f_{22}-f_{21}}{S_0u^2-S_0}-\frac{f_{21}-f_{20}}{S_0-S_0d^2}}{\frac{1}{2}S_0(u^2-d^2)}
\end{equation*}

\subsubsection{Theta}

Řecké písmeno theta, které vyjadřuje změnu ceny opce v čase, je dáno vztahem
\begin{equation*}
\Theta = \frac{f_{21}-f_{00}}{2 \cdot \delta t}
\end{equation*}

\subsubsection{Vega}

V případě řeckého písmene vega je třeba mít k dispozici dva binomické stromy - jeden pro standardní volatilitu $\sigma$ a druhý pro volatilitu $\sigma^*$. Vega je pak definována jako
\begin{equation*}
\nu = \frac{f^* - f}{\delta \sigma}
\end{equation*}
kde $f^*$ je hodnota opce vypočtená pro binomický strom s volatilitou $\sigma^*$ a $f$ hodnota opce vypočtená pro binomický strom s volatilitou $\sigma$. Ostatní parametry obou uvažovaných binomických stromů jsou shodné.

\subsubsection{Rho}

Podobným způsobem lze vypočítat také řecké písmeno rho. Opět jsou zapotřebí dva binomické stromy, tentokráte pro dvě rozdílné bezrizikové úrokové sazby $r$ a $r*$.
\begin{equation*}
\rho = \frac{f^* - f}{\delta r}
\end{equation*}

\subsection{Ostatní případy}

\subsubsection{Výnosová míra generovaná pokladovým aktivem}

Jestliže chceme vypočítat hodnoty opce, kdy podkladové aktivum generuje výnosovou míru $q$\footnote{Příkladem takovéhoto aktiva může být bankovní účet se spojitým úročením.}, je postup zcela identický pouze s tím rozdílem, že pravděpodobnost růstu ceny aktiva v časovém horizontu $\delta t$ je definována jako
\begin{equation*}
p = \frac{e^{(r-q)\delta t}-d}{u-d}
\end{equation*}

\subsubsection{Výplata dividend}

Další možnou modifikaci představuje situace, kdy podkladovým aktivem je akcie, ze které je majiteli vyplácena dividenda. Právě o výši vyplácené dividendy by se v tzv. ex-dividendový den měla snížit cena akcie. Před tímto dnem tedy pro cenu akcie v uzlu ($i$,$j$) platí
\begin{equation*}
Su^jd^{i-j},~j=0, 1, ..., i
\end{equation*}
Po tomto datu je pak cena akcie dána vztahem
\begin{equation*}
S(1-\delta)u^jd^{i-j},~j=0, 1, ..., i
\end{equation*}
kde $\delta$ představuje dividendový výnos. Hodnoty $p$, $u$ a $d$ jsou vypočteny podle (15.2), (15.3) a (15.4). Tento vztah lze poměrně snadno použít také v případě opakované výplaty dividendy. Je-li $\delta_i$ celkovým dividendovým výnosem mezi časem 0 až $i\delta t$, modifikuje se vztah do tvaru
\begin{equation*}
S(1-\delta_i)u^jd^{i-j},~j=0, 1, ..., i
\end{equation*}
\begin{center}
	\begin{pspicture}(0,0)(10,10.5)
		\rput(4.5,0.5){Binomický strom}
                \rput(4.5,0.0){dopad dividendového výnosu $\delta$ na počet uzlů}

		\psline[arrows=-*](0.5,6.0)(2.5,7.0)
                \psline[arrows=-*](0.5,6.0)(2.5,5.0)

                \psline[arrows=-*](2.5,7.0)(4.5,8.0)
                \psline[arrows=-*](2.5,7.0)(4.5,6.0)
                \psline[arrows=-*](2.5,5.0)(4.5,6.0)
                \psline[arrows=-*](2.5,5.0)(4.5,4.0)

                \psline[arrows=-*](4.5,7.8)(6.5,8.8)
                \psline[arrows=-*](4.5,7.8)(6.5,6.8)
                \psline[arrows=-*](4.5,5.8)(6.5,6.8)
                \psline[arrows=-*](4.5,5.8)(6.5,4.8)
                \psline[arrows=-*](4.5,3.8)(6.5,4.8)
                \psline[arrows=-*](4.5,3.8)(6.5,2.8)

                \psline[arrows=-*](4.5,8.0)(4.5,7.8)
                \psline[arrows=-*](4.5,6.0)(4.5,5.8)
                \psline[arrows=-*](4.5,4.0)(4.5,3.8)

                \psline[arrows=-*](6.5,8.8)(8.5,9.8)
                \psline[arrows=-*](6.5,8.8)(8.5,7.8)
                \psline[arrows=-*](6.5,6.8)(8.5,7.8)
                \psline[arrows=-*](6.5,6.8)(8.5,5.8)
                \psline[arrows=-*](6.5,4.8)(8.5,5.8)
                \psline[arrows=-*](6.5,4.8)(8.5,3.8)
                \psline[arrows=-*](6.5,2.8)(8.5,3.8)
                \psline[arrows=-*](6.5,2.8)(8.5,1.8)

                \psline[linestyle=dotted](2.5,2.5)(2.5,8.5)
                \psline[linestyle=dotted](4.5,2.5)(4.5,8.5)
                \rput(3.5,2.8){\tiny{ex-dividenový den}}

                \rput(0.0,5.8){\tiny{$S_0$}}
                
                \rput(2.0,7.2){\tiny{$S_0u$}}
                \rput(2.0,4.8){\tiny{$S_0d$}}

                \rput(5.4,7.8){\tiny{$S_0u^2(1-\delta)$}}
                \rput(5.3,5.8){\tiny{$S_0(1 - \delta)$}}
                \rput(5.4,3.8){\tiny{$S_0d^2(1 - \delta)$}}

                \rput(7.4,8.8){\tiny{$S_0u^2(1-\delta)$}}
                \rput(7.4,6.8){\tiny{$S_0u(1 - \delta)$}}
                \rput(7.4,4.8){\tiny{$S_0d(1 - \delta)$}}
                \rput(7.4,2.8){\tiny{$S_0d^2(1 - \delta)$}}
               

	\end{pspicture}
\end{center}

Namísto dividendového výnosu $\delta$ je možné použít přímo výši dividendy $D$. Spadá-li ex-dividendový den mezi $k \delta t$ a $(k+1) \delta t$, pak pro $i \le k$ je cena akcie rovna
\begin{equation*}
S_0 u^j d^{i-j},~j=0, 1, ..., i
\end{equation*}
pro $i=k+1$ 
\begin{equation*}
Su^jd^{i-j}-D,~j=0, 1, ..., i
\end{equation*} 
a pro $i=k+2$
\begin{equation*}
(Su^jd^{i-j}-D)u,~j=0, 1, ..., i
\end{equation*}
resp.
\begin{equation*}
(Su^jd^{i-j}-D)d,~j=0, 1, ..., i
\end{equation*}
V případě výplaty dividend $D$ platí, že pro $i=k+m$ existuje $m(k+2)$ uzlů namísto $k+m+1$. Počet uzlů binomického stromu je tedy vyšší, než když uvažujeme dividendový výnos $\delta$.
\begin{center}
	\begin{pspicture}(0,0)(10,10.5)
                \rput(4.5,0.0){dopad dividendy $D$ na počet uzlů}

		\psline[arrows=-*](0.5,6.0)(2.5,7.0)
                \psline[arrows=-*](0.5,6.0)(2.5,5.0)

                \psline[arrows=-*](2.5,7.0)(4.5,8.0)
                \psline[arrows=-*](2.5,7.0)(4.5,6.0)
                \psline[arrows=-*](2.5,5.0)(4.5,6.0)
                \psline[arrows=-*](2.5,5.0)(4.5,4.0)

                \psline[arrows=-*](4.5,7.8)(6.5,8.8)
                \psline[arrows=-*](4.5,7.8)(6.5,6.8)
                \psline[arrows=-*](4.5,5.6)(6.5,6.6)
                \psline[arrows=-*](4.5,5.6)(6.5,4.6)
                \psline[arrows=-*](4.5,3.4)(6.5,4.4)
                \psline[arrows=-*](4.5,3.4)(6.5,2.4)

                \psline[arrows=-*](4.5,8.0)(4.5,7.8)
                \psline[arrows=-*](4.5,6.0)(4.5,5.6)
                \psline[arrows=-*](4.5,4.0)(4.5,3.4)

                \psline[arrows=-*](6.5,8.8)(8.5,9.8)
                \psline[arrows=-*](6.5,8.8)(8.5,7.8)
                \psline[arrows=-*](6.5,6.8)(8.5,7.8)
                \psline[arrows=-*](6.5,6.8)(8.5,5.8)
                \psline[arrows=-*](6.5,6.6)(8.5,7.6)
                \psline[arrows=-*](6.5,6.6)(8.5,5.6)
                \psline[arrows=-*](6.5,4.6)(8.5,5.6)
                \psline[arrows=-*](6.5,4.6)(8.5,3.6)
                \psline[arrows=-*](6.5,4.4)(8.5,5.4)
                \psline[arrows=-*](6.5,4.4)(8.5,3.4)
                \psline[arrows=-*](6.5,2.4)(8.5,3.4)
                \psline[arrows=-*](6.5,2.4)(8.5,1.4)

                \psline[linestyle=dotted](2.5,2.5)(2.5,8.5)
                \psline[linestyle=dotted](4.5,2.5)(4.5,8.5)
                \rput(3.5,2.8){\tiny{ex-dividenový den}}

                \rput(0.0,5.8){\tiny{$S_0$}}
                \rput(2.0,7.2){\tiny{$S_0u$}}
                \rput(2.0,4.8){\tiny{$S_0d$}}

                \rput(5.3,7.8){\tiny{$S_0u^2 - D$}}
                \rput(5.2,5.6){\tiny{$S_0 - D$}}
                \rput(5.3,3.4){\tiny{$S_0d^2 - D$}}
                
	\end{pspicture}
\end{center}
Celou problematiku však lze zjednodušit následující úvahou. Rozdělme cenu akcie na dvě složky - "nejistou" část a část tvořenou současnou hodnotou všech budoucích dividend vyplacených do splatnosti opce. Uvažujme pouze jeden ex-dividend den $\tau$, kde $k \delta t \le \tau \le (k+1) \delta t$. Platí
\begin{equation*}
S = S^*,~ i \delta t > \tau
\end{equation*}
\begin{equation*}
S = S^* + De^{-r(\tau -i\delta t)}, ~ i \delta t \le \tau
\end{equation*}
Rizikovou složku $S^*$ akcie je pak možné modelovat klasickým způsobem. V případě, že budeme uvažovat více ex-dividendových dní, lze výše uvedené rovnice snadno modifikovat. 

\section{Trinomický strom}

V případě binomického stromu vedou z každého uzlu dvě větve. Cena podkladového aktiva se tak může z výchozí úrovně $S$ zvýšit na $Su$ nebo snížit na $Sd$, kde $u>1$ a $0<d<1$.

Alternativu k binomickým stromům pak představují trinomické stromy. Z každého uzlu trinomického stromu vedou tři větve.
\begin{center}
	\begin{pspicture}(0,0)(7,4)
		\rput(3.5,0.5){Trinomický strom}
		\rput(3.5,0){změna ceny v časovém intervalu $\delta t$}

		\psline[arrows=->](1.5,2.5)(5.5,3.5)
                \psline[arrows=->](1.5,2.5)(5.5,2.5)
		\psline[arrows=->](1.5,2.5)(5.5,1.5)
		\rput(3.5,3.5){$p_u$}
                \rput(3.5,2.7){$p_m$}
		\rput(3.5,1.5){$p_d$}
		\rput(1,2.5){$S$}
		\rput(6,3.5){$Su$}
                \rput(6,2.5){$S$}
		\rput(6,1.5){$Sd$}
	\end{pspicture}
\end{center}
Parametry $p_u$, $p_m$ a $p_d$ představují pravděpodobnosti pro jednotlivé větve. Uvažujme akcii s nulovým dividendovým výnosem. Hodnoty těchto parametrů vypočtené tak, aby odpovídaly střední hodnotě a směrodatné odchylce ceny podkladového aktiva při zanedbání členů řádu $\sqrt{\delta t}$ a vyšší, jsou
\begin{equation*}
u=e^{\sigma \sqrt{3 \delta t}}
\end{equation*}
\begin{equation*}
d = \frac{1}{u}
\end{equation*}
\begin{equation*}
p_u = \sqrt{\frac{\delta t}{12 \sigma^2}}(r-\frac{\sigma^2}/{2}+\frac{1}{6}
\end{equation*}
\begin{equation*}
p_m = \frac{2}{3}
\end{equation*}
\begin{equation*}
p_d = -\sqrt{\frac{\delta t}{12 \sigma^2}}(r-\frac{\sigma^2}{2}+\frac{1}{6} 
\end{equation*}
V případě akcie s dividendovým výnosem $q$ stačí ve výše uvedených rovnicích nahradit $r$ výrazem $r-q$. Samotná konstrukce trinomického stromu je analogická konstrukci binomického stromu.

Lze dokázat, že oceňování opcí pomocí trinomického stromu je shodné s oceňováním rozdílovou metodou, které je popsána v kapitole 15.5.

\section{Implikovaný binomický strom}

Konstrukce klasického binomického stromu je založena na předpokladu, že volatilita podkladového aktiva je konstantní. V kapitole 12 jsme ukázali, že v praxi tomu tak není a že volatilita je funkcí ceny pokladového aktiva a zbytkové splatnosti opce. Tento funkční vztah je vyjádřen pomocí tzv. volatility surface. Binomický strom zkonstruovaný na základě volatility surface je nazýván implikovaným binomickým stromem. Tento strom opouští předpoklad konstatní volatility a modeluje správně cenu pokladového aktiva ve všech svých uzlech. Implikovaný binomický strom nám tedy umožňuje si vytvořit představu o skutečném pravděpodobnostním rozdělení ceny pokladového aktiva.
\begin{center}
	\begin{pspicture}(0,0)(11.5,5.0)
		\rput(3,0.0){\small{(a) Klasický binomický strom}}

		\psline[arrows=->](0.5,0.5)(5.5,0.5)
                \psline[arrows=->](0.5,0.5)(0.5,4.5)

                \rput(5.0,0.7){\tiny{čas}}
                \rput(1.2,4.3){\tiny{cena akcie}}

                \psline(0.50,2.50)(1.25,2.80)
                \psline(0.50,2.50)(1.25,2.20)
                
                \psline(1.25,2.80)(2.00,3.10)
                \psline(1.25,2.80)(2.00,2.50)
                \psline(1.25,2.20)(2.00,2.50)
                \psline(1.25,2.20)(2.00,1.90)

                \psline(2.00,3.10)(2.75,3.40)
                \psline(2.00,3.10)(2.75,2.80)
                \psline(2.00,2.50)(2.75,2.80)
                \psline(2.00,2.50)(2.75,2.20)
                \psline(2.00,1.90)(2.75,2.20)
                \psline(2.00,1.90)(2.75,1.60)

                \psline(2.75,3.40)(3.50,3.70)
                \psline(2.75,3.40)(3.50,3.10)
                \psline(2.75,2.80)(3.50,3.10)
                \psline(2.75,2.80)(3.50,2.50)
                \psline(2.75,2.20)(3.50,2.50)
                \psline(2.75,2.20)(3.50,1.90)
                \psline(2.75,1.60)(3.50,1.90)
                \psline(2.75,1.60)(3.50,1.30)

                \psline(3.50,3.70)(4.25,4.00)
                \psline(3.50,3.70)(4.25,3.40)
                \psline(3.50,3.10)(4.25,3.40)
                \psline(3.50,3.10)(4.25,2.80)
                \psline(3.50,2.50)(4.25,2.80)
                \psline(3.50,2.50)(4.25,2.20)
                \psline(3.50,1.90)(4.25,2.20)
                \psline(3.50,1.90)(4.25,1.60)
                \psline(3.50,1.30)(4.25,1.60)
                \psline(3.50,1.30)(4.25,1.00)

		\rput(8.5,0.0){\small{(b) Implikovaný binomický strom}}

		\psline[arrows=->](6.0,0.5)(11.0,0.5)
                \psline[arrows=->](6.0,0.5)(6.0,4.5)

                \rput(10.5,0.7){\tiny{čas}}
                \rput(6.7,4.3){\tiny{cena akcie}}

                \psline(6.00,2.50)(6.75,2.80)
                \psline(6.00,2.50)(6.75,2.20)
                
                \psline(6.75,2.80)(7.50,3.00)
                \psline(6.75,2.80)(7.50,2.50)
                \psline(6.75,2.20)(7.50,2.50)
                \psline(6.75,2.20)(7.50,1.85)

                \psline(7.50,3.00)(8.25,3.10)
                \psline(7.50,3.00)(8.25,2.70)
                \psline(7.50,2.50)(8.25,2.70)
                \psline(7.50,2.50)(8.25,2.15)
                \psline(7.50,1.85)(8.25,2.15)
                \psline(7.50,1.85)(8.25,1.45)

                \psline(8.25,3.10)(9.00,3.15)
                \psline(8.25,3.10)(9.00,2.95)
                \psline(8.25,2.70)(9.00,2.95)
                \psline(8.25,2.70)(9.00,2.50)
                \psline(8.25,2.15)(9.00,2.50)
                \psline(8.25,2.15)(9.00,1.85)
                \psline(8.25,1.45)(9.00,1.85)
                \psline(8.25,1.45)(9.00,1.05)

                \psline(9.00,3.15)(9.75,3.17)
                \psline(9.00,3.15)(9.75,3.00)
                \psline(9.00,2.95)(9.75,3.00)
                \psline(9.00,2.95)(9.75,2.65)
                \psline(9.00,2.50)(9.75,2.65)
                \psline(9.00,2.50)(9.75,2.15)
                \psline(9.00,1.85)(9.75,2.15)
                \psline(9.00,1.85)(9.75,1.40)
                \psline(9.00,1.05)(9.75,1.40)
                \psline(9.00,1.05)(9.75,0.60)
                
	\end{pspicture}
\end{center}

Uvažujme implikovaný binomický strom s konstatní délkou kroku $\delta t$ a předpokládejme, že jsme tento strom zkonstruovali do $n$-tého kroku. Situaci zachycuje následující obrázek.
\begin{center}
	\begin{pspicture}(0,0)(7.0,8.5)
          \rput(3.5,0.5){Výsek z implikovaného binomického stromu}
          \rput(3.5,0.0){(krok $n$ a $n+1$)}

          \rput(0.0,8.0){\small{uzel}}

          \psline[arrows=-*](2.0,2.5)(5.0,3.0)
          \psline[arrows=-*](2.0,2.5)(5.0,2.0)
          \psline[linestyle=dashed](2.0,2.5)(5.2,2.5)
          \psline[arrows=-*](2.0,3.5)(5.0,4.0)
          \psline[arrows=-*](2.0,3.5)(5.0,3.0)
          \psline[linestyle=dashed](2.0,3.5)(5.2,3.5)
          \rput(1.3,2.5){\tiny{$S_{n,0}$}}
          \rput(1.3,3.5){\tiny{$S_{n,1}$}}
          \rput(6.0,2.0){\tiny{$S_{n+1,0}$}}
          \rput(6.0,3.0){\tiny{$S_{n+1,1}$}}
          \rput(6.0,4.0){\tiny{$S_{n+1,2}$}}
          \rput(0.0,2.5){\tiny{$(n,0)$}}
          \rput(0.0,3.5){\tiny{$(n,1)$}}
          \rput(2.0,2.8){\tiny{$\pi_{n,0}$}}
          \rput(2.0,3.8){\tiny{$\pi_{n,1}$}}
          \rput(3.0,2.9){\tiny{$p_{n,0}$}}
          \rput(3.0,3.9){\tiny{$p_{n,1}$}}

          \psline[arrows=-*](2.0,5.0)(5.0,5.5)
          \psline[arrows=-*](2.0,5.0)(5.0,4.5)
          \psline[linestyle=dashed](2.0,5.0)(5.2,5.0)
          \rput(1.3,5.0){\tiny{$S_{n,i}$}}
          \rput(0.0,5.0){\tiny{$(n,i)$}}
          \rput(6.0,5.5){\tiny{$S_{n+1,i+1}$}}
          \rput(6.0,4.5){\tiny{$S_{n+1,i}$}}
          \rput(2.0,5.3){\tiny{$\pi_{n,i}$}}
          \rput(3.0,5.4){\tiny{$p_{n,i}$}}
          \rput(6.2,5.0){\tiny{realizační cena}}

          \psline[arrows=-*](2.0,6.5)(5.0,7.0)
          \psline[arrows=-*](2.0,6.5)(5.0,6.0)
          \psline[linestyle=dashed](2.0,6.5)(5.2,6.5)
          \psline[arrows=-*](2.0,7.5)(5.0,8.0)
          \psline[arrows=-*](2.0,7.5)(5.0,7.0)
          \psline[linestyle=dashed](2.0,7.5)(5.2,7.5)
          \rput(1.3,6.5){\tiny{$S_{n,n-1}$}}
          \rput(1.3,7.5){\tiny{$S_{n,n}$}}
          \rput(6.0,6.0){\tiny{$S_{n+1,n-1}$}}
          \rput(6.0,7.0){\tiny{$S_{n+1,n}$}}
          \rput(6.0,8.0){\tiny{$S_{n+1,n+1}$}}
          \rput(0.0,6.5){\tiny{$(n,n-1)$}}
          \rput(0.0,7.5){\tiny{$(n,n)$}}
          \rput(2.0,6.8){\tiny{$\pi_{n,n-1}$}}
          \rput(2.0,7.8){\tiny{$\pi_{n,n}$}}
          \rput(3.5,7.0){\tiny{$p_{n,n-1}$}}
          \rput(3.0,7.9){\tiny{$p_{n,n}$}}

          \psline[linestyle=dotted](2.0,1.7)(2.0,8.2)
          \psline[linestyle=dotted](5.0,1.7)(5.0,8.2)
          \psline[arrows=<->](2.0,1.8)(5.0,1.8)
          \rput(3.5,1.6){\small{$\delta t$}}
          \rput(2.0,1.4){\small{$n$}}
          \rput(5.0,1.4){\small{$n+1$}}

        \end{pspicture}
\end{center}

Definujme $r$ jako bezrizikovou úrokovou míru mezi kroky $n$ a $n+1$ a $F_{n,i}$ jako forwardovou cenu podkladového aktiva.
\begin{equation}
F_{n,i} = S_{n,i}e^{r \delta t}
\end{equation}
Dále definujme $p_{n,i}$ jako pravděpodobnost přesunu z uzlu $(n,i)$ do uzlu $(n+1, i + 1)$ a $\pi_{n,i}$ jako pravděpodobnost přesunu z uzlu $(0,0)$ do uzlu $(n,i)$ diskontovanou bezrizikovou sazbou přes časové období $0$ až $n \delta t$. V případě klasického binomického stromu, který je kromě konstatní volatility charakteristický také konstantní pravděpodobností $p_{n,i}=p$, a za předpokladu konstantní bezrizikové sazby $r$, tedy platí $\pi_{n,i}=e^{-rn \delta t}\binom{n}{n-i}p^n(1-p)^{n-i}$.

\subsection{Konstrukce implikovaného binomického stromu}

Z výše uvedeného obrázku vyplývá, že pro přesun z $n$-tého do $n+1$-ního kroku je třeba vypočítat $2n+1$ parametrů. Konkrétně se jedná o
\begin{itemize}
\item $n+1$ cen pokladového aktiva $S_{n+1,i}$
\item $n$ pravděpodobností $p_{n,i}$
\end{itemize}

\subsubsection{Hodnota opce dle klasického binomického stromu}

Volatility surface definuje volatilitu podkladového aktiva jako funkci realizační ceny a doby do splatnosti opce. S využitím takto stanovené volatility jsme schopni zkonstruovat klasický binomický strom a ten použít pro ocenění evropské kupní popř. prodejní opce.

Nechť $c(K, t_{n+1})$ resp. $p(K, t_{n+1})$ jsou hodnota prodejní resp. kupní opce s realizační hodnotou $K$ a splatností v čase $t_{n+1}$. Hodnota těchto opcí vypočtená dle klasického binomického stromu je
\begin{equation}
c(K, t_{n+1}) = \Pi_{i=1}^{n+1}e^{-r_i \delta t} \sum_{i = 0}^{n+1} \max(S_{0,0}u^id^{(n+1)-i}-K,0)\binom{n+1}{i}p^i(1-p)^{(n+1)-i}
\end{equation}
\begin{equation}
p(K, t_{n+1}) = \Pi_{i=1}^{n+1}e^{-r_i \delta t} \sum_{i = 0}^{n+1}\max(K-S_{0,0}u^id^{(n+1)-i},0)\binom{n+1}{i}p^i(1-p)^{(n+1)-i}
\end{equation}
Parametr $r_i$ představuje bezrizikovou úrokovou míru platnou mezi kroky $i-1$ a $i$ a parametry $p$, $u$ a $d$ jsou dány rovnicemi (15.2), (15.3) a (15.4). Volatilita $\sigma$, která figuruje v rovnicích (15.3) a (15.4), je odvozena z volatility surface pro realizační cenu $K$ a splatnost $t_{n+1}$.

\subsubsection{Hodnota opce dle implikovaného binomického stromu}

Hodnotu evropské kupní popř. prodejní opce lze také vypočíst pomocí implikovaného binomického stromu. Hodnota opce vypočtená podle klasického a podle implikovaného binomického stromu musí být shodná. Zásadní rozdíl mezi oběma binomickými stromy je ten, že klasický binomický strom je zkonstruován na základě volatility odvozené z volatility surface, která je pro celý strom konstantní. Tento strom tak lze použít pouze pro ocenění jedné konkrétní opce, pro kterou byla tato volatilita odvozena. Naproti tomu s pomocí implikovaného binomického stromu je možné ocenit libovolnou opci, protože konstrukce tohoto stromu zohledňuje rozdílnou volatilitu pro rozdílnou realizační cenu a zbytkovou splatnost. Rovnice pro výpočet ceny evropské kupní resp. prodejní podle implikovaného binomického stromu jsou
\begin{equation*}
c(K, t_{n+1}) = e^{-r \delta t} \sum_{j=0}^{n+1}\bigg(\pi_{n,j}p_{n,j}+\pi_{n, j + 1}(1-p_{n,j+1})\bigg)\max(S_{n+1,j+1}-K,0)
\end{equation*} 
\begin{equation*}
p(K, t_{n+1}) = e^{-r \delta t} \sum_{j=0}^{n+1}\bigg(\pi_{n,j}p_{n,j}+\pi_{n, j + 1}(1-p_{n,j+1})\bigg)\max(K-S_{n+1,j+1},0)
\end{equation*}
Je-li realizační cena $K$ rovna $S_{n,i}$, mohou být výše uvedené rovnice po oddělení prvního in-the-money uzlu\footnote{Tímto uzlem je uzel $(n,i)$. Hodnota kupní opce v kroku $n+1$ je pro všechny nižší uzly rovna nule. Cena podkladového aktiva v těchto uzlech je totiž menší než realizační cena $S_{n,i}$. Hodnota prodejní opce se tedy skládá z horní větve vedoucí z uzlu $(n,i)$ a z uzlů nacházejících se nad touto větví. V případě prodejní opce je tomu přesně naopak.} vyjádřeny ve tvaru
\begin{equation*}
c(K, t_{n+1}) = e^{-r \delta t} \bigg( \pi_{n,i}p_{n,i}(S_{n+1,i+1}-S_{n,i}) + \sum_{j=i+1}^{n+1}\pi_{n,j}[(p_{n,j}S_{n+1,j+1}+ (1-p_{n,j})S_{n+1,j})-S_{n,i}] \bigg)
\end{equation*}
\begin{equation*}
p(K, t_{n+1}) = e^{-r \delta t} \bigg( \pi_{n,i}(1-p_{n,i})(S_{n,i}-S_{n+1,i}) + \sum_{j=0}^{i-1}\pi_{n,j}[S_{n,i}-(p_{n,j}S_{n+1,j+1}+ (1-p_{n,j})S_{n+1,j})] \bigg)
\end{equation*}
Vzhledem k tomu, že forwardovou cenu $F_{n,i}$ je možné kromě (15.5) vypočíst také ze vztahu
\begin{equation}
F_{n,i} = p_{n,i}S_{n+1,i+1} + (1-p_{n,i})S_{n+1,i}
\end{equation}
lze tyto rovnice dále upravit do tvaru
\begin{equation}
c(K, t_{n+1}) = e^{-r \delta t} \bigg( \pi_{n,i}p_{n,i}(S_{n+1,i+1}-S_{n,i}) + \sum_{j=i+1}^{n+1}\pi_{n,j}(F_{n,j}-S_{n,i}) \bigg)
\end{equation}
\begin{equation}
p(K, t_{n+1}) = e^{-r \delta t} \bigg( \pi_{n,i}(1-p_{n,i})(S_{n,i}-S_{n+1,i}) + \sum_{j=0}^{i-1}\pi_{n,j}(S_{n,i}-F_{n,j}) \bigg)
\end{equation}

\subsubsection{Výpočet parametrů $S_{n+1,j}$ a $p_{n,j}$}

Protože hodnota kupní resp. prodejní opce je dána rovnicí (15.6) resp. (15.7) a forwardová cena $F_{n,j}$ rovnicí (15.5), obsahují (15.9) a (15.10) následující neznámé
\begin{itemize}
\item $n+1$ cen pokladového aktiva $S_{n+1,j}$
\item $n$ pravděpodobností $p_{n,j}$ přechodu z uzlu $(n,j)$ do uzlu $(n+1,j+1)$
\end{itemize}
Celkem je tedy třeba vypočíst $2n+1$ neznámých. Pro výpočet těchto neznámých máme k dispozici
\begin{itemize}
\item $n$ rovnic forwardových cen (15.8)
\item $n$ rovnic cen kupních resp. prodejních opcí daných rovnicemi (15.9) resp. (15.10)\footnote{Nejedná o $2n$ rovnic, jak by se mohlo na první pohled zdát - tj. $n$ rovnic pro cenu kupní a $n$ rovnic pro cenu prodejní opce. Protože ceny kupní a prodejní opce jsou ``propojeny'' skrze put-call paritu, máme k dispozici pouze $n$ nezávislých rovnic.}
\end{itemize}
Jako $(2n+1)$-ní podmínku, která je nezbytná pro výpočet výše uvedených neznámých, zvolme ``vycentrování'' implikovaného binomického stromu na klasický binomický strom. Pro lichá $n$, kdy je počet uzlů v $(n+1)$-ním kroku taktéž lichý, platí
\begin{equation}
S_{n+1,(n+1)/2} = S_{0,0}
\end{equation}
Naproti tomu pro sudá $n$ má tato podmínka podobu rovnice
\begin{equation}
S_{n+1,n/2} = \frac{S_{0,0}^2}{S_{n+1, n/2 + 1}}
\end{equation}
Z rovnice (15.9) lze s využitím substituce za $p_{n,i}$ dle (15.8) vyjádřit cenu podkladového aktiva $S_{n+1,i+1}$ jako
\begin{equation}
S_{n+1,i+1}=\frac{S_{n+1,i}[e^{r\delta t}c(S_{n,i}, t_{n+1})-\Psi_c]-\pi_{n,i}S_{n,i}(F_{n,i} - S_{n+1,i})}{[e^{r \delta t}c(S_{n+1,i},t_{n+1})-\Psi_c]-\pi_{n,i}(F_{n,i}-S_{n+1,i})}
\end{equation}
kde
\begin{equation*}
\Psi_c = \sum_{j = i + 1}^{n+1}\pi_{n,i}(F_{n,j}-S_{n,i})
\end{equation*}
Podobně lze z rovnice (15.10) vyjádřit cenu podkladového aktiva $S_{n+1,i}$ jako
\begin{equation}
S_{n+1,i} = \frac{S_{n+1,i+1}[e^{r \delta t}p(S_{n,i},t_{n+1})-\Psi_p]+\pi_{n,i}S_{n,i}(F_{n,i}-S_{n+1,i+1})}{[e^{r \delta t}p(S_{n,i},t_{n+1})-\Psi_p]+\pi_{n,i}(F_{n,i}-S_{n+1,i+1})}
\end{equation}
kde
\begin{equation*}
\Psi_p = \sum_{j = 0}^{i-1}\pi_{n,i}(S_{n,i}-F_{n,j})
\end{equation*}


Ceny podkladového aktiva pro $(n+1)$-ní krok implikovaného binomického stromu začínáme počítat od uzlu $(n+1, i + 1)$, kde $i=n/2$ resp. $i = (n+1)/2$ pro sudá resp. lichá $n$. Ceny pokladového aktiva tak počítáme od ``centrálního'' uzlu $(n+1)$-ního kroku binomického stromu.

Nejdříve uvažujme situaci, kdy je $n$ liché. Počet uzlů v $(n+1)$-ním kroce je taktéž lichý. Proto je hodnota podkladového aktiva $S_{n+1,i}$ v centrálním uzlu s ohledem na (15.11) rovna výchozí ceně $S_{0,0}$. Cenu $S_{n+1,i+1}$ lze vypočíst dle rovnice (15.13). Následně je možné získat pravděpodobnost $p_{n,i}$ z rovnice (15.8). Tímto způsobem postupně odvodíme ceny a pravděpodobnosti až do nejvyššího uzlu $(n+1, n+1)$. Cenu podkladového aktiva pro uzly $(n+1,j)$, které se nachází pod centrálním uzlem $(n+1,i)$, lze dopočítat dle rovnice (15.14).

V případě, že je $n$ sudé, je počet uzlů v $(n+1)$-ním kroce také sudý. ``Centrální'' uzel proto neexistuje. Pro výpočet ceny $S_{n+1,i+1}$ je tedy třeba modifikovat rovnici (15.13) s využitím vztahu (15.12). Cena podkladového aktiva $S_{n+1,i+1}$ je tak dána rovnicí
\begin{equation*}
S_{n+1,i+1}=\frac{S_{0,0}[e^{r \delta t}c(S_{0,0},t_{n+1})+ \pi_{n,i}S_{0,0}-\Psi_c]}{\pi_{n,i}F_{n,i}-e^{r \delta t}c(S_{0,0},t_{n+1})+\Psi_c}
\end{equation*}
Cena $S_{n+1,i}$ je následně dopočtena podle (15.12). Ceny podkladového aktiva pro uzly nad uzlem $(n+1,i+1)$ je možné dopočítat dle (15.13). Ceny podkladového aktiva pro uzly pod uzlem $(n+1,i)$ jsou pak dány rovnicí (15.14). Pro výpočet pravděpodobností $p_{n,j}$ lze opět využít vztahu (15.8).

Další podmínkou, která musí být při konstrukci binomického stromu dodržena, je
\begin{equation}
F_{i, j} < S_{i+1, j+1} < F_{i, j + 1}
\end{equation}
Tato podmínka zaručuje, že v rámci implikovaného binomického stromu neexistuje možnost arbitráže. Jestliže modelová ceny pokladového aktiva $S_{i+1, j+1}$ tuto podmínku nesplňuje, definujeme tuto cenu jako
\begin{equation*}
S_{i+1, j + 1} = S_{i+1,j} \cdot \frac{S_{i, j-1}}{S_{i, j}}
\end{equation*}

\subsection{Modelový příklad}

Uvažujme akcii s nulovým dividendovým výnosem, jejíž spotová cena je 100 USD. Předpokládejme, že roční implikovaná volatilita at-the-money evropské opce je 10\% a že se zvyšuje lineárně o 0.05 procentního bodu za každý jeden dolar poklesu ceny podkladového aktiva\footnote{Tímto způsobem jsme právě definovali volatility surface.}. Dále předpokládejme, že bezriziková úroková sazba je konstatní pro všechny splatnosti a rovna 3\% v ročním vyjádření.

Následující grafy zobrazují implikovaný binomický strom včetně klíčových hodnot nezbytných pro jeho konstrukci.

\begin{center}
	\begin{pspicture}(0,0)(12.5,6.0)
		\rput(6.25,0.0){\small{Implikovaný binomický strom}}

                \psline(0.50,3.00)(2.50,3.50)
                \psline(0.50,3.00)(2.50,2.50)
                
                \psline(2.50,3.50)(5.00,4.00)
                \psline(2.50,3.50)(5.00,3.00)
                \psline(2.50,2.50)(5.00,3.00)
                \psline(2.50,2.50)(5.00,2.00)

                \psline(5.00,4.00)(7.50,4.50)
                \psline(5.00,4.00)(7.50,3.50)
                \psline(5.00,3.00)(7.50,3.50)
                \psline(5.00,3.00)(7.50,2.50)
                \psline(5.00,2.00)(7.50,2.50)
                \psline(5.00,2.00)(7.50,1.50)

                \psline(7.50,4.50)(10.00,5.00)
                \psline(7.50,4.50)(10.00,4.00)
                \psline(7.50,3.50)(10.00,4.00)
                \psline(7.50,3.50)(10.00,3.00)
                \psline(7.50,2.50)(10.00,3.00)
                \psline(7.50,2.50)(10.00,2.00)
                \psline(7.50,1.50)(10.00,2.00)
                \psline(7.50,1.50)(10.00,1.00)

                \psline(10.00,5.00)(12.00,5.50)
                \psline(10.00,5.00)(12.00,4.50)
                \psline(10.00,4.00)(12.00,4.50)
                \psline(10.00,4.00)(12.00,3.50)
                \psline(10.00,3.00)(12.00,3.50)
                \psline(10.00,3.00)(12.00,2.50)
                \psline(10.00,2.00)(12.00,2.50)
                \psline(10.00,2.00)(12.00,1.50)
                \psline(10.00,1.00)(12.00,1.50)
                \psline(10.00,1.00)(12.00,0.50)
                
                \rput(0.50,2.80){\tiny{[0,0]}}
                \rput(2.50,3.30){\tiny{[1,1]}}
                \rput(2.50,2.30){\tiny{[1,0]}}
                \rput(5.00,3.80){\tiny{[2,2]}}
                \rput(5.00,2.80){\tiny{[2,1]}}
                \rput(5.00,1.80){\tiny{[2,0]}}
                \rput(7.50,4.30){\tiny{[3,3]}}
                \rput(7.50,3.30){\tiny{[3,2]}}
                \rput(7.50,2.30){\tiny{[3,1]}}
                \rput(7.50,1.30){\tiny{[3,0]}}
                \rput(10.00,4.80){\tiny{[4,4]}}
                \rput(10.00,3.80){\tiny{[4,3]}}
                \rput(10.00,2.80){\tiny{[4,2]}}
                \rput(10.00,1.80){\tiny{[4,1]}}
                \rput(10.00,0.80){\tiny{[4,0]}}
                \rput(12.00,5.30){\tiny{[5,5]}}
                \rput(12.00,4.30){\tiny{[5,4]}}
                \rput(12.00,3.30){\tiny{[5,3]}}
                \rput(12.00,2.30){\tiny{[5,2]}}
                \rput(12.00,1.30){\tiny{[5,1]}}
                \rput(12.00,0.30){\tiny{[5,0]}}

                \rput(0.50,3.20){\tiny{100.00}}
                \rput(2.50,3.80){\tiny{110.52}}
                \rput(2.50,2.80){\tiny{90.48}}
                \rput(5.00,4.20){\tiny{120.30}}
                \rput(5.00,3.20){\tiny{100.00}}
                \rput(5.00,2.20){\tiny{79.31}}
                \rput(7.50,4.80){\tiny{130.07}}
                \rput(7.50,3.80){\tiny{110.57}}
                \rput(7.50,2.80){\tiny{90.44}}
                \rput(7.50,1.80){\tiny{71.32}}
                \rput(10.00,5.20){\tiny{139.16}}
                \rput(10.00,4.20){\tiny{120.36}}
                \rput(10.00,3.20){\tiny{100.00}}
                \rput(10.00,2.20){\tiny{79.29}}
                \rput(10.00,1.20){\tiny{58.86}}
                \rput(12.00,5.80){\tiny{147.73}}
                \rput(12.00,4.80){\tiny{130.10}}
                \rput(12.00,3.80){\tiny{110.63}}
                \rput(12.00,2.80){\tiny{90.39}}
                \rput(12.00,1.80){\tiny{71.19}}
                \rput(12.00,0.80){\tiny{54.34}}

	\end{pspicture}
\end{center}
\begin{center}
	\begin{pspicture}(0,0)(10.5,5.5)
		\rput(5.25,0.0){\small{Pravděpodobnost $p_{i,j}$ přechodu na vyšší uzel}}

                \psline(0.50,3.00)(2.50,3.50)
                \psline(0.50,3.00)(2.50,2.50)
                
                \psline(2.50,3.50)(5.00,4.00)
                \psline(2.50,3.50)(5.00,3.00)
                \psline(2.50,2.50)(5.00,3.00)
                \psline(2.50,2.50)(5.00,2.00)

                \psline(5.00,4.00)(7.50,4.50)
                \psline(5.00,4.00)(7.50,3.50)
                \psline(5.00,3.00)(7.50,3.50)
                \psline(5.00,3.00)(7.50,2.50)
                \psline(5.00,2.00)(7.50,2.50)
                \psline(5.00,2.00)(7.50,1.50)

                \psline(7.50,4.50)(10.00,5.00)
                \psline(7.50,4.50)(10.00,4.00)
                \psline(7.50,3.50)(10.00,4.00)
                \psline(7.50,3.50)(10.00,3.00)
                \psline(7.50,2.50)(10.00,3.00)
                \psline(7.50,2.50)(10.00,2.00)
                \psline(7.50,1.50)(10.00,2.00)
                \psline(7.50,1.50)(10.00,1.00)
                
                \rput(0.50,2.80){\tiny{[0,0]}}
                \rput(2.50,3.30){\tiny{[1,1]}}
                \rput(2.50,2.30){\tiny{[1,0]}}
                \rput(5.00,3.80){\tiny{[2,2]}}
                \rput(5.00,2.80){\tiny{[2,1]}}
                \rput(5.00,1.80){\tiny{[2,0]}}
                \rput(7.50,4.30){\tiny{[3,3]}}
                \rput(7.50,3.30){\tiny{[3,2]}}
                \rput(7.50,2.30){\tiny{[3,1]}}
                \rput(7.50,1.30){\tiny{[3,0]}}
                \rput(10.00,4.80){\tiny{[4,4]}}
                \rput(10.00,3.80){\tiny{[4,3]}}
                \rput(10.00,2.80){\tiny{[4,2]}}
                \rput(10.00,1.80){\tiny{[4,1]}}
                \rput(10.00,0.80){\tiny{[4,0]}}

                \rput(0.50,3.20){\tiny{0.625}}
                \rput(2.50,3.80){\tiny{0.682}}
                \rput(2.50,2.80){\tiny{0.671}}
                \rput(5.00,4.20){\tiny{0.684}}
                \rput(5.00,3.20){\tiny{0.623}}
                \rput(5.00,2.20){\tiny{0.542}}
                \rput(7.50,4.80){\tiny{0.724}}
                \rput(7.50,3.80){\tiny{0.682}}
                \rput(7.50,2.80){\tiny{0.669}}
                \rput(7.50,1.80){\tiny{0.715}}
                \rput(10.00,5.20){\tiny{0.751}}
                \rput(10.00,4.20){\tiny{0.685}}
                \rput(10.00,3.20){\tiny{0.623}}
                \rput(10.00,2.20){\tiny{0.546}}
                \rput(10.00,1.20){\tiny{0.373}}

	\end{pspicture}
\end{center}
\begin{center}
	\begin{pspicture}(0,0)(12.5,6.0)
		\rput(6.25,0.0){\small{Diskontovaná pravděpodobnost $\pi_{i,j}$}}

                \psline(0.50,3.00)(2.50,3.50)
                \psline(0.50,3.00)(2.50,2.50)
                
                \psline(2.50,3.50)(5.00,4.00)
                \psline(2.50,3.50)(5.00,3.00)
                \psline(2.50,2.50)(5.00,3.00)
                \psline(2.50,2.50)(5.00,2.00)

                \psline(5.00,4.00)(7.50,4.50)
                \psline(5.00,4.00)(7.50,3.50)
                \psline(5.00,3.00)(7.50,3.50)
                \psline(5.00,3.00)(7.50,2.50)
                \psline(5.00,2.00)(7.50,2.50)
                \psline(5.00,2.00)(7.50,1.50)

                \psline(7.50,4.50)(10.00,5.00)
                \psline(7.50,4.50)(10.00,4.00)
                \psline(7.50,3.50)(10.00,4.00)
                \psline(7.50,3.50)(10.00,3.00)
                \psline(7.50,2.50)(10.00,3.00)
                \psline(7.50,2.50)(10.00,2.00)
                \psline(7.50,1.50)(10.00,2.00)
                \psline(7.50,1.50)(10.00,1.00)

                \psline(10.00,5.00)(12.00,5.50)
                \psline(10.00,5.00)(12.00,4.50)
                \psline(10.00,4.00)(12.00,4.50)
                \psline(10.00,4.00)(12.00,3.50)
                \psline(10.00,3.00)(12.00,3.50)
                \psline(10.00,3.00)(12.00,2.50)
                \psline(10.00,2.00)(12.00,2.50)
                \psline(10.00,2.00)(12.00,1.50)
                \psline(10.00,1.00)(12.00,1.50)
                \psline(10.00,1.00)(12.00,0.50)

                \rput(0.50,2.80){\tiny{[0,0]}}
                \rput(2.50,3.30){\tiny{[1,1]}}
                \rput(2.50,2.30){\tiny{[1,0]}}
                \rput(5.00,3.80){\tiny{[2,2]}}
                \rput(5.00,2.80){\tiny{[2,1]}}
                \rput(5.00,1.80){\tiny{[2,0]}}
                \rput(7.50,4.30){\tiny{[3,3]}}
                \rput(7.50,3.30){\tiny{[3,2]}}
                \rput(7.50,2.30){\tiny{[3,1]}}
                \rput(7.50,1.30){\tiny{[3,0]}}
                \rput(10.00,4.80){\tiny{[4,4]}}
                \rput(10.00,3.80){\tiny{[4,3]}}
                \rput(10.00,2.80){\tiny{[4,2]}}
                \rput(10.00,1.80){\tiny{[4,1]}}
                \rput(10.00,0.80){\tiny{[4,0]}}
                \rput(12.00,5.30){\tiny{[5,5]}}
                \rput(12.00,4.30){\tiny{[5,4]}}
                \rput(12.00,3.30){\tiny{[5,3]}}
                \rput(12.00,2.30){\tiny{[5,2]}}
                \rput(12.00,1.30){\tiny{[5,1]}}
                \rput(12.00,0.30){\tiny{[5,0]}}

                \rput(0.50,3.20){\tiny{1.000}}
                \rput(2.50,3.80){\tiny{0.607}}
                \rput(2.50,2.80){\tiny{0.364}}
                \rput(5.00,4.20){\tiny{0.401}}
                \rput(5.00,3.20){\tiny{0.425}}
                \rput(5.00,2.20){\tiny{0.116}}
                \rput(7.50,4.80){\tiny{0.266}}
                \rput(7.50,3.80){\tiny{0.380}}
                \rput(7.50,2.80){\tiny{0.216}}
                \rput(7.50,1.80){\tiny{0.052}}
                \rput(10.00,5.20){\tiny{0.187}}
                \rput(10.00,4.20){\tiny{0.324}}
                \rput(10.00,3.20){\tiny{0.258}}
                \rput(10.00,2.20){\tiny{0.105}}
                \rput(10.00,1.20){\tiny{0.014}}
                \rput(12.00,5.80){\tiny{0.136}}
                \rput(12.00,4.80){\tiny{0.260}}
                \rput(12.00,3.80){\tiny{0.255}}
                \rput(12.00,2.80){\tiny{0.150}}
                \rput(12.00,1.80){\tiny{0.052}}
                \rput(12.00,0.80){\tiny{0.009}}

	\end{pspicture}
\end{center}

Ilustrujme si konstrukci implikovaného binomického stromu na výpočtu cen podkladového aktiva pro uzly v posledním pátém kroce.

\subsubsection{Centrální uzly}

Vzhledem k tomu, že pátý krok má sudý počet uzlů, neexistuje jeden centrální uzel. Konstrukci implikovaného binomického stromu tak zahájíme výpočtem cen podkladového aktiva pro uzly $(5,3)$ a $(5,2)$.\\

Cenu podkladového aktiva $S_{5,3}$ vypočteme jako
\begin{equation*}
S_{5,3} = \frac{S_{0,0}[e^{r \delta t}c(S_{0,0},5)+ \pi_{4,2}S_{0,0}-\Psi_c]}{\pi_{4,2}F_{4,2}-e^{r \delta t}c(S_{0,0},5)+\Psi_c}
\end{equation*}
Připomeňme, že hodnota evropské kupní opce $c(S_{0,0},5)$ vypočtená na základě klasického binomického stromu je
\begin{equation*}
c(S_{0,0}, 5) = e^{-5r} \sum_{i = 0}^5 \max(S_{0,0}u^id^{5-i}-S_{0,0},0)\binom{5}{i}p^i(1-p)^{5-i}
\end{equation*}
Protože se realizační cena opce rovná spotové ceně podkladového aktiva, je implikovaná volatilita rovna 10.00\%. Hodnoty parametrů $u$, $d$ a $p$ jsou dány rovnicemi (15.2), (15.3) a (15.4). Hodnota uvažované opce je tedy rovna 17.07 USD. Další parametr, který je nutné vypočíst, je $\Psi_c$. Ten je pro $S_{5,3}$ definován jako
\begin{equation*}
\Psi_c = \pi_{4,3}(F_{4,3}-S_{4,2}) + \pi_{4,4}(F_{4,4}-S_{4,2})
\end{equation*}
a je tedy roven 15.87. Dosazením do rovnice získáme cenu podkladového aktiva $S_{5,3}$.
\begin{equation*}
S_{5,3} = \frac{100[1.03 \cdot 17.07 + 0.258 \cdot 100 - 15.87]}{0.258 \cdot (100 \cdot 1.03) - 1.03 \cdot 17.07 + 15.87} = 110.66
\end{equation*}
Cena podkladového aktiva v uzlu $(5,3)$ je tedy 110.66 USD. Rozdíl oproti ceně 110.63 USD delarované v grafu implikovaného binomického stromu je způsoben zaokrouhlováním během výpočtu.\\

Cena podkladového aktiva v druhém centrálním uzlu $(5,2)$ je dána rovnicí (15.12) a je tedy rovna 90.37 USD.\\

Posledním krokem při výpočtu centrálních uzlů je kontrola podmínky neexistence arbitráže. Tato podmínka zformulovaná do nerovnosti (15.15) je splněna pro oba uvažované uzly.

\subsubsection{Horní větve implikovaného binomického stromu}

Nyní přistoupíme k výpočtu cen podkladového aktiva pro uzly nad uzlem $(5,3)$. Cena pokladového aktiva pro tyto uzly se vypočte dle rovnice (15.13).\\ 

Nejprve vypočteme cenu podkladového aktiva pro uzel $(5,4)$. Hodnota opce $c(120.36,5)$ při volatilitě 8.98\% je rovna 6.27 USD. Dalším parametrem nezbytným pro výpočet ceny $S_{5,4}$ je $\Psi_c$. Hodnota tohoto parametru je 4.30. Cena podkladového aktiva v uzlu $(5,4)$ je tedy dána rovnicí
\begin{equation*}
S_{5,4} = \frac{110.66(1.03 \cdot 6.27 - 4.30) - 0.324 \cdot 120.35(123.97 - 110.66)}{1.03 \cdot 6.27 - 4.30 - 0.324 \cdot(123.97 - 110.66)} = 130.06
\end{equation*}
Nakonec je třeba zkontrolovat podmínku neexistence arbitráže definovou nerovnosti (15.15). Tato nerovnost je pro vypočtené $S_{5,3}$ splněna.\\

Pro uzel $(5,5)$ je hodnota opce $c(139.16,5)$ při volatilitě 8.04\% rovna 1.17 USD. Vzhledem k tomu, že se jedná o ``nejvyšší'' uzel, nabývá sumační parametr $\Psi_c$ hodnotu nula. Konkrétní podoba rovnice pro výpočet ceny podkladového aktiva je tedy
 \begin{equation*}
S_{5,5} = \frac{130.06(1.03 \cdot 1.17 - 0) - 0.187 \cdot 139.16(143.33 - 130.06)}{1.03 \cdot 1.17 - 0 - 0.187 \cdot(143.33 - 130.06)} = 147.75
\end{equation*}
Nyní je třeba zkontrolovat splnění podmínky (15.15). Vzhledem k tomu, že $F_{4,5}$ není definováno, je tato podmínka modifikována do podoby
\begin{equation*}
F_{4,4} < S_{5,5}
\end{equation*}
Protože $F_{4,4}$ je rovno 143.33 USD, je tato podmínka splněna.

\subsubsection{Dolní větve implikovaného binomického stromu}

Zkonstruovali jsme implikovaný binomický strom v uzlech $(5,2)$ až $(5.5)$. Zbývá tedy vypočíst ceny podkladového aktiva pod uzlem $(5,2)$. Ceny podkladového aktiva pro tyto uzly jsou dány rovnicí (15.14).\\

Konstrukce dolních větví implikovaného binomického stromu začíná uzlem $(5,1)$. Nejprve je třeba vypočíst hodnotu opce $p(79.29,5)$ a sumačního parametru  . Hodnota opce $p(79.29,5)$ pro implikovanou volatilitu 11.04\% je rovna 0.64 USD; hodnota parametru $\Psi_p$ je pak rovna 0.267. Rovnice pro výpočet ceny podkladového aktiva v uzlu $(5,1)$ má tedy tvar
\begin{equation*}
S_{5,1} = \frac{90.37 \cdot(1.03 \cdot 0.64 - 0.267) + 0.105 \cdot 79.29 \cdot(81.67 - 90.37)}{1.03 \cdot 0.64 - 0.267 + 0.105 \cdot(81.67 - 90.37)} = 70.95
\end{equation*}
Před přesunem na vyšší větev binomického stromu je třeba zkontrolovat splnění podmínky (15.15). Vzhledem k tomu, že $F_{4,0} = 60.63$ a $F_{4,1}=81.67$ je tato podmínka splněna.\\

V případě uzlu $(5,0)$ je hodnota opce $p(58.86,5,5)$ při volatilitě 12.06\% rovna 0.04 USD. Protože se jedná o ``nejnižší'' uzel, je hodnota sumačního parametru $\Psi_c$ rovna nule. Cena podkladového aktiva je tak rovna
\begin{equation*}
S_{5,0} = \frac{70.95 \cdot (1.03 \cdot 0.04 - 0) + 0.014 \cdot 58.86 \cdot (60.63 - 70.95)}{1.03 \cdot 0.04 - 0 + 0.014 \cdot (60.63 - 70.95)} = 54.04
\end{equation*}
Při kontrole splnění podmínky neexistence arbitráže nastává analogická situace jako v případě uzlu $(5,5)$. Podmínka má proto podobu
\begin{equation*}
S_{5,0} < F_{4,0}
\end{equation*}
Vzhledem k tomu, že je $F_{4,0}$ rovno 60.63 USD, je tato podmínka splněna.

\subsubsection{Výpočet parametrů $p_{i,j}$ a $\pi_{i,j}$}

Jestliže bychom chtěli konstruovat implikovaný binomický strom pro šestý krok, je zapotřebí vypočíst pravděpodobnosti $p_{4,j}$ a $\pi_{5,j}$, které jsou definovány rovnicemi 
\begin{equation*}
p_{4,j} = \frac{F_{4,j} - S_{5,j}}{S_{5,j+1} - S_{5,j}}
\end{equation*}
resp.
\begin{equation*}
\pi_{5,j} = e^{-r \delta t}(p_{4,j}\pi_{4,j} + (1-p_{4,j+1})\pi_{4,j+1})
\end{equation*}
Další postup by pak byl analogický výše uvedenému postupu.

\section{Simulace Monte Carlo}

Postup při použití metody Monte Carlo je následující:
\begin{itemize}
\item vygenerovat náhodný vývoj ceny akcie pro zvolené časové období
\item vypočítat výnos z opce
\item opakovat kroky (1) a (2) za účelem vygenerování potřebného vzorku
\item zprůměrovat výnosy z opce s cílem určit očekávaný opční výnos
\item diskontovat očekávaný opční výnos bezrizikovou úrokovou sazbou
\end{itemize}

Jestliže rozdělíme životnost opce do $N$ časových intervalů délky $\delta t$, lze změnu ceny akcie mezi těmito intervaly definovat jako
\begin{equation}
S(t+\delta t)-S(t)= \hat{\mu} S(t)\delta t dt + \sigma S(t) \epsilon \sqrt{\delta t} 
\end{equation}
V praxi je však mnohem přesnější simulovat $\ln S$. Zatímco (15.16) platí pouze pro $\delta t$ blízká nule, (15.17) platí pro všechna $\delta t$.
\begin{equation}
\ln S(t+\delta t)- \ln S(t)= \big(\hat{\mu} - \frac{\sigma^2}{2} \big) \delta t + \sigma \epsilon \sqrt{\delta t}
\end{equation}

Nevýhodou metody Monte Carlo jsou zvýšené požadavky na výpočetní výkon. Naopak výhodou je relativní jednoduchost ocenění i poměrně komplikovaných struktur, které by bylo velice těžké uchopit analyticky.

Alternativu k metodě Monte Carlo představuje binomický strom. Výhodou binomického stromu je, že bere v potaz všechny možné ceny, které s ohledem na parametry modelu přicházejí v úvahu. Binomický strom je také v porovnání s metodou Monte Carlo zpravidla méně náročný na výpočetní výkon.

\subsection{Modelování korelovaných náhodných veličin}

Uvažujme situaci, kdy výplata z finačního derivátu závisí na $n$ proměnných $\theta_i$ ($1 \le i \le n$). Definujme $s_i$ jako volatilitu $\theta_i$, $\hat{m_i}$ jako očekávanou míru růstu $\theta_i$ and $\rho_{ij}$ jako míru korelace mezi $\theta_i$ a $\theta_j$. Vývoj $\theta_i$ v čase lze pak modelovat pomocí
\begin{equation*}
\theta_i(t + \delta t) - \theta_i(t) = \hat{m_i}\theta_i(t)\delta_t + s_i\theta_i(t)\epsilon_i\sqrt{\delta t}
\end{equation*}
Problém může nastat v případě $\epsilon_i$, které představuje náhodný výběr normovaného normálního rozdělení. Jestliže totiž uvažujeme $n$ parametrů $\theta_i$, které jsou vzájemně korelovány, je nutné tuto korelaci při konstrukci $\epsilon_i$ zohlednit. V prvním kroku je potřeba vygenerovat $n$ vzájemně nezávislých náhodných výběrů normovaného normálního rozdělení $x_i$. $\epsilon_i$ lze pak vypočítat na základě rovnice
\begin{equation}
\epsilon_i = \sum_{k=1}^i \alpha_{ik}x_k,
\end{equation}
přičemž musí být splněna podmínka
\begin{equation}
\sum_{k=1}^i \alpha_{ik}^2 = 1
\end{equation}
a pro všechna $j < i$ také podmínka
\begin{equation}
\sum_{k=1}^i \alpha_{ik} \alpha_{jk}=\rho_{ij}
\end{equation}
kde $\rho_{ij}$ představuje korelaci mezi $\epsilon_i$ a $\epsilon_j$.

\subsubsection{Postup výpočtu}

Jednotlivá $\epsilon_i$ jsou vypočtena metodou bootstrapping. To znamená, že pro výpočet hodnoty $\epsilon_i$ je třeba znalosti všech hodnot $\epsilon_j$, kde $j < i$.

Pro $i=1$ nejprve dle (15.19) vypočteme $\alpha_{11}=1$. Z (15.18) vyplývá
\begin{equation*}
\epsilon_1 = x_1
\end{equation*}
Po té můžeme přistoupit k určení náhodné veličiny $\epsilon_2$. S pomocí (15.20) odvodíme $\alpha_{21}=\rho_{21}$. Dosazením $\alpha_{21}$ do (15.19) získáme $\alpha_{22}=\sqrt{1-\rho_{21}}$. Hodnota  $\epsilon_2$ je tedy v souladu s (15.18) dána vztahem
\begin{equation*}
\epsilon_2 = \rho_{21}x_1 + \sqrt{1-\rho_{21}}x_2
\end{equation*}
V případě ostatních $\epsilon$ je postup analogický. Tento postup je znám jako Choleskyho dekompozice.

\subsubsection{Interní konzistence korelační matice}

Jestliže jednotlivá $\alpha$ získaná Choleskyho dekompozicí nejsou z množiny reálných čísel, není uvažovaná korelační matice tvořená korelacemi $\rho_{ij}$ interně konzistentní. Takováto korelační matice tedy nemůže existovat. Pouze připomeňme, že aby korelační matice byla interně konzistentní, musí být pozitivně semidefinitivní.

\subsection{Počet vzorků}

Vypovídací schopnost modelu Monte Carlo závisí na počtu vygenerovaných hodnot. Uvažujme $M$ vygenerovaných hodnot. Nechť $\mu$ je očekávaná hodnota finančního derivátu a $\omega$ jeho směrodatná odchylka vypočtená na základě nasimulovaných hodnot. Standardní chyba odhadu očekávané hodnoty $\mu$ je dána
\begin{equation*}
\frac{\omega}{\sqrt{M}}
\end{equation*}
Je-li $f$ skutečná očekávaná cena derivátu, pak je 95\% interval spolehlivosti pro $f$ dán vztahem
\begin{equation*}
\mu - \frac{1.96 \omega}{\sqrt{M}}<f< \mu + \frac{1.96\omega}{\sqrt{M}}
\end{equation*}

\subsection{Řecká písmena}

Metodu Monte Carlo je možné použít také pro výpočet řeckých písmen. Uvažujme odhad ceny derivátu $\hat{f}$, který jsme získali pro cenu pokladového aktiva $x$. Dále uvažujme odhad ceny derivátu $\hat{f^*}$ pro cenu pokladového aktiva $x^*$. Definujme $\delta x$ jako $x^* - x$. Delta je pak definována jako
\begin{equation*}
\Delta = \frac{\hat{f^*}-\hat{f}}{\delta x}
\end{equation*}
Pro simulaci $\hat{f}$ a $\hat{f^*}$ je třeba použít stejnou délku intervalu $\delta t$. Obdobným způsobem lze vypočíst také ostatní řecká písmena.

\section{Rozdílová metoda výpočtu ceny opce}

Uvažujme evropskou prodejní opci, kde pokladovým aktivem je akcie s nulovým dividendovým výnosem. Z dřívějších kapitol víme, že takováto opce musí splňovat diferenciální rovnici
\begin{equation*}
\frac{\partial f}{\partial t}+ rS \frac{\partial f}{\partial S} + \frac{1}{2} \sigma^2 S^2 \frac {\partial^2 f}{\partial S^2} = rf
\end{equation*} 
Nechť je $T$ životnost opce. Rozdělme $T$ na $N$ stejných intervalů délky $\delta t = T/N$. Celkový počet časových bodů je tedy $N+1$ a konkrétně se jedná o
\begin{equation*}
0, \delta t, 2\delta t,...,T
\end{equation*}
Dále definujme $S_{max}$ jako cenu pokladové akcie, pro kterou je hodnota prodejní opce blízká nule. Uvažujme $M+1$ možných cen akcie
\begin{equation*}
0, \delta S, 2 \delta S, ...,S_{max}
\end{equation*} 
Definujme $\delta S$ jako $S_{max}/M$.
Časové a cenové body tak vytvářejí matici o $(N~+~1) \times (M+1)$ bodech.
\begin{center}
	\begin{pspicture}(0,0)(11,11)
		\rput(5.5,0.0){Matice $(N+1) \times (M+1)$}
 
		\psline[arrows=->](1.0,1.0)(10.0,1.0)
                \psline[arrows=->](1.0,1.0)(1.0,10.0)
                
                \psdots(1.0,1.0)(2.0,1.0)(3.0,1.0)(4.0,1.0)(5.0,1.0)(6.0,1.0)(7.0,1.0)(8.0,1.0)(9.0,1.0)
                \psdots(1.0,2.0)(2.0,2.0)(3.0,2.0)(4.0,2.0)(5.0,2.0)(6.0,2.0)(7.0,2.0)(8.0,2.0)(9.0,2.0)
                \psdots(1.0,3.0)(2.0,3.0)(3.0,3.0)(4.0,3.0)(5.0,3.0)(6.0,3.0)(7.0,3.0)(8.0,3.0)(9.0,3.0)
                \psdots(1.0,4.0)(2.0,4.0)(3.0,4.0)(4.0,4.0)(5.0,4.0)(6.0,4.0)(7.0,4.0)(8.0,4.0)(9.0,4.0)
                \psdots(1.0,5.0)(2.0,5.0)(3.0,5.0)(4.0,5.0)(5.0,5.0)(6.0,5.0)(7.0,5.0)(8.0,5.0)(9.0,5.0)
                \psdots(1.0,6.0)(2.0,6.0)(3.0,6.0)(4.0,6.0)(5.0,6.0)(6.0,6.0)(7.0,6.0)(8.0,6.0)(9.0,6.0)
                \psdots(1.0,7.0)(2.0,7.0)(3.0,7.0)(4.0,7.0)(5.0,7.0)(6.0,7.0)(7.0,7.0)(8.0,7.0)(9.0,7.0)
                \psdots(1.0,8.0)(2.0,8.0)(3.0,8.0)(4.0,8.0)(5.0,8.0)(6.0,8.0)(7.0,8.0)(8.0,8.0)(9.0,8.0)
                \psdots(1.0,9.0)(2.0,9.0)(3.0,9.0)(4.0,9.0)(5.0,9.0)(6.0,9.0)(7.0,9.0)(8.0,9.0)(9.0,9.0)

                \rput(1.0,0.8){\tiny{$0$}}
                \rput(2.0,0.8){\tiny{$\delta t$}}
                \rput(3.0,0.8){\tiny{$2\delta t$}}
                \rput(9.0,0.8){\tiny{$T$}}
                \rput(10,1.2){\small{čas}}

                \rput(0.8,1.0){\tiny{$0$}}
                \rput(0.7,2.0){\tiny{$\delta S$}}
                \rput(0.7,3.0){\tiny{$2\delta S$}}
                \rput(0.5,9.0){\tiny{$S_{max}$}}
                \rput(3.0,10.0){\small{cena pokladové akcie}}

	\end{pspicture}
\end{center}
Bod $(i,j)$ odpovídá časovému bodu $i\delta t$ a cenovému bodu $j \delta S$. Dále definujme proměnnou $f_{i,j}$, která odpovídá hodnotě opce v bodě $(i,j)$.

Pro vnitřní bod $(i,j)$ platí
\begin{equation*}
\frac{\partial f}{\partial S} = \frac{f_{i,j+1}-f_{i,j}}{\delta S}
\end{equation*}
\begin{equation*}
\frac{\partial f}{\partial t}=\frac{f_{i+1,j}-f{i,j}}{\delta t}
\end{equation*}
\begin{equation*}
\frac{\partial^2 f}{\partial S^2}=\frac{f_{i,j+1}+f_{i,j-1}-2f_{i,j}}{\delta S^2}
\end{equation*}
Doplněním do výše uvažované diferenciální rovnice s využitím vztahu $S=j \delta S$ získáváme pro $j = 1,2,...,M-1$ a $i = 0, 1, ..., N-1$
\begin{equation*}
\frac{f_{i+1,j}-f_{i,j}}{\delta t}+rj\delta S \frac{f_{i,j+1}-f_{i,j-1}}{2 \delta S}+\frac{1}{2}\sigma^2 j^2 \delta S^2 \frac{f_{i,j+1}+f_{i,j-1}-2f_{i,j}}{\delta S^2}=rf_{i,j}
\end{equation*}
Tuto rovnici je pak možné dále upravit do tvaru
\begin{equation}
a_j f_{i,j-1} + b_jf_{i,j} + c_jf_{i,j+1} = f_{i+1,j}
\end{equation}
kde $a_j = \frac{1}{2}rj\delta t-\frac{1}{2} \sigma^2j^2 \delta t$, $b_j = 1 + \sigma^2j^2\delta t + r \delta t$ a $c_j = -\frac{1}{2}rj\delta t - \frac{1}{2}\sigma^2j^2\delta t$.
Abychom mohli tento vztah řešit, je zapotřebí určit cenu opce v "limitních" případech. Ta je dána rovnicemi (15.22), (15.23) a (15.24). Hodnota prodejní opce v čase $T$ je
\begin{equation}
f_{N,j}=\max(K-j \delta S, 0),~j=0,1,..., N
\end{equation}
Hodnota této opce, je-li cena podkladové akcie nulová, je $K$.
\begin{equation}
f_{i,0}=K,~i=0,1,...,N
\end{equation}
Dále předpokládáme, že cena prodejní opce je nulová, je-li $S=S_{max}$
\begin{equation}
f_{i,M}=0,~i=0,1,...,N
\end{equation}
Výše uvedené tři rovnice definují hodnotu uvažované prodejní opce podél tří hran matice $(N+1) \times (M+1)$. Pro výpočet ceny opce ve zbývajících bodech stačí použít rovnici (15.21). Nejprve je vypočtena hodnota opce v čase $T - \delta$. Rovnice (15.21) pro $i = N - 1$ je
\begin{equation}
a_j f_{N-1,j-1}+b_jf_{N-1,j}+c_jf_{N-1,j+1}=f_{N,j}
\end{equation}
kde $j=1,2,...,M-1$. Pravá strana soustavy $M+1$ těchto rovnic je dána (15.22). Z rovnic (15.23) a (15.24) vyplývá
\begin{equation*}
f_{N-1,0} = K
\end{equation*}
\begin{equation*}
f_{N-1,M} = 0
\end{equation*}
Řešením (15.25) pak získáme postupně $M+1$ neznámých - $f_{N-1,1}$, $f_{N-1,2}$, ..., $f_{N-1,M-1}$. Po té jsou hodnoty $f_{N-1,j}$ porovnány s $K-j \delta S$. Jestliže $f_{N-1,j}~<~K - j \delta S$, je optimální uplatnit opci předčasně v $T - \delta t$ a hodnota $f_{N-1,j}$ je tak rovna $K - j \delta t$. Analogickým způsobem jsou postupně vypočteny hodnoty také pro ostatní časové body. Výsledkem je matice hodnot $f_{0,1}$, $f_{0,2}$, ..., $f_{0,M-1}$, které odpovídají hodnotám opce pro ceny $\delta S$, $2 \delta S$, ..., $M-1 \delta S$ podkladové akcie. Pro nulovou cenu pokladové akcie je hodnota prodejní opce rovna $K$ a pro cenu $S_{max}$ je pak tato hodnota nulová.

\chapter{Exotické opce}

\section{Nestandardní americké opce}

V případě nestandardních amerických opcí může být právo uplatnit opci omezeno na určitá předem dohodnutá data, popř. může být toto právo omezeno pouze na určitou část životnosti opce. Další možností je změna realizační ceny v průběhu splatnosti opce. Příkladem finančních instrumentů, které mohou implementovat některou z těchto modifikací, jsou akciové waranty. Prakticky jedinou možností, jak ocenit tento druh opcí, je pomocí binomického stromu popř. metodou Monte Carlo.

\section{Forwardové opce}

Forwardové opce jsou opce, které neplatí okamžitě, ale od určitého v budoucnu sjednaného okamžiku.

Uvažujme evropskou forwardovou at-the-money kupní opci, která je sjednána v čase $T_0$, začíná v čase $T_1$ a je splatná v čase $T_2$. Hodnota at-the-money kupní opce je propocionální ceně pokladového aktiva. Cena forwardové opce je tedy dána vztahem
\begin{equation*}
c_{T_1}^f = e^{-rT_1}\hat{E} \Bigg[ c \frac{S_1}{S_0}\Bigg],
\end{equation*}
kde $\hat{E}[x]$ představuje očekávanou hodnotu náhodné veličiny $x$ v rizikově neutrálním světě, $c$ je cena odpovídající evropské kupní opce v čase $T_1$ a $S_0$ je spotovou cenou podkladového aktiva. Jediná veličina, kterou neznáme, je tak $S_1$, resp. $\hat{E}[S_1]$. V rizikově neutrálním světě však musí platit $\hat{E}[S_1] = S_0e^{(r-q)T_1}$. Po úpravách tak získávame
\begin{equation*}
c_{T_1}^f  = ce^{-qT_1}
\end{equation*}

\section{Složené opce}

Existují čtyři základní typy složených opcí (compound option): (a) kupní opce na kupní opci, (b) prodejní opce na prodejní opci, (c) kupní opce na prodejní opci a (d) prodejní opce na kupní opci. Každá složená opce má dvě realizační ceny a dvě data, kdy je možné uplatnit opční právo.

Uvažujme kupní opci na kupní opci. V čase $T_1$ má držitel složené opce možnost uplatnit první opční právo. V tomto případě zaplatí první z realizačních cen, $K_1$, a obdrží prodejní opci. Tato opce pak dává držiteli právo koupit za druhou realizační cenu, $K_2$, podkladové aktivum v čase $T_2$. Tato složená opce bude v čase $T_1$ uplatněna pouze v případě, že cena opce je vyšší než první realizační cena $K_1$. Hodnota evropské kupní opce na kupní opci je v čase $T_0$
\begin{equation*}
S_0e^{-qT_2}M(a_1, b_1; \sqrt{T_1 / T_2}) - K_2e^{-rT_2}M(a_2, b_2; \sqrt{T_1 / T_2})-e^{-rT_1}K_1N(a_2),
\end{equation*}
kde
\begin{equation*}
a_1 = \frac{\ln(S_0/S^*)+(r-q+\sigma^2/2)T_1}{\sigma \sqrt{T_1}},~a_2 = a_1 - \sigma \sqrt{T_1}
\end{equation*}
\begin{equation*}
b_1 = \frac{\ln(S_0/K_2)+(r-q+\sigma^2/2)T_2}{\sigma \sqrt{T_2}},~b_2 = b_1 - \sigma \sqrt{T_2}
\end{equation*}
Funkce $M$ je kumulativní dvojrozměrné normální rozdělení. Proměnná $S^*$ je cena aktiva v čase $T_1$, pro kterou se cena opce v čase $T_1$ rovná $K_1$. Podobně lze vyjádřit také cenu evropské prodejní opce na kupní opci
\begin{equation*}
K_2e^{-rT_2}M(-a_2, b_2; -\sqrt{T_1/T_2})-S_0e^{-qT_2}M(-a_1, -b_1; -\sqrt{T_1 / T_2})+e^{-rT_1}K_1N(-a_2)
\end{equation*}
cenu evropské kupní opce na prodejní opci
\begin{equation*}
K_2e^{-rT_2}M(-a_2, -b_2; \sqrt{T_1/T_2})-S_0e^{-qT_2}M(-a_1, -b_1; \sqrt{T_1 / T_2})-e^{-rT_1}K_1N(-a_2)
\end{equation*}
a cenu evropské prodejní opce na prodejní opci
\begin{equation*}
S_0e^{-qT_2}M(a_1, -b_1; -\sqrt{T_1 / T_2}) - K_2e^{-rT_2}M(a_2, -b_2; -\sqrt{T_1 / T_2})+e^{-rT_1}K_1N(a_2)
\end{equation*}

\section{Opce s volbou}

Pro opci s volbou (chooser option) je specifické, že se držitel této opce může po určité době rozhodnout, zda má být opce prodejní nebo kupní. Jsou-li obě tyto opce evropské a mají-li stejnou realizační cenu, je možné cenu této opce určit na základě put-call parity.
\begin{equation*}
\max(c,p) = \max(c,c+Ke^{-r(T_2 - T_1)}-S_1e^{-q(T_2 - T_1)})=
\end{equation*}
\begin{equation*}
=c+e^{-q(T_2 - T_1)} \max(0,Ke^{-(r-q)(T_2 - T_1)}-S_1)
\end{equation*}
Na opci s volbou lze v zásadě pohlížet jako na portfolio, které se skládá z
\begin{itemize}
\item kupní opce s realizační cenou $K$ a maturitou $T_2$
\item $e^{-q(T_2 - T_1)}$ prodejní opce s realizační cenou $Ke^{-(r-q)(T_2 - T_1)}$ a splatností $T_1$
\end{itemize}

\section{Bariérová opce}

V případě bariérové opce záleží výplata z opce na tom, zda-li cena podkladového aktiva dosáhne popř. nedosáhne určité hranice v průběhu životnosti opce. Základní dělení bariérových opcí je na knock-out a knock-in opce. Knock-out opce je v zásadě klasickou opcí, které však přestane existovat v okamžiku, kdy podkladové aktivum dosáhne určité hranice. Naopak knock-in opce je regulérní opcí, které však vznikne teprve v okamžiku, kdy pokladové aktium tuto předem dohodnutou hranici dotkne. V souvislosti s bariérovými opcemi se pak mluví o down-and-out, down-and-in, up-and-out a up-and-in kupních opcích a down-and-out, down-and-in, up-and-out a up-and-in prodejních opcích. Označení těchto opcí souvisí s tím, zda-li je bariéra protnuta shora nebo zdola, a zda-li tímto prodejní popř. kupní opce vzniká nebo zaniká.

Down-and-out a up-and-out opce jsou knock-out opce - opce zanikne, jestliže cena podkladového aktiva protne stanovenou hranici. Jak název napovídá, v případě down-and-out opce je tato hranice protnuta shora\footnote{To mimojiné znamená, že výchozí cena pokladového aktiva musí být nad touto hranicí.}, v případě up-and-out opce pak zdola\footnote{Bariéra je zvolena tak, abych se nacházela nad spotovou cenou podkladového aktiva v době sjednání opce.}. Down-and-in a up-and-in opce pak patří do rodiny knock-in opcí - vzniknou, jestliže cena podkladového aktiva protne stanovenou hranici. Směr protnutí této bariéry je pak analogický jako v případě odpovídajících knock-out opcí.

Velmi důležitým faktorem v případě bariérových opcí je stanové frekvence, se kterou se bude sledovat protnutí bariéry. Níže uvedené vzorce předpokládají, že vývoj cen je sledován spojitě; existují však také modely pro diskrétní intervaly sledování vývoje cen pokladového aktiva.

\subsection{Down-and-in a down-and-out opce}

Hodnota down-and-in kupní opce v čase nula je za předpokladu $H \le K$ dána vztahem
\begin{equation*}
c_{di} = S_0e^{-qT}(H/S_0)^{2\lambda}N(y)-Ke^{-rT}(H/S_0)^{2\lambda-2}N(y-\sigma \sqrt{T}),
\end{equation*}
kde
\begin{equation*}
\lambda = \frac{r-q+\sigma^2/2}{\sigma^2}
\end{equation*}
\begin{equation*}
y=\frac{\ln[H^2/(S_0K)]}{\sigma \sqrt{T}}+\lambda \sigma \sqrt{T}
\end{equation*}
Protože se cena kupní opce rovná součtu ceny down-and-in a down-and-out kupní opce, platí pro down-and-out kupní opci
\begin{equation*}
c_{do}=c-c_{di}
\end{equation*}
Je-li naopak $H \ge K$, je cena down-and-out opce
\begin{equation*}
c_{do} = S_0 N(x_1)e^{-qT}-Ke^{-rT}N(x_1 - \sigma \sqrt{T})-S_0e^{-qT}(H/S_0)^{2 \lambda}N(y_1)+Ke^{-rT}(H/S_0)^{2 \lambda -2}N(y_1 - \sigma \sqrt{T}),
\end{equation*}
kde
\begin{equation*}
x_1 = \frac{\ln(S_0/H)}{\sigma \sqrt{T}}+\lambda \sigma \sqrt{T}
\end{equation*}
\begin{equation*}
y_1 = \frac{\ln(H/S_0)}{\sigma \sqrt{T}}+\lambda \sigma \sqrt{T} 
\end{equation*}
Cenu down-and-in opce pak lze dopočítat ze vztahu
\begin{equation*}
c_{di} = c - c_{di}
\end{equation*}
V případě up-and-in kupní opce je její cena pro $H \ge K$ dána
\begin{equation*}
c_{ui} = S_0 N(x_1)e^{-qT}-Ke^{-rT}N(x_1 - \sigma \sqrt{T})-S_0e^{-qT}(H/S_0)^{2\lambda}[N(-y)-N(-y_1)]
\end{equation*}
\begin{equation*}
+Ke^{-rT}(H/S_0)^{2\lambda-2}[N(-y+\sigma \sqrt{T})-N(-y_1 + \sigma \sqrt{T})]
\end{equation*}
Z této ceny je možné dopočítat cenu kupní up-and-out opce
\begin{equation*}
c_{uo} = c - c_{ui}
\end{equation*}

\subsection{Up-and-in a up-and-down opce}

Cena up-and-in prodejní opce je pro $H \ge K$ dána
\begin{equation*}
p_{ui} = -S_0e^{-qT}(H/S_0)^{2 \lambda}N(-y)+Ke^{-rT}(H/S_0)^{2 \lambda - 2}N(-y+\sigma \sqrt{T})
\end{equation*}
a cena up-and-out prodejní opce pak vztahem
\begin{equation*}
p_uo = p - p_{uo}
\end{equation*}
Je-li $H \le K$, jsou odpovídající ceny
\begin{equation*}
p_{uo} = -S_0N(-x_1)e^{-qT}+Ke^{-rT}N(-x_1+\sigma \sqrt{T})+S_0e^{-qT}(H/S_0)^{2\lambda}N(-y_1)
\end{equation*}
\begin{equation*}
-Ke^{-rT}(H/S_0)^{2 \lambda - 2}N(-y_1+\sigma)
\end{equation*}
a
\begin{equation*}
p_{ui}=p-p_{uo}
\end{equation*}
V případě down-and-in and down-and-out prodejní opce platí pro $H \ge K$
\begin{equation*}
p_{do} = 0
\end{equation*}
\begin{equation*}
p_{di} = p
\end{equation*}
Je-li $H \le K$ platí pro ceny těchto akcií
\begin{equation*}
p_{di}=-S_0N(-x_1)e^{-qT}+Ke^{-rT}N(-x_1 + \sigma \sqrt{T})+S_0e^{-qT}(H/S_0)^{2 \lambda}[N(y)-N(y_1)]
\end{equation*}
\begin{equation*}
-Ke^{-rT}(H/S_0)^{2 \lambda -2}[N(y-\sigma \sqrt{T})-N(y_1 - \sigma \sqrt{T})]
\end{equation*}
a
\begin{equation*}
p_{do} = p - p_{di}
\end{equation*}

\section{Binární opce}

Výplata z binárních opcí je vázána na splnění určité podmínky. Je-li tato podmínka splněna v případě tzv. cash-or-nothing binárních opcí, je vyplacena konstatní částka; jestliže nedojde ke splnění podmínky, je výplata z opce nulová.

Uvažujme kupní cash-or-nothing opci. Výplata z této opce je nulová, je-li cena pokladového aktiva v čase $T$ pod realizační cenou $K$. V opačném případě je vyplacena částka $Q$. Pravděpodobnost, že cena podkladového aktiva přesáhne realizační cenu, je $N(d_2)$. Cena této binární opce je tudíž
\begin{equation*}
c_{con} = Qe^{-rT}N(d_2)
\end{equation*}
Hodnota odpovídající prodejní binární opce je pak 
\begin{equation*}
p_{con} = Qe^{-rT}N(-d_2)
\end{equation*}
Další možnou modifikací je, že namísto fixní částky $Q$ bude dodáno samotné aktivum. V tomto případě hovoříme o tzv. asset-or-nothing opci.
\begin{equation*}
c_{aon} = S_0e^{-qT}N(d_1)
\end{equation*}
\begin{equation*}
p_{aon} = S_0e^{-qT}N(-d_1)
\end{equation*}

\section{Zpětná opce}

Výplata ze zpětné opce (lookback option) závisí na minimální nebo maximální ceně aktiva v průběhu životnosti opce. Výplata z evropské zpětné kupní opce představuje rozdíl mezi konečnou a minimální cenou podkladového aktiva. Výplata evropské prodejní opce pak naopak představuje rozdíl mezi konečnou a maximalní cenou podkladového aktiva. Zpětná kupní opce tak představuje způsob, jak může kupující získat podkladové aktivum za nejnižší cenu, které bylo dosaženo v průběhu životnosti opce. Podobně v případě prodejní zpětné opce může vlastník opce prodat pokladové aktivum za nejvyšší cenu dosaženou v průběhu životnosti opce.

Hodnota evropské kupní zpětné opce je dána
\begin{equation*}
S_0e^{-qT}N(a_1)-S_0e^{-qT}\frac{\sigma^2}{2(r-q)}N(-a_1)-S_{min}e^{-rT}\Bigg( N(a_2) - \frac{\sigma^2}{2(r-q)}e^{Y_1}N(-a_3)\Bigg)
\end{equation*}
kde
\begin{equation*}
a_1 = \frac{\ln(S_0/S_{min})+(r-q+\sigma^2/2)T}{\sigma sqrt{T}}
\end{equation*}
\begin{equation*}
a_2 = a_1 - \sigma \sqrt{T}
\end{equation*}
\begin{equation*}
a_3 = \frac{\ln(S_0/S_{min})+(-r+q+\sigma^2/2)T}{\sigma \sqrt{T}}
\end{equation*}
\begin{equation*}
Y_1 = \frac{2(r-q-\sigma^2/2)\ln(S_0/S_{min})}{\sigma^2}
\end{equation*}
Cena evropská prodejní zpětné opce je dána vztahem
\begin{equation*}
S_{max}e^{-rT}\Bigg( N(b_1) - \frac{\sigma^2}{2(r-q)}e^{Y_2}N(-b_3) \Bigg)+S_0e^{-qT} \frac{\sigma^2}{2(r-q)}N(-b_2)-S_0e^{-qT}N(b_2)
\end{equation*}
kde
\begin{equation*}
b_1 = \frac{\ln(S_{max}/S_0)+(r-q+\sigma^2/2)T}{\sigma \sqrt{T}}
\end{equation*}
\begin{equation*}
b_2 = b_1 - \sigma \sqrt{T}
\end{equation*}
\begin{equation*}
b_3 = \frac{\ln(S_{max}/S_0)+(-r+q+\sigma^2/2)T}{\sigma \sqrt{T}}
\end{equation*}
\begin{equation*}
Y_2 = \frac{2(r-q-\sigma^2/2)\ln(S_{max}/S_0)}{\sigma^2}
\end{equation*}
$S_{min}$ resp. $S_{max}$ jsou minimální resp. maximální cena pokladového aktiva od počátku životnosti opce do data výpočtu její ceny\footnote{Jestliže cenu této opce počítame na začátku její životnosti, platí $S_{min} = S_0$ resp. $S_{max}=S_0$.}. Výše uvedené rovnice předpokládají, že jsou ceny opce sledovány na spojité bázi.

\section{Opce s výzvou}

V případě opcí s výzvou (shout option) se jedná o modifikovanou evropskou opci. Její držitel může jednou v průběhu životnosti opce vyzvat jejího vypisovatele. Na konci životnosti pak dostane vyšší částku z (a) standardní výplaty klasické evropské opce nebo (b) vnitřní hodnotou opce v době výzvy. Z tohoto pohledu je opce s výzvou do určité míry podobná zpětné opci. Uvažujme kupní opci s výzvou. Pro určení ceny opce s výzvou je důležité si uvědomit, že je-li výzva podána v čase $\tau$, platí pro výplatu z této opce v čase $T$
\begin{equation*}
\max(0,S_T - S_{\tau})+(S_T - K)
\end{equation*}
Cena této opce je proto v čase $\tau$ rovna součtu $S_{\tau}-K$ a klasické evropské kupní opce s realizační cenou $S_{\tau}$. Postup ocenění opce s výzvou bude představen v  kapitole 17.

\section {Asijská opce}

V případě asijské opce závisí výplata na průměrné ceně pokladového aktiva za určité období po dobu životnosti opce. Průměrná cena této akcie je nižší než cena klasické akcie.

Stejně jako v případě bariérových opcí je v praxi důležitá perioda, s jakou budou sledovány ceny pokladového aktiva. Níže uvedené vzorce předpokládají, že průměrná cena je počítána ze spojitých cen.

\subsection{Geometrický průměr}

Jestliže budeme pod pojmem "průměrná cena" chápat geometrický průměr, je možné asijskou opci ocenit jako klasickou opci s volatilitou $\sigma \sqrt{3}$ a dividendovým výnosem
\begin{equation*}
\tfrac{1}{2} \Bigg( r + q + \frac{\sigma^2}{6} \Bigg)
\end{equation*}

\subsection{Aritmetický průměr}

V případě, kdy je průměrná cena definována aritmetickým průměrem, nejsou analytické rovnice pro výpočet ceny asijských opcí k dispozici. Nicméně je možné tuto cenu poměrně přesně aproximovat. Nejprve je třeba určit první a druhý moment pravděpodobnostního rozdělení aritmetického průměru
\begin{equation*}
M_1 = \frac{e^{(r-q)T}-1}{(r-q)T}S_0
\end{equation*}
\begin{equation*}
M_2 = \frac{2e^{[2(r-q)+\sigma^2]T}S_0^2}{(r-q+\sigma^2)(2r-2q+\sigma^2)T^2}+\frac{2S_0^2}{(r-q)T^2}\Bigg( \frac{1}{2(r-q)+\sigma^2}-\frac{e^{(r-q)T}}{r-q+\sigma^2}\Bigg)
\end{equation*}
Asijskou opci pak můžeme ocenit stejně jako opci na futures s parametry
\begin{equation*}
F_0 = M_1
\end{equation*}
\begin{equation*}
\sigma^2 = \frac{1}{T}\ln \Bigg( \frac{M_2}{M_1^2} \Bigg)
\end{equation*}

\section{Opce na výměnu aktiv}

Klasickým příkladem opce s výměnou aktiv jsou měnové opce.

Uvažujme evropskou opci, která umožňuje svému držiteli získat v čase $T$ aktivum s hodnotu $V_T$ výměnou za aktivum s cenou $U_T$.Předpokládejme, že tyto dvě aktiva s cenami $U$ a $V$ sledují geometrický Brownův pohyb s volatilitami $\sigma_U$ a $\sigma_V$. Dále předpokládejme, že tyto aktiva generují výnos $q_U$ a $q_V$ a jejich vzájemná korelace je $\rho$. Cena takovéto opce v čase nula je
\begin{equation*}
V_0e^{-q_VT}N(d_1)-U_0e^{-q_UT}N(d_2),
\end{equation*}
kde
\begin{equation*}
d_1 = \frac{\ln(V_0/U_0)+(q_U - q_V + \hat{\sigma}^2)/T}{\hat{\sigma}\sqrt{T}}
\end{equation*}
\begin{equation*}
d_2 = d_1 - \hat{\sigma}\sqrt{T}
\end{equation*}
\begin{equation*}
\hat{\sigma}=\sqrt{\sigma_U^2 + \sigma_V^2 - 2\rho \sigma_U \sigma_V}
\end{equation*}
Je dobré si uvědomit, že opci na získání lepšího popř. horšího ze dvou aktiv lze replikovat pomocí pozice v jednom z aktiv a opcí na výměnu tohoto aktiva za druhé.
\begin{equation*}
\min{U_T, V_T} = V_T - \max(V_T - U_T, 0)
\end{equation*}
\begin{equation*}
\max(U_T, V_T) = U_T + \max(V_T, U_T,0)
\end{equation*}

\section{Portfoliové opce}

Opce, které zahrnují alespoň dvě riziková aktiva, jsou označovány jako duhové opce (rainbow option). Klasickým příkladem jsou dluhopisové futures, tak jak byly popsány dříve, kdy má krátká strana právo vybrat konkrétní dluhopis z předem daného balíku státních dluhopisů. Pravděpodobně nejpopulárnější duhovou opcí je tzv. portfoliová opce (portfolio option). V jejím případě záleží výplata na ceně podkladového portfolia, které se skládá z několika aktiv. Tyto opce je možné oceňovat pomocí metody Monte Carlo za předpokladu, že aktiva v portfoliu popsat pomocí korelovaného Brownova pohybu. Rychlejším avšak méně přesným přístupem je vypočítat první dva momenty portfolia v době maturity opce a předpokládat, že jeho cena sleduje lognormální rozdělení. Cena takovéto opce pak může být vypočtena na základě rovnic pro výpočet futures opcí.
\begin{equation*}
c = e^{-rT}[F_0 N(d_1)-K N(d_2)]
\end{equation*}
\begin{equation*}
p = e^{-rT}[K N(-d_2)- F_0 N(-d_1)]
\end{equation*}
kde
\begin{equation*}
d_1 = \frac{\ln(F_0/K)+\sigma^2T/2}{\sigma \sqrt{T}}
\end{equation*}
\begin{equation*}
d_2 = \frac{\ln(F_0/K)-\sigma^2T/2}{\sigma \sqrt{T}}=d_1 - \sigma \sqrt{T}
\end{equation*}
\begin{equation*}
F_0 = M_1
\end{equation*}
\begin{equation*}
\sigma^2 = \frac{1}{T} \ln {M_2}{M_1^2}
\end{equation*}

\chapter{Rozšíření modelů a numerických postupů}

\section{CEV model}

CEV (Constant elasticity of variance) model předpokládá, že rizikově neutrální proces pro vývoj ceny akcií je popsán rovnicí
\begin{equation*}
dS = (r-q)Sdt + \sigma S^\alpha dz
\end{equation*}
kde $r$ je bezriziková úroková sazba, $q$ dividendový výnos, $dz$ Wienerův proces, $\sigma$ volatilita a $\alpha$ pozitivní konstanta. Cena akcie pak má volatilitu $\sigma S^{\alpha - 1}$. To umožňuje nastavit model tak, aby volatilita byla funkcí ceny akcie\footnote{Je-li $\alpha < 1$, volatilita roste s poklesem ceny akcie a obráceně. Je-li $\alpha = 1$, pak se jedná o klasický model, který byl popsán výše.}.

Vzorce pro evropské opce jsou následující:

\begin{itemize}
\item $0 < \alpha < 1$
\end{itemize}
\begin{equation*}
c = S_0 e^{-qT}[1-\chi^2(a,b+2,c)]-Ke^{-rT}\chi^2(c,b,a)
\end{equation*}
\begin{equation*}
p = Ke^{-rT}[1-\chi^2(c,b,a)]-S_0e^{-qT}\chi^2(a,b+2,c)
\end{equation*}
\begin{itemize}
\item $\alpha > 1$
\end{itemize}
\begin{equation*}
c = S_0 e^{-qT}[1-\chi^2(c,-b,a)]-Ke^{-rT}\chi^2(a,2-b,c)
\end{equation*}
\begin{equation*}
p = Ke^{-rT}[1-\chi^2(a,2-b,c)]-S_0e^{-qT}\chi^2(c,-b,a)
\end{equation*}
kde
\begin{equation*}
a = \frac{K^{2(1-\alpha)}}{(1-\alpha)^2\sigma^2T}, \quad b = \frac{1}{1-\alpha}, \quad c = \frac{(Se^{(r-q)T})^{2(1-\alpha)}}{(1-\alpha)^2\sigma^2T}
\end{equation*}
$\chi^2(z,v,k)$ je kumulativní pravděpodobnost, že proměnná s necentrálním $\chi^2$ rozdělením s parametrem necentrality $v$ a $k$ stupni volnosti je menší než $z$.

Pro $0 < \alpha < 1$ generuje model CEV pravděpodobnostní rozdělení podobné tomu, jaké je možné pozorovat u akcií, tj. s těžkým levým koncem. Pro $\alpha > 1$ generuje tento model naopak pravděpodobnostní rozdělení s těžším pravým koncem.

\section{Skokový model}

Tento model představený Mertonem předpokládá možnost skokové změny ceny aktiva. Změna ceny podkladového aktiva je v rámci tohoto modelu dána rovnicí
\begin{equation*}
\frac{d S}{S} = (\mu - \lambda k)dt + \sigma dz + dp,
\end{equation*}
kde $\mu$ představuje očekávaný výnos z aktiva očištěný o dividendový výnos, $\lambda$ průměrný počet skoků v roce a $k$ je průměrná velikost skoku vyjádřená jako procentní část ceny aktiva. Pravděpodobnost cenového skoku v čase $\delta t$ je $\lambda \delta t$. Průměrná výše skoku v procentním vyjádření z ceny aktiva je tak $\lambda k$. Parametr $dz$ představuje Wienerův proces a $dp$ Poissonův proces. Oba tyto procesy jsou nezávislé. Klíčovým předpokladem modelu je, že skoková složka ceny představuje nesystematické riziko (tj. riziko, které není oceněno trhem). To znamená, že portfolio, které eliminuje nejistotu vyplývající z Brownova pohybu, musí investorovi přinášet výnos odpovídající bezrizikové sazbě.

Uvažujme specifickou situaci, kdy je logaritmus procentní změny ceny aktiva náhodnou veličinou s normálním rozdělením. Předpokládejme, že směrodatná odchylka tohoto rozdělení je $s$. Cena evropské opce je pak definována jako
\begin{equation*}
\sum^\infty_{n=0}\frac{e^{-\lambda^T}(\lambda' T)^n}{n!}f_n,
\end{equation*}
kde $\lambda' = \lambda(1+k)$. $f_n$ je cena opce vypočtena na základě Black-Scholes modelu s dividendovým výnosem $q$, směrodatnou odchylkou
\begin{equation*}
\sigma^2_n = \sigma^2 + \frac{n s^2}{T}
\end{equation*}
a bezrizikovou sazbou
\begin{equation*}
r_n = r - \lambda k + \frac{n \gamma}{T},
\end{equation*}
kde $\gamma = ln(1+k)$.

Skokový model je charakteristický těžším levým i pravým koncem pravděpodobnostního rozdělení ceny podkladového aktiva, které je charakteristické pro měnové opce.

\section{Stochastické modely volatility}

Je-li volatilita funkcí času, je proces, který sleduje vývoj cen akcií, dán vztahem
\begin{equation*}
dS = (r - q)S dt + \sigma(t)S dz
\end{equation*}
Výše uvedená rovnice předpokládá, že volatilita v čase je predikovatelná. V praxi se však volatility mění stochasticky. Hull a White uvažovali níže uvedený model změny ceny podkladového aktiva
\begin{equation*}
\frac{d S}{S} = (r - q)dt + \sqrt{V}dz_s
\end{equation*}
\begin{equation*}
d V = a(V_L - V)dt + \xi V^\alpha dz_v
\end{equation*}
kde $a$, $V_L$ a $\alpha$ jsou konstanty a $dz_s$ a $dz_y$ jsou Wienerovým procesem. Proměnná $V$ představuje v tomto modelu směrodatnou odchylku ceny aktiva. Tato směrodatná odchylka s vrací s intenzitou $a$ k dlouhodobé hladině volatility $\sigma$.

Autoři modelu porovnali tento model s Black-Scholes modelem.  Směrodatná odchylka v Black-Scholes modelu byla nastavena jako rovna očekávané směrodatné odchylce po dobu životnosti opce. Hull a White prokázali, že je-li volatilita stochastická ale nekorelovaná s cenou aktiva, je cena evropské opce rovna ceně podle Black-Scholes modelu integrované nad pravděpodobnostním rozdělením průměrné volalitilty po dobu životnosti opce. Pro evropskou kupní opci tedy např. platí
\begin{equation*}
c^* = \int^{\infty}_0 c(\overline{V})g(\overline{V})d\overline{V},
\end{equation*}
kde $\overline{V}$ je průměrná hodnota rozptylu, $c$ je cena opce z Black-Scholes modelu vyjádřená jako funkce $\overline{V}$ a $g$ je hustota pravděpodobnosti proměnné $\overline{V}$ v rizikově neutrálním světě. Vzájemným porovnáním vyšlo najevo, že Black-Scholes model nadhodnocuje at-the-money opce a podhodnocuje opce, které jsou deep-in-the-money nebo deep-out-of-the-money.

V případě, že jsou volatilita a cena akcie vzájemně korelovány, je analytické odvození ceny opce problematické. Řešení je však možné získat pomocí simulace Monte-Carlo.

\section{IVF model}
IVF (Implied volatility function) model vyvinutý Dermanem, Kanim a Rubinsteinem je konstruován tak, aby umožňoval přesnou kalibraci na dnešní ceny všech evropských opcí. Stochastický proces pro cenu podkladového aktiva je dán rovnicí

\begin{equation*}
dS = [r(t) - q(t)]S dt + \sigma(S,t)S dz,
\end{equation*}
kde $r(t)$ je forwardová úroková sazba pro kontrakt s splatností v čase $t$ a $q(t)$ je dividendový výnos jako funkce času. Směrodatná odchylka $\sigma(S,t)$ je funkcí $S$ a $t$ a je vybrána tak, aby modelové ceny všech evropských opcí byly v souladu s trhem. Volatilita podkladového aktiva pak může být počítána analyticky jako
\begin{equation*}
\sigma^2(K,T)=2\frac{\partial c_{mkt}/\partial T + q(T)c_{mkt} + K[r(T)-q(T)]\partial c_{mkt}/\partial K}{K^2(\partial^2 c_{mkt}/\partial K^2)}
\end{equation*}
kde $c_{mkt}(K,T)$ je trží cena evropské kupní opce s realizační cenou $K$ a maturitou $T$. Je-li na trhu dostatek kotací, je možné pomocí výše uvedené funkce odhadnout průběh funkce $\sigma(S,t)$.

\section{Deriváty s vazbou na historický vývoj cen}

Jedná se o deriváty, u kterých výplata závisí na historickém vývoji ceny podkladového aktiva a nikoliv pouze na jeho konečné ceně. Klasickým příkladem těchto opcí jsou asijské a zpětné opce. Nejjednodušším způsobem, jak ocenit tyto opce je metoda Monte Carlo.

\subsection{Zpětné opce}

Uvažujme americkou zpětnou opci. V případě, že se investor rozhodne tuto opci uplatnit, bude mu vyplacen rozdíl mezi nejvyšší cenou od okamžiku sjednání opce a spotovou cenou v okamžiku jejího uplatnění.

Definujme $G(t)$ jako maximimum ceny akcie dosažené do času $t$.
\begin{equation*}
Y(t)=\frac{G(t)}{S(t)}
\end{equation*}
V prvním kroku platí $Y(0)=1$. Z logiky věci také vyplývá, že je to minimální hodnota pro $Y(t)$. Pravděpodobnost růstu $Y(t)$ je rovna $1-p$; pravděpodobnost poklesu je rovna $p$. Výplatu ze zpětné opce v peněžních jednotkách lze vyjádřit jako
\begin{equation*}
SY - S
\end{equation*}
Definujme $f_{i,j}$ jako hodnotu zpětné americké opce v $j$-tém uzlu v čase $i \delta t$ a $Y_{i,j}$ jako hodnotu $Y$ v příslušném uzlu. Hodnotu $f_{i,j}$ pro $j \ge 1$ lze vyjádřit jako
\begin{equation*}
f_{i,j} = \max \Big( Y_{i,j}-1, e^{-r \delta t}[(1-p)f_{i+1,j+1}d+pf_{i+1, j-1}u]\Big)S_{i,j}
\end{equation*}
a pro $j = 1$
\begin{equation*}
f_{i,j} = \max \Big( Y_{i,j} - 1, e^{-r \delta t}[(1-p)f_{i+1,j+1}d + pf_{i+1, j}u] \Big)S_{i,j}
\end{equation*}

\subsection{Bariérová opce}
Uvažujme up-and-down bariérovou opci. Tuto opci lze ocenit stejným způsobem jako klasickou opci s tím rozdílem, že v případě uzlů nad stanovenou bariérou je hodnota opce rovna nule.

Výše uvažovanou bariérovou opci lze oceňit pomocí binomického resp. trinomického stromu. Problémem je však relativně pomalá konvergence k požadovanému výsledku. Důvodem je vysoký počet požadovaných časových period, který je nezbytný k získání přijatelně přesného výsledku. Důvodem je, že skutečná bariéra se liší od uvažované bariéry v rámci stromu. Částečným řešením tohoto problému je zavedení tzv. vnitřní a vnější beriéry, mezi kterými leží skutečná bariéra. Pro obě bariéry se vypočte cena opce v jednotlivých uzlech.
\begin{center}
	\begin{pspicture}(0,0)(7,8)
		\rput(4,0.0){Binomický strom: vnější, vnitřní a skutečná bariéra}

		\psline(0,4)(1,4.5)
		\psline(0,4)(1,3.5)
		
		\psline(1,4.5)(2,5)
		\psline[linewidth=0.5mm](1,4.5)(2,4)
		\psline(1,3.5)(2,4)
		\psline(1,3.5)(2,3)
		
		\psline[linewidth=0.5mm](2,5)(3,5.5)
		\psline(2,5)(3,4.5)
		\psline[linewidth=0.5mm](2,4)(3,4.5)
		\psline(2,4)(3,3.5)
		\psline(2,3)(3,3.5)
		\psline(2,3)(3,2.5)
		
		\psline(3,5.5)(4,6)
		\psline[linewidth=0.5mm](3,5.5)(4,5)
		\psline(3,4.5)(4,5)
		\psline[linewidth=0.5mm](3,4.5)(4,4)
		\psline(3,3.5)(4,4)
		\psline(3,3.5)(4,3)
		\psline(3,2.5)(4,3)
		\psline(3,2.5)(4,2)
		
		\psline(4,6)(5,6.5)
		\psline(4,6)(5,5.5)
		\psline[linewidth=0.5mm](4,5)(5,5.5)
		\psline(4,5)(5,4.5)
		\psline[linewidth=0.5mm](4,4)(5,4.5)
		\psline(4,4)(5,3.5)
		\psline(4,3)(5,3.5)
		\psline(4,3)(5,2.5)
		\psline(4,2)(5,2.5)
		\psline(4,2)(5,1.5)
		
		\psline(5,6.5)(6,7)
		\psline(5,6.5)(6,6)
		\psline(5,5.5)(6,6)
		\psline[linewidth=0.5mm](5,5.5)(6,5)
		\psline(5,4.5)(6,5)
		\psline[linewidth=0.5mm](5,4.5)(6,4)
		\psline(5,3.5)(6,4)
		\psline(5,3.5)(6,3)
		\psline(5,2.5)(6,3)
		\psline(5,2.5)(6,2)
		\psline(5,1.5)(6,2)
		\psline(5,1.5)(6,1)
		
		\rput(7.0,5.1){\tiny{vnější bariéra}}
		\rput(7.0,3.9){\tiny{vnitřní bariéra}}
		
		\psline[linewidth=0.5mm](0,4.8)(7.5,4.8)
		\rput(7.0,4.6){\tiny{skutečná bariéra}}
		
	\end{pspicture}
\end{center}
\begin{center}
	\begin{pspicture}(0,0)(7,8)
		\rput(4,0.0){Trinomický strom: vnější, vnitřní a skutečná bariéra}

		\psline(0,4)(1,4.5)
		\psline(0,4)(1,3.5)
		\psline(0,4)(1,4)
		
		\psline(1,4.5)(2,5)
		\psline(1,4.5)(2,4)
		\psline(1,4.5)(2,4.5)
		\psline(1,3.5)(2,4)
		\psline(1,3.5)(2,3)
		\psline(1,3.5)(2,3.5)
		\psline(1,4)(2,4.5)
		\psline(1,4)(2,3.5)
		\psline(1,4)(2,4)
		
		\psline(2,5)(3,5.5)
		\psline(2,5)(3,4.5)
		\psline(2,5)(3,5)
		\psline(2,4)(3,4.5)
		\psline(2,4)(3,3.5)
		\psline(2,4)(3,4)
		\psline(2,3)(3,3.5)
		\psline(2,3)(3,2.5)
		\psline(2,3)(3,3)
		\psline(2,4.5)(3,5)
		\psline(2,4.5)(3,4)
		\psline(2,4.5)(3,4.5)
		\psline(2,3.5)(3,4)
		\psline(2,3.5)(3,3)
		\psline(2,3.5)(3,3.5)
		
		\psline(3,5.5)(4,6)
		\psline(3,5.5)(4,5)
		\psline(3,5.5)(4,5.5)
		\psline(3,4.5)(4,5)
		\psline(3,4.5)(4,4)
		\psline(3,4.5)(4,4.5)
		\psline(3,3.5)(4,4)
		\psline(3,3.5)(4,3)
		\psline(3,3.5)(4,3.5)
		\psline(3,2.5)(4,3)
		\psline(3,2.5)(4,2)
		\psline(3,2.5)(4,2.5)
		\psline(3,5)(4,5.5)
		\psline(3,5)(4,4.5)
		\psline(3,5)(4,5)
		\psline(3,4)(4,4.5)
		\psline(3,4)(4,3.5)
		\psline(3,4)(4,4)
		\psline(3,3)(4,3.5)
		\psline(3,3)(4,2.5)
		\psline(3,3)(4,3)
		
		\psline(4,6)(5,6.5)
		\psline(4,6)(5,5.5)
		\psline(4,6)(5,6)
		\psline(4,5)(5,5.5)
		\psline(4,5)(5,4.5)
		\psline(4,5)(5,5)
		\psline(4,4)(5,4.5)
		\psline(4,4)(5,3.5)
		\psline(4,4)(5,4)
		\psline(4,3)(5,3.5)
		\psline(4,3)(5,2.5)
		\psline(4,3)(5,3)
		\psline(4,2)(5,2.5)
		\psline(4,2)(5,1.5)
		\psline(4,2)(5,2)
		\psline(4,5.5)(5,6)
		\psline(4,5.5)(5,5)
		\psline(4,5.5)(5,5.5)
		\psline(4,4.5)(5,5)
		\psline(4,4.5)(5,4)
		\psline(4,4.5)(5,4.5)
		\psline(4,3.5)(5,4)
		\psline(4,3.5)(5,3)
		\psline(4,3.5)(5,3.5)
		\psline(4,2.5)(5,3)
		\psline(4,2.5)(5,2)
		\psline(4,2.5)(5,2.5)
		
		\psline(5,6.5)(6,7)
		\psline(5,6.5)(6,6)
		\psline(5,6.5)(6,6.5)
		\psline(5,5.5)(6,6)
		\psline(5,5.5)(6,5)
		\psline(5,5.5)(6,5.5)
		\psline(5,4.5)(6,5)
		\psline(5,4.5)(6,4)
		\psline(5,4.5)(6,4.5)
		\psline(5,3.5)(6,4)
		\psline(5,3.5)(6,3)
		\psline(5,3.5)(6,3.5)
		\psline(5,2.5)(6,3)
		\psline(5,2.5)(6,2)
		\psline(5,2.5)(6,2.5)
		\psline(5,1.5)(6,2)
		\psline(5,1.5)(6,1)
		\psline(5,1.5)(6,1.5)
		\psline(5,6)(6,6.5)
		\psline(5,6)(6,5.5)
		\psline(5,6)(6,6)
		\psline(5,5)(6,5.5)
		\psline(5,5)(6,4.5)
		\psline(5,5)(6,5)
		\psline(5,4)(6,4.5)
		\psline(5,4)(6,3.5)
		\psline(5,4)(6,4)
		\psline(5,3)(6,3.5)
		\psline(5,3)(6,2.5)
		\psline(5,3)(6,3)
		\psline(5,2)(6,2.5)
		\psline(5,2)(6,1.5)
		\psline(5,2)(6,2)

		\psline[linewidth=0.5mm](0,5)(7.5,5)
		\rput(7.0,5.2){\tiny{vnější bariéra}}
                \psline[linewidth=0.5mm](0,4.5)(7.5,4.5)
		\rput(7.0,4.3){\tiny{vnitřní bariéra}}
		
		\psline[linewidth=0.5mm](0,4.9)(7.5,4.9)
		\rput(7.0,4.7){\tiny{skutečná bariéra}}
		
	\end{pspicture}
\end{center}
Obecný postup předpokládá, že vnější bariéra je skutečnou bariérou, protože k překročení uvažované bariéry dojde právě v uzlech, které tvoří vnější bariéru. V případě, že délka kroku stromu je $\delta t$, je vertikální vzdálenost mezi uzly na vnější a vnitřní bariéře řádu $\sqrt{\delta t}$. To znamená, že oceňovací chyba z titulu rozdílu mezi vnější a vnitřní bariérou je také řádu $\sqrt{t}$. Určitého zpřesnění výsledku je možné dosáhnout tak, že oceníme opci pro vnější a vnitřní bariéru a výsledné hodnoty zprůměrujeme. 

\section{Opce na dvě korelovaná aktiva}

Jedná se o situaci, kdy je hodnota opce závislá na dvou podkladových aktivech, která jsou vzájemně korelovaná. 

\subsection{Transformace proměnných}

Předpokládejme, že tato dvě aktiva sledují proces
\begin{equation*}
dS_1 = (r-q_1)S_1 dt + \sigma_1 S_1 dz_1
\end{equation*}
\begin{equation*}
d S_2 = (r-q_2)S_2dt+\sigma_2 S_2 dz_2
\end{equation*}
kde korelace mezi Wienerovým procesem $dz_1$ a $dz_2$ je $\rho$. Platí tedy
\begin{equation*}
d \ln S_1 = (r-q_1-\sigma_1^2/2)dt + \sigma_1 dz_1 
\end{equation*}
\begin{equation*}
d \ln S_2 = (r-q_2-\sigma_2^2/2)dt + \sigma_2 dz_2
\end{equation*}
Definujme dále dvě nekorelované proměnné
\begin{equation*}
x_1 = \sigma_2 \ln S_1 + \sigma_1 \ln S_2
\end{equation*}
\begin{equation*}
x_2 = \sigma_2 \ln S_1 - \sigma_1 \ln S_2
\end{equation*}
Tyto proměnné sledují proces
\begin{equation*}
dx_1 = [\sigma_2(r-q_1-\sigma^2_1/2)+\sigma_1(r-q_2-\sigma^2_2/2)]dt+\sigma_1 \sigma_2 \sqrt{2(1+\rho)}dz_A
\end{equation*}
\begin{equation*}
dx_2 = [\sigma_2(r-q_1-\sigma^2_1/2)-\sigma_1(r-q_2-\sigma^2_2/2)]dt+\sigma_1 \sigma_2 \sqrt{2(1-\rho)}dz_A
\end{equation*}
kde $dz_A$ a $dz_B$ jsou nekorelované Wienerovy procesy. Proměnné $x_1$ a $x_2$ mohou být modelovány pomocí dvou nezávislých binomických stromů. V čase $\delta t$ má $x_i$ pravděpodobnost $p_i$ růstu o $h_i$ a pravděpodobnost $1-p_i$ poklesu o $h_i$. Proměnné $h_i$ a $p_i$ jsou vybrány tak, aby binomické stromy dávaly správné hodnoty pro první dva momenty pravděpodobnostního rozdělení $x_1$ a $x_2$. Protože jsou vzájemně nekorelované, je možné oba binomické stromy zkombinovat do jednoho trojrozměrného stromu. Pro každý uzel binomického stromu je možné vypočítat $S_1$ a $S_2$ z $x_1$ a $x_2$ jako
\begin{equation*}
S_1 = e^{\frac{x_1 + x_2}{2\sigma_2}}
\end{equation*}
\begin{equation*}
S_2 = e^{\frac{x_1 - x_2}{2\sigma_1}}
\end{equation*}

\subsection{Modifikace binomického stromu}

Další možností je přímo modelovat strom pro obě aktiva. Opět uvažujme aktiva $S_1$ a $S_2$ s mírou korelace $\rho$. Z každého uzlu $(S_1, S_2)$ je možné se přesunout do uzlů $(S_1u_1,S_2A)$, $(S_1u_1,S_2B)$, $(S_1d_1,S_2C)$ a $(S_2d_1, S_2D)$, kde každý uzel má přiřazenu pravděpodobnost 0.25. Pro tyto uzly platí
\begin{equation*}
u_1 = e^{(r-q_1-\sigma^2_1/2)\delta t + \sigma_1 \sqrt{\delta t}}
\end{equation*}
\begin{equation*}
d_1 = e^{(r-q_1-\sigma^2_1/2)\delta t - \sigma_1 \sqrt{\delta t}}
\end{equation*}
\begin{equation*}
A = e^{(r-q_2-\sigma^2_2/2)\delta t + \sigma_2\sqrt{\delta t}(\rho+\sqrt{1-\rho^2})}
\end{equation*}
\begin{equation*}
B = e^{(r-q_2-\sigma^2_2/2)\delta t + \sigma_2\sqrt{\delta t}(\rho+\sqrt{1-\rho^2})}
\end{equation*}
\begin{equation*}
C = e^{(r-q_2-\sigma^2_2/2)\delta t - \sigma_2\sqrt{\delta t}(\rho-\sqrt{1-\rho^2})}
\end{equation*}
\begin{equation*}
C = e^{(r-q_2-\sigma^2_2/2)\delta t - \sigma_2\sqrt{\delta t}(\rho+\sqrt{1-\rho^2})}
\end{equation*}

\subsection{Modifikace pravděpodobnosti}

Poslední alternativou konstrukce trojrozměrného binomického stromu pro $S_1$ a $S_2$ je úprava pravděpodobnosti o vzájmenou korelaci mezi aktivy.

Jestliže uvažujeme dvě nekorelovaná aktiva $S_1$ a $S_2$, je pravděpodobnostní matice těchto aktiv následující
\begin{center}
\begin{tabular}{c c c}
\textbf{} &
\multicolumn{2}{c}{\textbf{Pohyb $S_2$}}\\
\hline
\multicolumn {1}{l|}{\textbf{Pohyb $S_1$}} &
\textbf{Pokles} &
\textbf{Růst} \\
\hline
\multicolumn {1}{l|}{\textbf{Růst}} & 0.25 & 0.25 \\ 
\multicolumn {1}{l|}{\textbf{Pokles}} & 0.25 & 0.25 \\
\hline
\end{tabular}
\end{center}
\begin{center}
\small{Pravděpodobnost pohybu cen nekorelovaných aktiv $S_1$ a $S_2$}
\end{center}
V případě, že míra korelace mezi aktivy $S_1$ a $S_2$ je $\rho$, je třeba pravděpodonosti upravit níže uvedeným způsobem
\begin{center}
\begin{tabular}{c c c}
\textbf{} &
\multicolumn{2}{c}{\textbf{Pohyb $S_2$}}\\
\hline
\multicolumn {1}{l|}{\textbf{Pohyb $S_1$}} &
\textbf{Pokles} &
\textbf{Růst} \\
\hline
\multicolumn {1}{l|}{\textbf{Růst}} & 0.25(1-$\rho$) & 0.25(1+$\rho$) \\ 
\multicolumn {1}{l|}{\textbf{Pokles}} & 0.25(1+$\rho$) & 0.25(1-$\rho$) \\
\hline
\end{tabular}
\end{center}
\begin{center}
\small{Pravděpodobnost pohybu cen korelovaných aktiv $S_1$ a $S_2$}
\end{center}

\section{Simulace Monte Carlo a americké opce}

Metoda Monte Carlo je ideální pro oceňování opcí, jejichž výplata se odvíjí od historického vývoje ceny podkladového aktiva\footnote{Jedná se zejména o asijské a zpětné opce.}, a případy, kdy do ceny opce vstupuje velké množství náhodných proměnných.

K tomu, aby bylo možné ocenit americkou opci, je nutné v každém čase, kdy je možné opci uplatnit, rozhodnout, zda-li je výhodnější tak učinit či nikoliv. Uvažujme tříroční prodejní opci na akci s nulovým dividendových výnosem, která může být uplatněna vždy na konci každého ze tří roků. Bezriziková úroková sazba je 6\% p.a. (složené úročení), spotová cena pokladové akcie 1.00 a realizační cena je 1.10. Následující tabulka zachycuje osm nasimulovaných hodnot podkladového aktiva, na kterých budeme demonstrovat postup výpočtu.
\begin{center}
\begin{tabular}{c c c c c}
\hline
\textbf{Simulace} & $t=0$ & $t=1$ & $t=2$ & $t=3$ \\
\hline
1 & 1.00 & 1.09 & 1.08 & 1.34 \\
2 & 1.00 & 1.16 & 1.26 & 1.56 \\
3 & 1.00 & 1.22 & 1.07 & 1.03 \\
4 & 1.00 & 0.93 & 0.97 & 0.92 \\
5 & 1.00 & 1.11 & 1.56 & 1.52 \\
6 & 1.00 & 0.76 & 0.77 & 0.90 \\
7 & 1.00 & 0.92 & 0.84 & 1.01 \\
8 & 1.00 & 0.88 & 1.22 & 1.34 \\
\hline
\end{tabular}
\end{center}
\begin{center}
\small{Simulace vývoje cen pokladové akcie}
\end{center}
Jestliže by majitel uvažovanou opci neuplatnil na konci prvního ani druhého roku, byla by opce na konci třetího roku at-the-money pro simulace 3, 4, 6 a 7. Uplatnění opcí by přineslo cash-flow ve výši 0.07, 0.18, 0.20 a 0.09.

Přesuňme se nyní na konec druhého roku. Předpokládejme, že opce nebyla uplatněna. Je-li opce na konci druhého roku at-the-money, musí se její majitel rozhodnout, zda-li tuto opci uplatní nebo nikoliv. Z výše uvedené tabulky je zřejmé, že se jedná o simulace 1, 3, 4, 6 a 7. Pro tyto simulace uvažujme vztah
\begin{equation*}
V = a + bS + cS^2,
\end{equation*}
kde $S$ je cena podkladového aktiva dle simulace na konci druhého roku a $V$ je cena neuplatnění opce diskontovaná ke konci druhého roku. Hodnoty parametru $V$ jsou $0.00e^{-0.06}$, $0.07e^{-0.06}$, $0.18e^{-0.06}$, $0.20e^{-0.06}$ a $0.09e^{-0.06}$\footnote{Pro každou simulaci je porovnána cena na konci třetího roku s realizační cenou a příslušný rozdíl je diskontován zpětně ke konci druhého roku.}. Tyto hodnoty se pak použijí pro výpočet parametrů $a$, $b$ a $c$ pomocí metody nejmenších čtverců.
\begin{equation*}
\sum^5_{i=1}(V_i - a -bS_i-cS^2_i)^2 \xrightarrow{} \min
\end{equation*}
Řešením výše uvedené podmínky je
\begin{equation*}
V = -1.070 + 2.983S - 1.813S^2
\end{equation*}
Dle této rovnice vychází regresní hodnota $V$ pro simulace 1, 3, 4, 6 a 7 jako 0.0369, 0.0461, 0.1176, 0.1520 and 0.1565. Odpovídající hodnoty dané realizační cenou a cenou akcie na konci druhého roku podle výše uvedené tabulky jsou 0.02, 0.03, 0.13, 0.33 a 0.26. Z toho vyplývá, že na konci druhého roku by majitel opce měl realizovat své právo v případě simulací 4, 6 a 7.

Stejný postup je pak aplikován také na konec prvního roku. Simulace, které jsou na konci prvního roku in-the-money, jsou 1, 4, 6 a 7. Hodnoty $V$ pro tyto simulace jsou rovny $0.00e^{-0.06}$, $0.13e^{-0.06}$, $0.33e^{-0.06}$, $0.26e^{-0.06}$ a $0.00e^{-0.06}$. Hodnoty parametrů $a$, $b$ a $c$ vypočtené metodou nejmenších čtverů jsou 2.038, -3.335 a 1.356. Rovnice pro výpočet regresní hodnoty $V$ má tedy tvar
\begin{equation*}
V = 2.038 - 3.335S + 1.356S^2
\end{equation*}
Jestliže opce nebude na konci prvního roku uplatněna, je její hodnota pro simulace 1, 4, 6 a 7 rovna 0.0139, 0.1092, 0.2866, 0.1175 a 0.1533. Hodnota opce v případě uplatnění bude pro uvažované simulace rovna 0.01, 0.17, 0.34, 0.18 a 0.22. Majitel tedy uplatní opci v případě simulací 4, 6, 7 a 8.

Na konci prvního roku tedy majitel uplatnil opci pro simulace 4, 6, 7 a 8. Na konci druhého roku nebyla opce uplatněna pro žádnou ze simulací\footnote{V případě uzlů 4,6 a 7 byla opce uplatněna již na konci prvního roku.}. Na konci třetího roku byla uplatněna opce pro simulaci 3\footnote{Pro simulace 4, 6 a 7 byla opce uplatněna již v prvním roce.}. Výplatu z opce shrnuje následující tabulka.
\begin{center}
\begin{tabular}{c c c c }
\hline
\textbf{Simulace} & $t=1$ & $t=2$ & $t=3$ \\
\hline
1 & 0.00 & 0.00 & 0.00 \\
2 & 0.00 & 0.00 & 0.00 \\
3 & 0.00 & 0.00 & 0.07 \\
4 & 0.17 & 0.00 & 0.00 \\
5 & 0.00 & 0.00 & 0.00 \\
6 & 0.34 & 0.00 & 0.00 \\
7 & 0.18 & 0.00 & 0.00 \\
8 & 0.22 & 0.00 & 0.00 \\
\hline
\end{tabular}
\end{center}
\begin{center}
\small{Výplata z americké prodejní opce}
\end{center}
Výsledná hodnota opce je tedy
\begin{equation*}
\frac{1}{8}(0.07e^{-0.06 \cdot 3} + 0.34e^{-0.06 \cdot 1} + 0.18e^{-0.06 \cdot 1} + 0.18e^{-0.06 \cdot 1} + 0.22e^{-0.06 \cdot 1})=0.1144
\end{equation*}
Protože je výsledná hodnota opce větší než 0.10 je výhodné tuto opci okamžitě uplatnit.

\chapter{Martingaly a míry}

Až dosud jsme uvažovali konstantní úrokové sazby po celou dobu životnosti opce. V rámci rizikově neutrálního světa, ve kterém jsme se v předchozích kapitolách ``pohybovali'', hrála klíčovou roli bezriziková sazba. Ta se používala k diskontování cash-flow a dále se předpokládalo, že očekávaný výnos všech finančních instrumentů je roven právě bezrizikové úrokové sazbě\footnote{Připomeňme, že v uvažovaném rizikově neutrálním světě neexistovala prémie za riziko.}. V následující kapitole předpoklad konstantních úrokových sazeb opustíme. Uvolnění tohoto předpokladu má za následek, že existuje mnoho paralelních rizikově neutrálních světů. A právě martingaly a míry jsou nezbytné pro správné pochopení rizikově neutrálního oceňování. Martingale lze zjednodušeně definovat jako stochastický proces s nulovou očekávanou střední hodnotou. Míru pak lze chápat jako jednotku, kterou měříme hodnotu cenných papírů.

\section{Tržní cena rizika}

\subsection{Jeden rizikový parametr}

Uvažujme proměnnou $\theta$, která sleduje proces
\begin{equation*}
\frac{d \theta}{\theta}=mdt + sdz,
\end{equation*}
kde $dz$ je Wienerův proces, $m$ je očekávaná míra růstu a $s$ volatilita proměnné $\theta$. Dále uvažujme hodnoty $f_1$ a $f_2$ dvou derivátů, které jsou závislé pouze na $\theta$ a $t$. Tyto hodnoty sledují procesy
\begin{equation*}
\frac{d f_1}{f_1}=\mu_1 dt + \sigma_1 dz
\end{equation*}
\begin{equation*}
\frac{d f_2}{f_2}=\mu_2 dt + \sigma_2 dz
\end{equation*}
kde $\mu_1$, $\mu_2$, $\sigma_1$ a $\sigma_2$ jsou funkcemi $\theta$ a $t$. $dz$ je ten samý Wienerův proces, jaký figuruje v procesu proměnné $\theta$, protože se jedná o jediný zdroj nejistoty pro $f_1$ a $f_2$. Diskrétní verze výše uvedených procesů jsou
\begin{equation}
\delta f_1 = \mu_1 f_1 \delta t + \sigma_1 f_1 \delta z
\end{equation}
\begin{equation}
\delta f_2 = \mu_2 f_2 \delta t + \sigma_2 f_2 \delta z 
\end{equation}
Wienerův proces $\delta z$ je možné eliminovat pomocí portfolia, které se skládá z $\sigma_2 f_2$ prvního derivátu a $-\sigma_1 f_1$ druhého derivátu. Je-li $\Pi$ hodnota portfolia, pak platí
\begin{equation*}
\Pi = (\sigma_2 f_2)f_1 - (\sigma_1 f_1)f_2
\end{equation*}
\begin{equation}
\delta \Pi = \sigma_2 f_2 \delta f_1 - \sigma_1 f_1 \delta f_2
\end{equation}
Dosadíme-li (18.1) a (18.2) do (18.3), dostáváme
\begin{equation}
\delta \Pi = (\mu_1 \sigma_2 f_1 f_2 - \mu_2 \sigma_1 f_1 f_2)\delta t
\end{equation}
Protože je portfolio $\Pi$ bezrizikové, musí generovat výnos, který odpovídá bezrizikové sazbě $r$.
\begin{equation*}
\delta \Pi = r \Pi \delta t
\end{equation*}
Dalšími úpravami výše uvedeného vztahu získáme
\begin{equation*}
(\mu_1 \sigma_2 f_1 f_2 - \mu_2 \sigma_1 f_1 f_2)\delta t = r\bigg((\sigma_2 f_2)f_1 - (\sigma_1 f_1)f_2 \bigg) \delta t
\end{equation*}
\begin{equation*}
\mu_1 \sigma_2 - \mu_2 \sigma_1 = r \sigma_2 - r \sigma_1
\end{equation*}
Definujme $\lambda$ jako
\begin{equation*}
\lambda = \frac{\mu_1 - r}{\sigma_1} = \frac{\mu_2 - r}{\sigma_2}
\end{equation*}
Vypustíme-li indexy, dokázali jsme , že je-li $f$ cenou derivátu závislého na $\theta$ a $t$, kde
\begin{equation}
\frac{d f}{f}=\mu dt + \sigma dz
\end{equation}
pak
\begin{equation}
\lambda = \frac{\mu - r}{\sigma}
\end{equation}
Parametr $\lambda$ je tzv. tržní cena rizika $\theta$. $\lambda$ závisí na $\theta$ popř. také na $t$, ale není závislé na konstrukci příslušného derivátu. V daný časový okamžik tak musí být $(\mu - r)/\sigma$ shodné pro všechny deriváty, které jsou závislé pouze na $\theta$ a $t$.

Směrodatná odchylka $\sigma$ ceny derivátu $f$ je v (18.5) definovaná koeficient $dz$. Paradoxem je, že tato směrodatná odchylka může být jak kladná tak záporná. Jestliže je směrodatná odchylka $s$ parametru $\theta$ je kladná a $f$ je pozitivně korelované s $\theta$, pak je $\sigma$ také pozitivní. Je-li však $f$ negativně korelováno s $\theta$, je také $\sigma$ záporné. Směrodatná odchylka $\sigma$ ceny derivátu $f$ ve statistickém slova smyslu je tedy definována jako $\arrowvert \sigma \arrowvert$.

Rovnici (18.6) lze přepsat do tvaru
\begin{equation*}
\mu - r = \lambda \sigma
\end{equation*}
Tento vztah lze chápat jako kompenzaci za podstupované riziko a proměnnou $\sigma$ jako množství rizika $\theta$ v ceně derivátu $f$. Na pravé straně rovnice proto násobíme množství rizika $\theta$ jeho cenou $\lambda$.\\

\noindent \textbf{Poznámka:} Výše uvedené by mělo platit pro investiční aktiva. V případě aktiv, která slouží jako výrobní vstup, toto však platit nemusí\footnote{Jedná se především o komodity.}.\\

Jak již bylo zmíněno výše, cena derivátu $f$ sleduje proces
\begin{equation}
df = \mu f dt + \sigma f dz
\end{equation}
Hodnota $\mu$ závisí na rizikových preferencích investorů. V rizikově neutrálním světě, kde je tržní cena rizika rovna nule, je $\lambda$ nulové. V tomto případě platí $\mu = r$, a proto lze výše uvedenou rovnici přepsat do tvaru
\begin{equation*}
df = r f dt + \sigma f dz
\end{equation*}
Definujeme-li v souladu s výše uvedeným textem $\lambda$ jako tržní cenu rizika $\theta$, pak lze definovat $\mu$ jako
\begin{equation*}
\mu = r + \lambda \sigma
\end{equation*}
Rovnici (18.7) je tak možné přepsat do tvaru
\begin{equation}
df = (r + \lambda \sigma)f dt + \sigma f dz
\end{equation}
Tržní cena příslušného rizika tedy ovlivňuje hodnotu očekávaného výnosu všech cenných papírů závislých na tomto riziku. Při přesunu z jedné cenové úrovně rizika na jinou se tak mění očekávaný výnos cenného papíru, avšak jeho volatilita zůstává stejná. Výběrem konkrétní tržní ceny rizika je tedy definována pravděpodobnostní míra. Pro určitou hodnotu rizika tak lze získat obraz ``reálného'' světa, kdy modelované výnosové míry odpovídají skutečným výnosovým mírám.

\subsection{Vícero rizikových parametrů}

Uvažujme proměnné $\theta_1$, $\theta_2$,...,$\theta_n$, které jsou popsány náhodným procesem
\begin{equation*}
\frac{d \theta_i}{\theta_i} = m_i dt + s_i dz_i
\end{equation*}
kde $m_i$ je očekávaná míra růstu proměnné $\theta_i$ a $s_i$ je její směrodatnou odchylku. Dále uvažujme náhodnou veličinu $f$, která představuje hodnotu cenného papíru a která je závislá na $\theta_i$ a čase $t$. Dle lemy 18A sleduje náhodná veličina $f$ stochastický proces
\begin{equation*}
\frac{d f}{f} = \mu dt + \sum^n_{i=1}\sigma_i dz_i
\end{equation*}
kde $\mu$ je očekávaný výnos z cenného papíru a $\sigma_i dz_i$ představuje výnos, který kompenzuje rizikovou složku z $\theta_i$. Lze dokázat, že pro celkovou rizikovou složku výnosu platí
\begin{equation*}
\mu - r = \sum^n_{i=1}\lambda_i \sigma_i
\end{equation*}
Výraz $\lambda_i \sigma_i$ na pravé straně rovnice vyjadřuje dodatečný výnos, který je vyžadován investory jako kompenzace za závislost hodnoty $f$ na $\theta_i$. Celkový rizikový výnos je pak dán součtem přes všechna $\theta_i$. $\lambda_i \sigma_i$ může být i záporné a to v případě, kdy $\theta_i$ má za následek snížení rizika portfolia typického investora.

\section{Martingaly}

Martingale je stochastický proces s nulovým trendem. Jestliže je určitá náhodná veličina popsána martingalem, pak sleduje proces
\begin{equation*}
d \theta = \sigma dz
\end{equation*}
Proměnná $\sigma$ může být sama o sobě náhodná. Jestliže je proměnná $\theta$ martingalem, platí 
\begin{equation*}
E[\theta_T]=\theta_0
\end{equation*}
Z definice martingalu tedy vyplývá, že jeho očekávaná hodnota v libovolný okamžik je rovna jeho současné hodnotě. Výše uvedenou definici lze interpretovat tak, že změny $\theta$ ve velmi malém časovém intervalu lze popsat normálním rozdělením s nulovou střední hodnotou.

\subsection{Martingale míra}

Uvažujme hodnoty dvou instrumentů $f$ a $g$, které závisí na společném ``zdroji'' nejistoty. Dále předpokládejme, že oba instrumenty nepřinášejí investorům po námi uvažovaný časový interval žádný výnos. Definujme $\phi$ jako
\begin{equation*}
\phi = \frac{f}{g}
\end{equation*}
Tímto jsme definovali tzv. martingale míru. Hodnotu $f$ tak namísto v peněžních jednotkách vyjadřujeme jako násobek hodnoty $g$.

Lze dokázat, že při nemožnosti arbitráže existuje úroveň ceny rizika, pro kterou je $\phi$ martingalem. Navíc platí, že pro daný instrument $g$ dělá stejná cena rizika z $\phi$ martingale pro všechna $f$. Touto cenou rizika je v našem případě směrodatná odchylka $g$. Jinými slovy je-li tržní cena rizika rovna směrodatné odchylce $g$, je podíl $f/g$ martingalem pro všechna $f$.

Použijeme-li (18.8), můžeme za podmínky, že třžní cena rizika je rovna směrodatné odchylce $g$, vyjádřit $df$ a $dg$ jako
\begin{equation*}
df = (r + \sigma_g \sigma_f)f dt + \sigma_f fdz
\end{equation*}
\begin{equation*}
dg = (r + \sigma_g^2)g dt + \sigma_g gdz
\end{equation*}
Tyto dva vztahy lze s využitím It\^o lemmy dále upravit do tvaru
\begin{equation*}
d \ln f = (r + \sigma_g \sigma_f - \sigma_f^2/2)dt + \sigma_f dz
\end{equation*}
\begin{equation*}
d \ln g = (r + \sigma_g^2/2)dt + \sigma_g dz
\end{equation*}
Rozdílem obou výše uvedených rovnice je
\begin{equation}
d \Bigg( \ln \frac{f}{g} \Bigg) = - \frac{(\sigma_f - \sigma_g)^2}{2}dt + (\sigma_f - \sigma_g)dz 
\end{equation}
S opětovným použitím It\^o lemmy lze (18.9) upravit do tvaru
\begin{equation*}
d \Bigg( \frac{f}{g} \Bigg) = (\sigma_f - \sigma_g) \frac{f}{g}dz
\end{equation*}
čímž jsme dokázali, že $f/g$ je martingale\footnote{Výše uvedený proces je martingale, protože postrádá trend. Očekávaná hodnota $dz$ je nulová, a proto je očekávaná hodnota $d(f/g)$ rovnež nulová.}. Svět, kde je tržní cena rizika rovna $\sigma_g$, nazýváme forwardově rizikově neutrální s ohledem k $g$. Z toho mimojiné vyplývá
\begin{equation*}
\frac{f_0}{g_0}=E_g\Bigg[ \frac{f_T}{g_T} \Bigg]
\end{equation*}
neboli
\begin{equation}
f_0 = g_0 E_0 \Bigg[ \frac{f_T}{g_T} \Bigg]
\end{equation}
kde $E_0$ označuje očekávanou hodnotu ve světě, který je forwardově neutrální ve vztahu k $g$.

\subsection{Použití alternativních jmenovatelů v ekvivalentní martingale míře}

V přechozí kapitole jsme jako jmenovatele při definici ekvivalentní martingale míři použili $g$, které reprezentovalo cenu obecného finančního instrumentu. V následujícím textu použijeme některé konkrétní investiční instrumenty.\\

\subsubsection{Účet peněžního trhu}

Účet peněžního trhu je investiční instrument, jehož hodnota je 1 USD v čase nula a který svému majiteli generuje výnos odpovídající bezrizikové úrokové sazbě $r$. Proměnná $r$ může být stochastická.
\begin{equation*}
dg = r g\ dt
\end{equation*}
Proměnná $g$ je stochastická a její volatilita je tak nulová. Tímto jsme definovali svět, kde je tržní hodnota rizika nula. Jedná se o tradiční rizikově neutrální svět, tak jak jsme ho definovali dříve.
\begin{equation*}
f_0 = g_0 \hat{E} \Bigg[ \frac{f_T}{g_T} \Bigg]
\end{equation*}
Vzhledem k tomu, že
\begin{equation*}
g_0 = 1
\end{equation*}
a
\begin{equation*}
g_T = e^{\int^T_0 r\ dt}
\end{equation*}
lze $f_0$ vyjádřit jako
\begin{equation*}
f_0 = \hat{E}\Big[ e^{- \int^T_0 r\ dt} f_T \Big]
\end{equation*}
Jestliže je krátkodobá sazba $r$ fixní, lze $f_0$ vyjádřit jako
\begin{equation*}
f_0 = e^{-rT}\hat{E}[f_T]
\end{equation*}

\subsubsection{Diskontní dluhopis}

Definujme $P(t,T)$ jako cenu diskontního dluhopisu v čase $t$ za předpokladu, že tento dluhopis generuje cash-flow 1 USD v čase $T$. Dále definujme $E_T$ jako očekávání ve světe, který je forwardově rizikově neutrální vzhledem k $P(t,T)$. Vzhledem k tomu, že $g_T = P(T,T) = 1$ a $g_0 = P(0,T)$\footnote{$P(0,T)$ lze chápat jako diskontní faktor pro časový horizont 0 až $T$.}, platí podle (18.10)
\begin{equation*}
f_0 = P(0,T)E_T[f_T]
\end{equation*}
Zatímco v předchozím případě bylo diskontování součástí očekávané hodnoty\footnote{Výjimkou byla situace, kdy jsme uvažovali konstantní bezrizikovu míru. V tomto případě bylo možné diskontování z očekávané hodnoty ``vyjmout''.}, je diskontování pro diskontní dluhopis od očekávané hodnoty oddělené. Tento fakt značně zjednodušuje navazující úvahy.

Je-li $F$ forwardovou cenou $f$ pro kontrakt se splatností v čase $T$, je $F$ za předpokladu neexistence arbitráže dáno vztahem
\begin{equation*}
F = \frac{f_0}{P(0,T)}
\end{equation*}
z čehož vyplývá
\begin{equation*}
F = E_T[f_T]
\end{equation*}
Tímto jsme dokázali, že ve světě, který je forwardově rizikově neutrální ve vztahu k $P(t,T)$, je forwardová cena finačního instrumentu $f$ rovna jeho očekávané spotové ceně.\\

\subsubsection{Forwardové sazby}

Definujme $R(t,T_1,T_2)$ jako forwardovou úrokovou sazbu platnou v čase $t$ na období $T_1$ až $T_2$. Protože je forwardová úroková sazba implikována forwardovou cenou odpovídajícího dluhopisu, platí
\begin{equation*}
\frac{1}{1+(T_2 - T_1)R(t,T_1, T_2)}=\frac{P(t,T_2)}{P(t,T_1)}
\end{equation*}
resp.
\begin{equation*}
R(t, T_1, T_2) = \frac{1}{T_2 - T_1} \Bigg( \frac{P(t, T_1)-P(t, T_2)}{P(t, T_2)}\Bigg)
\end{equation*}
Jestliže definujeme $f$ jako
\begin{equation*}
f = \frac{1}{T_2 - T_1}[P(t,T_1) - P(t, T_2)]
\end{equation*}
a $g$ jako
\begin{equation*}
g = P(t,T_2)
\end{equation*}
lze dokázat, že $R(t,T_1,T_2)$ je martingalem ve světě, který se forwardově rizikově neutrální s ohledem na $P(t,T_2)$.
\begin{equation*}
f_0 = g_0 E_2 \Bigg[ \frac{f_{T_1}}{g_{T_1}}\Bigg]
\end{equation*}
\begin{equation*}
\frac{1}{T_2 - T_1}[P(T_1,T_1) - P(T_1, T_2)] = P(0, T_2) E_2 \Bigg[\frac{\frac{1}{T_2 - T_1}[P(T_1,T_1) - P(T_1, T_2)]}{P(T_1, T_2)}\Bigg]
\end{equation*}
\begin{equation*}
R(0,T_1,T_2) = E_2[R(T_1, T_1, T_2)]
\end{equation*}

\subsubsection{Anuita}

Uvažujme swap, který začíná v budoucím čase $T$ s platbami v čase $T_1$, $T_2$, ..., $T_N$. Definujme $T_0 = T$. Předpokládejme, že jistina tohoto swapu má hodnotu 1 USD. Dále uvažujme forwardovou swapovou sazbu $s(t)$ v čase $t$\footnote{Swapovou sazbou $s(t)$ rozumíme sazbu fixní nohy, pro kterou je hodnota úrokového swapu nulová.}, kde $t \le T$. Hodnota fixní nohy swapu je
\begin{equation*}
s(t)A(t)
\end{equation*}
kde
\begin{equation*}
A(t) = \sum^{N-1}_{i=0}(T_{i+1}-T_i)P(t, T_{i+1})
\end{equation*}
Pro úrokový swap platí, že je-li k poslední platbě v době splatnosti swapu přidán nominál, je hodnota plovoucí nohy na počátku kontraktu rovná výši tohoto nominálu. Hodnota 1 USD v čase $t$ obdrženého v čase $T_N$ je $P(t, T_N)$ a podobně hodnota 1 USD v čase $t$ obdrženého v čase $T_0$ je $P(t, T_0)$. Hodnota plovoucí swapové nohy v čase $t$ je tak
\begin{equation*}
P(t,T_0) - P(t,T_N)
\end{equation*}
Jestliže platí předpoklad rovnosti mezi hodnotou fixní a plovoucí nohy, získáváme
\begin{equation*}
s(t)A(t) = P(t,T_0) - P(t,T_N)
\end{equation*}
Parametr $s(t)$ je tak možné vyjádřit jako
\begin{equation*}
s(t) = \frac{P(t,T_0) - P(t,T_N)}{A(t)}
\end{equation*}
Jestliže $f$ definujeme jako $P(t,T_0) - P(t,T_N)$ a $g$ jako $A(t)$, lze s použítím martingale míry vyjádřit $s(t)$ jako
\begin{equation*}
s(t) = E_A[s(T)]
\end{equation*}
Ve světě, který je rizikově neutrální vzhledem k $A(t)$, je tak očekávaná budoucí swapová sazba rovna současné swapové sazbě. Pro libovolný cenný papír $f$ tedy platí
\begin{equation*}
f_0 = A(0)E_A\Bigg[\frac{f_T}{A(T)} \Bigg]
\end{equation*}

\section{Rozšíření o vícero nezávislých faktorů}

Předpokládejme, že procesy, které sledují $f$ a $g$, jsou
\begin{equation*}
df = rf\ dt + \sum^n_{i=1}\sigma_{f,i}fdz_i
\end{equation*}
\begin{equation*}
dg = rg\ dt + \sum^n_{i=1}\sigma_{g,i}fdz_i
\end{equation*}
Podobně jako v případě jednoho rizikového faktoru, můžeme definovat jiné světy, které jsou interně konzistení, pomocí
\begin{equation*}
df = \Bigg( r + \sum^n_{i=1}\lambda_i \sigma_{f,i} \Bigg)f\ dt+ \sum^n_{i=1}\sigma_{f,i}f\ dz_i
\end{equation*}
\begin{equation*}
dg = \Bigg( r + \sum^n_{i=1}\lambda_i \sigma_{g,i} \Bigg)g\ dt + \sum^n_{i=1}\sigma_{g,i}g\ dz_i
\end{equation*}
Jeden z těchto světů je svět reálný. Dále definujme svět, který je forwardově rizikově neutrální k $g$, jako svět, kde $\lambda_i = \sigma_{g,i}$. S využitím It\^o lemmy a toho, že $dz_i$ jsou vzájemně nekorelované, lze dokázat, že $f/g$ je martingalem.

\section{Aplikace}

\subsection{Modifikace Black-Scholes modelu}

V následující kapitole si ukážeme rozšíření Black-Scholes modelu za předpokladu, že úrokové sazby jsou stochastické. Uvažujme evropskou kupní opci se splatností v čase $T$, kde je podkladovým aktivem akcie, která negeneruje žádný dividendový výnos. Hodnota této opce je dána vztahem
\begin{equation*}
c = P(0,T)E_T[\max(S_T - K,0)]
\end{equation*}
Definujme $R$ jako zero sazbu pro splatnost $T$. Platí tedy
\begin{equation*}
c = e^{-RT}E_T[\max(S_T-K,0)]
\end{equation*}
Jestliže má $S_T$ lognormální rozdělení v námi uvažovaném forwardově neutrálním světě a směrodatná odchylka $\ln(S_T)$ je rovna $s$, pak v souladu se závěry kapitoly 10.3 platí
\begin{equation*}
E_T[\max(S_T-K,0)] = E_T(S_T)N(d_1)-KN(d_2)
\end{equation*}
kde
\begin{equation*}
d_1 = \frac{\ln[E_T(S_T)/K]+s^2/2}{s}
\end{equation*}
\begin{equation*}
d_2 = \frac{\ln[E_T(S_T)/K]-s^2/2}{s}
\end{equation*}
$E_T(S_T)$ je forwardová cena akcie pro kontrakt s maturitou v čase $T$. Při neexistenci arbitráže tedy platí
\begin{equation*}
E_T(S_T) = S_0e^{RT}
\end{equation*}
Hodnota výše uvažované opce je pro $s = \sigma \sqrt{T}$ tedy rovna
\begin{equation*}
c = S_0N(d_1)-Ke^{-RT}N(d_2)
\end{equation*}
kde
\begin{equation*}
d_1 = \frac{\ln(S_0/K)+(R+\sigma^2/2)/T}{s}
\end{equation*}
\begin{equation*}
d_2 = \frac{\ln(S_0/K)+(R-\sigma^2/2)/T}{s}
\end{equation*}
Jediným rozdílem oproti standardnímu Black-Scholes modelu je nahrazení bezrizikové úrokové sazby $r$ zero sazbou $R$.

\subsection{Opce s výměnnou jednoho aktiva za druhé}

Uvažujme opci, která nám umožňuje vyměnit aktivum s hodnotou $U$ za aktivum s hodnotou $V$. Předpokládejme, že žádné z těchto aktiv negeneruje výnos.

Pro martingale míru zvolme jako jmenovatele aktivum $g$ s hodnotou $U$ a jako čitatele aktivum $f$ s hodnotou $V$. Z rovnice (18.10) vyplývá
\begin{equation*}
V_0 = U_0E_U \bigg( \frac{V_T}{U_T} \bigg)
\end{equation*}
kde $E_U$ představuje očekávání ve světě, který je forwardově rizikově neutrální s ohledem k $U$.

Jestliže definujeme $f$ v rovnici (18.10) jako $f_T = \max(V_T - U_T,0)$, platí
\begin{equation}
f_0 = U_0 E_U \Bigg[ \frac{\max(V_T - U_T,0)}{U_T} \Bigg]
\end{equation}
neboli
\begin{equation*}
f_0 = U_0 E_U \Bigg[ \max \Bigg( \frac{V_T}{U_T} -1 ,0 \Bigg) \Bigg]
\end{equation*}
Volatilita proměnné $V/U$ je definována jako
\begin{equation*}
\hat{\sigma}^2 = \sigma^2_U + \sigma^2_V - 2 \rho \sigma_U \sigma_V
\end{equation*}
Pomocí Black-Scholes modelu lze $f_0$ vyjádřit jako
\begin{equation}
f_0 = U_0 \Bigg[ E_U \Bigg( \frac{V_T}{U_T} \Bigg) N(d_1) - N(d_2) \Bigg]
\end{equation}
kde
\begin{equation*}
d_1 = \frac{\ln(V_0/U_0)+ \hat{\sigma}^2T/2}{\hat{\sigma} \sqrt{T}}
\end{equation*}
\begin{equation*}
d_2 = d_1 - \hat{\sigma} \sqrt{T}
\end{equation*}
Rovnici (18.12) lze s využitím (18.11) dále upravit do tvaru
\begin{equation*}
f_0 = V_0 N(d_1) - U_0 N(d_2)
\end{equation*}
Jestliže aktiva $f$ a $g$ generují výnos $q_f$ a $q_g$, změní se výše uvedená rovnice do tvaru
\begin{equation*}
f_0 = e^{-q_V T} V_0 N(d_1) - e^{-q_U T}U_0N(d_2)
\end{equation*}
kde
\begin{equation*}
d_1 = \frac{\ln(V_0/U_0)+(q_U - q_V + \hat{\sigma}^2/2)T}{\hat{\sigma}\sqrt{T}}
\end{equation*}
\begin{equation*}
d_2 = d_1 - \hat{\sigma} \sqrt{T}
\end{equation*}

\section{Změna forwardově rizikového světa}

Ve světě, který je forwardově rizikově neutrální s ohledem na finanční instrument $g$, sleduje cena aktiva $f$ proces
\begin{equation*}
df = \bigg( r + \sum^n_{i=1} \sigma_{g,i} \sigma_{f,i} \bigg)fdt + \sum^n_{i=1} \sigma_{f,i}fdz_i
\end{equation*}
Podobně ve světě, který je rizikově neutrální k jinému aktivu $h$, platí pro proces sledovaný aktivem $f$ rovnice
\begin{equation*}
df = \bigg( r + \sum^n_{i=1} \sigma_{h,i} \sigma_{f,i} \bigg)fdt + \sum^n_{i=1} \sigma_{f,i}fdz_i
\end{equation*}
Jestliže se přesuneme ze světa, který je rizikově neutrální k $g$, do světa, který je rizikově neutrální k $h$, je změna očekávaného růstu ceny aktiva $f$ dána vztahem
\begin{equation*}
\sum^n_{i=1}(\sigma_{h,i}-\sigma_{g,i})\sigma_{f,i}
\end{equation*}
Uvažujme další proměnnou $v$, která je funkcí ceny obchodovaných finančních instrumentů. Definujme $\sigma_{v,i}$ jako $i$-tou komponentu volatility $v$. S využitím lemy A je možné určit, co se stane s procesem proměnné $v$, jestliže dojde ke změně forwardově rizikového světa. Ta má, jak již bylo zmíněno, za následek změnu očekávaného růstu ceny pokladového aktiva. Lze dokázat, že očekávaný růst $v$ reaguje na tyto změny stejně jako očekávaný růst cen obchodovaných aktiv, jejichž je $v$ funkcí. Tato změna je tedy rovna
\begin{equation*}
\alpha_v = \sum^n_{i=1}(\sigma_{h,i} - \sigma_{g,i})\sigma_{v,i}
\end{equation*}

Definujme $w = h/g$ a $\sigma_{w,i}$ jako $i$-tou komponentu volatility proměnné $w$. Z It\^o lemy vyplývá
\begin{equation*}
\sigma_{w,i} = \sigma_{h,i} - \sigma_{g,i}
\end{equation*}
resp.
\begin{equation}
\alpha_v = \sum^{n}_{i=1} \sigma_{w,i}\sigma_{v,i} = \rho \sigma_v \sigma_w
\end{equation}
kde $\rho$ je korelace mezi $v$ a $w$. Parametr $\alpha_v$ odpovídá změně očekávané míry růstu proměnné $v$ v důsledku změny forwardově rizikového světa. Proměnná $w$ je tzv. poměrový ukazatel. Změna očekávané míry růstu proměnné $v$ je tedy rovna kovarianci mezi procentní změnou $v$ a procentní změnou poměrového ukazatele $w$.

\section{Kvanta}

Kvantum je měnový derivát, který zahrnuje dvě měny. Výplata z tohoto derivátu je vázána na vývoj jedné měny, avšak samotná výplata je realizována ve druhé z měn.

Uvažujme kvantum, které generuje výplatu v měně $X$ v čase $T$. Předpokládejme, že výplata závisí na hodnotě aktiva $V$ vyjádřené v čase $T$ v měně $Y$. Definujme
\begin{center}
\begin{tabular}{l l}
$F(t)$ & forwardová cena aktiva $V$ v čase $t$ denominovaná v měně $Y$ pro\\
 & účely kontraktu, který maturuje v čase $T$\\
$V_T$ & cena $V$ v čase $T$\\
$P_X(t,T)$ & hodnota diskontního dluhopisu v čase $t$ denominovaná v měně $X$;\\
 & nominální hodnota dluhopisu je 1 $X$ a dluhopis je splatný v čase $T$\\
$P_Y(t,T)$ & hodnota diskontního dluhopisu v čase $t$ denominovaná v měně $Y$;\\
 & nominální hodnota dluhopisu je 1 $Y$ a dluhopis je splatný v čase $T$\\
$E_X[\ast]$ & očekávání v čase nula ve světě, který je forwardově rizikově neutrální\\
 & vzhledem k $P_X(t,T)$\\
$E_Y[\ast]$ & očekávání v čase nula ve světě, který je forwardově rizikově neutrální\\
 & vzhledem k $P_Y(t,T)$\\
$G(t)$ & forwardový měnový kurz v čase $t$ (počet jednotek $Y$ získaných výměnou\\
 & za jednu jednotku $X$) v rámci forwardového kontraktu splatného v čase $T$\\
$\sigma_F$ & volatilita $F(t)$\\
$\sigma_G$ & volatilita $G(t)$\\
$\rho$ & korelace mezi $F(t)$ a $G(t)$\\
$S_T$ & spotový měnový kurz v čase $T$ ($S_T = G(T)$)
\end{tabular}
\end{center}
Víme, že platí
\begin{equation*}
E_Y[V_T] = F_0
\end{equation*}
a chceme znát $E_X[V_T]$. Jestliže zaměníme svět $P_Y(t,T)$ za $P_X(t,T)$, je poměrový ukazatel roven
\begin{equation*}
G(t) = \frac{P_X(t,T)}{P_Y(t,T)}
\end{equation*}
Z (18.13) vyplývá, že změna forwardově rizikově neutrálního světa má za následek změnu očekávané míry růstu $F(t)$ o
\begin{equation*}
\rho  \sigma_F \sigma_G
\end{equation*}
To znamená, že přibližně platí
\begin{equation*}
E_X[F(T)] = E_Y[F(T)]e^{\rho  \sigma_F \sigma_G T}
\end{equation*}
nebo-li také
\begin{equation*}
E_X[V_T] = F(0)e^{\rho  \sigma_F \sigma_G T}
\end{equation*}
protože $V_T = F(T)$ a $E_T[V_T]=F(0)$. Tento vztah je pak přibližně roven
\begin{equation*}
E_X[V_T]=F(0)(1 + \rho  \sigma_F \sigma_G T)
\end{equation*}

\subsection{Tradiční rizikově neutrální svět}

Koncept forwardově rizikově neutrálního světa, který jsme až dosud používali, je vhodný pro situace, kdy je výplata realizována pouze v jeden daný časový okamžik. V ostatních případech je lepší použít tradiční rizikově neutrální svět.

Uvažujme proces, který sleduje proměnná $V$ v rizikově neutrálním světě s ohledem na měnu $Y$. Předpokládejme, že chceme odhadnout odpovídající proces v rizikově neutrálním světě s ohledem na měnu $X$. Definujme
\begin{center}
\begin{tabular}{l l}
$S$ & spotový měnový kurz (počet jednotek $Y$ získaných výměnou za jednu\\
 & jednotku $X$)\\
$\sigma_S$ & volatilita $S$\\
$\sigma_V$ & volatilita $V$\\
$\rho$ & korelace mezi $S$ a $V$\\
\end{tabular}
\end{center}
Ve výše popsaném příkladě dochází ke změně z peněžního trhu měny $Y$ na peněžní trh měny $X$ (v obou případech jsou odpovídající veličiny vyjádřeny v měně $X$). Podle It\^o lemy je možné dokázat, že poměrový ukazatel je roven $\sigma_S$. Změna očekávané míry růstu $V$ je proto rovna
\begin{equation*}
\rho \sigma_V \sigma_S
\end{equation*}
Tržní cena rizika se tedy zvýší z nuly na $\rho \sigma_V \sigma_S$.

\section{Siegelův paradox}

Uvažujme dvě měny $X$ a $Y$ a forwardově rizikově neutrální svět vzhledem k $Y$. Definujme spotový měnový kurz $S$ jako počet jednotek $Y$, které je možné získat výměnou za jednu jednotku $X$. Protože je na měnový kurz možné nahlížet podobně jako na akcii nesoucí dividendový výnos odpovídající bezrizikové úrokové sazbě, sleduje $S$ v rizikově neutrálním světě vzhledem k $Y$ proces
\begin{equation*}
dS = (r_Y - r_X)S dt + \sigma_S S dz
\end{equation*}
kde $r_X$ a $r_Y$ jsou úrokové míry pro měny $X$ a $Y$. Z It\^o lemy vyplývá, že $1/S$ sleduje proces
\begin{equation}
d(1/S) = (r_x - r_Y + \sigma^2_S)(1/S)dt - \sigma_S(1/S)dz
\end{equation}
Vzhledem k tomu, že očekávaná míra růstu $S$ je $r_Y - r_X$, intuice napovídá, že by míra růstu $1/S$ měla být $r_X - r_Y$. Podle (18.14) je však tato míra rovna $r_X - r_Y + \sigma^2_S$.

Abychom porozumněli tomuto paradoxu, je třeba si uvědomit, že rovnice (18.14) je platná ve světě, který je rizikově neutrální vzhledem k měně $Y$. V předchozí kapitole jsme ukázali, že přechod z $S$ na $1/S$\footnote{Tato změna je ekvivalentní změně podkladové měně.} a s tím související změna světů má za následek změnu očekávané míry růstu o $\rho \sigma_V \sigma_S$. V našem případě platí $V = 1/S$, $\sigma_V = \sigma_S$ a $\rho = -1$. Tato změna je tedy po úpravách rovna $-\sigma^2_S$. Rovnice (18.14) se tedy v rizikově neutrálním světě vzhledem k $X$ změní na
\begin{equation*}
d(1/S) = (r_x - r_Y)(1/S)dt - \sigma_S(1/S)dz
\end{equation*}

\section{Lema 18A - Zobecnění It\^o lemy} 

\subsection{Varianta 1}

Uvažujme funkci $f$ proměnných $x_1$, $x_2$, ..., $x_n$ a času $t$. Předpokládejme, že $x_i$ sleduje It\^o proces s trendem $a_i$ a volatilitou $b^2_i$, tj.
\begin{equation}
d x_i = a_i dt + b_i d z_i
\end{equation}
kde $d z_i$ představuje Wienerův proces. Každé $a_i$ a $b_i$ může být funkcí všech $x_i$ a $t$. Taylorovým rozvojem $f$ získáváme
\begin{equation}
\delta f = \sum^n_{i=1}\frac{\partial f}{\partial x_i} \delta x_i + \frac{\partial f}{\partial t} \delta t + \frac{1}{2} \sum^n_{i = 1} \sum^n_{j = 1} \frac{\partial^2 f}{\partial x_i \partial x_j} \delta x_i \delta x_j + \frac{1}{2} \sum^n_{i=1} \frac{\partial^2 f}{\partial x_i \partial t} \delta x_i \delta t + ...
\end{equation}
Diskretizací rovnice (18.15) získáme
\begin{equation*}
\delta x_i = a_i \delta t + b_i \epsilon_i \sqrt{\delta t}
\end{equation*}
Při odvození základní formy It\^o lemy v kapitole 9 jsme dokázali, že
\begin{equation*}
\lim \limits_{\delta t \to 0} \delta x^2_i = b^2_i dt 
\end{equation*}
Podobně platí
\begin{equation}
\lim \limits_{\delta t \to 0} \delta x_i \delta x_j = b_i b_j \rho_{ij} dt
\end{equation}
kde $\rho_{ij}$ představuje korelaci mezi $d z_i$ a $d z_j$. S tím, jak se $\delta t$ blíží limitně nule, jsou první tři členy (18.16) řádu $\delta t$, ostatní členy jsou vyššího řádu.
\begin{equation*}
d f = \sum^n_{i=1} \frac{\partial f}{\partial x_i} d x_i + \frac{\partial f}{\partial t} dt + \frac{1}{2} \sum^n_{i=1} \sum^n_{j=1}\frac{\partial^2 f}{\partial x_i \partial x_j}b_i b_j \rho_{ij}dt
\end{equation*}
Substitucí podle (18.15) a (18.17) získáme zobecněnou It\^o lemu.
\begin{equation*}
d f = \bigg( \sum^n_{i=1}\frac{\partial f}{\partial x_i} a_i + \frac{\partial f}{\partial t} + \frac{1}{2} \sum^n_{i=1} \sum^n_{j=1}\frac{\partial^2 f}{\partial x_i \partial x_j}b_i b_j \rho_{ij} \bigg) dt + \sum^n_{i=1}\frac{\partial f}{\partial x_i} b_i d z_i
\end{equation*}

\subsection{Varianta 2}

Alternativní přístup pro zobecnění It\^o lemy předpokládá, že $f$ je závislé na jedné proměnné $x$, jejíž proces obsahuje vícero Wienerových procesů.
\begin{equation*}
dx = a dt + \sum^{m}_{i=1}b_idz_i
\end{equation*}
V tomto případě platí
\begin{equation*}
\delta f = \frac{f}{x}\delta x + \frac{f}{t}\delta t + \frac{1}{2}\frac{\partial^2f}{\partial x^2f}\delta x^2 + \frac{1}{2}\frac{\partial^2f}{\partial x \partial t}\delta x \delta t + ...
\end{equation*}
\begin{equation*}
\delta x = a \delta t + \sum^{m}_{i=1}b_i\epsilon_i \sqrt{\delta t}
\end{equation*}
\begin{equation*}
\lim_{\delta t \to 0} \delta x^2_i = \sum^{m}_{i=1} \sum^{m}_{j=1}b_i b_j \rho_{ij}dt
\end{equation*}
kde $\rho_{ij}$ představuje korelaci mezi $dz_i$ a $dz_j$. Kombinací výše uvedených tří vztahů tak získáváme
\begin{equation*}
df = \Bigg( \frac{\partial f}{\partial x}a + \frac{\partial f}{\partial t} + \frac{1}{2}\frac{\partial^2 f}{\partial x^2} \sum^m_{i=1}\sum^m_{j=1}b_ib_j\rho_{ij} \Bigg)dt + \frac{\partial f}{\partial x}\sum^{m}_{i=1}b_idz_i
\end{equation*}

\subsection{Varianta 3}

Nejobecnější variantou zobecnění It\^o lemy představuje situace, kdy $f$ je funkcí $n$ proměnných $x_i$, z nichž každá sleduje proces
\begin{equation*}
d x_i = a_i \delta t + \sum^{m}_{k=1}b_{ik}dz_k
\end{equation*}
Podobným způsobem jako v předchozím případě lze odvodit
\begin{equation*}
df = \Bigg( \sum^n_{i=1} \frac{\partial f}{\partial x_i}a_i + \frac{\partial f}{\partial t} + \frac{1}{2}\sum^n_{i=1}\sum^n_{j=1}\frac{\partial^2 f}{\partial x_i \partial dx_j} \sum^m_{k=1}\sum^m_{l=1}b_{ik}b_{jl}\rho_{kl} \Bigg)dt + \sum^n_{i=1}\frac{\partial f}{\partial x_i}\sum^{m}_{k=1}b_{ik}dz_k
\end{equation*}

\section{Lema 18B - Vícenásobné zdroje nejistoty}

Uvažujme $n$ stochastických proměnných, které sledují Wienerův proces. Dále uvažujme $n+1$ obchodovaných aktiv, jejichž ceny závisí na některých popř. všech těchto $n$ stochastických proměnných. Definujme $f_j$ jako cenu $j$-tého aktiva. Předpokládejme, že tato aktiva negenerují žádný výnos ve formě dividend, [roků popř. kupónů. Cena aktiva $f_j$ pak dle lemy 18A sleduje proces
\begin{equation}
d f_j = \mu_j f_j dt + \sum^n_{i=1} \sigma_{ij}f_j d z_i
\end{equation}
Protože existuje $n+1$ obchodovaných aktiv a $n$ Wienerových procesů, je možné vytvořit bezrizikové portfolio $\Pi$
\begin{equation*}
\Pi = \sum^{n+1}_{j=1}k_j f_j
\end{equation*}
kde $k_j$ představuje váhu $j$-tého aktiva v portfoliu. Parametr $k_j$ musí být zvolen tak, aby se stochastické komponenty výnosové míry vzájemně vyrušily. S ohledem na (18.18) to znamená, že musí platit
\begin{equation}
\sum^{n+1}_{j=1} k_j \sigma_{ij} f_j = 0
\end{equation}
Výnos generovaný portfoliem je dán
\begin{equation*}
d \Pi = \sum^{n+1}_{j=1} k_j \mu_j f_j dt
\end{equation*}
a náklady na jeho pořízení jsou
\begin{equation*}
\sum^{n+1}_{j=1} k_j f_j
\end{equation*}
Za předpokladu nulové možnosti arbitráže proto musí platit
\begin{equation*}
\sum^{n+1}_{j=1}k_j \mu_j f_j = r \sum^{n+1}_{j=1} k_j f_j
\end{equation*}
neboli
\begin{equation}
\sum^{n+1}_{j=1} k_j f_j (\mu_j - r) = 0
\end{equation}
Na rovnice (18.19) a (18.20) je možné pohlížet jako na $n+1$ homogenních lineárních rovnic. Aby existovalo řešení těchto rovnic, musí platit
\begin{equation}
f_j(\mu_j - r) = \sum^{n}_{i=1} \lambda_i \sigma_{ij} f_j
\end{equation}
což je ekvivalentní
\begin{equation}
\mu_j - r = \sum^n_{i=1} \lambda_i \sigma_{ij}
\end{equation}
Po vypuštění indexu $j$ je zřejmé, že pro libovolné aktivum $f$ závislé na $n$ stochastických proměnných, platí
\begin{equation}
df = \mu f dt + \sum^{n}_{i=1} \sigma_i f d z_i
\end{equation}
kde
\begin{equation}
\mu - r = \sum^n_{i=1} \lambda_i \sigma_i
\end{equation}

\chapter{Úrokové deriváty - standardní tržní modely}

Oceňování úrokových derivatů je složitější, než ocenění měnových nebo akciových derivátů. Jedním z důvodů je to, že vývoj úrokových sazeb není dán stochastickým procesem ale řadou externích faktorů jako například měnovou politikou centrální banky. V případě, že bychom chtěli přesto na úrokovou sazbu pohlížet jako na náhodnou veličinu, je třeba modelově popsat celou zero křivku. Volatilita v různých bodech této křivky přitom může být různá. Úrokové sazby získané z tohoto modelu jsou navíc použity nejen pro diskontování ale také pro definování výplaty derivátu.

\section{Rozšíření Black-Scholes modelu}

Uvažujme evropskou kupní opci na proměnnou, jejíž hodnota je $V$. Definujme
\begin{center}
\begin{tabular}{l l}
$T$ & zbytková splatnost opce\\
$F$ & forwardová cena $V$ v čase $T$\\
$F_0$ & hodnota $F$ v čase $T_0 = 0$\\
$K$ & realizační cena opce\\
$P(t,T)$ & cena diskontního dluhopisu v čase $t$ s výplatou 1 USD v čase $T$\\
$V_T$ & hodnota $V$ v čase $T$\\
$\sigma$ & volatilita náhodné veličiny $F$\\
\end{tabular}
\end{center}
Pomocí Black-Scholes modelu lze vypočítat výplatu opce za předpokladu, že:
\begin{itemize}
\item $V_T$ má lognormální rozdělení se standardní směrodatnou odchylkou $\ln V_T$ rovnou $\sigma \sqrt{T}$
\item očekávaná hodnota $V_T$ je rovna $F_0$
\end{itemize}
Očekávanou výplatu je nutné diskontovat $T$-roční bezrizikovou sazbou tak, že ji vynásobíme $P(0,T)$. Výplata z dané opce očekávaná v čase $T$ je rovna $\max(V_T - K, 0)$. Při odvozování Black-Scholes modelu jsme dokázali, že očekávaná výplata je za předpokladu lognormální rozdělení dána rovnicí
\begin{equation*}
E[V_T]N(d_1) - KN(d_2)
\end{equation*}
kde $E[V_T]$ je očekávaná hodnota $V_T$ a
\begin{equation*}
d_1 = \frac{\ln(E[V_T]/K) + \sigma^2T/2}{\sigma \sqrt{T}}
\end{equation*}
\begin{equation*}
d_2 = \frac{\ln(E[V_T]/K)-\sigma^2T/2}{\sigma \sqrt{T}}= d_1 - \sigma \sqrt{T}
\end{equation*}
Protože předpokládáme $E[V_T] = F_0$ a diskontujeme bezrizikovou úrokovou sazbou, je hodnota opce rovna
\begin{equation}
c = P(0,T)[F_0N(d_1)-KN(d_2)]
\end{equation}
kde
\begin{equation*}
d_1 = \frac{\ln(F_0/K)+\sigma^2T/2}{\sigma \sqrt{T}}
\end{equation*}
\begin{equation*}
d_2 = \frac{\ln(F_0/K)-\sigma^2T/2}{\sigma \sqrt{T}} = d_1 - \sigma \sqrt{T}
\end{equation*}
Podobně cena odpovídající put opce je dána
\begin{equation}
p = P(0,T)[KN(-d_2)-F_0N(-d_1)]
\end{equation}
Jediný rozdíl mezi výše odvozenými rovnicemi a rovnicemi (10.14) a (10.15) je ten, že očekávaná cena proměnné $V$ je zde vyjádřena jako forwardová cena narozdíl od futures ceny.

Dále je možné rozšířit Black-Scholes model o situace, kdy je výplata vypočtena z hodnoty proměnné $V$ v čase $T$, ale samotná výplata je realizována v pozdějším čase $T^*$.
\begin{equation*}
c = P(0,T^*)[F_0N(d_1)-KN(d_2)]
\end{equation*}
\begin{equation*}
p = P(0,T^*)[KN(-d_2)-F_0N(-d_1)]
\end{equation*}
Black-Scholes model nepředpokládá, že $V$ nebo $F$ jsou popsány Brownovým pohybem. Jediné, co se vyžaduje, je aby $V_T$ sledovalo lognormální rozdělení v čase $T$. Parametr $\sigma$ je nazýván volatilitou proměnné $F$ nebo také forwardovou volatilitou $V$. Jedinou úlohou tohoto parametru je definovat směrovanou odchylku $\ln V_T$ jako $\sigma \sqrt{T}$. Parametr $\sigma$ nám tak neříká nic o volatilitě $\ln V$ v jiném čase než $T$. 

\subsection{Teoretické zdůvodnění modelu}

Aplikace Black-Scholes modelu předpokládá konstantní popř. deterministické úrokové sazby. V tomto případě, jak bylo vysvětleno v kapitole 3, je forwardová cena rovna futures ceně. Odvozené rovnice tak odpovídají rovnicím (10.14) a (10.15). Jestliže jsou však úrokové sazby stochastické povahy, nabízejí se následující otázky
\begin{itemize}
\item Proč bychom měli předpokládat, že $E[V_T]$ je rovno forwardové ceně $F_0$? Forwardová cena přeci není totéž jako futures cena.
\item Proč při diskontování očekávané výplaty ignorujeme, že úrokové sazby mohou být stochastické?
\end{itemize}
V následujícím textu ukážeme s využítím poznatků z kapitoly 18.2.2, že výše odvozené rovnice nejsou pouhou aproximací rovnic (10.14) a (10.15) a že je možné předpokládat $E[V_T] = F_0$, jestliže současně diskontujeme budoucí cash-flow pomocí zero křivky.

\section{Opce na dluhopisy}

Opce na dluhopisy představují pro svého majitele právo koupit nebo prodat určitý bond k určitému datu za předem dohodnutou cenu.

\subsection{Evropské dluhopisové opce}

V případě opce na dluhopis předpokládáme, že cena dluhopisu má v době splatnosti lognormální rozdělení. Na základě tohoto předpokladu je pak možné použít rovnice (19.1) popř. (19.2) pro ocenění opce. Proměnná $\sigma$ je pak definována jako směrodatná odchylka logaritmu ceny dluhopisu v době splatnosti opce. $F_0$ je možné vypočítat na základě vztahu
\begin{equation*}
F_0 = \frac{B_0 - I}{P(0,T)}
\end{equation*}
kde $B_0$ je cena dluhopisu v čase nula a $I$ je současná hodnota kupónů, které budou vyplaceny po dobu životnosti opce. Pro účely výpočtu forwardové ceny $F_0$ zahrnuje spotová cena $B_0$ naběhlý úrok stejně jako realizační cena $K$ v rovnicích (19.1) a (19.2).

Následující graf zobrazuje vývoj směrodatné odchylky logaritmu ceny dluhopisu v čase.
\begin{center}
	\begin{pspicture}(0,0)(7,4)
		\rput(3.5,0.0){Směrodatná odchylka logaritmu ceny dluhopisu v čase}

                \psline[arrows=->](0.5,1.0)(6.5,1.0)
                \psline[arrows=->](0.5,1.0)(0.5,3.5)

                \rput(6.5,0.7){\small{čas}}
                \rput(2.2,3.5){\small{směrodatná odchylka}}

                \pscurve[linewidth=0.5mm](0.5,1.0)(2.0,2.5)(5.0,1.0)

                \rput(5.0,0.7){\tiny{splatnost dluhopisu}}
                
	\end{pspicture}
\end{center}
Vzhledem k tomu, že je současná cena dluhopisu známá, vykazuje její logaritmus nulovou směrodatnou odchylku. Také v době splatnosti dluhopisu je tato směrodatná odchylka nulová, protože cena dluhopisu je rovněž známa a je rovna jeho nominální hodnotě. Mezi těmito dvěma časovými okamžiky směrodatná odchylka nejprve roste a po té klesá. Volatilita $\sigma$, která by měla být použita pro ocenění evropské opce, je
\begin{equation*}
\sigma = \frac{\sigma_m}{\sqrt{T}}
\end{equation*}
kde $\sigma_m$ je směrodatná odchylka logaritmu ceny dluhopisu v době splatnosti opce a $T$ je zbytková splatnost uvažované opce.

\subsection{Vnořené opce na dluhopis}

V zásadě existují dva druhy dluhopisů s tzv. vnořenou opcí, která umožňuje zpětný odkup nebo prodej dluhopisu před jeho splatností. V případě, že emitent má právo na zpětný odkup dluhopisu za předem stanovenou cenu, hovoříme o tzv. svolatelném dluhopisu. Jestliže naopak investor má právo na prodej dluhopisu emitentovi za předem určenou cenu, hovoříme o tzv. vratném dluhopisu. Cena odpovídající opce je pak zpravidla kotována ve formě počtu bazických bodů, o které se zvýší (svolatelný dluhopis) popř. sníží (vratný dluhopis) výnos podkladového dluhopisu.
Vedle dluhopisů mohou vnořenou opci obsahovat také vklady a půjčky klientů bank, přičemž princip je shodný jako u dluhopisů.

\subsection{Volatilita výnosové míry}

Volatility, které jsou kotovány pro dluhopisové opce, jsou velice často volatilitami výnosových měr spíše než volatilitami ceny dluhopisu. Pro výpočet cenové volatility z volatilit výnosových měr se pak používá koncept durace. Změnu forwardové ceny dluhopisu na změnu forwardových výnosových měr lze vyjádřit vztahem
\begin{equation*}
\frac{\delta F}{F} \approx -D \delta y_F
\end{equation*}
resp.
\begin{equation*}
\frac{\delta F}{F} \approx -D y_F \frac{\delta y_F}{y_F}
\end{equation*}
 Volatilita proměnné je vyjádřena pomocí směrodatné odchylky její procentní změny. Volatilita ceny dluhopisu tak může být z volatility výnosové míry aproximována pomocí
\begin{equation*}
\sigma = D y_0 \sigma_y
\end{equation*}
kde $y_0$ je počáteční hodnota $y_F$.

\subsection{Teoretické zdůvodnění modelu}

Jednou z možností, které jsme uvažovali v předchozí kapitole, byl svět forwardově rizikově neutrální s ohledem na diskontní dluhopis se splatností v čase $T$. To znamená, že současná hodnota jakéhokoliv finančního instrumentu je rovna jeho očekávané hodnotě v čase $T$ přenásobené současnou hodnotou diskontního dluhopisu se splatností v čase $T$. Očekávaná cena libovolného finančního instrumentu je tak rovna jeho forwardové ceně. Cena kupní dluhopisové opce se splatností v čase $T$ je
\begin{equation}
c = P(0,T)E_T[\max(B_T - K, 0)]
\end{equation}
Za předpokladu, že cena dluhopisu má lognormální rozdělení a standardní směrodatnou odchylku $\sigma \sqrt{T}$, lze (19.3) vyjádřit jako
\begin{equation*}
c = P(0,T)[E_T(B_T)N(d_1)-KN(d_2)]
\end{equation*}
kde
\begin{equation*}
d_1 = \frac{\ln[E_T[B_T]/K]+\sigma^2T/2}{\sigma \sqrt{T}}
\end{equation*}
\begin{equation*}
d_2 = \frac{\ln[E_T[B_T]/K]-\sigma^2T/2}{\sigma \sqrt{T}} = d_1 - \sigma \sqrt{T}
\end{equation*}
Protože pro očekávanou cenu uvažovaného dluhopisu platí $E_T(B_T) = F_0$, lze rovnici (19.3) dále upravit do tvaru
\begin{equation*}
c = P(0,T)[F_0N(d_1)-KN(d_2)]
\end{equation*}
která je shodná s rovnicí (19.1).

\section{Úrokový cap, floor a collar}

\subsection{Úrokový cap}

Úrokový cap zaručuje, že úroková sazba nepřesáhne určitou předem stanovenou úroveň. Poskytuje tak ochranu před růstem úrokových sazeb.

Uvažujme úvěr, kdy výše splátky je v pravidelných intervalech určena podle předem dohodnuté floatové úrokové sazby. Úrok je tak vypočten na základě aktuální referenční sazby. V případě úrokového capu je však navíc dohodnut tzv. úrokový strop. Jestliže referenční sazba přesáhne tento strop, je úrok vypočten na základě stropové sazby; v opačném případě je pro výpočet rozhodující referenční sazba.\\

Uvažujme cap s životností $T$, jistinou $L$ a úrokovým stropem $R_K$. Předpokládejme, že k přehodnocení úrokových sazeb dochází v čase $t_1$, $t_2$, ..., $t_n$ (tzv. nulovací období). Dále definujme $t_{n+1}$ jako $T$ a $R_k$ jako referenční sazbu platnou pro $t_k$ až $t_{k+1}$ známou v čase $t_k$. Výplata z daného capu v čase $t_{k+1}$, kde $k = 1, 2, ..., n$, je dána vztahem
\begin{equation}
L \delta_k \max(R_k - R_K,0)
\end{equation}
kde $\delta_k = t_{k+1} - t_k$. Perioda úročení sazeb $R_k$ a $R_K$ odpovídá periodě $\delta_k$. Výše uvedený vzorec představuje výplatu prodejní opce na refereční sazbu. Cap lze pak chápat jako portfolio $n$ takovýchto opcí. Jednotlivé prodejní opce jsou pak označovány jako caplety.

Cap lze také chápat jako portfolio prodejních bondových opcí. Výplatu z capu v čase $t_{k+1}$ je totiž možné vyjádřit také jako
\begin{equation*}
\frac{L \delta_k}{1+R_k \delta_k} \max(R_k - R_K, 0)
\end{equation*}
Tento vztah lze elementárními úpravami převést do tvaru
\begin{equation*}
\max \Big( L - \frac{L(1 + R_K \delta_k)}{1 + \delta_k R_K}, 0 \Big)
\end{equation*}
Výraz $\frac{L(1 + R_K \delta_k)}{1 + \delta_k R_k}$ představuje hodnotu diskontního dluhopisu v čase $t_k$ za předpokladu, že tento dluhopis generuje v čase $t_{k+1}$ výplatu $L(1+R_K \delta_k)$. Výše uvedený vztah tak lze interpretovat jako výplatu z prodejní opce na diskontní dluhopis se splatností v čase $t_k$ za předpokladu, že nominální hodnota dluhopisu je $L(1 + R_K \delta_k)$ a realizační cena je rovna $L$.\\

Caplet odpovídající sazbě platné v čase $t_k$ generuje v čase $t_{k+1}$ výplatu danou rovnicí (19.4). Jestliže je sazba $R_k$ dána lognormálním rozdělením se směrodatnou odchylkou $\sigma_k$, lze současnou hodnotu tohoto capletu vyjádřit jako
\begin{equation}
L \delta_k P(0,t_{k+1})[F_kN(d_1)-R_KN(d_2)]
\end{equation}
kde $F_k$ představuje forwardovou sazbu pro periodu $t_k$ až $t_{k+1}$ a
\begin{equation*}
d_1 = \frac{\ln(F_k/R_k)+\sigma_k^2 t_k/2}{\sigma_k \sqrt{t_k}}
\end{equation*}
\begin{equation*}
d_2 = \frac{\ln(F_k/R_k)-\sigma_k^2 t_k/2}{\sigma_k \sqrt{t_k}}=d_1 - \sigma_k \sqrt{t_k}
\end{equation*}
Cena celého capu je pak dána součtem jednotlivých capletů. Při výpočtu ceny capletu je také možné používat odlišnou volatilitu s ohledem na různé časové období, která tyto caplety pokrývají. V tomto případě hovoříme o tzv. spotových volatilitách. Alternativou k tomuto přístupu je pak použít pro všechny caplety stejnou volatilitu, avšak tu rozlišit na úrovni capu podle jeho životnosti. Tyto volatility jsou označovány jako flat volatility. Implikované volatility kotované na trhu jsou pak zpravidla flat volatility. Pokladové instrumenty, z nichž jsou tyto implikované volatility počítány, jsou zpravidla at-the-money\footnote{To znamená, že limitní sazba je rovna swapové sazbě pro swapový obchod, který má stejné výplatní dny jako uvažovaný cap.}.

Následující graf zobrazuje typický profil spotové a flat volatility v závislosti na čase. Flat volatilita má povahu průměru spotových volatilit, a proto je méně volatilní.
\begin{center}
	\begin{pspicture}(0,0)(8,4.5)
		\rput(4.0,0.0){Cap - spotová a flat volatilita}

                \psline[arrows=->](0.5,1.0)(7.5,1.0)
                \psline[arrows=->](0.5,1.0)(0.5,4.0)

                \rput(7.5,0.7){\small{splatnost}}
                \rput(2.2,4.0){\small{směrodatná odchylka}}

                \pscurve(0.5,2.5)(2.0,3.5)(4.5,2.0)(6.5,1.5)
                \pscurve(0.5,2.5)(1.8,3.0)(3.8,2.5)(5.0,2.4)(6.5,2.45)

                \rput(6.5,1.3){\tiny{spotová volatilita}}
                \rput(6.5,2.25){\tiny{flat volatilita}}
                
	\end{pspicture}
\end{center}

\subsection{Úrokový floor}

Floor generuje výplatu v případě, že referenční floatová úroková sazba klesne pod určitou stanovenou hranici. Floor tak představuje ochranu před poklesem úrokových sazeb. Podobně jako v případě capu je možné definovat výplatu z úrokového flooru v čase $t_{k+1}$ jako
\begin{equation}
L \delta_k \max(R_K - R_k, 0)
\end{equation}
Floor je tak možné rozložit na sérii prodejních úrokových opcí popř. na sérii kupních opcí na diskontní dluhopisy. Tyto dílčí opce pak nazýváme floorlets.
Současná hodnota floorletu, jehož výplata je definována (19.6), je dána vztahem
\begin{equation}
L \delta_k P(0,t_{k+1})[R_KN(-d_2)-F_kN(-d_1)]
\end{equation}
což analogie k  rovnici (19.5) pro caplet. Stejně jako v případě úrokového capu představuje proměnná $F_k$ forwardovou sazbu pro periodu $t_k$ až $t_{k+1}$ a $\sigma_k$ volatilitu logaritmu náhodné proměnné $R_k$.

\subsection{Put-call parita}

Podobně jako v případě prodejní a kupní opce platí pro cap a floor tzv. put-call parita.
\begin{center}
\textit{cena cap = cena floor + cena úrokového swapu} 
\end{center}
Podmínkou platnosti výše uvedeného vztahu je, že cap i floor mají stejnou realizační cenu $R_K$ a že v rámci swapu dochází k výměně floatové úrokové sazby za fixní sazbu $R_K$. Dále musí mít všechny tři uvažované instrumenty stejná data plateb. Uvedená rovnice tedy mimojiné znamená, že úrokový swap je možné replikovat pomocí dlouhé pozice v capu a krátké pozice ve flooru.

\subsection{Úrokový collar}

Collar je kombinací capu a flooru. Collar tedy generuje výplatu v případě, že referenční floatová úroková sazba opustí stanovené pásmo. Collar je velice často konstruován tak, aby se hodnota capu rovnala hodnotě flooru a cena collar v době jeho sjednání tak byla nulová.

\subsection{Teoretické zdůvodnění modelu}

Je možné dokázat, že Black-Scholes model pro ocenění capletu je interně konzistentní s světem, který je forwardově rizikově neutrální vzhledem k diskontnímu dluhopisu se splatností v čase $t_{k+1}$. Pro tento forwardově rizikově neutrální svět bylo prokázáno, že
\begin{itemize}
\item současná hodnota libovolného investičního instrumentu je rovna očekávané hodnotě v čase $t_{k+1}$ vynásobená současnou cenou diskontního dluhopisu se splatností v čase $t_{k+1}$
\item očekávaná hodnota úrokových sazeb v období $t_k$ až $t_{k+1}$ je rovna forwardové sazbě
\end{itemize}
Cena capletu, který generuje případnou výplatu v čase $t_{k+1}$, je
\begin{equation*}
L \delta_t P(0,t_{k+1})E_{k+1}[\max(R_k - R_K, 0)]
\end{equation*}
Podobně jako v případě klasické evropské opce lze dokázat, že tento vztah lze také vyjádřit jako
\begin{equation}
L \delta_t P(0, t_{k+1})[E_{k+1}[R_k]N(d_1) - R_KN(d_2)]
\end{equation}
kde
\begin{equation*}
d_1 = \frac{\ln[E_{k+1}[R_k]/R_k]+\sigma_k^2 t_k/2}{\sigma_k \sqrt{t_k}}
\end{equation*}
\begin{equation*}
d_2 = \frac{\ln[E_{k+1}[R_k]/R_k]-\sigma_k^2 t_k/2}{\sigma_k \sqrt{t_k}}=d_1 - \sigma_k \sqrt{t_k}
\end{equation*}
Druhý z výše uvedených předpokladů implikuje vztah
\begin{equation}
E_{k+1}[R_k]=F_k
\end{equation}
Spojením (19.8) a (19.9) pak získáváme (19.5).

\section{Evropské swapové opce}

Swapové opce neboli swapce jsou opce na úrokové swapy. Swapce dávají právo svému vlastníkovi vstoupit do úrokového swapu v daném časovém okamžiku v budoucnosti a jedná se tak do určité míry o alternativu k tzv. forwardovým swapům. Jestliže obchodník nakoupí swapci, zajišťuje se proti případnému růstu fixních sazeb.

\subsection{Vztah k dluhopisovým opcím}

Úrokový swap je lze chápat jako dohodu o výměně fixního dluhopisu za floatový. Swapci je proto možné interpretovat jako opci na výměnu fixního dluhopisu za floatový. Jestliže swapce dává svému majiteli možnost zaplatit fixní a obdržet floatový dluhopis, hovoříme o prodejní swapci, v opačném případě pak o kupní swapci.
\\

Swapová sazba pro určitou splatnost v daný čas je fixní sazba, která by byla v rámci swapového kontraktu nabídnuta výměnou za floatovou sazbu LIBOR. Black-Scholes model, pomocí kterého obvykle oceňujeme evropské swapce, předpokládá, že odpovídající swapová sazba v době splatnosti swapce má lognormální rozdělení.

Uvažujme swapci, v rámci které máme právo zaplatit sazbu $s_K$ a získat LIBOR. Kontrakt bude trvat $n$ let a začne za $T$ roků. Předopokládejme, že v rámci tohoto kontraktu bude provedeno $m$ plateb za rok a že pokladový kapitál, od kterého se odvíjí úroky, je roven $L$. Dále předpokládejme, že swapová sazba pro $n$-letý swap, jehož splatnost se shoduje se splatností swapce, je $s_T$. Výplata z námi uvažované swapce v době splatnosti je dána vztahem
\begin{equation*}
\frac{L}{m}\max(s_T-s_K, 0)
\end{equation*}
Peněžní toky jsou obdrženy $m$ krát ročně po dobu $n$ let životnosti swapového kontraktu. Nechť jsou data těchto plateb $T_1$, $T_2$,..., $T_{mn}$. Každou takovoutu platbu v čase $T_i$ lze pak považovat za výplatu z kupní opce na $s_T$ s realizační cenou $s_K$.
\begin{equation*}
\frac{L}{m}P(0, T_i)[s_0N(d_1) - s_K N(d_2)]
\end{equation*}
kde
\begin{equation*}
d_1 = \frac{\ln(s_0/s_K)+\sigma^2/2 T}{\sigma \sqrt{T}}
\end{equation*}
\begin{equation*}
d_2 = \frac{\ln(s_0/s_K)-\sigma^2/2 T}{\sigma \sqrt{T}}= d_1 - \sigma \sqrt{T} 
\end{equation*}
a $s_0$ forwardová swapová sazba. Celková hodnota swapce je tedy rovna
\begin{equation*}
\sum^{mn}_{i=1} \frac{L}{m} P(0, T_1)[s_0N(d_1)-s_KN(d_2)]
\end{equation*}
Definujme $A$ jako hodnotu kontraktu, který vyplácí $1/m$ v čase $T_i$ ($1 \leq i \leq mn$). Hodnota odpovídající swapce je pak rovna
\begin{equation}
LA[s_0N(d_1) - s_KN(d_2)]
\end{equation}
kde
\begin{equation*}
A = \frac{1}{m} \sum^{mn}_{i=1}P(0, T_i)
\end{equation*}
Jestliže swapce dává vlastníkovi možnost obdržet fixní sazbu $s_K$ namísto toho, aby jí platil, je hodnota swapce dána vztahem
\begin{equation*}
LA[s_KN(-d_2) - s_0N(-d_1)]
\end{equation*}

Brokeři obvykle poskytují tabulku s implikovanými volatilitami pro evropské swapce. Instrumenty, na jejichž základě byla implikovaná volatilita vypočtena, jsou obvykle at-the-money\footnote{To znamená, že realizační swapová sazba je rovna forwardové swapové sazbě.}. Frekvence plateb podkladového swapu je zpravidla šest měsíců pro floatovou nohu a jeden rok pro fixní nohu.

\subsection{Teoretické zdůvodnění modelu}

Je možné dokázat, že Black-Scholes model pro ocenění swapcí je interně konzistentní se světem, který je forwardově rizikově neutrální s ohledem k anuitě $A$. V tomto světě totiž platí
\begin{itemize}
\item současná hodnota libovolného aktiva se rovna součinu současné hodnoty anuity a očekávané hodnotě výrazu
\begin{center}
\textit{(hodnota aktiva v čase T) / (hodnota anuity v čase T)}
\end{center}.
\item očekávaná hodnota swapové sazby v čase $T$ je rovna forwardové swapové sazbě (viz. kapitola 18.2.2, Anuita)
\end{itemize}
Výplata a čase $T$ ze swapce, kde vlastník má právo zaplatit $s_K$ a obdržet floatovou sazbu, je rovna součinu anuity a
\begin{equation*}
\frac{L}{m}\max(s_T - s_K, 0)
\end{equation*}
Hodnota této swapce je tedy
\begin{equation*}
LAE_A[\max(s_T - s_K, 0)]
\end{equation*}
Tento výraz lze dále upravit do tvaru
\begin{equation*}
LA[E_A(s_T)N(d_1) - s_KN(d_2)]
\end{equation*}
kde
\begin{equation*}
d_1 = \frac{\ln(E_A[s_T]/s_K)+\sigma^2T/2}{\sigma \sqrt{T}}
\end{equation*}
\begin{equation*}
d_2 = \frac{\ln(E_A[s_T]/s_K)-\sigma^2T/2}{\sigma \sqrt{T}} = d_1 - \sigma \sqrt{T}
\end{equation*}
Dle druhého z výše uvedených bodů platí $E_A[s_T] = s_0$. S využitím této substituce lze vzorec pro výpočet hodnoty swapce upravit do tvaru (19.10). Za předpokladu, že očekávané swapové sazby jsou rovny forwardovým swapovým sazbám, je tedy možné pro účely diskontování považovat úrokové sazby za konstantní.

\section{Zobecnění}

Dosud jsme představili tři různé verze Black-Scholes modelu - jeden pro dluhopisové opce, jeden pro capy a jeden pro swapce. Tyto modely jsou sice interně konzistentní, však nejsou konzistentní vzájmně\footnote{Například když jsou budoucí ceny dluhopisů lognormální, budoucí zero sazby a swapové sazby tuto podmínku nesplňují.}. Následujících několik bodů představuje zobecnění výše popsaných modelů.
\begin{itemize}
\item Uvažujme aktivum, které generuje výplatu v čase $T$ v závislosti na jeho ceně v čase $T$. Současná hodnota tohoto aktiva je dána součinem $P(0,T)$ a očekávané výplaty. Toto tvrzení však platí pouze za předpokladu, že očekáváná výplata je vypočtena pro svět, kde očekávaná hodnota podkladového aktiva je rovna jeho forwardové ceně.
\item Uvažujme aktivum, které generuje výplatu v čase $T_2$ v závislosti na úrokové míře pro splatnost $T_2$ platnou pro časový okamžik $T_1$. Jeho současná hodnota je rovna součinu $P(0,T_2)$ a očekávané výplaty. Nezbytným předpokladem opět je, že očekávaná výplata je vypočtena pro svět, kde se očekávaná hodnota podkladové úrokové sazby rovná forwardové úrokové sazbě.
\item Uvažujme aktivum, které generuje výplatu ve formě anuity. Předpokládejme, že výše anuity v čase $T$ je funkcí swapové sazby pro $n$ roční swap, který začíná v čase $T$. Dále předpokládejme, že tato anuita trvá po $n$ let a výplaty pro danou anuity se shodují s výplatami v rámci swapu. Hodnota odpovídajícího aktiva je rovna součinu $A$ a očekávané roční výplaty. Podmínky, pro které je toto tvrzení platné, jsou (a) $A$ je současná hodnota anuity, kde výplaty jsou ve výši 1 USD za rok a (b) očekávaní jsou vztažena ke světu, kde očekávaná budoucí swapová sazba je rovna forwardové swapové sazbě.
\end{itemize}

\section{Konvexita}

Forwardová výnosová míra dluhopisu je definována jako výnosová míra implikovaná jeho forwardovou cenou. Předpokládejme, že $B_T$ je cena bondu v čase $T$ a  $y_T$ je odpovídající výnosová míra. Vztah mezi $B_T$ a $y_T$ je definován pomocí
\begin{equation*}
B_T = G(y_T)
\end{equation*}
Definujme $F_0$ jako forwardovou cenu dluhopisu v čase nula pro kontrakt se splatností v čase $T$ a $y_0$ jako odpovídající forwardovou výnosovou míru. Platí
\begin{equation*}
F_0 = G(y_0)
\end{equation*}
Funkce $G$ je nelineární. To znamená, že je-li očekávaná budoucí cena dluhopisu rovna jeho forwardové ceně, není očekávaná budoucí výnosová míra dluhopisu rovna forwardové výnosové míře. Abychom toto tvrzení ilustrovali na příkladě, uvažujme dluhopis, pro který existují pouze dvě možné ceny a to $B_1$ a $B_2$ a jim odpovídající výnosové míry $y_1$ a $y_2$. Jestliže pravděpodobnost realizace obou cen je shodná, je očekávaná cena tohoto dluhopisu dána průměrem cen $B_1$ a $B_2$, tj. cenou $B_a$. Očekávaná výnosová míra $y_a$ však není prostým průměrem výnosových měr $y_1$ a $y_2$.
\begin{center}
	\begin{pspicture}(0,0)(6,6)
		\rput(3,0.5){Vztah mezi cenou dluhopisu a výnosou mírou}

		\psline[arrows=->](0.5,1.5)(5.5,1.5)
		\psline[arrows=->](0.5,1.5)(0.5,5.5)
		
		\rput(1.5,5.8){\small cena dluhopisu}
		\rput(5,1.2){\small výnosová míra}

		\pscurve[linewidth=0.5mm](0.7,5.5)(2.4,2.7)(5.5,1.8)

		\psline[linewidth=0.1mm, linestyle=dashed](1.5,1.5)(1.5,3.8)
		\psline[linewidth=0.1mm, linestyle=dashed](0.5,3.8)(1.5,3.8)
		\rput(1.5,1.3){\tiny $y_1$}
		\rput(0.2,3.8){\tiny $B_1$}

		\psline[linewidth=0.1mm, linestyle=dashed](2,1.5)(2,3.1)
		\psline[linewidth=0.1mm, linestyle=dashed](0.5,3.1)(2,3.1)
		\rput(2,1.3){\tiny $y_a$}
		\rput(0.2,3.1){\tiny $B_a$}

		\psline[linewidth=0.1mm, linestyle=dashed](3,1.5)(3,2.4)
		\psline[linewidth=0.1mm, linestyle=dashed](0.5,2.4)(3,2.4)
		\rput(3,1.3){\tiny $y_2$}
		\rput(0.2,2.4){\tiny $B_2$}

	\end{pspicture}
\end{center}

\subsection{Matematické zdůvodnění}

Uvažujme derivát, který poskytuje výplatu závislou na výnosové míře dluhopisu v čase $T$. Z rovnice
\begin{equation*}
d f = (r + \lambda \sigma)f dt + \sigma f dz
\end{equation*}
víme, že tento derivát můžeme ocenit (a) výpočtem očekávané výplaty ve světě, který je forwardově rizikově neutrální vzhledem k diskontnímu dluhopisu se splatností v čase $T$ a (b) následným diskontováním současnou bezrizikovou mírou pro splatnost $T$. Víme, že očekávaná hodnota dluhopisu je v námi uvažovaném světě rovna jeho forwardové ceně. Proto potřebujeme znát hodnotu očekávané výnosové míry dluhopisu za předpokladu, že se očekávaná cena dluhopisu rovná jeho forwardové ceně. V lemě 19A je ukázáno, že přibližné vyjádření očekávané výnosové míry dluhopisu je rovno
\begin{equation*}
E_T[y_T] = y_0 - \frac{1}{2}y^2_0 \sigma^2_y T \frac{G''(y_0)}{G'(y_0)}
\end{equation*}
kde $E_T$ označuje očekávání ve světě, který je forwardově rizikově neutrální vzhledem k $P(t,T)$ a $\sigma_y$ je volatilita forwardové výnosové míry. Z výše uvedeného vyplývá, že očekávané výplaty můžeme diskontovat současnou bezrizikovou sazbou pro splatnost $T$ za předpokladu, že očekávaná výnosová míra dluhopisu je rovna
\begin{equation*}
y_0 - \frac{1}{2}y^2_0 \sigma^2_y T \frac{G''(y_0)}{G'(y_0)}
\end{equation*}
a nikoliv pouze $y_0$. Rozdíl $-\frac{1}{2}y^2_0 \sigma^2_y T \frac{G''(y_0)}{G'(y_0)}$ je tzv. konvexní úprava.

\subsection{Časová korekce}

V této kapitole se budeme zabývat situací, kdy derivát generuje výplatu v čase $T_2$ na základě hodnoty proměnné $v$ pozorované v dřívějším časovém okamžiku $T_1$. Definujme
\begin{center}
\begin{tabular}{l l}
$v_1$ & hodnota proměnné $v$ v čase $T_1$\\
$F$ & forwardová cena proměnné $v$ kontraktu se\\
 & splatností v čase $T_1$\\
$E_1[v_1]$ & očekávaná hodnota $v_1$ ve světě, který je forwardově\\
 & rizikově neutrální vzhledem k $P(t, T_1)$\\
$E_2[v_1]$ & očekávaná hodnota $v_1$ ve světě, který je forwardově\\
 & rizikově neutrální vzhledem k $P(t, T_2)$\\
$G$ & forwardová cena diskontního dluhopisu s životností od $T_1$\\
 & do $T_2$\\
$R$ & forwardová úroková sazba pro časové období mezi $T_1$ a $T_2$\\
 & pro úrokovou frekvenci $m$\\
$R_0$ & dnešní hodnota $R$\\
$\sigma_F$ & volatilita náhodné veličiny $F$\\
$\sigma_G$ & volatilita náhodné veličiny $G$\\
$\sigma_R$ & volatilita náhodné veličiny $R$\\
$\rho$ & korelace mezi náhodnými veličinami $F$ a $G$\\
\end{tabular}
\end{center}
Jestliže se přesuneme ze světa, který je forwardově rizikově neutrální vzhledem k $P(t,T_1)$, do světa, který je forwardově neutrální vzhledem k $P(t, T_2)$, platí
\begin{equation*}
G = \frac{P(t, T_2)}{P(t, T_1)}
\end{equation*}
Míra růstu náhodné veličiny $v$ se změní o
\begin{equation*}
\alpha_v = - \rho \sigma_G \sigma_F
\end{equation*}
(Náhodné veličiny $G$ a $F$ jsou vzájemně dokonale negativně korelovány, proto záporné je korelace mezi $G$ a $R$ rovna $-\rho$). Protože
\begin{equation*}
G = \frac{1}{(1 + R/m)^{m(T_2 - T_1)}}
\end{equation*}
může být vztah mezi volatilitami náhodných veličin $G$ a $R$ vypočten na základě It\^o lemy jako
\begin{equation*}  
\sigma_G = \frac{\sigma_R R(T_2 - T_1)}{1 + R/m}
\end{equation*}
Z toho vyplývá, že $\alpha_v$ je možné vyjádřit jako
\begin{equation*}  
\alpha_v = - \frac{\rho \sigma_F \sigma_F R(T_2 - T_1)}{1 + R/m}
\end{equation*}
V rámci aproximace můžeme předpokládat, že $R$ je konstatní na úrovni $R_0$. Tímto získáváme vztah
\begin{equation*}
E_2[v_1] = E_1[v_1]e^{-\frac{\rho \sigma_F \sigma_R R_0 T_1 (T_2 - T_1)}{1 + R_0/m}}
\end{equation*}
Tato rovnice umožňuje úpravu forwardové ceny tak, aby byl zohledněn časový posun mezi okamžikem, ke kterému pozorujeme hodnotu podkladové veličiny $v$, a okamžikem, ke kterému je realizována výplata zavisející na této hodnotě.

\section{Lema 19A - Odvození konvexity}

Uvažujme derivát, jehož výplata v čase $T$ závisí na dluhopisovém výnosu pozorovaném v tomto čase. Definujme
\begin{center}
\begin{tabular}{l l}
$y_0$ & dnešní forwardová výnosová míra dluhopisu pro forwardový\\
 & kontrakt se splatností v čase $T$\\
$y_T$ & výnosová míra dluhopisu v čase $T$\\
$B_T$ & cena dluhopisu v čase $T$\\
$\sigma_y$ & volatilita výnosové míry dluhopisu\\
\end{tabular}
\end{center}
Nechť platí
\begin{equation*}
B_T = G(y_T)
\end{equation*}
Taylorovým rozvojem $G(y_T)$ podle $y_T = y_0$ získáme aproximaci
\begin{equation*}
B_T = G(y_0) + (y_T - y_0)G'(y_0)+ \frac{1}{2}(y_T - y_0)^2 G''(y_0)
\end{equation*}
Jestliže do této rovnice zapracujeme očekávání ve světě, který je forwardově rizikově neutrální vzhledem k diskontnímu dluhopisu se splatností v čase $T$, přejde tato rovnice do tvaru
\begin{equation*}
E_T[B_T]=G(y_0)+E_T[y_T - y_0]G'(y_0) + \frac{1}{2} E_T[(y_T - y_0)^2]G''(y_0)
\end{equation*}
Z definice vyplývá, že $G(y_0)$ a $E_T[B_T]$ jsou forwardovou cenou dluhopisu. Rovnice se tedy dále zjednoduší na
\begin{equation*}
E_T[y_T - y_0]G'(y_0) + \frac{1}{2} E_T[(y_T - y_0)^2]G''(y_0) = 0
\end{equation*}
Vzhledem k tomu, že platí $E_T[(y_T - y_0)^2] \approx \sigma^2_y y^2_0 T$, získáváme
\begin{equation*}
E_T[y_T] \approx y_0 - \frac{1}{2} \sigma^2_y y^2_0 T \frac{G''(y_0)}{G'(y_0)}
\end{equation*}

\chapter{Modelování úrokových sazeb}

Výše uvedené modely pro oceňování derivátů předpokládaly, že pravděpodobnostní rozdělení úrokových sazeb, cen dluhopisů a dalších veličin je lognormální. Ačkoliv je předpoklad lognormálního rozdělení všeobecně přijímán, má svá omezení - např. není schopen postihnout vývoj úrokových sazeb v čase. To znamená, že tento přístup není možné aplikovat na ocenění amerických úrokových opcí. V této kapitole se pokusíme nabídnout alternativní přístup a budeme se zabývat modelováním krátkodobých úrokových sazeb.

\section{Rovnovážné modely}

Rovnovážné modely zpravidla začínají předpoklady o ekonomických proměnných v návaznosti na než definují proces, který sleduje krátkodobá úroková sazba $r$. Tato úroková sazba je platná po nekonečně malý časový interval v čase $t$. Následně se zkoumá vazba mezi vývojem úrokové sazby a cenou dluhopisů a opcí. Předpokládá se, že tyto ceny závisí výhradně na vývoji úrokové sazby. V tradičním pojetí rizikově neutrálního světa platí, že hodnota derivátu, který v čase $T$ generuje výplatu $f_T$, je v čase $t$ dána
\begin{equation*}
\hat{E} \bigg[ e^{-\bar{r}(T-t)}f_T\bigg]
\end{equation*}
kde $\bar{r}$ představuje průměrnou hodnotu $r$ v časovém intervalu od $t$ do $T$ a $\hat{E}$ je očekávanou hodnotou v tradičním rizikově neutrálním světě. Pouze připomeňme, že v tomto světě platí
\begin{equation*}
P(t,T)=\hat{E}\bigg[ e^{-\bar{r}(T-t)}\bigg]
\end{equation*}
Jestliže $R(t,T)$ je úroková míra s kontinuálním úročením v čase $t$ pro časový interval $T - t$, pak
\begin{equation*}
P(t,T) = e^{-R(t,T)(T-t)}
\end{equation*}
\begin{equation}
R(t,T)=-\frac{1}{T-t}\ln P(t,T)
\end{equation}
\begin{equation*}
R(t,T)=-\frac{1}{T-t}\ln \hat{E}\bigg[ e^{-\bar{r}(T-t)}\bigg]
\end{equation*}
Tato rovnice umožňuje získat kompletní časovou strukturu úrokových sazeb na základě znalosti hodnoty $r$ v daném čase a procesu, který tato náhodná veličina sleduje. Z této rovnice tak vyplývá, že je-li definován proces pro $r$, je definována také zero křivka a její vývoj v čase.

\subsection{Rovnovážné modely s jedním parametrem}

V případě rovnovážných modelů s jedním parametrem zahrnuje náhodný proces $r$ pouze jeden zdroj nejistoty.
\begin{equation*}
dr = m(r)dt + s(r)dz
\end{equation*}
Parametr $m$ a $s$ jsou funkcí $r$, avšak jsou nezávislé na $t$. Tyto modely předpokládají, že se všechny sazby pohybují stejným směrem v rámci krátkého časového intervalu, i když tento posun nemusí být vždy stejný.

\subsubsection{Rendelman-Bartterův model}

V rámci Rendelman-Bartterova modelu sleduje náhodná veličina $r$ proces
\begin{equation*}
dr = \mu r dt + \sigma r dz
\end{equation*}
kde $\mu$ a $\sigma$ jsou konstanty. Náhodná veličina $r$ tedy sleduje tzv. Brownův pohyb, podobně jak je tomu v případě modelování cen akcií. Jeden zásadní rozdíl mezi vývojem cen akcií a úrokových sazeb je ten, že sazby mají tendenci se v delším časovém horizontu vracet k dlouhodobému průměru. Rendelman-Bartterův model však tuto vlastnost v sobě zakomponovánu nemá.

\subsubsection{Vašíčkův model}

V rámci Vašíčkova modelu sleduje náhodná veličina $r$ proces
\begin{equation}
dr = a(b-r)dt + \sigma dz
\end{equation}
kde $a$, $b$ a $\sigma$ jsou konstanty. Tento model zahrnuje návrat úrokových sazeb k dlouhodobému průměru - ty se vrací k úrovni $b$ mírou $a$. Vašíček dokázal, že (20.1) může být použita k výpočtu ceny diskontovaného dluhopisu v časovém okamžiku $t$, jestliže tento dluhopis v čase $T$ generuje výplatu 1 USD.
\begin{equation*}
P(t,T)=A(t,T)e^{-B(t,T)r(t)}
\end{equation*}
Ve výše uvedené rovnici představuje $r(t)$ hodnotu $r$ v čase $t$,
\begin{equation*}
B(t,T)= \frac{1-e^{-a(T-t)}}{a}
\end{equation*}
a
\begin{equation*}
A(t,T)=e^{\frac{(B(t,T)-T+t)(a^2b-\sigma^2/2)}{a^2}-\frac{\sigma^2B(t,T)^2}{4a}}
\end{equation*}
Jestliže $a=0$, dostáváme $B(t,T)=T-t$ a $A(t,T)=e^{\sigma^2(T-t)^3/6}$. Dosazením do rovnice (20.1) získáme
\begin{equation*}
R(t,T)=-\frac{1}{T-t}\ln A(t,T)+\frac{1}{T-t}B(t,T)r(t)
\end{equation*}
Pomocí sazby $r(t)$ lze tedy vypočítat celou výnosovou křivku a volbou parametrů $a$, $b$ a $\sigma$ lze ovlivnit její tvar.\\

Pomocí Vašíčkova modelu lze také oceňovat opce na diskontní dluhopis. Cenu evropské kupní opce se splatností v čase $s$ na diskontní dluhopis s nominálem $L$ a se splatností v čase $T$ lze určit podle
\begin{equation*}
c = L P(0,s)N(h)-K P(0,T) N(h - \sigma_P)
\end{equation*}
kde
\begin{equation*}
h = \frac{1}{\sigma_P} \ln \frac{L P(0,s)}{P(0, T)K} + \frac{\sigma_P}{2}
\end{equation*}
a
\begin{equation*}
\sigma_P = \frac{\sigma}{a}(1-e^{a(s-T)})\sqrt{\frac{1-e^{-2aT}}{2a}}
\end{equation*}
Cena evropské prodejní opce je pak dána
\begin{equation*}
p = KP(0,T)N(-h + \sigma_P) - L P(0,s)N(-h)
\end{equation*}

Cenu opcí na kupónové dluhopisy lze odvodit z cen opcí na diskontní dluhopis pomocí jednoparametrových modelů, jakým je např. Vašíčkův model, kde jsou všechny sazby pozitivně korelovány s $r$. Uvažujme evropskou kupní opci s realizační cenou $K$ a splatností $T$ na kupónový dluhopis. Předpokládejme, že tento dluhopis generuje $n$ plateb po té, co příslušná opce zmaturuje. Označme $i$-tou platbu, ktrá nastane v čase $s_i$ ($s_i \ge T$), jako $c_i$. Definujme
\begin{center}
\begin{tabular}{l l}
$r_K$ & hodnota krátkodobé sazby $r$ v čase $T$, pro kterou se hodnota dluhopisu\\
 & s nenulovým kupónem rovná realizační ceně $K$\\
$K_i$ & hodnota diskontního dluhopisu v čase $T$ za předpokladu, že tento\\
 & dluhopis generuje výplatu 1 USD v době splatnosti $s_i$ a rovnosti\\
 & sazeb $r$ a $r_K$
\end{tabular}
\end{center}
Jestliže jsou ceny dluhopisu vyjádřeny jako funkce proměnné $r$ (jak je tomu v případě Vašíčkova modelu), je možné iterativně vypočíst $r_K$ např. pomocí Newton-Raphsonovy metody. Nechť je $P(T, s_i)$ cenou diskontního dluhopisu v čase $T$ za předpokladu, že tento dluhopis generuje výplatu 1 USD v čase $s_i$. Výplata z uvažované opce je tak dána rovnicí
\begin{equation*}
\max \bigg( 0, \sum^n_{i=1} c_i P(T, s_i) - K\bigg)
\end{equation*}
Protože jsou všechny sazby rostoucí funkcí $r$, jsou ceny dluhopisů klesající funkcí $r$. To znamená, že kupónový dluhopis, který má v čase $T$ hodnotu vyšší než realizační cenu $K$, by měl být uplatněn v případě, že $r < r_K$. Dále platí, že diskontní dluhopis se splatností v $s_i$, který je podkladem pro kupónový dluhopis, má v čase $T$ hodnotu vyšší než $c_i K_i$ tehdy a jen tehdy, platí-li $r<r_K$. Z toho vyplývá, že výplata z opce je
\begin{equation*}
\sum_{i=1}^n c_i \max [0, P(T, s_i) - K_i]
\end{equation*}
Evropská kupní opce na kupónový dluhopis je tedy součtem $n$ opcí na podkladové diskontní dluhopisy. Obdobný postup lze aplikovat na evropskou prodejní opci.

\subsubsection{Cox-Ingersollův-Rossův model}

V případě Vašíčkova modelu může dojít k tomu, že krátkodobá úroková míra $r$ bude záporná. Cox, Ingersoll a Ross navrhli model, v rámci kterého jsou úrokové sazby vždy nezáporné.
\begin{equation*}
dr = a(b-r)dt + \sigma \sqrt{r}dz
\end{equation*}
Tento model, stejně jako Vašíčkův, má v sobě zabudovaný návrat úrokových sazeb k dlouhodobému průměru $b$ tempem, které určuje parametr $a$. Směrodatná odchylka je však propocionální k $\sqrt{r}$. To znamená, že dojde-li k růstu krátkodobé úrokové sazby, dojde také k růstu směrodatné odchylky. Ceny dluhopisů jsou dány stejným vztahem jako v případě Vašíčkova modelu
\begin{equation*}
P(t,T) = A(t,T)e^{-B(t,T)r}
\end{equation*}
avšak rovnice pro $B(t,T)$ a $A(t,T)$ jsou definovány odlišně.
\begin{equation*}
B(t,T)=\frac{2(e^{\gamma(T-t)}-1)}{(\gamma + a)(e^{\gamma (T-t)} - 1) + 2 \gamma}
\end{equation*}
\begin{equation*}
A(t,T)=\Bigg( \frac{2 \gamma e^{(a + \gamma)(T - t)/2}}{(\gamma + a)(e^{\gamma (T - t)} - 1) + 2 \gamma} \Bigg)^{\frac{2ab}{\sigma^2}}
\end{equation*}
\begin{equation*}
\gamma = \sqrt{a^2 + 2 \sigma^2}
\end{equation*}

Cox, Ingersoll a Ross definovali také rovnice pro výpočet evropských kupní a prodejních opcí na diskontní dluhopis. Ty však obsahují integrály přes chi-kvadrát rozdělení, a proto je jejich nasazení v praxi relativně problematické. Cena opcí na kupónový dluhopis se dá určit obdobně jako v případě Vašíčkova modelu.

\subsection{Rovnovážné modely se dvěma parametry}

\subsubsection{Brennan-Schwartzův model}

V tomto modelu konverguje krátkodobá úroková sazba k dlouhodobé úrokové sazbě. Dlouhodobá sazba sleduje stochastický proces a je definována v rámci modelu jako výnos perpetuitního bondu, který generuje 1 USD ročně. S pomocí It\^o lemmy lze pak z procesu, který sleduje cena tohoto dluhopisu, určit proces, který sleduje jeho výnosová míra. Podmínkou kalibrace modelu je, že tento dluhopis je obchodován. Dále platí, že očekávaný výnos tohoto dluhopisu je v rizikově neutrálním světě roven bezrizikové úrokové sazbě.

\subsubsection{Longstaff-Schwartzův model}

Tento model je zvláštní tím, že vychází z obecného rovnovážného modelu ekonomiky, na základě kterého odvozuje model úrokových sazeb, ve kterém je zapracována stochastická volatilita.

\subsection{Nearbitrážní modely}

Nevýhodou rovnovážných modelů je, že nemusí vždy odpovídat dnešní struktuře úrokových sazeb. Vhodným výběrem parametrů je možné určité shody dosáhnout, však tato shoda ve většině případů není dokonalá.

Nearbitrážní model je zkonstruován tak, aby byl konzistentní se strukturou úrokových sazeb v době své kalibrace. V případě rovnovážného modelu jsou "dnešní" úrokové sazby výstupem; v případě nearbitrážního modelu jsou tyto sazby vstupem. Dalším rozdílem je tzv. trend krátkodobé sazby (tj. koeficient parametru $dt$) - u rovnovážných modelů není odpovídající parametr funkcí času; u nearbitrážních modelů je naopak tento parametr velice často funkcí času.

\subsubsection{Ho-Lee model}

Rovnice popisující model je
\begin{equation*}
dr = \theta (t) dt + \sigma dz
\end{equation*}
kde $\sigma$ je směrodatná odchylka krátkodobé úrokové sazby, kterou v rámci modelu považujeme za konstatní, a $\theta (t)$ je funkcí času, která je zvolena tak, aby model odpovídal výchozí struktuře úrokových sazeb. Funkce $\theta (t)$ vyjadřuje průměrný směr, kterým se pohybuje $r$ v čase $t$, a je nezávislá na úrovni $r$.

Proměnná $\theta (t)$ může být vypočtena analyticky jako
\begin{equation*}
\theta (t) = F_t(0,t)+\sigma^2 t
\end{equation*}
kde $F(0,t)$ je forwardová sazba v čase nula pro splatnost $t$. Dolní index $t$ značí derivaci s ohledem na $t$. Jako aproximaci je možné použít vztah $\theta (t) = F(0,t)$\footnote{To znamená, že průměrný směr, kterým se bude krátkodobá úroková sazba bude pohybovat v budoucnu, je rovna sklonu forwardové křivky.}.

V Ho-Lee modelu je možné diskontní dluhopis a evropské opce ocenit analyticky. Cena diskontního dluhopisu v čase $t$ je
\begin{equation}
P(t,T)=A(t,T)e^{-r(t)(T-t)}
\end{equation}
kde
\begin{equation*}
\ln A(t,T) = \ln \frac{P(0,T)}{P(0,t)}-(T-t)\frac{\partial \ln P(0,t)}{\partial t}-\frac{1}{2}\sigma^2t(T-t)^2
\end{equation*}
Ve výše uvedené rovnici značí čas nula dnešek. Čas $T$ a $t$ jsou obecné časy v budoucnu, kde $T \ge t$. Tato rovnice tedy definuje cenu diskotního dluhopisu v budoucím čase $t$ v kontextu krátkodobé sazby v čase $t$ a dnešních cen dluhopisů s tím, že dnešní cena dluhopisů může být vypočtena na základě současné zero křivky.

Po zbytek této kapitoly bude $R(t)$ popř. pouze $R$ označovat úrokovou sazbu pro periodou $\delta t$. Rovnici (20.3) lze také vyjádřit ve tvaru
\begin{equation}.
P(t,T)= hat{A}(t,T)e^{-R(t)(T-t)}
\end{equation}
kde
\begin{equation*}
\ln \hat{A}(t,T)=\ln \frac{P(0,T)}{P(0,t)}-\frac{T-t}{\delta t}\ln \frac{P(0,t+\delta t)}{P(0,t)}-\frac{1}{2}\sigma^2 t(T-t)[(T-t)-\delta t]
\end{equation*}
V praxi se většinou cena dluhopisu určuje ve formě $R$ spíše než $r$, a proto je rovnice (20.3) použitelnější než (20.4).

Evropská kupní opce na diskontní dluhopis s maturitou v čase $s$, která je splatná v čase $T$, má v čase nula hodnotu
\begin{equation*}
LP(0,s)N(h) - KP(0,T)N(h-\sigma_p)
\end{equation*}
$L$ představuje nominální hodnotu dluhopisu, $K$ realizační cenu, $h$ je definováno jako
\begin{equation*}
h = \frac{1}{\sigma_p}\ln{LP(0,s)}{P(0,T)K}+\frac{\sigma_p}{2}
\end{equation*}
a $\sigma_p$ jako
\begin{equation*}
\sigma_p = \sigma(s-T)\sqrt{T}
\end{equation*}
Cena ekvivalentní prodejní opce je pak rovna
\begin{equation*}
KP(0,T)N(-h+\sigma_p)-LP(0,s)N(-h)
\end{equation*}

Nevýhodou Ho-Lee modelu je, že neposkytuje uživateli flexibilitu při volbě struktury volatilit. Změny všech úrokových sazeb mají v krátkém časovém okamžiku stejnou směrodatnou odchylku. Další nevýhodou je, že tento model v sobě nemá zabudovaný mechanismus návratu k dlouhodobé úrokové míře.

\subsubsection{Hull-White model}

Hull-Whitův model je rozšířením Vašíčkova modelu o mechanismus, který zajišťuje návrat úrokové míry k dlouhodobému průměru.

Uvažujme
\begin{equation*}
dr = [\theta(t) - ar]dt+\sigma dz
\end{equation*}
kde $a$ a $\sigma$ jsou konstanty. Funkce $\theta(t)$ může být vypočtena z počáteční struktury úrokových sazeb.
\begin{equation}
\theta(t) = F_t(0,t) + aF(0,t) + \frac{\sigma^2}{2a}(1-e^{-2at})
\end{equation}
Poslední člen ve výše uvedené rovnici je zpravidla zanedbatelný. Jeho vynecháním se proces, který sleduje $r$, zjednoduší na
\begin{equation*}
r(t) = F_t(0,t) + a[F(0,t)-r]
\end{equation*}
To znamená, že $r$ sleduje v průměru sklon křivky počátečních forwardových sazeb. Když se $r$ odlišuje od této křivky, konveruje zpět k této křivce rychlostí $a$.

Cena diskontního dluhopisu v čase $t$ podle Hull-Whitova modelu je dána rovnicí
\begin{equation*}
P(t,T)=A(t,T)e^{-B(t,T)r(t)}
\end{equation*}
kde
\begin{equation*}
B(t,T) = \frac{1-e^{a(T-t)}}{a}
\end{equation*}
a
\begin{equation*}
\ln A(t,T) = \ln \frac{P(0,T)}{P(0,t)}-B(t,T)\frac{\partial \ln P(0,t)}{\partial t}-\frac{1}{4a^3}\sigma^2(e^{-aT}-e^{-at})^2(e^{2at}-1)
\end{equation*}
Stejně jako v případě Ho-Lee modelu je relevantnější vztáhnout hodnotu dluhopisu k $R(t)$, tj. sazbě, která odpovídá periodě $\delta t$ v čase $t$.
\begin{equation*}
P(t,T) = \hat{A}(t,T)e^{-\hat{B}(t,T)R(t)}
\end{equation*}
kde
\begin{equation*}
\ln \hat{A}(t,T) = \ln \frac{P(0,T)}{P(0,t)}-\frac{B(t,T)}{B(t,t+\delta t)}\ln \frac{P(0,t+\delta t)}{P(0,t)}-\frac{\sigma^2}{4a}(1-e^{-2at})B(t,T)[B(t,T)-B(t,t + \delta t)]
\end{equation*}
\begin{equation*}
\hat{B}(t,T)=\frac{B(t,T)}{B(t,t+\delta t)}\delta t
\end{equation*}

Cena evropské kupní opce, která maturuje v čase $T$, na diskontní dluhopis se splatností v čase $s$ je v čase nula dána vztahem
\begin{equation*}
LP(0,s)N(h)-KP(0,T)N(h-\sigma_p)
\end{equation*}
kde $L$ představuje nominále dluhopisu a $K$ je realizační cenou. Ostatní parametry jsou definovány jako
\begin{equation*}
h = \frac{1}{\sigma_p} \ln \frac{LP(0,s)}{P(0,T)K}+\frac{\sigma_p}{2}
\end{equation*}
\begin{equation*}
\sigma_p = \frac{\sigma}{a}[1-e^{-a(s-T)}]\sqrt{\frac{1-e^{-2aT}}{2a}}
\end{equation*}
Cena odpovídající prodejní opce se vypočte podle
\begin{equation*}
KP(0,T)N(-h + \sigma_p)-LP(0,s)N(-h)
\end{equation*}

Volatilita je v Hull-Whitovém modelu determinována parametry $a$ a $\sigma$. Volatilita ceny diskontního dluhopisu, který je splatný v čase $T$, je definována v čase $t$ jako
\begin{equation*}
\frac{\sigma}{a}(1-e^{-a(T-t)})
\end{equation*}
Směrodatná odchylka odpovídající výnosové míry je pak dána vztahem
\begin{equation*}
\frac{\sigma}{a(T-t)}(1-e^{-a(T-t)})
\end{equation*}
Volatilita forwardové sazby s maturitou v čase $T$ je
\begin{equation*}
\sigma e^{-a(T-t)}
\end{equation*}
Parametr $\sigma$ představuje volatilitu krátkodobých úrokových sazeb. Parametr $a$ představuje míru, se kterou volatility cen dluhopisů rostou resp. míru se kterou volatilita úrokových sazeb klesá s narůstající splatností.

\subsection{Opce na kupónové dluhopisy}

V předchozí kapitole jsme ukázali na příkladě Vašíčkova modelu, že je možné opci na kupónový dluhopis rozložit na sérii dílčích opcí na diskontní dluhopis. V následujícím textu aplikujeme tento postup Ho-Lee a Hull-Whitův model.

V případě Vašíčkova modelu jsme vypočetli hodnotu krátkodobé úrokové sazby, $r = r_K$, pro kterou se cena kupónového dluhopisu rovnala realizační ceně. V návaznosti na to jsme uvažovali, že opce na kupónový dluhopis je ekvivalentní portfoliu opcí na diskontní dluhopisy, na které lze rozložit původní kupónový dluhopis. Realizační hodnota pro každou z těchto opcí je hodnota odpovídajícího diskontního dluhopisu pro $r = r_K$.

Postup v případě Ho-Lee na Hull-Whitova modelu je identický. Pro účely těchto modelů je však vhodnější pracovat s úrokovou sazbou $R$ pro časový interval $\delta t$ než s aktuální krátkodobou sazbou $r$. Vhodnou volbou pro $\delta t$ je časový interval mezi splatností opce a prvním následujícím kupónem pro podkladový dluhopis. Nejprve vypočteme sazbu $R_K$, pro kterou je cena kupónového dluhopisu rovna realizační ceně. Hodnota uvažované opce je rovna portfoliu opcí na diskontované dluhopisy, na které je možné rozložit původní kupónový dluhopis. Realizační hodnota pro každou z těchto dílčích opcí je rovna hodnotě odpovídajícího diskontního dluhopisu za předpokladu $R = R_K$.

\section{Úrokové stromy}

Úrokový strom je diskrétním znázorněním stochastického procesu, který sledují úrokové sazby. Jestliže je časový posun roven $\delta t$, jsou sazby uvažované v rámci úrokového stromu definovány jako složeně úročené pro časový interval $\delta t$. Obvykle předpokládáme, že sazba $R$ pro časový interval $\delta t$ sleduje stejný náhodný proces jako úroková sazba $r$ v odpovídajícím spojitém modelu. V praxi se často osvědčuje použít namísto klasického binomického stromu strom trinomický.

\subsection{Příklad trinomického úrokového stromu}

Předpokládejme, že pravděpodobnost růstu úrokové sazby je 0.25, pravděpodobnost setrvalého stavu je 0.50 a pravděpodobnost poklesu je 0.25 pro každý krok v rámci trinomického stromu. Tento strom je použit pro ocenění derivátu, který na generuje výplatu
\begin{equation*}
\max[100(R-0.11),0]
\end{equation*}
na konci druhého kroku. Definujme $R$ jako sazbu pro časový interval $\delta t$.
\begin{center}
	\begin{pspicture}(0,0)(7,7)
	  \rput(3.5,0){Úrokový derivát oceněný pomocí trinomického úrokového stromu}

	  \rput(0.2,2.8){\tiny{10\%}}
	  \rput(0.2,2.6){\tiny{0.35}}

	  \rput(3.4,4.4){\tiny{12\%}}
          \rput(3.4,4.2){\tiny{1.11}}
          \rput(3.4,3.4){\tiny{10\%}}
          \rput(3.4,3.2){\tiny{0.23}}
          \rput(3.4,2.4){\tiny{8\%}}
          \rput(3.4,2.2){\tiny{0.0}}
              
          \rput(6.8,5.1){\tiny{14\%}}
          \rput(6.8,4.9){\tiny{3.0}}
          \rput(6.8,4.1){\tiny{12\%}}
          \rput(6.8,3.9){\tiny{1.0}}
          \rput(6.8,3.1){\tiny{10\%}}
          \rput(6.8,2.9){\tiny{0.0}}
          \rput(6.8,2.1){\tiny{8\%}}
          \rput(6.8,1.9){\tiny{0.0}}
          \rput(6.8,1.1){\tiny{6\%}}
          \rput(6.8,0.9){\tiny{0.0}}
          
          \rput(1.2,3.1){\tiny{A}}
          \rput(4.2,4.1){\tiny{B}}
          \rput(4.2,3.1){\tiny{C}}
          \rput(4.2,2.1){\tiny{D}}
          \rput(6.4,4.8){\tiny{E}}
          \rput(6.4,3.8){\tiny{F}}
          \rput(6.4,2.8){\tiny{G}}
          \rput(6.4,1.8){\tiny{H}}
          \rput(6.4,0.8){\tiny{I}}

	  \psline[arrows=->](0.5,3)(3.5,2)
          \psline[arrows=->](0.5,3)(3.5,4)
          \psline[arrows=->](0.5,3)(3.5,3)
		
          \psline[arrows=->](3.5,2)(6.5,3)
          \psline[arrows=->](3.5,2)(6.5,1)
          \psline[arrows=->](3.5,2)(6.5,2)
          
          \psline[arrows=->](3.5,4)(6.5,3)
          \psline[arrows=->](3.5,4)(6.5,5)
          \psline[arrows=->](3.5,4)(6.5,4)

          \psline[arrows=->](3.5,3)(6.5,2)
          \psline[arrows=->](3.5,3)(6.5,4)
          \psline[arrows=->](3.5,3)(6.5,3)             		
	\end{pspicture}
\end{center}
V koncových uzlech trinomického stromu se hodnota derivátu rovná výplatě, kterou generuje. Například pro uzel E platí, že hodnota derivátu je rovna $100 \cdot (0.14 - 0.11) = 3$. V přecházejících uzlech je hodnota derivátu vypočtena oceňováním "pozpátku", tak jak je popsáno v kapitole 15.1.2. V uzlu B je jednoroční úroková sazba 12\%. Jestliže použijeme tuto sazbu pro diskontování pro výpočet hodnoty derivátu v uzlu B z hodnot derivátu v uzlech E, F, G, dostaneme $(0.25 \cdot 3 + 0.5 \cdot 1 + 0.25 \cdot 0)e^{-0.12 \cdot 1} = 1.11$.

\subsection{Nestandardní větvení}
V některých případech se může ukázat žádoucí modifikovat standardní větvení, které je použito ve výše uvedeném obrázku. V rámci tohoto větvení se uvažuje, že úroková sazba může během skoku klesnout, zůstat stejná nebo vzrůst.
\begin{center}
	\begin{pspicture}(0,0)(9,4)
		\rput(4.5,0){Alternativní větvení}
		\rput(1.5,0.5){(a) standardní}
		\rput(4.5,0.5){(b) rostoucí}
		\rput(7.5,0.5){(c) klesající}

		\psline[arrows=->](0.5,2.5)(2.5,3.5)
		\psline[arrows=->](0.5,2.5)(2.5,2.5)
		\psline[arrows=->](0.5,2.5)(2.5,1.5)
		
		\psline[arrows=->](3.5,1.5)(5.5,3.5)
                \psline[arrows=->](3.5,1.5)(5.5,2.5)
                \psline[arrows=->](3.5,1.5)(5.5,1.5)
          
 		\psline[arrows=->](6.5,3.5)(8.5,3.5)
		\psline[arrows=->](6.5,3.5)(8.5,2.5)
		\psline[arrows=->](6.5,3.5)(8.5,1.5)             		
	\end{pspicture}
\end{center}
Alternativní větvení jsou vhodná např. v situacích, kdy je návrat k dlouhodobé úrokové míře v rámci uvažovaného úrokového modelu příliš rychlý resp. příliš pomalý.

\subsection{Obecná metodika úrokových stromů}

Následující kapitola bude vycházet z Hull-Whitova modelu úrokových sazeb. V tomto modelu platí, že krátkodobá úroková sazba sleduje náhodný proces
\begin{equation*}
dr = [\theta(t)-ar]dt + \sigma dz
\end{equation*}
Předpokládejme, že  krok v rámci úrokového stromu je konstatní a roven $\delta t$. Dále předpokládejme, že sazba $R$ pro období $\delta t$ sleduje stejný proces jako $r$.
\begin{equation*}
dR = [\theta(t) - aR]dt + \sigma dz
\end{equation*}

\subsubsection{První krok}

Prvním krokem při konstrukci úrokového stromu je zkonstruovat strom proměnné $R^*$, která má nulovou počáteční hodnotu a která sleduje proces
\begin{equation*}
d R^* = -aR^* dt + \sigma dz
\end{equation*}
Tento proces je symetrický kolem $R^* = 0$ a proměnná $R^*(t + \delta t) - R^*(t)$ má normální rozdělení. Jestliže zanedbáme členy vyššího řádu než $\delta t$, je očekáváná hodnota $R^*(t + \delta t) - R^*(t)$ rovna $-aR^*(t)$ a rozptyl této náhodné veličiny roven $\sigma^2 \delta t$.

Definujme $\delta R$ jako mezery mezi úrokovými sazbami v rámci úrokového stromu a nechť platí
\begin{equation*}
\delta R = \sigma \sqrt{3 \delta t}
\end{equation*}
Nechť $(i,j)$ je uzel, kde $t = i \delta t$ a $R^* = j \delta R$ ($i$ musí být kladné, $j$ může být kladné i záporné).

Další podmínkou, kterou je nutné splnit je, aby pravděpodobnost každé větve vedoucí z uvažovaného uzlu byla kladná. Toho lze docílit vhodnou volbou větvení. Ve většině případů je aplikovatelné standardní větvení. Pro $a > 0$ je pro dostatečně velká $j$ nutné přejít na větvení klesající. Je-li naopak $a < 0$ je nutné pro dostatačně záporná $j$ použít rostoucí větvení. Definujme $j_{max}$ jako hodnotu $j$, pro kterou je nutné přejít ze standardního na klesající větvení a hodnotu $j_{min}$ jako hodnotu $j$, pro kterou je nutné přejít ze standardního na rostoucí větvení. Lze dokázat, že výsledné pravděpodobnosti jsou vždy kladné, jestliže nastavíme $j_{max}$ na hodnotu odpovídající nejmenšímu celému číslu většímu než $0.184/(a \delta t)$ a $j_{min}$ na hodnotu $-j_{max}$.

Definujme $p_u$, $p_d$ a $p_m$ jako pravděpodobnosti, že se úroková sazba bude sledovat jednu ze tří uvažovaných větví. Tyto pravděpodobnosti by měly být vybrány tak, aby odpovídaly očekávané změně a rozptylu $R^*$ v následujícím časovém intervalu $\delta t$. Další podmínkou je, že součet těchto pravděpodobností musí být roven 1. Tímto získáváme tři rovnice o třech neznámých.

Jak již bylo zmíněno, očekávaná změna $R^*$ v čase $\delta t$ je rovna $-aR^* \delta t$ a rozptyl této změny je roven $\sigma^2 \delta t$. V uzlu $(i,j)$ platí $R^* = j \delta R$. Jestliže se použije standardní větvení, musí pravděpodobnosti $p_u$, $p_m$ a $p_d$ v uzlu $(i,j)$ splňovat následující tři rovnice.
\begin{equation*}
p_u \delta R - p_d \delta R = -a j \delta R \delta t
\end{equation*}
\begin{equation*}
p_u \delta R^2 + p_d \delta R^2 = \sigma^2 \delta t + a^2 j^2 \delta R^2 \delta t^2
\end{equation*}
\begin{equation*}
p_u + p_m + p_d = 1
\end{equation*}
Jestliže $\delta R = \sqrt{3 \delta t}$, získáváme
\begin{equation*}
p_u = \frac{1}{6} + \frac{a^2 j^2 \delta^2 - a j \delta t}{2}
\end{equation*}
\begin{equation*}
p_m = \frac{2}{3}- a^2 j^2 \delta t^2
\end{equation*}
\begin{equation*}
p_d = \frac{1}{6} + \frac{a^2 j^2 \delta t^2 + a j \delta t}{2}
\end{equation*}
Jestliže bychom použili rostoucí větvení, byly pravděpodobnosti definovány jako
\begin{equation*}
p_u = \frac{1}{6} + \frac{a^2 j^2 \delta t^2 + a j \delta t}{2}
\end{equation*}
\begin{equation*}
p_m = - \frac{1}{3}-a^2 j^2 \delta t^2 - 2aj \delta t
\end{equation*}
\begin{equation*}
p_d = \frac{2}{6} + \frac{a^2 j^2 \delta t^2 + 3aj \delta t}{2}
\end{equation*}
Pro klesající větvení musí pravděpodobnosti splňovat podmínky
\begin{equation*}
p_u = \frac{2}{6} + \frac{a^2 j^2 \delta t^2 - 3aj \delta t}{2}
\end{equation*}
\begin{equation*}
p_m = - \frac{1}{3}-a^2 j^2 \delta t^2 + 2aj \delta t
\end{equation*}
\begin{equation*}
p_d = \frac{1}{6} + \frac{a^2 j^2 \delta t^2 - a j \delta t}{2}
\end{equation*}

\subsubsection{Druhý krok}

V kruhém kroku je třeba konvertovat strom pro $R^*$ na strom pro $R$. Definujme
\begin{equation*}
\alpha (t) = R(t) - R^*(t)
\end{equation*}
Protože $R$ a $R^*$ jsou definována jako
\begin{equation*}
dR = [\theta(t) - aR]dt + \sigma dz
\end{equation*}
\begin{equation*}
dR^* = -aR^*dt + \sigma dz
\end{equation*}
$d \alpha(t)$ je tedy možné definovat jako
\begin{equation*}
d \alpha = [\theta(t) - a \alpha(t)]dt
\end{equation*}
Jestliže budeme ignorovat rozdíl mezi $r$ a $R$, pak lze s pomocí (20.5) vyjádřit $\alpha (t)$ jako
\begin{equation}
\alpha(t) = F(0,t) + \frac{\sigma^2}{2a^2}(1-e^{-at})^2
\end{equation}
Pro $a$ limitně se blížící nule platí
\begin{equation*}
\alpha (t) = F(0,t) + \frac{\sigma^2 t^2}{2}
\end{equation*}
Vzhledem k $R(t) = R^*(t) + \alpha(t)$, je možné použít rovnici (20.6) pro vytvoření stromu pro $R$.

\subsubsection{Alternativní přístup}

Ačkoliv je výše uvedená rovnice pro většinu účelů dostatečná, není zcela konzistentní s počáteční strukturou úrokových sazeb. Alternativním přístupem je iterativní výpočet parametrů $\alpha$.

Definujme $\alpha(i \delta t)$ jako rozdíl mezi $R$ a $R^*$ v čase $i \delta t$. Definujme $Q_{i,j}$ jako současnou hodnotu investičního instrumentu, který generuje 1 USD v případě, že je dosaženo uzlu $(i,j)$ a 0 USD v ostatních případech. Hodnoty $Q_{0,0}$ je tedy rovna 1. Hodnoty ostatních $Q_{i,j}$ jsou dány pravděpodobnostmi $p_u$, $p_m$ a $p_d$. Parametr $\alpha_0$ je zvolen tak, aby diskotní dluhopis se splatností v čase $\delta t$ byl oceněn správně. Jestliže je tedy délka kroku v rámci stromu rovna jednomu roku, odpovídá $\alpha_0$ jednoroční zero sazbě.

Předpokládejme, že $Q_{i,j}$ bylo vypočteno pro $i \le m$ ($m \ge 0$). Dalším krokem je výpočet $\alpha_m$, tak aby výsledný strom správně oceňoval diskontní dluhopisy se splatností v čase $(m+1) \delta t$. Úroková sazba v uzlu $(m,j)$ je $\alpha_m + j \delta R$. Cena diskontního dluhopisu je tedy rovna
\begin{equation*}
P_{m+1} = \sum^{n_m}_{j=-n_m} Q_{m,j}e^{-(\alpha_m + j \delta R) \delta t}
\end{equation*}
kde $n_m$ je počet uzlů na každé straně centrálního uzlu v čase $m \delta t$. Hodnota $\alpha_m$, které je řešením výše uvedené rovnice, je rovna
\begin{equation*}
\alpha_m = \frac{\ln \sum^{n_m}_{j = -n_m}Q_{m,j}e^{-j \delta R \delta t} - \ln P_{m+1}}{\delta t}
\end{equation*}
Po té, co je vypočteno $\alpha_m$, je možné dopočítat $Q_{i,j}$ pro $i = m + 1$.
\begin{equation*}
Q_{m+1,j} = \sum_k Q_{m,k}q(k,j)e^{-(\alpha_m + k \delta R) \delta t}
\end{equation*}
kde $q(k,j)$ představuje pravděpodobnost přesunu z uzlu $(m,k)$ do uzlu $(m+1,j)$;  sumace je provedena přes všechna nenulová $k$.

\subsection{Kalibrace}

Až dosud jsme uvažovali, že hodnoty parametrů $a$ a $\sigma$, ať už se jedná o konstanty nebo o funkce času, jsou známé. V následujícím textu se budeme zabývat možnostmi jejich výpočtu.

Parametry volatility jsou určeny z tržních dat na aktivně obchodované opce. Tyto opce nazýváme kalibračními instrumenty. Obecně platí, že kalibrační nástroj by měl být vybrán tak, aby se v co největší míře shodoval s oceňovaným instrumentem. Prvním krokem kalibrace je volba vhodného měřítka dobré shody. Oblíbeným měřítkem je
\begin{equation*}
\sum_{i=1}^n (U_i - V_i)^2
\end{equation*}
kde $U_i$ je tržní cena $i$-tého kalibračního instrumentu a $V_i$ je cena daná modelem pro tento instrument. Volbou parametrů modelu je pak možné dosáhnout minimalizace výše uvedeného výrazu.

Jestliže jsou $a$ a $\sigma$ konstanty, jsou k dispozici pouze dva zdroje volatility. To v řadě případů nestačí a je nutné uvažovat $a$ a $\sigma$ jako funkci času. Vhodným způsobem, jak modelovat $a$ a $\sigma$ jako funkce času je použít krokovou funkci. Uvažujme například situaci, kdy $a$ bude konstatní a $\sigma$ funkcí času. Můžeme zvolit časy $t_1$, $t_2$, ...,$t_n$ a předpokládat, že $\sigma (t) = \sigma_0$ pro $t \le t_1$, $\sigma (t) = \sigma_i$ pro $t_i <t \le t_{i+1}$ ($1 \le i \le n - 1$) a $\sigma (t) = \sigma_n$ pro $t > t_n$. Celkově tedy budeme mít k dispozici $n+2$ parametrů volatility a to $a$, $\sigma_0$, $\sigma_1$, ..., $\sigma_n$. Počet parametrů volatility musí být vždy menší než počet kalibračních nástrojů. Minimalizace měřítka dobré shody je možné dosáhnout pomocí Levenberg-Marquardtovy procedury.

Jsou-li $a$ i $\sigma$ funkcí času, je měřítko dobré shody často doplňováno o tzv. penalizační funkci. Měřítko dobré shody pak může vypadat například následovně
\begin{equation*}
\sum^{n}_{i=1}(U_i - V_i)^2 + \sum^n_{i=1}w_{1,i}(\sigma_i - \sigma_{i-1})^2 + \sum^{n-1}_{i=1}w_{2,i}(\sigma_{i-1}+\sigma_{i+1}-2 \sigma_i)^2
\end{equation*}
Druhý člen představuje penalizaci velkých změn parametru $\sigma$ mezi jednotlivými kroky. Třetí člen představuje penalizaci za vysokou křivost. Vhodné hodnoty parametrů $w_{1,i}$ a $w_{2,i}$ jsou dány empirickou zkušeností a jsou voleny tak, aby poskytly přijatelnou úroveň hladkosti funkce parametru $\sigma$.

\subsection{Zajištění a jednoparametrový model}

Ačkoliv velice často používáme jednoparametrový model při oceňování úrokových derivátů, není tento model vhodný pro účely zajištění se. V praxi se velice často berou v úvahu nejen změny, které mohou v rámci modelu nastat, ale také změny, které nastat nemohou. Hovoříme o tzv. vnějším modelovém zajištění a jedná se o běžnou praxi mezi obchodníky. Platí totiž, že jednofaktorové modely, jsou-li vhodně aplikovány, mohou poskytovat relativně rozumné ocenění, avšak pro účely zajištění se je třeba uvažovat více faktorů. Klasickým příkladem je předpoklad konstatní volatility v rámci Black-Scholes modelu, kdy však obchodníci sledují hodnotu řeckého písmene vega pro své portfolio a zajišťují se proti možným změnám volatility.

\subsection{Forwardové a futures sazby}

Jak bylo zmíněno v kapitole 5.9, nejsou forwardové a futures sazba totožné. Platí, že futures sazba je vždy vyšší než forwardová sazba.

Uvažujme futures kontrakt na úrokovou sazbu, který trvá od $t_1$ do $t_2$. V souladu s výše uvedeným je v rámci Ho-Lee modelu  spojitá forwardová úroková sazba pro danou periodu rovna spojité futures úrokové sazbě méně $\sigma^2 t_1 t_2 / 2$. Typická hodnota $\sigma$ je 0.012. Pro eurodolarové futures obchodované na CME platí $t_2 = t_1 + 0.25$. Typická korekce, kterou musíme provést pro splatnosti 2, 4, 6, 8 a 10 let je proto 3, 12, 27, 48 a 74 bazických bodů.

V rámci Hull-Whitova modelu je hodnota korekce, kterou je třeba odečíst od spojité futures úrokové sazby, abychom získali spojitou forwardovou úrokovou sazbu, rovna
\begin{equation*}
\frac{B(t_1, t_2)}{t_2 - t_1}[B(t_1, t_2)(1 - e^{-2at} + 2aB(0,t)^2]\frac{\sigma^2}{4a}
\end{equation*}
kde $B$ je definováno jako
\begin{equation*}
B(t,T)=\frac{1 - e^{a(T-t)}}{a}
\end{equation*}

\section{Swapové obchody}

\subsection{Deriváty swapových obchodů}

V předchozích kapitolách jsme uvažovali pouze základní (tzv. plain vanila) typy swapových obchodů. Tyto základní typy obchodů je pak možné nejrůznějším způsobem modifikovat. Například u některých swapových obchodů se nominál mění deterministicky v čase. Swapy, kde je nominál rostoucí funkcí času, nazýváme rostoucí swapy (step-up swap). Swapy, jejichž nominál je naopak klesající funkcí času, označujeme jako amortizační swapy (amortizing swap). Tento typ swapů je pak možné např. použít pro zajištění anuitně spláceného uvěru. Další možnou modifikací je různý nominál nebo frekvence plateb na každé ze swapových nohou. Referenční floatová sazba také nemusí být vždy LIBOR. Například v rámci tzv. basis swapu dochází k výměně cash-flow počítaného podle jedné floatové sazby za cash-flow počítané podle jiné floatové sazby.

Následující podkapitoly popisují základní modifikace klasického úrokového swapového obchodu.

\subsubsection{Složený swap}

V případě složeného swapu (compounding swap) existuje pouze jedno výplatní datum pro fixní a floatovou nohu. Výplata je realizována v den splatnosti swapu. Úroky jsou namísto toho, aby byly průběžně vypláceny, složeně úročeny do konce životnosti příslušného swapu. Jako příklad uvažujme složený swap, jehož fixní sazba je 6\% a úrok, namísto toho, aby byl vyplácen, je dále úročen fixní sazbou 6.3\%. Floatová sazba uvažovaného swapu je LIBOR navýšený o 20 bazických bodů a úrok z floatové nohy je úročen sazbou LIBOR plus 10 bazických bodů.

Tento typ swapu je možné ocenit za překpokladu, že se budoucí úrokové sazby budou rovny forwardovým sazbám. Uvažujme $t_0$ jako datum bezprostředně předcházející datu ocenění. Předpokládejme, že výplatní data po datu ocenění jsou $t_1$, $t_2$,... , $t_n$. Definujme $\tau_i = t_{i+1} - t_i$ kde $0 \le i \le n - 1$ a následující proměnné jako

\begin{center}
\begin{tabular}{l l}
$L$ & nominál floatové swapové nohy\\
$Q_i$ & hodnota floatové nohy složeně úročené k $t_i$ ($Q_0$ je známo)\\
$R_i$ & LIBOR sazba pro časový horizont $t_i$ až $t_{i+1}$ ($R_0$ je známo)\\
$F_i$ & forwardová sazba aplikovaná na periodu $t_i$ až $t_{i+1}$ pro $i \ge 1$ (všechny\\
$ $ & hodnoty jsou známy)\\
$s_1$ & spread nad refereční sazbu LIBOR, který je použit pro výpočet úroku\\
$s_2$ & spread nad refereční sazbu LIBOR, kterým je složeně úročen úrok\\
\end{tabular}
\end{center}
Výpočet hodnoty fixní nohy je triviální, protože cash-flow generované touto nohou je dopředu známo. Hodnota floatové nohy v čase $t_1$ je známa.
\begin{equation*}
Q_1 = Q_0[1+(R_0 + s_2)\tau_0]+L(R_0 + s_1)\tau_0
\end{equation*}
Hodnota floatové nohy v čase $t_2$ není známá a závisí na $R_1$.
\begin{equation*}
Q_2 = Q_1[1 + (R_1 + s_2)\tau_1]+L(R_1 + s_1)\tau_1
\end{equation*}
Je však možné vstoupit do dvou FRA
\begin{itemize}
\item v rámci prvního FRA vyměnit $R_1 + s_2$ za $F_1 + s_2$ ve vztahu k nominálu $Q_1$
\item v rámci druhého FRA vyměnit $R_1 + s_1$ za $F_1 + s_1$ ve vztahu k nominálu $L$
\end{itemize}
S využitím výše zmíněných FRA lze hodnotu floatové nohy v čase $t_2$ vyjádřit jako
\begin{equation*}
Q_2 = Q_1[1+(F_1 + s_2)\tau_1]+L(F_1 + s_1)\tau_1
\end{equation*}
Obecně pak platí
\begin{equation*}
Q_{i+1} = Q_i[1+(F_i + s_2)\tau_i] + L(F_i + s_i)\tau_i
\end{equation*}
Výsledná hodnota složeného swapu je tak dána současnou hodnotou rozdílu fixní a floatové nohy.

\subsubsection{Dopředný swap}

U klasického swapového obchodu je výše referenční floatové sazby pozorovaná k jednomu výplatnímu datu použita pro určení výše cash-flow k následujícímu výplatnímu datu. V případě dopředného swapu je k výpočtu výše cash-flow pro daný výplatní den použita hodnota refereční sazby pozorovaná v tento den.\\

Uvažujme datum $t_i$ pro stanovení floatové sazby, kde $i = 0, 1,...,n$ a $\tau_i = t_{i+1}-t_i$. Definujme $R_i$ jako LIBOR sazbu platnou pro periodu $t_i$ až $t_{i+1}$, $F_i$ jako forwardovou hodnotu $R_i$ a $\sigma_i$ jako volatilitu forwardové sazby\footnote{Hodnota $\sigma_i$ je typicky získávána z cen capletů.}. V případě dopředného swapu je platba na floatové noze v čase $t_i$ založena na $R_i$ spíše než na $R_{i-1}$ jak je tomu u klasického swapu. Jak již bylo vysvětleno dříve, je nutné pro účely stanovení cash-flow provést konvexní úpravu forwardové sazby. Výpočet cash-flow z floatové nohy by měl být založen na předpokladu, že forwardová sazba je rovna
\begin{equation*}
F_i + \frac{F^2_i \sigma^2_i \tau_i t_i}{1 + F_i \tau_i}
\end{equation*}
spíše než $F_i$. Pro pro účely diskontování je však použita sazba $F_i$.

\subsubsection{CMS swap}

CMS swap (constant maturity swap) je úrokový swap, kde je jako refereční floatová sazba zvolena swapová sazba plain vanilla úrokového swapu s určitou splatností (např. swapová sazba z pětiletého úrokového swapu). Podobně jako u úrokového swapu dochází také u CMS swapu zpravidla k fixaci úrokových sazeb na floatové noze. To znamená, že cash-flow generované floatovou nohou k určitému datu se odvíjí od swapové sazby přecházejícího výplatního data.\\

Předpokládejme, že sazby na floatové noze jsou fixovány k datům $t_0, t_1, t_2, ...$ a že odpovídající platby jsou realizovány k datům $t_1, t_2, t_3,...$. Cash-flow generované floatovou nohou ve výplatní den $t_{i+1}$ je tak rovno $\tau_i L S_i$, kde $\tau_i = t_{i+1}-t_i$ a $S_i$ je referenční swapová sazba v čase $t_i$. Dále předpokládejme, že $y_i$ je forwardová hodnota swapové sazby $S_i$. Abychom správně ocenili platbu v čase $t_{i+1}$, je žádoucí vzít v potaz konvexitu a to, že platby vypočtené na základě $S_i$ jsou realizovány až v čase $t_{i+1}$. Uvažovaná swapová sazba je tak rovna
\begin{equation*}
S_i = y_i - \frac{1}{2}y^2_i \sigma^2_{y,t_i}\frac{G''(y_i)}{G'(y_i)}-\frac{y_i \tau_i F_i \rho_i \sigma_{y,i}\sigma_{F,i}t_i}{1 + F_i \tau_i}
\end{equation*}
kde druhý sčítanec na pravé straně rovnice představuje úpravu z titulu konvexity a třetí sčítanec pak úpravu z titulu časového posunu plateb. Ve výše uvedené rovnici figuruje $\sigma_{y,i}$ jako volatilita forwardové swapové sazby, $F_i$ je současnou forwardovopu úrokovou sazbou pro období $t_i$ až $t_{i+1}$ a $\sigma_{F,i}$ je její volatilitou. Proměnná $\rho_i$ představuje korelaci mezi forwardovou swapovou sazbou a forwardovou úrokovou sazbu. Funkce $G_i(x)$ je cenou v čase $t_i$ pro dluhopis s výnosovou mírou $x$. Tento modelový dluhopis vyplácí kupóny odpovídající sazbě $y_i$ a má stejnou životnost a frekvenci plateb jako úrokový swap, od kterého je odvozena CMS swapová sazba. Volatilita $\sigma_{y,i}$ může být odvozena ze swapcí, volatilita $\sigma_{F,i}$ z cen capletů a korelace $\rho_i$ může být vypočtena na základě historických dat.

\subsubsection{Akciový swap}

U akciového swapu dochází ke směně výnosu z akciového indexu za pevnou popř. pohyblivou úrokovou sazbu. Tento swap tak umožňuje portfolio manažerům zvýšit nebo snížit expozici na index bez toho aniž by prodávali nebo nakupovali akcie. Výnos akciového indexu je zpravidla počítán za předpokladu, že dividendy jsou reinvestovány zpět do akcií tvořících index.\\

Uvažujme swap, je jehož rámci dochází ke směně výnosu z akciového indexu za pohyblivou refereční sazbu. Takovýto swap má na začátku své životnosti nulovou tržní hodnotu. Důvodem je to, že finanční instituce může velice snadno replikovat cash-flow z akciového swapu tak, že si půjčí příslušný nominál ve výplatní den za sazbu odpovídající LIBOR, použije ji na nákup akciového indexu do dalšího výplatního dne a případné dividendy reinvestuje zpět do indexu. Tato úvaha mimojiné implikuje, že tržní hodnota takového swapu je bezprostředně po každém výplatním dni nulová.\\

Pro ocenění akciového swapu mezi dvěma výplatními daty definujme
\begin{center}
\begin{tabular}{l l}
$R_0$ & floatová sazba aplikovaná k příštímu výplatnímu datu\\
 & (fixována v poslední výplatní den)\\
$L$ & nominální hodnota akciového swapu\\
$\tau_0$ & čas mezi výplatním dnem a příštím výplatním dnem\\
$\tau$ & časová perioda mezi současností a následujícím\\
 & platebním dnem\\
$E_0$ & hodnota akciového indexu k poslednímu datu fixace\\
$E$ & současná hodnota akciového indexu\\
$R$ & sazba LIBOR platná pro období mezi současností\\
 & a příštím výplatním dnem\\
\end{tabular}
\end{center}
Jestliže si půjčíme $\frac{E}{E_0}L$ za sazbu $R$ na časové období $\tau$ a tuto částku investujeme do akciového indexu, jedná se výměnu $\frac{E_1}{E_0}L$ za $\frac{E}{E_0}L(1 + R \tau)$, která proběhne k příštímu výplatnímu dni. Vzhledem k tomu, že tato výměna je pro případného investora spojena s nulovými náklady, je její současná  hodnota rovna nule. V rámci akciového swapu dojde k příštímu výplatnímu dni v faktické výměně $\Big( \frac{E_1}{E_0} - 1 \Big)L$ za $R_0 L \tau$, což je identické s výměnou $\frac{E_1}{E_0}$ za  $L(1 + R_0 \tau_0)$. Porovnáním obou směn získáme, že hodnota akciového swapu pro stranu, která obdrží cash-flow z floatové nohy, je rovna
\begin{equation*}
L(1 + R_0 \tau_0)-L \frac{E}{E_0}(1 + R \tau)
\end{equation*}
neboli
\begin{equation*}
L \frac{1 + R_0 \tau_0}{1 + R \tau} - L \frac{E}{E_0}
\end{equation*}
Podobně, hodnota akciového swapu pro stranu, která obdrží cash-flow z fixní nohy, je rovna
\begin{equation*}
L \frac{E}{E_0}(1 + R \tau) - L(1 + R_0 \tau_0)
\end{equation*}

\subsubsection{Akruální swap}

Akruální swap je swap, kdy úrok z jedné nohy ``nabíhá'' pouze tehdy, nachází-li se refereční sazba určitém předem stanoveném koridoru.\\

Uvažujme swap v rámci kterého je fixní sazba $Q$ směňována za tříměsíční LIBOR na konci čtvrtletí. Předpokládejme, že cash-flow z fixní nohy je vypláceno pouze v případě, že tříměsíční LIBOR pro daný den nepřesáhne 8\% p.a. Nechť je nominál swapu $L$. V případě klasického swapu by fixní noha generovala ve výplatní den cash-flow $QLn_1/n_2$, kde $n_1$ je počet dní v předcházejícím čtvrtletí a $n_2$ je počet dní v roce. V případě akruálního swapu je toto cash-flow rovno $QLn_3/n_2$, kde $n_3$ je počet dní v předchozím čtvrtletí, kdy referenční sazba nepřesáhla 8\%. Cash-flow z akruálního swapu je tak možné replikovat pomocí klasického swapu a sérií binárních opcí (jednou pro každý den života swapového obchodu). Binární opce by generovaly cash-flow $QL/n_2$ za každý den, kdy je referenční sazba nad stanoveným limitem 8\%.

Zobecněme výše uvažovaný příklad. Uvažujme limit pro refereční sazbu $R_K$ (tj. 8\% ve výše uvažovaném příkladě). Předpokládejme, že platby jsou směněny každých $\tau$ let. Dále uvažujme den $i$ v rámci života swapu. Definujme $t_i$ jako čas do dne $i$. Nechť $\tau$-roční sazba LIBOR v den $i$ je $R_i$. Úrok pak nabíhá v případě, že $R_i < R_K$. Definujme $F_i$ jako forwardovou sazbu a $\sigma_i$ jako volatilitu $F_i$. Ve světě, který je forwardově rizikově neutrální vzhledem k dluhopisu se splatností v $t_i + \tau$, je za předpokladu lognormálního rozdělení sazeb pravděpodobnost, že sazba LIBOR bude větší než $R_K$, rovna $N(d_2)$, kde
\begin{equation*}
d_2 = \frac{\ln(F_i/R_K)-\sigma^2_i t_i/2}{\sigma_i \sqrt{t_i}}
\end{equation*}
Výplata z binární opce je realizována ve výplatní den akruálního swapu, který následuje den $i$. Označme tento den jako $s_i$. Ve světě, který je forwardově rizikově neutrální s ohledem na diskontní dluhopis se splatností v čase $s_i$, je pravděpodobnost, že sazba LIBOR je větší než $R_K$, dána $N(d^*_2)$, kde $d^*_2$ je počítáno podle stejného vzorce jako $d_2$ s tím, že je třeba provést drobnou korekci $F_i$, která odráží rozdíl mezi $t_i + \tau$ a $s_i$. Hodnota binární opce odpovídající dni $i$ je pak
\begin{equation*}
\frac{QL}{n_2}P(0,s_i)N(d^*_2)
\end{equation*}
Celková hodnota binárních opcí je pak získána součtem výše uvedeného výrazu přes všechny dny životnosti swapu. Časová korekce (náhrada $d_2$ za $d^*_2$) je tak malá, že je v praxi často ignorována.

\subsubsection{Svolatelný swap}

Svolatelný swap (cancelable swap) je  klasický swap, kde jedna strana má právo ukončit obchod v jeden popř. vícero výplatních dní. Ukončení swapového obchodu je ekvivaletní k souběžnému uzavření identického protiswapu. Jestliže je existuje pouze jeden den, ke kterému je možné obchod ukončit, je možné svolatelný swap replikovat pomocí klasického swapu a pozicí v evropské swapové opci.

\subsubsection{Svolatelný složený swap}

Někdy je možné složené swapy ukončit v určitý výplatní den. V tomto případě hovoříme o tzv. svolatelném složeném swapu (cancelable compounding swap). V den ukončení generuje floatová noha složeně úročené cash-flow do dne svolání swapového obchodu. Fixní noha pak generuje složeně úročené cash-flow, které odpovídá fixní sazbě do téhož dne.\\

Předpokládejme nejprve, že floatová sazba je LIBOR a že úroky jsou úročeny také sazbou LIBOR. Dále předpokládejme, že na konci životnosti swapu je výplácen nominál na fixní i floatové noze. Tento předpoklad, který nezmění hodnotu swapu, má za následek, že hodnota floatové nohy je ve výplatní den vždy rovna nominálu swapového obchodu. Pro účely rozhodování se o zrušení swapového obchodu je tedy možné se soustředit pouze na fixní nohu.

Nejprve je třeba zkonstruovat úrokový strom tak, jak bylo popsáno v předchozí kapitole. Pak se fixní noha oceňuje pomocí tohoto stromu od jeho konce směrem k vrcholu stejně jako v případě amerických opcí. V každém uzlu, ve kterém může být swap zrušen, se kontroluje, zda-li je jeho zrušení optimální. Jesliže jsme stranou, která plní z fixní nohy, je cílem minimalizovat hodnotu fixní nohy. Jestliže plníme z floatové nohy, je naopak cílem hodnotu fixní nohy maximalizovat.

Jesliže je floatová sazba definovaná jako LIBOR navýšená o spread, je možné odečíst cash-flow odpovídající tomuto spreadu od fixní nohy. Swap pak lze ocenit stejně jako ve výše uvedeném případě.

Jestliže je složené úročení cash-flow prováděné sazbou LIBOR navýšenou o spread, je postup následující:
\begin{itemize}
\item vypočteme hodnotu floatové nohy pro každé datum, ke kterému je možné zrušit obchod, za předpokladu, že budoucí úrokové sazby jsou rovny forwardovým sazbám
\item vypočteme hodnotu floatové nohy pro každé datum, ke kterému je možné zrušit obchod, za předpokladu, že floatová sazba je LIBOR a že úroky jsou složeně úročeny opět sazbou LIBOR
\item definujeme rozdíl mezi hodnotou swapu podle bodu (1) a (2) jako hodnotu spreadu pro dané datum, ke kterému je možné zrušit obchod
\item posuzujeme vnořenou opci stejným způsobem jako ve výše popsaném postupu; v případě, že se budete rozhodovat o zrušení opce, odečtěte hodnotu spreadu od hodnot vypočtených pro fixní nohu
\end{itemize}

\chapter{Kreditní riziko}

Hodnocením kreditního rizika korporátních dluhopisů se zabývají společnosti Moody's a S\&P. Například podle ratingové stupnice používané společností Moody's je nejlepší rating Aaa. U dluhopisů s tímto ratingem se předpokládá téměř nulová pravděpodobnost vzniku defaultní události. Jinými slovy investování do těchto dluhopisů by mělo obnášet nulové kreditní riziko. Další ratingové stupně jsou Aa, A, Baa, Ba, B a Caa. Pouze ratingové stupně Baa a vyšší jsou považovány za investiční stupně. Ratingová stupnice společnosti S\&P je velmi podobná výše uvedené. Odpovídající ratingové stupně jsou AAA, AA, A, BBB, BB, B a CCC.

Obchodníci shromažďují data pro aktivně obchodované dluhopisy, ze kterých pak pro každou ratingovou skupinu počítají zero křivku. Tyto křivky jsou pak použity při oceňování jiných dluhopisů ve stéjné ratingové skupině. Následující graf zobrazuje spread zero křivek pro různé ratingové skupiny proti bezrizikové zero křivce zkonstruované ze státních dluhopisů.
\begin{center}
	\begin{pspicture}(0,0)(10,7)
		\rput(5,0.5){Spread zero křivek pro vybrané ratingové skupiny}
		\rput(5,0.0){proti bezrizikové zero křivce}

		\psline[arrows=->](0.5,1.2)(9.5,1.2)
		\psline[arrows=->](0.5,1.2)(0.5,6.5)
		\rput(0.3,1.0){$0$}
		\rput(9.0,1.5){\small{$splatnost$}}
		\rput(1.2,6.5){\small{$spread$}}
		
		\pscurve(1.0,2.0)(4.0,2.2)(9.0,2.5)
		\pscurve(1.0,2.4)(4.0,3.0)(9.0,3.5)
		\pscurve(1.0,2.8)(4.0,3.8)(9.0,4.4)
		\pscurve(1.0,3.5)(3.5,4.7)(9.0,5.7)
		
		\rput(9.0,2.7){\tiny{$Aaa/AAA$}}
		\rput(9.0,3.7){\tiny{$Aa/AA$}}
		\rput(9.0,4.6){\tiny{$A/A$}}
		\rput(9.0,6.0){\tiny{$Baa/BBB$}}

	\end{pspicture}
\end{center}
Tento spread se zvětšuje s zhoršujícím se ratingových stupněm a prodlužující se splatností. Spread také roste rychleji se splatností pro horší ratingové stupně než pro lepší ratingové stupně.

\section{Očekávaná ztráta z dluhopisů}

Prvním krokem při odhadovaní pravděpodobnosti defaultu dluhopisů je výpočet očekávané ztráty z korporátních dluhopisů pro různé splatnosti. Obvyklým předpokladem je, že současná hodnota očekávané ztráty se rovná rozdílu ceny ekvivalentního bezrizikového dluhopisu a ceny uvažovaného korporátního dluhopisu. Vyšší výnosová míra korporátního dluhopisu tak v rámci této úvahy představuje kompenzaci za případné ztráty z titulu defaultní události.

\begin{center}
\begin{tabular}{c c c c}
\textbf{Splatnost} &
\textbf{Bezriziková} & 
\textbf{Korporátní} &
\textbf{Očekávaná ztráta}\\
\textbf{(roky)} &
\textbf{zero křivka} & 
\textbf{zero křivka} &
\textbf{(\% z nezdefautované hodnoty)}\\
\hline
 1 & 5.00 & 5.25 & 0.2497\\
 2 & 5.00 & 5.50 & 0.9950\\
 3 & 5.00 & 5.70 & 2.0781\\
 4 & 5.00 & 5.85 & 3.3428\\
 5 & 5.00 & 5.95 & 4.6390\\
\hline 
\end{tabular}
\end{center}

Hodnota ročního bezrizikového dluhopisu s nominální hodnotu $100$ je $100e^{-0.05} = 95.1229$. Hodnota obdobného korporátního dluhopisu je $100e^{-0.0525}=94.8854$. Současná hodnota očekávané ztráty z titulu defaulní události je tedy $95.1229 - 94.8854 = 0.2375$. Očekávaná relativní ztráta je tedy $0.2375/95.1229 = 0.2497\%$.\\

\section{Pravděpodobnost defaultu za předpokladu nulové míry náhrady}

Definujme
\begin{center}
\begin{tabular}{l l}
$y(T)$   & výnosová míra $T$-ročního korporátního diskontního dluhopisu\\
$y^*(T)$  & výnosová míra $T$-ročního bezrizikového diskontního dluhopisu\\
$Q(T)$    & pravděpodobnost defaultu společnosti v časovém období 0 až $T$\\
\end{tabular}
\end{center}

Hodnota $T$-ročního bezrizikového diskontního dluhopisu s nominální hodnotou $100$ je $100e^{-y^*(T)T}$ a hodnota obdobného korporátního dluhopisu je $100e^{-y(T)T}$. Očekávaná ztráta z titulu defaultní události za předpokladu nulové míry náhrady je proto
\begin{equation*}
100(e^{-y^*(T)T}-e^{-y(T)T})
\end{equation*}
Výpočet pravděpodobnosti $Q(T)$ je tedy relativně snadný. Korporátní dluhopis bude mít v době splatnosti za předpokladu nulové míry náhrady s pravděpodobností $Q(T)$ hodnotu nula a s pravděpodobností $1-Q(T)$ hodnotu 100. Platí tedy
\begin{equation*}
{Q(T) \cdot 0 + [1-Q(T)] \cdot 100}e^{-y^*(T)T} = 100[1-Q(T)]e^{-y^*(T)T}
\end{equation*}

Výnosová míra uvažovaného korporátního dluhopisu je $y(t)$, a proto zároveň platí
\begin{equation*}
100e^{-y(T)T}=100[1-Q(T)]e^{-y^*(T)T}
\end{equation*}
Pravděpodobnost $Q(T)$ je tedy možné vyjádřit jako

\begin{equation*}
Q(T) = \frac{e^{-y^*(T)T}-e^{-y(T)T}}{e^{-y^*(T)T}}
\end{equation*}

\subsection{Kvantifikace pravděpodobnosti defaultu}

Existují dva způsoby kvantifikování pravděpodobnosti defaultu - míra hazardu a hustota pravděpodobnosti defaultu

Míra hazardu $h(t)$ v čase $t$ je definována tak, že $h(t) \delta t$ je pravděpodobnost defaultu v časovém období $t$ až $t + \delta t$ za podmínky, že před tímto obdobím nedošlo k defaultu. Hustota pravděpodobnosti defaultu $q(t)$ je definována tak, že $q(t) \delta t$ je nepodmíněná pravděpodobnost defaultu v období $t$ až $t + \delta t$ tak, jak se jeví v časovém bodě nula. Jak míra hazardu, tak hustota mohou být použity k popisu pravděpodobnosti defaultu - obě poskytují shodnou informaci. Platí totiž
\begin{equation*}
q(t) = h(t)e^{-\int^{t}_0 h(\tau) d \tau}
\end{equation*}
V následujícím textu budeme spíše než míru hazardu používat hustotu pravděpodobnosti defaultu.

\begin{center}
\begin{tabular}{c c c}
\textbf{Rok} &
\textbf{Kumulativní pravděpodobnost} & 
\textbf{Pravděpodobnost defaultu}\\
\textbf{} &
\textbf{defaultu (\%)} & 
\textbf{v roce (\%)}\\
\hline
 1 & 0.2497 & 0.2497\\
 2 & 0.9950 & 0.7453\\
 3 & 2.0781 & 1.0831\\
 4 & 3.3428 & 1.2647\\
 5 & 4.6390 & 1.2962\\
\hline 
\end{tabular}
\end{center}

Na výše uvedené tabulce budeme demonstrovat souvislost mezi pravděpodobností defaultu a mírou hazardu. Uvažujme pátý rok. Nepodmíněná pravděpodobnost defaultu je dle této tabulky 1.2962\%. Pravděpodobnost, že default nenastane v průběhu prvních čtyř let je pak $1 - 0.033428 = 96.66572\%$. Míra hazardu pro čtyři roky je tedy $0.012962/0.966572 = 1.3410\%$.

\subsection{Míra náhrady}

Až dosud jsme uvažovali nulovou míru náhrady (recovery rate). To znamená, že v případě defaultní události klesla hodnota dluhopisu na nulu. V praxi je však běžnější, že investor získá nazpět alespoň část prostředků. Procento nárokované hodnoty, které takto investor získá, nazýváme mírou náhrady.

Označme míru náhrady jako $R$. Alternativně k výše uvedené rovnici můžeme pravděpodobnost defaultu $Q(T)$ definovat jako
\begin{equation}
Q(T) = \frac{e^{-y^*(T)T} - e^{-y(T)T}}{(1-R)e^{-y^*(T)T}}
\end{equation}

\subsubsection{Přiblížení se realitě}

Rovnice (21.1) je velice často používána jako rychlá aproximace pravděpodobnosti defaultu. Tato rovnice však v sobě nezahrnuje naběhlý úrok a předpokládá, že cena diskontního korporátního dluhopisu je kotovaná na trhu nebo alespoň odvoditelná na základě modelu. V praxi musí být většinou pravděpodobnosti defaultu vypočteny na základě kupónových dluhopisů.

V následujícím textu si popíšeme způsob výpočtu pravděpodobnosti defaultu. Uvažujme skupinu $N$ kupónových dluhopisů, které spadají do stejné ratingové skupiny. Předpokládejme, že splatnost $i$-tého dluhopisu je $t_i$ a že platí $t_1 < t_2 < ... < t_N$.

Definujme
\begin{center}
\begin{tabular}{l l}
$B_j$     & dnešní cena $j$-tého dluhopisu\\
$G_j$     & dnešní cena $j$-tého dluhopisu za předpokladu\\
          & nulové pravděpodobnosti defaultu\\
$F_j(t)$  & forwardová cena $j$-tého dluhopisu v rámci\\
          & forwardového kontraktu se splatností v čase $t$\\
          & ($t < t_j$) za předpokladu nulové pravděpodobnosti defaultu\\
$v(t)$    & diskontní faktor pro čas $t$\\
$C_j(t)$  & částka nárokovaná majiteli $j$-tého dluhopisu za\\
          & předpokladu defaultu v čase $t$\\
$R_j(t)$  & míra náhrady pro majitele $j$-tého dluhopisu\\
          & v případě defaultu v čase $t$ ($t < t_j$)\\
$\alpha_{ij}$  & současná hodnota ztráty z titulu defaultu\\
          &$j$-tého dluhopisu v čase $t_i$\\
$p_i$     & pravděpodobnost defaultu v čase $t_i$\\
\end{tabular}
\end{center}

Pro zjednodušení předpokládejme, že úrokové sazby jsou deterministické a že jak míra náhrady tak nárokovaná částka jsou známy s jistotou.
Cena nezdefaultované hodnoty $j$-tého dluhopisu v čase $t$ je $F_j(t)$. Jestliže dojde v čase $t$ k defaultu, získá investor část nárokované částky $C_j(t)$ zpět. Tato částka je dána mírou náhrady $R_j(t)$. Z toho vyplývá
\begin{equation}
\alpha_{ij} = v(t_i)[F_j(t_i)-R_j(t_i)C_j(t_i)]
\end{equation} 
Pravděpodobnost ztráty $\alpha_{ij}$ je $p_i$. Celková současná hodnota ztrát z $j$-tého dluhopisu je tedy dána vztahem
\begin{equation*}
G_j - B_j = \sum^{j}_{i=1}p_i \alpha_{ij}
\end{equation*}
Tato rovnice umožňuje, aby příslušné pravděpodobnosti byli vypočteny rekurzivně. První pravděpodobnost $p_1$ lze vypočíst na základě
\begin{equation*}
p_1 = \frac{G_1 - B_1}{\alpha_{11}}
\end{equation*}
Ostatní pravděpodobnosti jsou definovány jako
\begin{equation}
p_j = \frac{G_j - B_j - \sum^{j-1}_{i=1}p_i \alpha_{ij}}{\alpha_{jj}}
\end{equation}

Jak již bylo řečeno dříve, všechny ze skupiny $N$ dluhopisů musí spadat do stejné ratingové skupiny. To by mělo zajistit, že budou z hlediska pravděpodobnosti defaultu představovat homogenní skupinu. Pro zjednodušení budeme předpokládat, že míra náhrady bude stejná pro všechny dluhopisy emitované danou společností a bude nezávislá na čase. Označme tuto míru náhrady jako $\hat{R}$. Rovnice (21.2) se tedy změní na
\begin{equation*}
\alpha_{ij} = v(t_i)[F_j(t_i)-\hat{R}C_j(t_i)]
\end{equation*}

\subsubsection{Default v libovolném časovém okamžiku}

Rovnice (21.3) předpokládá, že k defaultu může dojít pouze v době splatnosti dluhopisu. Nyní tento předpoklad opustíme. Definujme $q(t)$ jako hustotu pravděpodobnosti defaultu. Předpokládejme, že $q(t)$ je konstantní a rovno $q_i$ pro $t_{i-1} < t < t_i$. Jestliže definujeme současnou hodnotu ztráty jako
\begin{equation*}
\beta_{ij} = \int^{t_i}_{t_{i-1}}v(t)[F_j(t)-\hat{R}C_j(t)]dt
\end{equation*}
pak rovnice
\begin{equation}
q_j = \frac{G_j - B_j - \sum^{j-1}_{i=1}q_i \beta_{ij}}{\beta{jj}}
\end{equation}
je zobecněním (21.3).

Parameter $\beta_{ij}$ může být odhadnut pomocí Simpsonova pravidla, které se používá pro výpočet hodnoty určitých integrálů. 

Následující tabulky představují porovnání hodnot pravděpodobnosti defaultu vypočtených za předpokladu, že dluhopis může zdefaultovat pouze v okamžiku splatnosti, a předpokladu, že k defaultu může dojít v libovolný okamžik.

\begin{center}
\begin{tabular}{c c c c}
\textbf{Rok} &
\textbf{Bezriziková} &
\textbf{Kupón} &
\textbf{Výnosová}\\
\textbf{} &
\textbf{zero křivka (\%)} &
\textbf{(\%)} &
\textbf{míra (\%)}\\
\hline
1  & 5.0 & 7.0 & 6.60\\
2  & 5.0 & 7.0 & 6.70\\
3  & 5.0 & 7.0 & 6.80\\
4  & 5.0 & 7.0 & 6.90\\
5  & 5.0 & 7.0 & 7.00\\
10 & 5.0 & 7.0 & 7.20\\
\hline
\multicolumn{4}{c}{\footnotesize{\textit{Informace o korporátním dluhopisu}}}\\
\end{tabular}
\end{center}

\begin{center}
\begin{tabular}{c c c}
\textbf{Rok} &
\textbf{Nárokovaná částka} &
\textbf{Nárokovaná částka}\\
\textbf{} &
\textbf{(bez naběhlého úroku)} &
\textbf{(včetně naběhlého úroku)}\\
\hline
1  & 0.0224 & 0.0224\\
2  & 0.0249 & 0.0247\\
3  & 0.0273 & 0.0269\\
4  & 0.0297 & 0.0291\\
5  & 0.0320 & 0.0312\\
10 & 0.1717 & 0.1657\\
\hline
\multicolumn{3}{c}{\footnotesize{\textit{Pravděpodobnost defaultu za předpokladu, že k defaultu může dojít pouze}}} \\
\multicolumn{3}{c}{\footnotesize{\textit{v době splatnosti dluhopisu}}} \\
\end{tabular}
\end{center}

\begin{center}
\begin{tabular}{c c c}
\textbf{Rok} &
\textbf{Nárokovaná částka} &
\textbf{Nárokovaná částka}\\
\textbf{} &
\textbf{(bez naběhlého úroku)} &
\textbf{(včetně naběhlého úroku)}\\
\hline
0-1  & 0.0220 & 0.0219\\
1-2  & 0.0245 & 0.0242\\
2-3  & 0.0269 & 0.0264\\
3-4  & 0.0292 & 0.0285\\
4-5  & 0.0315 & 0.0305\\
5-10 & 0.295 & 0.0279\\
\hline
\multicolumn{3}{c}{\footnotesize{\textit{Pravděpodobnost defaultu za předpokladu, že k defaultu může dojít}}} \\
\multicolumn{3}{c}{\footnotesize{\textit{v libovolný časový okamžik}}} \\
\end{tabular}
\end{center}

\subsubsection{Nárokovaná částka a aditivita}

Jestliže budeme předpokládát, že forwardová cena dluhopisu $F_j(t)$ odpovídá částce $C_j(t)$ nárokované investory v případě defaultu, lze dokázat, že hodnota $B_j$ kupónového dluhupisu je rovna součtu hodnot pokladových diskontních dluhopisů. Tuto vlastnost označujeme jako aditivitu. Aditivita implikuje, že je teoreticky správné vypočítat zero křivku z likvidních dluhopisů a tu následně použít pro ocenění nelikvidních dluhopisů.

Jesliže přijmeme realističtější předpoklad, že $C_j(t)$ je rovno součtu nominální hodnoty a naběhlého úroku v čase $t$, výše uvedená aditivita neplatí. To v praxi znamená, že neexistuje zero křivka, kterou by bylo možné použít pro ocenění korporátních dluhopisů při dodržení předpokladů o pravděpodobnostech defaultu a míry náhrady.

\subsubsection{Asset swapy}

V praxi používají obchodníci velice často kotace asset swapů jako kvantifikaci pravděpodobnosti defaultu pro dluhopisy za předpokladu, že LIBOR křivka je bezriziková.

Předpokládejme, že investor vlastní fixní pětiletý korporátní dluhopis, který se momentálně obchoduje za par a vyplácí půlroční kupón 6\%. Dále předpokládejme, že LIBOR křivka je plochá na úrovni 4.5\% při půlročním úročení. V rámci úrokového swapu by tedy půlroční sazba 6\% byla vyměněna za sazbu LIBOR navýšenou o 150 bazických bodů. Investor tedy jako kompenzaci za kreditní riziko obdrží 150 bazických bodů vyplácených pololetně. Současná hodnota této kompenzace je v případě uvažovaného pětiletého dluhopisu rovna 6.65 USD na 100 USD nominální hodnoty. Hodnoty parametrů $B_j$ resp. $G_j$ je tedy 100 USD resp. 106.65 USD. Pro výpočet pravděpodobnosti defaultu je možné použít rovnice (21.3) a (21.4).

Dále uvažujme dluhopis, který má hodnotu 95 USD na 100 USD nominální hodnoty a pololetně vyplácí kupón výši 5\% p.a. Opět předpokládejme, že LIBOR křivka je plochá na úrovni 4.5\% při půlročním úročení. V tomto případě můžeme uvažovat, že investor nejprve zaplatí 5 USD na 100 USD nominální hodnoty a následně vymění kupón dluhopisu za LIBOR křivku navýšenou o 50 bazických bodů. Platba 5 USD na začátku životnosti swapu odpovídá částce 1.1279 USD za rok vyplácené pololetně po dobu pěti let. Investor tedy požaduje dalších 112.79 bazických bodů jako kompenzaci za tuto platbu. Výsledná platba z floatové nohy asset swapu by tedy byla LIBOR křivka navýšená o 162.79 bazických bodů. Uvažovaných 162.79 USD ročně vyplácených pololetně po dobu pěti let odpovídá ekvivalentu 7.22 USD v současné hodnotě. V tomto případě je tedy $B_j = 95$ a $G_j = 102.22$.

Jestliže máme k dispozici asset swap spready pro dluhopisy s různými splatnostmi, je možné použít rovnice (21.3) a (21.4) pro odhad pravděpodobností defaultu. Kreditní spread závisí na ratingové skupině dluhopisu a jeho splatnosti. Teoreticky by kreditní spread měl záviset také na výši kupónu. V praxi jsou však dluhopisy často málo likvidní a proto se předpokládá, že kreditní spread je stejný pro všechny výše kupónů. Také pravděpodobnosti defaultu jsou velice často určovány na základě historických dat spíše než na základě kreditních spreadů.

\section{Rizikově neutrální vs. reálný svět}

Uvažujme pětiletý dluhopis s nulovou mírou náhrady, jehož výnosové procento je 50 bazických bodů nad LIBOR křivkou a který spadá do ratingové skupiny A. Podle rovnice (21.1) je pravděpodobnost defaultu tohoto dluhopisu $1-e^{-0.005 \cdot 5} = 0.0247$ neboli 2.47\%. Podle historických dat je pravděpodobnost defaultu pouze 0.57\%. Za vypočtenou pravděpodobností defaultu se totiž skrývají dva základní předpoklady
\begin{itemize}
\item dluhopis je obchodován 50 bazických bodů nad bezrizikovou křivkou
\item míra náhrady je rovna nula
\end{itemize}
V praxi jsou dluhopisy ratingové skupiny A obchodovány často i s vyšší přirážkou než 50 bazických bodů a míra náhrady bývá vyšší než nula. Jestliže bychom zvýšili spread nebo míru náhrady, dojde v rámci výpočtu paradoxně k dalšímu navýšení pravděpodobnosti defaultu. Naše výpočty tedy implikují, že
\begin{itemize}
\item hodnota očekávaného cash-flow z korporátního dluhopisu je na konci pětiletého období je o 2.47\% nižší než v případě ekvivalentního bezrizikového dluhopisu
\item diskontní faktory jsou shodné pro korporátní i bezrizikový dluhopis  
\end{itemize}
Druhý z předpokladů platí v tzv. rizikově neutrálním světě, kde investoři nepožadují dodatečný výnos za podstupované riziko. Jestliže však budeme předpokládat, že
\begin{itemize}
\item hodnota očekávaného cash-flow z korporátního dluhopisu je na konci pětiletého období je o 0.57\% nižší než v případě ekvivalentního bezrizikového dluhopisu
\item sazba, ze které jsou vypočteny diskontní faktory, je o 0.38\% vyšší bezriziková sazba  
\end{itemize}
bude rozdíl v ocenění korporátního a bezrizikového dluhopisu opět 2.47 USD na 100 USD nominální hodnoty. Zvýšená diskontní sazba totiž znamená snížení hodnoty korporátního dluhopisu o přibližně $5 \cdot 0.38 = 1.9\%$. V kombinaci s očekávaným nižší cash-flow je výsledný rozdíl $0.019 + 0.0057 = 2.47\%$. Odhadovaná pravděpodobnost defaultu ve výši 0.57\% je tedy konzistení s reálným světem, kde investoři požadují kompenzaci za podstupované riziko.

Logickou otázkou je, zda-li při analýze kreditních rizik používat pravděpodobnost defaultu vypočtenou pro rizikově neutrální nebo reálný svět. Jestliže budeme oceňovat kreditní deriváty nebo odhadovat dopad rizika defaultu na oceňování instrumentů, je třeba použít rizikově neutrální pravděpodobnost defaultu. Důvodem je, že oceňovací techniky jsou založeny na předpokladech rizikově neutrálního světa. Jesliže však budeme připravovat scénáře s cílem kvantifikovat možné dopady ztrát z titulu defaultu, je vhodnější použít pravděpodobnost defaultu pro reálný svět.

\section{Využití cen akcií pro odhad pravděpodobnosti defaultu}

Až dosud jsme pravděpodobnosti defaultu společností odvozovali na základě jejich ratingu. Bohužel ne vždy je tento rating s dostatečnou periodicitou aktualizován.

V roce 1974 představil Merton model, ve kterém je akcie opcí na aktiva společnosti. Pro zjednodušení předpokládejme, že aktiva společnosti tvoří jeden diskontní dluhopis se splatností v čase $T$. Definujme
\begin{center}
\begin{tabular}{l l}
$V_0$          & dnešní hodnota aktiv společnosti\\
$V_T$          & hodnota aktiv společnosti v čase $T$\\
$E_0$          & dnešní hodnota akcie společnosti\\
$E_T$          & hodnota akcie společnosti v čase $T$\\
$D$            & objem úroků a jistiny splatných v čase $T$\\
$\sigma_V$     & volatilita aktiv\\
$\sigma_E$     & volatilita akcií\\
\end{tabular}
\end{center}
Jestliže $V_T < D$, je (alespoň z teoretického pohledu) pro společnost lepší zdefaultovat v čase $T$ - hodnota akcií je v takovémto případě rovna nula. Jestliže naopak $V_T > D$, společnost by měla v čase $T$ splatit svůj dluh - hodnota akcií je pak $V_T - D$. Platí tedy
\begin{equation*}
E_T = \max(V_T - D, 0)
\end{equation*}
Akcie je tedy v podání tohoto modelu kupní opcí na hodnotu aktiv s realizační cenou podpovídající požadované splátce dluhu. S použitím rovnice Black-Scholes pro výpočet ceny opcí pak dostáváme
\begin{equation}
E_0 = V_0N(d_1) - De^{-rT}N(d_2)
\end{equation}
kde
\begin{equation*}
d_1 = \frac{\ln V_0/D + (r+\sigma^2_V/2)T}{\sigma_V \sqrt{T}}
\end{equation*}
\begin{equation*}
d_2 = d_1 - \sigma_V \sqrt{T}
\end{equation*}
Hodnota dluhu dnes je $V_0 - E_0$. Rizikově neutrální pravděpodobnost, že společnost zdefaultuje je $N(-d_2)$. Pro výpočet této hodnoty je zapotřebí $V_0$ a $\sigma_V$. Jestliže je společnost veřejně obchodovatelná, je $E_0$ kotováno na trhu. Rovnice (21.5) tak představuje podmínku, kterou musí $V_0$ a $\sigma_V$ splňovat. Dále je třeba odhadnout $\sigma_E$ a to například na základě historického vývoje cen akcie. Z It\^o lemmy dále vyplývá
\begin{equation*}
\sigma_E E_0 = N(d_1)\sigma_V V_0
\end{equation*}
Tímto jsme získali druhou rovnici, která je zapotřebí pro výpočet parametrů $V_0$ a $\sigma_V$.

Až dosud jsme předpokládali, že všechen dluh společnosti je splatný v jeden okamžik. V praxi však k splácení dluhu dochází ve vícero různých okamžiků. To výše uvedený model poněkud komplikuje, nicméně v zásadě je stále možné uplatňovat opční oceňovancí princip pro odhad hodnot parametrů $V_0$ a $\sigma_V$. V návaznosti na to je pak možné odhadovat pravděpodobnost defaultu v různých časech.

Mezi pravděpodobnostmi defaultu dle Mertonova modelu a pravděpodobnostmi defaultu pozorovanými v praxi existuje rozdíl. Nicméně pravděpodobnosti vypočtené dle výše uvedeného modelu mohou být použity jako odhad skutečných pravděpodobností.

\section{Odhad výše ztráty z titulu defaultu}

Uvažujme předpokládanou výši ztráty v případě defaultu. Tato ztráta je označována jako LGD (loss given default). Jestliže tuto ztrátu vynásobíme pravděpdobnostní defaultu, získáme očekávanou výši ztráty. Předpokládaná ztráta z titulu defaultu je v případě půjčky obvykle definována jako
\begin{equation*}
V - R(L + A)
\end{equation*}
kde $L$ je nesplacená jistina, $A$ naběhlý úrok, $R$ míra náhrady a $V$ představuje nezdefautovanou hodnotu půjčky.

Předpokládaná ztráta z dluhopisu je počítána obdobným způsobem s tím, že $L$ je rovno nominální hodnotě dluhopisu. V případě derivátů je výpočet předpokládané ztráty komplikovanější. Pro tyto účely je možné rozdělit deriváty do tří skupin - (a) deriváty, které jsou vždy závazkem (např. prodaná opce), (b) deriváty, které vždy představují aktivum (např. nakoupené opce) a (c) deriváty, které mohou být jak závazkem, tak aktivem (např. forward). Deriváty spadající do první skupiny neimplikují žádné kreditní riziko - v případě defaultu protistrany nevzniká žádná škoda. Naproti tomu deriváty druhé skupiny s sebou nesou vždy kreditní riziko. V případě poslední skupiny záleží na momentální tržní hodnotě derivátu.

\subsection{Netting}

Další komplikací při odhadování předpokládané ztráty, je tzv. "netting". Tento pojem znamená, že jestliže prostistrana zdefaultuje na jeden kontrakt, musí zdefaultovat na všechny nezmaturované kontrakty. Princip nettingu se nejčastěji používá ve finančním sektoru.

Uvažujme finanční instituce A a B. Nechť finanční instituce A má s finanční institucí B uzavřeny tři konrakty, jejichž tržní hodnoty jsou -5, 2 a 8 miliónů USD z pohledu finanční instituce A. Jestliže by finanční instituce B zdefaultovala, zdefaultovala by pouze na poslední dva kontrakty. Ztráta pro společnost A by tak činila -10 miliónů USD. V případě, že by obě finanční instituce měly uzavřen netting, byla by ztráta pro společnost A pouze -5 miliónu USD.

Předpokládejme, že finanční instituce uzavřela $N$ derivátových obchodů s určitou protistranou. Dále předpokládejmě, že nezdefaultovaná hodnota $i$-tého kontraktu je $V_i$ a míra náhrady je $R$. Bez nettingu je hodnota tohoto portfolia 
\begin{equation*}
(1-R)\sum^N_{i=1}\max(V_i,0)
\end{equation*}
S nettingem je pak hodnota portfolia rovna
\begin{equation*}
(1-R)\max \Big(\sum^N_{i=1}V_i,0 \Big)
\end{equation*}

\subsection{Snižování expozice na kreditní riziko}

Jedním ze způsobů snižování expozice vůči kreditními riziku je tzv. kolateralizace. Uvažujme společnosti A a B. V rámci kolateralizace může být po společnosti A požadováno složení zástavy u společnosti B popř. třetí strany. Účelem této zástavy je krýt případné ztráty z titulu kreditního rizika. Výše požadované zástavy se periodicky reviduje s tím, jak se mění tržní hodnota kontraktu mezi společnostmi A a B.

Dalším způsobem, jak snížit kreditní riziko, jsou tzv. defaultní události. Příkladem takovéto události může být např. snížení ratingu externí ratingovou společností nebo pokles podílu vlastního mění pod stanovenou úroveň. V případě realizace defaultní události se kontrakt stává splatným popř. je zavázaná strana povinna zaplatit smluvní pokutu.

Za způsob snížení kreditního rizika lze považovat také diverzifikaci, která může být vedena v rovině jednotlivých společností, odvětví nebo geografických oblastí. Diverzifikace je také součástí podmínek stanovených nařízením BASEL, které upravuje kapitálové požadavky kladané na finanční instituce.

\section{Změna kreditního ratingu}

Čas od času může dojít k tomu, že se dluhopisy přesunou z jedné ratingové skupiny do jiné. Ratingové společnosti na základě historických map publikují tzv. ratingovou přechodovou matici. Tato matice vyjadřuje pravděpodobnost přesunu z jedné ratingové skupiny do jiné v rámci určitého časového období (obyvykle se jedná jeden rok).

\begin{center}
\begin{tabular}{c r r r r r r r r}
\textbf{Počáteční} &
\multicolumn{8}{c}{\textbf{Rating na konci roku}} \\
\textbf{rating} &
\multicolumn{1}{c}{\textbf{AAA}} &
\multicolumn{1}{c}{\textbf{AA}} &
\multicolumn{1}{c}{\textbf{A}} &
\multicolumn{1}{c}{\textbf{BBB}} &
\multicolumn{1}{c}{\textbf{BB}} &
\multicolumn{1}{c}{\textbf{B}} &
\multicolumn{1}{c}{\textbf{CCC}} &
\multicolumn{1}{c}{\textbf{Default}} \\
\hline
AAA       & 93.66 &  5.83 &  0.40 &  0.09 &  0.03 &  0.00 &  0.00 &  0.00 \\
AA        &  0.66 & 91.72 &  6.94 &  0.49 &  0.06 &  0.09 &  0.02 &  0.01 \\
A         &  0.07 &  2.25 & 91.76 &  5.18 &  0.49 &  0.20 &  0.01 &  0.04 \\
BBB       &  0.03 &  0.26 &  4.83 & 89.24 &  4.44 &  0.81 &  0.16 &  0.24 \\
BB        &  0.03 &  0.06 &  0.44 &  6.66 & 83.23 &  7.46 &  1.05 &  1.08 \\
B         &  0.00 &  0.10 &  0.32 &  0.46 &  5.72 & 83.62 &  3.84 &  5.94 \\
CCC       &  0.15 &  0.00 &  0.29 &  0.88 &  1.91 & 10.28 & 61.23 &  25.26 \\
Default   &  0.00 &  0.00 &  0.00 &  0.00 &  0.00 &  0.00 &  0.00 & 100.00 \\
\hline
\end{tabular}
\end{center}

Pro účely oceňování derivátů, jejichž výplata závisí na kreditním ratingu, je zapotřebí rizikově neutrální ratingové přechodové matice. Její konstrukce je relativně obtížná. Jeden z možných přístupů je založen na ceně dluhopisů. Pro zjednodušení uvažujme pouze tři ratingové skupiny - A, B a C. Písmenem D pak budeme označovat stav defaultu. Následující tabulka obsahuje hypotetické údaje o kumulativní rizikově neutrální pravděpodobnosti defaultu.
\begin{center}
\begin{tabular}{c r r r r r}
\textbf{Počáteční} &
\multicolumn{1}{c}{\textbf{1}} &
\multicolumn{1}{c}{\textbf{2}} &
\multicolumn{1}{c}{\textbf{3}} &
\multicolumn{1}{c}{\textbf{4}} &
\multicolumn{1}{c}{\textbf{5}} \\
\multicolumn{1}{c}{\textbf{rating}} &
\multicolumn{1}{c}{\textbf{rok}} &
\multicolumn{1}{c}{\textbf{rok}} &
\multicolumn{1}{c}{\textbf{rok}} &
\multicolumn{1}{c}{\textbf{rok}} &
\multicolumn{1}{c}{\textbf{rok}} \\
\hline
A &   0.67 &   1.33 &   1.99 &   2.64 &   3.29 \\
B &   1.66 &   3.29 &   4.91 &   6.50 &   8.08 \\
C &   3.29 &   6.50 &   9.63 &  12.69 &  15.67 \\
D & 100.00 & 100.00 & 100.00 & 100.00 & 100.00 \\
\hline
\end{tabular}
\end{center}  
Nechť $M$ je matice $4 \times 4$ obsahující rizikově neutrální pravděpodobnosti přechodu z jedné ratingové skupiny do jiné. To znamená, že např. $M_{11}$ představuje pravděpodobnost, že společnost s ratingem A zůstane v této ratingové skupině i po roce. Definujme $d_i$ jako vektor, který představuje $i$-tý sloupec výše uvedené tabulky. Tento vektor je tak kumulativní pravděpodobností defaultu pro všchny ratingové skupiny v $i$-tém roce. Z toho vyplývá, že $d_2 = Md_1$, $d_3 = Md_2 = M^2d_1$, $d_4 = Md_3=M^3d_1$ a $d_5=Md_4=M^4d_1$. Existuje devět neznámých členů matice $M$. Jedná se o členy $M_{ij}$ pro $1 \le 1, j \le 3$\footnote{Platí totiž $M_{i,4} = 1 - M_{i,1} - M_{i,2} - M_{i,3}$ a $M_4,j = 1$ pro $j = 4$. Pro ostatní $j$ jsou tyto členy rovny nule.}. Výše uvedená tabulka kumulativních pravděpodobností definuje patnáct rovnic, které musí platit (dluhopisy s rating A, B a C a defaultem v prvním až pátém roce). Proto si vybere uvažovaných devět členů pro minimalizaci součtu rozdílu čtverců pro patnáct vztahů, které jsou dány rozdílem $M^{i-1}d_1$ a odpovídajících členů vektoru $d_i$ ($1 \le i \le 5$). Výsledky jsou zobrazeny v následující tabulce.
\begin{center}
\begin{tabular}{c r r r r r}
\textbf{Počáteční} &
\multicolumn{5}{c}{\textbf{Rating na konci roku}} \\
\multicolumn{1}{c}{\textbf{rating}} &
\multicolumn{1}{c}{\textbf{A}} &
\multicolumn{1}{c}{\textbf{B}} &
\multicolumn{1}{c}{\textbf{C}} &
\multicolumn{1}{c}{\textbf{D}} \\
\hline
A &   98.4 &  0.9 &  0.0 &  0.0 &   0.7 \\
B &    0.5 & 97.1 &  0.7 &  0.7 &   1.7 \\
C &    0.0 &  0.0 & 96.7 & 96.7 &   3.3 \\
D &    0.0 &  0.0 &  0.0 &  0.0 & 100.0 \\
\hline
\end{tabular}
\end{center}
\textbf{Poznámka:} V tomto jsou výsledky dané rizikově neutrální přechodovou tabulkou přijatelné. V praxi toto bohužel není vždy pravidlem. Proto jsou velice často navrhovány metody, které rizikově neutrální přechodovou tabulku odvozují z přechodové tabulky vypočtené na základě historických dat.

\section{Korelace defaultních událostí}

\subsection{Míry korelace}

Korelace jsou v případě kreditního rizika používány pro vyjádření možnosti, že dvě společnosti zdefaultují přibližně v stejný okamžik. Korelace je v tomto případě vypočtena jako korelace mezi dvěma proměnnými, kde každá z těchto proměnných nabývá hodnoty nula nebo jedna podle toho, zda-li u zkoumané společnosti došlo ve sledovaném období k defaultu či nikoliv. Pro období nula až $T$ je tak možné korelaci defaultních událostí vyjádřit jako
\begin{equation*}
\beta_{AB}(T) = \frac{P_{AB}(T)-Q_A(T)Q_B(T)}{\sqrt{[Q_A(T)-Q_A(T)^2][Q_B(T)-Q_B(T)^2]}}
\end{equation*}
kde $P_{AB}(T)$ je sdružená pravděpodobnost, že společnost A a B zdefaultují v časovém období 0 až $T$. Obdobně $Q_A(T)$ resp. $Q_B(T)$ představuje kumulativní pravděpodobnost, že společnost A resp. B zdefaultují v časovém období 0 až $T$. Vypočtená korelace $\beta_{AB}$ obvykle závisí na uvažovaném časovém období - s rostoucí délkou se tato korelace zvyšuje.\\

Dále je možné korelaci defaultních událostí získat z pravděpodobnostního rozdělení času do defaultu. Předpokládejme, že $t_A$ a $t_B$ jsou časy do defaultu společností A a B. Proměnné $t_A$ a $t_B$ nejsou normálně rozdělené, nicméně funkce proměnných $t_A$ a $t_B$
\begin{equation*}
u_A(t_A) = N^{-1}[Q_A(t_A)]
\end{equation*} 
\begin{equation*}
u_B(t_B) = N^{-1}[Q_B(t_B)]
\end{equation*} 
mají normální rozdělení. Korelaci pak definujeme jako
\begin{equation*}
\rho_{AB} = corr[u_A(t_A), u_B(t_B)]
\end{equation*}
Předpokládáme, že proměnné $u_A(t_A)$ a $u_B(t_B)$ mají bivariální normální rozdělení. To znamená, že sdružená pravděpodobnostní funkce časů do defaultu může být popsána pomocí kumulativního pravděpodobnostního rozdělení $Q_A(t_A)$, $Q_B(t_B)$ a $\rho_{AB}$. Výše popsanou sérii předpokladů označujeme jako Gausovu kopulu.

Přístup pomocí Gausovy kopuly může být použit také v situaci vícero společností. Uvažujme $N$ společností. Označme čas do defaultu $i$-té společnosti jako $t_i$. Definujme $Q_i(t_i)$ jako kumulativní pravděpodobnostní rozdělení proměnné $t_i$ a
\begin{equation*}
u_i(t_i) = N^{-1}[Q_i(t_i)]
\end{equation*}
pro $1 \le i \le N$. Předpokládáme, že proměnné $u_i(t_i)$ mají multivariální normální rozdělení.

\subsection{Vztahy mezi jednotlivými měřítky korelace}

Definujme $M(a,b;\rho)$ jako pravděpodobnost v rámci standardizovaného bivariálního normálního rozdělení, že první proměnná bude menší než $a$ a druhá proměnná menší než $b$ při vzájemné korelaci $\rho$. Nechť $\rho_{AB}$ je defaultní korelací mezi A a B v modelu Gausovy kopuly. Předpokládejme, že $u_A(t)$ a $u_B(t)$ jsou transformované časy do defaultu pro společnosti A a B v modelu Gausovy kopule. Z toho vyplývá
\begin{equation*}
P_{AB}(T) = M(u_A(T), u_B(T); \rho_{AB})
\end{equation*}
a
\begin{equation*}
\beta_{AB}(T) = \frac{M(u_A(T),u_B(T); \rho_{AB})-Q_A(T)Q_B(T)}{\sqrt{[Q_A(T)-Q_A(T)^2][Q_B(T)-Q_B(T)^2]}}
\end{equation*}
Z výše uvedeného vyplývá, že je-li známo $Q_A(T)$ a $Q_B(T)$, lze vypočíst $\beta_{AB}(T)$ z $\rho_{AB}$ a naopak. Obvykle je $\rho_{AB}$ výrazně vyšší než $\beta_{AB}(T)$.

\subsection{Modelování korelace}

Pro modelování korelací defaulních událostí se používají dva typy modelů - makroekonomický model a strukturální model.

Makroekonomický model předpokládá, že míra hazardu pro různé společnosti sleduje stochastický proces a je korelována s makroekonomickými ukazately. Korelace mezi jednotlivými společnostmi je tak dána vývojem vybraných makroekonomických ukazatelů. Tyto modely jsou matematicky atraktivní a odráží závislost pravděpodobnosti defaultu na vývoji ekonomických cyklů. Makroekonomické modely je možné nakalibrovat tak, aby odpovídaly historickým defaultním pravděpodobnostem nebo rizikově neutrálním pravděpodobnostem defaultu odvozených z cen korporátních dluhopisů. Hlavní nevýhodou těchto modelů je omezený rozsah korelací, které je možné modelovat.

Strukturální model vychází z myšlenky, že společnost je souborem aktiv. Společnost pak zdefaultuje, jestliže cena jejích aktiv klesne pod určitou kritickou hranici. Korelace defaulních událostí pro jednotlivé společnosti jsou pak dány cenovou korelací mezi portfolii aktiv, které tyto společnosti vlastní. Předpokládá se, že ceny jednotlivých aktiv sledují stochastický proces. Výhodou těchto modelů je, že korelace je možné namodelovat v požadované výši. Nevýhodou je naopak fakt, že tyto modely nejsou konzistentní s pravděpodobnostmi defaultu vypočtenými z historických hodnot nebo cen dluhopisů.

\section{CVaR}

CVaR (Credit Value at Risk) představuje maximální ztrátu, která může s $X$\% pravděpodobností ve stanoveném časovém období vzniknout z titulu realizace kreditního rizika. V případě CVaR je zkoumané období zpravidla delší než u dříve diskutovaného VaR (běžně jeden rok oproti deseti dnům).

Ztráty z titulu kreditního rizika nevznikají pouze v případě, že dojde k defaultu společnosti. Ke ztrátám může dojít např. při poklesu ratingu společnosti, v důsledku čehož klesne tržní cena jí emitovaných dluhopisů. Při výpočtu CVaR je vhodné vycházet z historických hodnot spíše než z pravděpodobností defaultu odvozených na základě kotací korporátních dluhopisů.

\subsection{Credit Risk Plus}

Credit Risk Plus je metodologií, kterou vyvinula Credit Suisse Financial Products v roce 1997.

Předpokládejme, že finanční instituce má $N$ protistran určitého typu a že pravděpodobnost defaultu každé z těchto protistran do časového okamžiku $T$ je $p$. Pro očekávaný počet defaultů $\mu$ tedy platí
\begin{equation*}
\mu = Np
\end{equation*}
Jestliže je $p$ malé a jednotlivé defaulty vzájemně nezávislé, řídí se pravděpodobnost realizace $n$ defaultů ve zkoumaném období Poissonovým rozdělením.
\begin{equation*}
P(n) = \frac{e^{-\mu}\mu^n}{n!}
\end{equation*} 
Náhodná veličina představující počet defaultních událostí v portfoliu protistran může být kombinována s modelem výše ztráty na jednu událost. Tímto způsobem lze snadno odvodit pravděpodobnostní rozdělení celkové ztráty z titulu kreditního rizika.

V praxi musí každá finanční situace rozdělovat své protistrany do několika skupin. V tomto případě je třeba provést výše popsanou analýzu pro každou skupinu zvlášť a výsledky zkombinovat.

Výše popsaný postup je možné realtivně snadno implementovat formou Monte-Carlo pomocí následujících kroků
\begin{itemize}
\item simulace celkových defaultních sazeb
\item výpočet pravděpodobnosti defaultu pro jednotlivé kategorie protistran
\item simulace počtu defaultů v rámci kategorií
\item simulace výše ztráty pro každou defaultní událost
\item výpočet celkové ztráty
\end{itemize}
Opakovaným prováděním těchto kroků získáme vektor hodnot náhodné veličiny ztrát z titulu kreditních rizik. Na základě vektoru hodnot je pak možné sestrojit pravděpodobnostní rozdělení této náhodné veličiny.

\begin{center}
	\begin{pspicture}(0,0)(10,8)
		\rput(5.0,0.0){Pravděpodobnostní rozdělení ztráty z realizovaných kreditních rizik}

		\psline[arrows=->](0.5,1.0)(9.5,1.0)
		\psline[arrows=->](0.5,1.0)(0.5,7.5)
		
		\pscurve[linewidth=0.5mm, curvature=0.8 0.1 0](0.6,1.1)(0.7,1.3)(1.6,6.7)(1.9,7.0)(2.5,6.7)(5.0,2.5)(6.5,1.5)(9.0,1.1)
		
		\rput(0.3,0.8){$0$}
		\rput(9.5,0.8){\small{ztráta}}
	\end{pspicture}
\end{center}

\subsection{CreditMetrics}

J.P.Morgan vyvinul metodu pro výpočet CVaR nazývanou CreditMetrics. Podobně jako Credit Risk Plus využívá tento model metody Monte-Carlo. Pro ilustraci tohoto modelu předpokládejme, že nás zajímá pravděpodobnostní rozdělení kreditních ztrát pro období jednoho roku. V každém kroku simulace se nejprve nasimulují změny ratingu jednotlivých protistran v průběhu roku. Dále se simuluje vývoj rozhodujících tržních ukazatelů. Na základě nasimulovaných hodnot pak přeceníme portfolio. Výhodou tohoto modelu je velká flexibilita\footnote{Relativně snadno je možné např. implementovat netting.}, nevýhodou je naopak vysoká náročnost na výpočetní zdroje. V rámci modelu je využívána Gausova kopule k modelování změn ratingů mnoha protistran. Model předpokládá, že korelace mezi normálními proměnnými uvažovaných společností odpovídá korelaci cen jejich akcií.

\subsubsection{Modelování změny ratingu}

Uvažujme simulaci změny ratingu společnosti s ratingem AAA a BBB v rámci jednoho roku. Pro simulaci použijeme přechodovou matici z kapitoly 21.6 odvozenou na základě historických hodnot. Předpokládejme, že korelace mezi cenou akcií uvažovaných společností je 0.2. V každém kroku simulace vygenerujeme dvě proměnné $x$ a $y$ z normovaného normálního rozdělení a korelací 0.2. Proměnná $x$ představuje nový rating společnosti AAA, proměnná $y$ pak nový rating společnosti BBB. Protože
\begin{equation*}
N^{-1}(0.9366) = 1.527
\end{equation*} 
\begin{equation*}
N^{-1}(0.9366 + 0.0583) = 2.569
\end{equation*}
\begin{equation*}
N^{-1}(0.9366 + 0.0583 + 0.0040) = 3.062
\end{equation*}
zůstane společnost AAA ve stejné ratingové skupině, pokud $x < 1.527$. Jestliže $1.527 \le x < 2.569$, přesune se tato společnost do skupiny AA. Pokud $2.569 \le x < 3.062$, získá tato společnost rating A atd. Podobně pro společnost BBB platí
\begin{equation*}
N^{-1}(0.0003) = -3.432
\end{equation*} 
\begin{equation*}
N^{-1}(0.0003 + 0.0026) = -2.759
\end{equation*}
\begin{equation*}
N^{-1}(0.0003 + 0.0026 + 0.0483) = -1.633
\end{equation*}
Proto se společnost přesune do ratingové skupiny AAA pro $y < -3.432$, do skupiny AA pro $-3.432 \le y < -2.759$ a do skupiny A jestliže $-2.759 \le y < -1.633$.

Přínos Gausovy kopuly je, že umožňuje aplikovat multivariální normální rozdělení pro účely simulace migrace mezi ratingovými skupinami pro mnoho společností.

\chapter{Kreditní deriváty}

Kreditní deriváty jsou kontrakty, kde výplata záleží na vzniku předem definovaných kreditních událostí jedné nebo více entit. Typickými kreditními událostmi jsou vedle již zmiňovaného defaultu např. zhoršení kreditního ratingu, zpoždění s prováděním plateb atd.

\section{Kreditní swap}

Kreditní swap je ve své podstatě pojištěním proti vzniku defaultní události konkrétní protistrany. Majitel tohoto "pojištění" má v případě vzniku kreditní události právo prodat dluhopis emitovaný touto protistranou za jeho nominální hodnotu. Cena kreditního swapu má formu periodických plateb ve prospěch prodejce swapu prováděných až do konce jeho životnosti\footnote{V případě vzniku kreditní události je poslední platba provedena v poměrné části.}. Jestliže dojde ke kreditní události, je kreditní swapu vypořádán fyzickým dodáním nebo v hotovosti. V případě fyzického dodání získá prodejce dluhopis a majitel swapu hotovost ve výši jeho nominální hodnoty. Jestliže má narovnání formu hotovosti, je stanovený počet dní po vzniku kreditní události z dostupných kotací na trhu určena průměrná cena dluhopisu $Z$. Částka, kterou získá majitel swapu, je pak dána $(100 - Z)\%$ z nominální hodnoty dluhopisu.

Cena kreditních swapů, jak ukazuje následující tabulka, je kotována v bazických bodech a odvíjí se od kreditního ratingu a zbytkové doby splatnosti.
\begin{center}
\begin{tabular}{l l c c c c}
\multicolumn{2}{c}{\textbf{}} &
\multicolumn{4}{c}{\textbf{Splatnost}} \\
\textbf{Společnost} &
\textbf{Rating} &
\textbf{3 roky} &
\textbf{5 let} &
\textbf{7 let} &
\textbf{10 let} \\
\hline
Toyota Motor Corp     & Aa1/AAA   &  16/24  &  20/30  &  26/37  &  32/53  \\
Merrill Lynch         & Aa3/AA-   &  21/41  &  40/55  &  41/83  &  56/96  \\
Ford Motor Company    & A+/A-     &  59/80  &  85/100 &  95/136 & 118/159 \\
Enron                 & Baa1/BBB+ & 105/125 & 115/135 & 117/158 & 182/233 \\
Nissan Motor Co. Ltd. & Ba1/BB+   & 115/145 & 125/155 & 200/230 & 244/274 \\
\hline
\end{tabular}
\end{center}
Namísto toho, aby se banky zbavovaly části svých úvěrů, umožňují kreditní deriváty převod kreditního rizika z jedné společnosti na jinou. Kreditní deriváty tak umožňují aktivně řídit kreditní riziko a měnit jeho strukturu nejen z pohledu jednotlivých společnosti ale také odvětví nebo geografického rozložení.

\subsection{Ocenění}
Předpokládejme vzájemnou nezávislost defaultních událostí, úrokových sazeb a měr náhrady. Nechť je nárokovaná částka v případě vzniku kreditní události rovna součtu nominální hodnoty a naběhlého úroku. Dále předpokládejme, že ke vzniku kreditní události může dojít pouze v časové okamžiky $t_1, t_2, ..., t_n$. Definujme
\begin{center}
\begin{tabular}{l l}
$T$		& životnost kreditního swapu v letech \\
$p_i$		& rizikově neutrální pravděpodobnost defaultu v čase $t_i$ \\
$\hat{R}$	& očekávaná míra náhrady v rizikově neutrálním světě pro daný dluhopis \\
		& (předpokládá se, že míra náhrady je nezálislá na okamžiku defaultu) \\
$u(t)$		& současná hodnota periodických ročních plateb ve výší 1 USD vyplácených \\
		& v průběhu časového období nula až $t$ \\
$e(t)$		& současná hodnota plateb v čase $t$ rovna $t - t^*$ USD, kde $t^*$ představuje \\
		& okamžik platby, která bezprostředně předchází $t$ ($t$ a $t^*$ jsou vyjádřeny \\
		& v letech) \\
$v(t)$		& současná hodnota 1 USD vyplaceného v čase $t$ \\
$w$		& roční částka placená majitelem kreditního swapu na 1 USD nominální hodnoty \\
		& podkladového dluhopisu \\
$s$		& hodnota $w$, která má následek nulovou hodnotu kreditního swapu \\
$\pi$		& rizikově neutrální pravděpodobnost, že v průběhu životnosti kreditního swapu \\
		& nedojde ke vzniku kreditní události \\
$A(t)$		& naběhlý úrok z podkladového dluhopisu v čase $t$ jako procento nominální hodnoty \\
\end{tabular}
\end{center}
Hodnota $\pi$ může být odvozena na základě znalosti $p_i$. Platí totiž
\begin{equation*}
\pi = 1 - \sum^n_{i=1}p_i
\end{equation*}
Majitel kreditního swapu hradí ročně částku $w$ a to do konce životnosti swapu nebo do vzniku kreditní události. Současná hodnota těchto plateb je
\begin{equation*}
w \sum^n_{i=1}[u(t_i) + e(t_i)]p_i + w\pi u(T)
\end{equation*}
Jestliže dojde v čase $t_i$ ke kreditní události, rizikově neutrální hodnota podladového dluhopisu jako procenta jeho nominální hodnoty je $[1+A(t_i)]\hat{R}$. Rizikově neutrální hodnota výplaty generované kreditním swapem je proto
\begin{equation*}
1 - [1 + A(t_i)]\hat{R} = 1 - \hat{R} - A(t_i)\hat{R}
\end{equation*}
Současná hodnota očekávané výplaty generované tímto swapem je tak
\begin{equation*}
\sum^n_{i=1}[1 - \hat{R} - A(t_i)\hat{R}]p_i v(t_i)
\end{equation*}
Hodnota kreditního swapu z pohledu kupce je tak současná hodnota plateb generovaná swapem snížená o současnou hodnotu plateb hrazených kupcem.
\begin{equation*}
\sum^n_{i=1}[1 - \hat{R} - A(t_i)\hat{R}]p_i v(t_i) - w \sum^n_{i=1}u(t_i) + e(t_i)]p_i + w \pi u(T)
\end{equation*}
Spread kreditního spreadu $s$ odpovídá hodnotě $w$, pro kterou je níže uvedený výraz roven nule.
\begin{equation*}
s = \frac{\sum^n_{i=1}[1 - \hat{R} - A(t_i)\hat{R}]p_i v(t_i)}{\sum^n_{i=1}[u(t_i) + e(t_i)]p_i + \pi u(T)}
\end{equation*}
Hodnota parametru $s$ odpovídá roční částce, kterou musí hradit majitel nově uzavřeného kreditního spreadu.

Dále je možné analýzu rozšířit o předpoklad, že defaultní události může dojít v libovolném časovém okamžiku. Předpokládejme, že $q(t)$ je rizikově neutrální hustota pravděpodobnost defaultu v čase $t$. Výše uvedená rovnice pak přejde do tvaru
\begin{equation}
s = \frac{\int^T_0[1 - \hat{R} - A(t)\hat{R}]q(t)v(t)dt}{\int^T_0 q(t)[u(t) + e(t)]dt + \pi u(T)}
\end{equation}

\subsection{Argumenty neexistence arbitráže}

Faktory ovliňující výši spreadu kreditního swapu $s$ lze vysvětlit pomocí předpokladu neexistence arbitráže. Předpokládejme, že investor koupí $T$-roční dluhopis emitovaný danou protistranou za jeho nominální hodnotu a zároveň nakoupí odpovídající $T$-roční kreditní swap. Tímto způsobem investor eliminoval většinu kreditního rizika. Dále předpokládejme, že výnosová míra dluhopisu je $y$. Jestliže nedojde k defaultní události, získá investor roční výnos $y - s$. V případě defaultu získá investor do doby defaultu roční výnos $y - s$ a době defaultu pak získá nominální hodnotu dluhopisu.

Výše uvedené úvahy vedou k závěru, že kombinací dluhopisu a kreditního swapu získává investor bezrizikovou výnosovou míru $y - s$. Předpokládejme, že par výnosová míra\footnote{Par výnosovou mírou dluhopisu rozumíme takovou výnosovou míru, pro kterou je tržní hodnota dluhopisu rovna 100\% jeho nominální hodnoty.} bezrizikového státního dluhopisu je $x$. V případě neexistence arbitráže by tedy mělo platit $x = y - s$. Nicméně bližší analýza problému ukazuje, že by případná arbitráž nebyla zcela dokonolá a to z následujících důvodů:
\begin{itemize}
\item Kreditní swap generuje výplatu nominální hodnoty sníženou o hodnotu dluhopisu po vzniku defaultní události. Pro dokonalou arbitráž by bylo zapotřebí, aby kreditní swap navíc zahrnoval také naběhlý úrok z dluhopisu.
\item Dalším předpokladem dokonalé arbitráže je, že investor by musel být schopen získat bezrizikovou výnosovou míru $x$ v období mezi vznikem defaultní události a časem $T$. Jestliže však struktura úrokových sazeb není plochá nebo úrokové sazby jsou dány stochastickým procesem, budou se skutečné bezrizikové úrokové sazby v době defaultu odlišvat od $x$.
\end{itemize}
Definujme $s^* = y - x$. Také definujme průměrný naběhlý úrok $a$ z daného referečního dluhopisu a průměrný naběhlý úrok $a^*$ z par dluhopisu\footnote{Par dluhopis je takový dluhopis, jehož tržní hodnota je rovna jeho nominální hodnotě.} emitovaný stejnou entitou jako refereční dluhopis. Hull a White dokázali, že
\begin{equation*}
s = \frac{s^*(1 - \hat{R} -a\hat{R})}{(1 + \hat{R})(1 + a^*)}
\end{equation*}
poskytuje lepší odhad $s$ než samotné $s^*$.

\subsection{Implikované pravděpodobnosti defaultu}

Trh kreditních swapů je natolik likvidní, že je mnohými analytiky používán pro výpočet implikovaných pravděpodobností defaultu. Jedná se analogii k opčím trhům, kdy jsou ceny opcí použity pro výpočet implikovaných volatilit.

Předpokládejme, že kreditní swapy pro splatnosti $t_1, t_2, ..., t_n$ jsou $s_1, s_2, ..., s_n$. Uvažujme krokovou funkci pro pravděpodobnost defaultu a definujme $q_i$ jako hustotu pravděpodobnosti defaultu pro časový interval $t_{i-1}$ až $t_i$. Použijme dřívější označení s tím rozdílem, že $A_i(t)$ představuje naběhlý úrok na $i$-tý refereční dluhopis v čase $t$. S použitím (22.1) získáváme
\begin{equation*}
s_i = \frac{\sum^i_{k = 1} q_i \int^{t_k}_{t_{k-1}}[1 - \hat{R} - A_i(t)\hat{R}]v(t)dt}{\sum^i_{k=1}q_k \int^{t_k}_{t_{k-1}}[u(t) + e(t)]dt + u(t_i)[1 - \sum^i_{k=1}q_k(t_k - t_{k-1})]}
\end{equation*}
Parameter $q_i$ může být iteračně vypočten z výše uvedené rovnice. Definujme $\delta_k = t_k - t_{k-1}$ a
\begin{equation*}
\alpha_k = \int^{t_k}_{t_{k-1}} (1 - \hat{R})v(t)dt
\end{equation*}
\begin{equation*}
\beta_{k,i} = \int^{t_k}_{t_{k-1}} A_i(t) \hat{R} v(t) dt
\end{equation*}
\begin{equation*}
\gamma_{k} = \int^{t_k}_{t_{k-1}} [u(t) + e(t)]dt
\end{equation*}
Z toho vyplývá
\begin{equation*}
q_i = \frac{s_i u(t_i) + \sum^{i-1}_{k=1} q_k [s_i \gamma_k - s_i u(t_i) \delta_k - \alpha_k + \beta_{k,j}]}{\alpha_j - \beta_{i,j} - s_i \gamma_i + s_i u(t_i) \delta_i}
\end{equation*}

\subsection{Odhad míry náhrady}
Jediný parametrem, který není přímo pozorovatelný na trhu a který je nutný pro oceňování kreditních swapů, je míra náhrady. Naštěstí je ocenění kreditního swapu závislé na míře náhrady pouze okrajově. To je způsobeno tím, že dopad míry náhrady na ocenění kreditního swapu je dvojí. Míra náhrady totiž ovliňuje nejen odhad rizikově neutrální pravděpodobnosti defaultu ale také výši výplaty v případě vzniku defaultní události. Tyto dva faktory se do značné míry vzájemně eliminují.

\subsection{Binární kreditní swapy}

Nestandardní kreditní swapy mohou být naopak relativně senzitivní na změnu míry náhrady. Uvažujme binární kreditní swap (binary credit default swap). Binární kreditní swap je podobný klasickému kreditnímu swapu s tím rozdílem, že výplata v případě vzniku kreditní události je konstatní. V tomto případě tak míra náhrady ovlivňuje pouze pravděpodobnost defaultu, avšak nevstupuje do výpočtu výplaty. Například hodnota binárního kreditního swapu s mírou náhrady 50\% je přibližně o 80\% vyšší než v případě binárního kreditního swapu s mírou náhrady 10\%.

\subsection{Portfoliový kreditní swap}

V případě portfoliového kreditního swapu (basket credit default swap) není výplata odvozována od jednoho pokladového aktiva ale od koše podkladových aktiv. Tento swap pak může být nastaven tak, že k výplatě dochází vždy v případě defaultní události libovolného pokladového aktiva. Jedná se tedy o ekvivalent souboru kreditních swapů. Další možností výplata generovaná pouze v případě vzniku první defaultní události - po té swap zaniká.

\subsection{Risk defaultu na straně prodávajícího}

Až dosud jsme v našich úvahách předpokládali, že riziko defaultu společnosti, která prodává kreditní swap, je nulové. Uvažujme pozici majitele kreditního swapu bezprostředně po té, co došlo k defaultu referenční protistrany\footnote{Pojmem ``referenční protistrana'' budeme rozumět entitu, proti jejíž defaultu se chce majitel kreditního swapu primárně zajistit.}. Aby byl kupující zajištěn proti defaultu této protistrany, musí koupit nový kreditní swap od nového prodávajícího. Životnost tohoto swapu bude $T - t_D$, kde $T$ je splatnost původního kreditního swapu a $t_D$ je okamžik defaultu původního prodávajícího. Kupující tak ponese ztrátu v případě, jestliže roční cena nového kreditního swapu bude vyšší než cena původního kreditního swapu. Riziko defaultu prodávajícího tedy závisí na
\begin{itemize}
\item míře, s jakou trh očekává zhoršení ratingu referenční protistrany
\item korelaci pravděpodobnosti defaultu prodávajícího a referenční protistrany
\end{itemize} 
Dopad defaultu prodávajícího je možné kvantifikovat metodou Monte Carlo. Definujme $Y$ jako současnou hodnotu výplaty generované kreditním swapem a $C$ jako současnou hodnotu ročních plateb ve výši 1 USD od současnoti do konce životnosti kreditního swapu nebo do okamžiku defaultu. V každém cyklu simulace budeme modelovat čas defaultu prodávajícího a referenční protistrany. To umožňuje výpočet modelových hodnot parametrů $Y$ a $C$.

Jestliže nedojde k defaultu prodávajícího kreditního swapu ani refereční protistrany, pak $Y = 0$ a $C$ je rovno současné hodnotě anuity 1 USD vyplácené po celou dobu životnosti swapu. Jestliže dojde nejprve k defaultu referenční protistrany, pak $Y$ představuje výplatu z kreditního swapu a $C$ současnou hodnotu anuity 1 USD vyplácené od současnosti do okamžiku defaultu. Jestliže jako první zdefaultuje prodávající, pak $Y=0$ a $C$ je opět rovno současné hodnotě anuity 1 USD vyplácené do okamžiku defaultu. Výše kreditního spreadu je průměrnou hodnotou podílu $Y$ a $C$.

\section{Totální swap}

Totální swap (total return swap) je dohodou o výměně kompletních výnosů z dluhopisu nebo jiného podkladového aktiva za LIBOR sazbu navýšenou o spread. Totální swap tedy zahrnuje kupóny, úroky a zisk resp. ztrátu z pokladového aktiva po dobu životnosti swapu. Totální swap tak umožňuje přenést kreditní riziko. 

Totální swap je velmi často používán jako finanční nástroj. Jestliže investor potřebuje finanční prostředky na nákup dluhopisu, je možné namísto klasické půjčky řešit tuto situaci pomocí totálního swapu. Předpokládejme situaci, kdy finanční instituce uzavře s tímto investorem totální swap. Investor bude platit sazbu LIBOR navýšenou o spread a výměnou za to získá veškeré výnosy generované dluhopisem po dobu životnosti swapu. V praxi to pak zpravidla probíhá tak, že finanční instituce nakoupí dluhopis, ten drží a investorovi vyplácí kupóny po podobu životnosti totálního swapu. Na konci životnosti uvažovaného totálního swapu pak dojde navíc k platbě, která odráží změnu tržní hodnoty dluhopisu po dobu trvání kontraktu. Z pohledu investora je tedy lhostejné, zda-li získal úvěr, za který daný dluhopis nakoupil, nebo zda-li uzavřel totální swap. Z pohledu finanční instituce je však výhodnější totální swap a to z důvodu nižší expozice vůči kreditnímu riziku investora\footnote{Připomeňme, že pokladový dluhopis je v držení finanční instituce a že kupóny jsou vypláceny postupně proti platbě odpovídající sazbě LIBOR navýšené o dohodnutý spread.}. Z tohoto pohledu se totální swap velice blíží repo operacím\footnote{Repo operace je dohodou dvou zúčastněných stran o nákupu a zpětném prodeji pokladového aktiva. Objem pokladového aktiva je při obou transakcích totožný. Cena nákupu a prodeje jsou fixovány v okamžiku sjednání obchodu.}, jejichž cílem je také snížit kreditní riziko podstupované věřitelem.\\

Jestliže budeme předpokládat, že neexistuje riziko defaultu na žádné ze zúčastněných stran, byla by hodnota totálního swapu v libovolném časovém okamžiku rovna hodnotě pokladového dluhopisu snížené o současnou hodnotu plateb odvíjejících se od referenční sazby.

V okamžiku uzavření totálního swapu by jeho hodnota měla být rovna nule. To znamená, že současná hodnota plateb navázaných na refereční sazbu je rovna  tržní hodnotě dluhopisu v okamžiku uzavření obchodu. Tyto platby totiž představují ``splátku'' úvěru ve výši třžní hodnoty dluhopisu. Teoreticky by tak výše spreadu, o kterou se navyšuje referenční sazba LIBOR, měla být rovna nule. V praxi však finanční instituce požadují spread, který má kompenzovat kreditní riziko protistrany. Ta by totiž utrpěla ztrátu v případě defaultu investora a poklesu tržní hodnoty pokladového dluhopisu. Výše tohoto spreadu tak závisí na pravděpodobnosti defaultu investora a emitenta dluhopisu a korelaci mezi těmito dvěma pravděpodobnostmi.

\section{Opce na kreditní spread}

V případě opcí na kreditní spread závisí výplata na výši kreditního spreadu popř. na ceně defaultně senzitivního aktiva. Nejčastěji jsou tyto opce konstruovány tak, že v případě defaultu zanikají. Jestliže chce tedy investor současnou ochranu před zvýšením kreditního spreadu a defaultu protistrany, musí kromě opce na kreditní spread nakoupit také kreditní swap.\\

Výplata jednoho z typů opcí na kreditní spread je definován jako
\begin{equation*}
D max(K-S_T,0)
\end{equation*}
popř.
\begin{equation*}
D max(S_T - K,0)
\end{equation*}
kde $S_T$ je specifikovaný kreditní spread v době splatnosti opce a $K$ je realizační spread. Parametr $D$ představuje duraci, která je použita pro převod spreadu na cenu. Tato opce může být s příslušnou úpravou konvexity oceněna pomocí Black-Scholova modelu za předpokladu, že budoucí spready jsou v případě, že nedojde k defaultní události, lognormálně rozděleny. Klasický Black-Scholův vzorec musí být vynásoben pravděpodobností, že v průběhu životnosti opce nedojde ke vzniku defaultní situace.\\

Dalším typem opce na kreditní swap je evropská kupní popř. prodejní opce na aktivum, jehož cena je závislá na vzniku defaultní události. Klasickým příkladem takovéhoto aktiva je floatový dluhopis. V tomto případě je výplata dána vzorci
\begin{equation*}
max(K-S_T,0)
\end{equation*}
popř.
\begin{equation*}
max(S_T - K,0)
\end{equation*}
kde $S_T$ je cena aktiva v době splatnosti opce a $K$ je realizační cenou aktiva. Pro ocenění takovéto opce je opět možné použít Black-Scholův model za předpokladu, že cena pokladového aktiva má v době splatnosti opce za předpokladu, že nedojde k defaultní události, lognormální rozdělení. Stejně jako v předchozím případě musí být příslušný vzorec vynásoben pravděpodobností, že nedojde ke vzniku defaultní události.\\

Opce na kreditní spread jsou někdy součástí jiný produktů. Příklady těchto produktů jsou následující:
\begin{itemize}
\item garance, že spread nad refereční sazbou v případě půjčky nepřesáhne určitou limitní úroveň
\item předplatitelnou půjčku s fixním spreadem nad refereční sazbou
\item právo vstoupit do asset swapu
\item právo ukončit asset swap
\item právo vstoupit do kreditního swapu
\item právo ukončit kreditní swap
\end{itemize}

\section{Sekuritizované dluhopisy}
Sekuritizované dluhopisy, nebo-li CDO (collateralized debt obligation), jsou dluhové cenné papíry, které byly emitovány proti portfoliu úvěrů\footnote{Nejčastěji se jedná o spotřebitelské úvěry, úvěry z kreditních karet nebo hypotéky.}. Cash-flow generované z těchto pokladových úvěrů je rozděleno do několika skupin - tzv. tranší. Kromě cash-flow se mezi jednotlivé tranše dělí také potenciální ztráty z titulu realizace kreditního rizika. Kreditní riziko jednotlivých tranší je dáno způsobem, kterým se dělí případné kreditní ztráty mezi tranše, a korelací mezi pravděpodobnostmi defaultu jednotlivých dlužníků podkladových půjček.

Jako příklad uvažujme sekuritizovaný dluhopis, který se skládá ze čtyř tranší. První tranše představuje 5\% nominální hodnoty pokladových půjček a generuje výnos 35\%. První tranše zároveň absorvuje kreditní ztráty až do výše 5\% nominálu - jedná se tedy o poměrně rizikové aktivum. Druhá tranše pak představuje 10\% nominální hodnoty, absorvuje kreditní ztráty v rozmezí 5\% - 15\% nominálu a generuje výnos 15\%. Podobně třetí tranše představuje 10\% nominálu sekuritizovaného dluhopisu a na její vrub jdou kreditní ztráty v rozmezí 15\% - 25\% nominální hodnoty. Výnos této tranše je 7.5\%. Poslední čtvrtá tranše přestavuje objemově 75\% nominálu a absorvuje kreditní ztráty přesahující 25\% nominální hodnoty sekuritizovaného dluhopisu. Její výnos je 6\%.

Z výše uvedeného příkladu je patrné, že jednotlivé tranše se liší nejenom výnosem ale také kreditním rizikem. První tranše s sebou přináší nejvyšší výnos ale také nejvyšší kreditní riziko. Čtvrtá tranše naopak generuje nejnižší výnos a nejnižší kreditní riziko. Jednotlivé tranše je možné opatřit ratingem a prodávat samostatně. Sekuritizace tak představuje možnost, jak proti portfoliu půjček s nižším ratingem emitovat cenné papíry s vysokým ratingem. Sekuritizací půjček se zabývají finanční instituce - ty nejčastěji prodávají vyšší tranše s nižším kreditní rizikem a nižší tranše si ponechá ve své portfoliu.

\section{Úprava cen derivátů o riziko defaultu}
Úprava ceny derivátu o riziko defaultu může být poměrně komplikovaná a to z důvodu nettingu, existence zástavních instrumentů, definování tzv. defaultních událostí apod. V této kapitole budeme od těchto případných komplikací odhlížet.

Předpokládejme, že následující dvě skupiny proměnných jsou vzájmně nezávislé
\begin{itemize}
\item proměnné ovlivňující hodnotu derivátu v modelovém světě bez existence defaultu
\item proměnné ovlivňující výskyt defaultu protistrany a míra náhrady v případě defaultu
\end{itemize}
Dále předpokládejme, že nárokovaná částka v případě defaultu je rovna nezdefaultované hodnotě derivátu a definujme
\begin{center}
\begin{tabular}{l l}
$f(t)$		& hodnota derivátu v čase $t$, která zahrnuje možnost defaultu protistrany \\
$f^*(t)$	& hodnota totožného derivátu v čase $t$ v modelovém světě bez existence\\
		& defaultu\\
\end{tabular}
\end{center}

\subsection{Deriváty, které jsou aktivy}
Příkladem derivátu, který pro jednu ze zúčastněných stran představuje vždy aktivum, je např. opce.

Je možné dokázat, že
\begin{equation*}
f(0) = f^*(0)e^{-[y(T)-y^*(T)]T}
\end{equation*}
V této rovnici představuje $y(T)$ výnosovou míru z $T$-ročního diskontního dluhopisu emitovaného protistranou se stejným ratingem jako protistrana v námi uvažovaném derivátovém kontraktu. $y^*(T)$ reprezentuje výnosovou míru ekvivalentního bezrizikového dluhopisu. Jestliže dojde k defaultu, je ztráta jako procento nezdefaultované hodnoty stejná v případě dluhopisu i derivátu. Protože nezdefaultovaná hodnota dluhopisu a derivátu je nezávislá na pravděpodobnosti defaultu, je procentní snížení ceny derivátu z důvodů rizika defaultu stejné jako v případě dluhopisu. Hodnota dluhopisu je $e^{[-y(T)-y^*(T)]T}$ krát jeho nezdefaultovaná hodnota. Ten samý závěr platí také pro derivát.

Výše uvedená rovnice demonstruje možnost využití diskontní sazby $y-y^*$ pro výpočet hodnoty $f^*(0)$. Protože $f^*(t)$ je získáno diskontováním očekávaného cash-flow v rizikově neutrálním světě sazbou $y^*$, je možné $f$ analogicky vypočíst diskontováním očekávaného cash-flow v rizikově neutrálním světě sazbou $y^* + (y - y^*) = y$. Jedna z interpretací této rovnice tedy zní, že bylo použito vyšší diskontní sazby, která odráží riziko defaultu.

Je důležité si uvědomit, že bezriková sazba vstupuje do klasického ocenění derivátů dvěma způsoby a to (a) při definování očekávaného výnosu podkladového aktiva a (b) při diskontování tohoto očekávaného výnosu. Ve výše uvažovaného rovnici jsme však změnili pouze diskontní sazbu. To znamená, že např. při ocenění derivátů pomocí binomického stromu je třeba jako míru růstu podkladového aktiva použít bezrizikovou sazbu, ale pro diskontování očekávaného cash-flow je třeba použít sazbu, která zohledňuje riziko defaultu.

\subsection{Deriváty, které mohou být aktivy i pasivy}
Příkladem derivátu, který může být pro jednu ze zúčastněných stran jako aktivem tak pasivem, je např. forward a swap.

Nechť k defaultu může dojít pouze v čase $t_1, t_2, ..., t_n$. Definujme $v_i$ jako hodnotu kontraktu, který v čase $t_i$ vyplácí $max[f^*(t_i),0]$ a $u_i$ jako $u_i = p_i(1-R)$, kde $p_i$ je pravděpodobností defaultu v čase $t_i$ a $R$ je mírou náhrady. Z rovnice (21.1) vyplývá, že přibližné vyjádření $u_i$ je
\begin{equation}
u_i = e^{-[y(t_{i-1})-y^*(t_{i-1})]t_{i-1}}-e^{-[y(t_i)-y^*(t_i)]t_i}
\end{equation}
kde $y(t)$ resp. $y^*(t)$ jsou výnosovou mírou z korporátních resp. státních dluhopisů se splatností v čase $t$. Jestliže budeme, stejně jako v předchozí podkapitole, předpokládat, že nárokovaná částka z derivátů je rovna jejich nezdefaultované hodnotě, představuje $u_i$ procentní ztrátu z titulu defaultu v čase $t_i$. Proměnná $v_i$ představuje současnou hodnotu nároku v čase $t_i$. Z toho plyne
\begin{equation*}
f^*(0)-f(0)=\sum^n_{i=1}u_i v_i
\end{equation*}

\subsubsection{Demonstrace na příkladě měnového swapu}

Předpokládejme, že finanční instituce vstoupí do fixního měnového swapu. V rámci tohoto swapu obdrží úrokové platby v USD a platí úrokové platby v GBP. Nominály v obou měnách jsou vyměněny na konci životnosti swapu. Bližší údaje týkající se tohoto swapu jsou
\begin{center}
\begin{tabular}{l l}
životnost swapu			& 5 let \\
frekvence plateb		& roční \\
GBP úroková sazba		& 10\% ročně \\
USD úroková sazba		& 5\% ročně \\
GBP nominál			& 50 miliónů GBP \\
USD nominál			& 100 milónů USD \\
počáteční měnový kurz		& 2.000 \\
volatilita měnového kurzu	& 15\%
\end{tabular}
\end{center}
Předpokládejme, že USD a GBP zero křivky jsou ploché a konstantní po celou dobu životnosti měnového swapu. Dále předpokládejme, že jedno, dvou, tří, čtyř a pětiroční nezajištěné dluhopisy emitované uvažovanou protistranou by měli spread 25, 50, 70, 85 a 95 bazických bodů nad odpovídající bezrizikovou sazbou. Pro zjednodušení také předpokládejme, že k defaultu protistrany může dojít pouze v den, kdy jsou směňovány úrokové platby. To znamená, že $n=5$ a $t_1 = 1, t_2 = 2, ..., t_5 = 5$.

Protože úrokové sazby jsou konstantní, víme, že hodnota fiktivního podkladového GBP dluhopisu\footnote{Pouze připomeňme, že každý swap lze rozložit na dvojici fiktivních podkladových dluhopisů, které representují jednotlivé nohy.} je v libovolný platební den  $t_i$ 55 miliónů GBP. Podobně hodnota fiktivního pokladového USD dluhopisu je v libovolný platební den $t_i$ 105 miliónů USD. Hodnota swapu v čase $t_i$ je tedy $105 - 55S(t_i)$ USD, kde $S(t)$ je měnový kurz v čase $t$. Proměnná $v_i$ je hodnotou derivátu, který vyplácí
\begin{equation*}
v_i = max[105-55S(t_i),0]
\end{equation*}
\begin{equation*}
v_i = 55 max[105/55 - S(t_i),0]
\end{equation*}
miliónů dolarů v čase $t_i$. To je 55 násobek hodnoty hodnoty měnové prodejní opce s realizačním kurzem 105/55 = 1.90909, počátečním měnovým kurzem 2.000, domácí úrokovou sazbou 5\% při ročním úročení (4.897\% při spojitém úročení), cizoměnovou úrokovou sazbou 10\% při ročním úročení (9.531\% při spojitém úročení), volatilitou 15\% a maturitou $t_i$. Například hodnota této opce je pro $t_3$ rovna 0.246403 a $v_3$ je tedy rovno $55 \cdot 0.246403 = 13.5522$. Parametry $u_i$ je možné vypočíst podle (22.2).
\begin{center}
\begin{tabular}{c c c c}
\textbf{Maturita $t_i$} &
\textbf{$u_i$} &
\textbf{$v_i$} &
\textbf{$u_i v_i$} \\
\hline
1 & 0.002497 &  5.9758 & 0.0149 \\
2 & 0.007459 & 10.2140 & 0.0761 \\
3 & 0.010831 & 13.5522 & 0.1468 \\
4 & 0.012647 & 16.2692 & 0.2058 \\
5 & 0.012962 & 18.4967 & 0.2398 \\
\hline
Celkem & & & 0.6834 \\
\hline
\end{tabular}
\end{center}
Celkové ocenění rizika defaultu je tedy 0.6834 miliónů USD. Níže uvedená tabulka zobrazuje data pro analogickou kupní měnovou opci.
\begin{center}
\begin{tabular}{c c c c}
\textbf{Maturita $t_i$} &
\textbf{$u_i$} &
\textbf{$v_i$} &
\textbf{$u_i v_i$} \\
\hline
1 & 0.002497 & 5.9785 & 0.0149 \\
2 & 0.007459 & 5.8850 & 0.0439 \\
3 & 0.010831 & 5.4939 & 0.0595 \\
4 & 0.012647 & 5.0169 & 0.0634 \\
5 & 0.012962 & 4.5278 & 0.0587 \\
\hline
Celkem & & & 0.2404 \\
\hline
\end{tabular}
\end{center}
Celkové ocenění rizika defaultu je oproti prodejní opci výrazně nižší a to na úrovni 0.2404 miliónů USD. Tento příklad demonstruje obecné pravidlo, že finanční instituce má větší expozici na kreditní riziko, když obdrží platby v měně s nižší úrokovou sazbou a platí platby v měně s vyšší úrokovou sazbou. Důvod je ten, že měna s nižší úrokovou sazbou by měla vůči měně s vyšší úrokovou sazbou posilovat. To má v případě měnového swapu za následek relativní posilování hodnoty fiktivního podkladového dluhopisu v měně s nižší úrokovou sazbou vůči hodnotě fiktivního podkladového dluhopisu v měně s vyšší úrokovou sazbou.

Z celkových nákladů rizika defaultu je pak možné na anuitní bázi vypočíst vypočítat roční kreditní spread. Jestliže bychom např. uvažovali výše popsaný měnový swap, který by byl uzavřen ekvivalentním měnovým swapem, pak by celkové náklady rizika defaultu z pohledu finanční instituce byly 0.6834 + 0.2404 = 0.9238 miliónů USD nebo-li 0.924\% nominálu\footnote{Implicitní podmínkou je, že se protistrana původního měnového swapu neshoduje s protistranou, se kterou byl sjednán uzavírací obchod.}. To, při konstantní úrokové sazbě 5\%, odpovídá ročnímu kreditnímu spreadu 21 bazických bodů. Finanční instituce byt tak měla nabízet tento měnový swap dané protistraně s minimálním spreadem 21 bazických bodů.

\subsubsection{Úrokový vs. měnový swap}
Uvažujme výše popsanou dvojici vzájemně se kompenzujících měnových swapů. Dále uvažujme podobnou dvojici vzájemně se kompenzujících úrokových swapů\footnote{Za podobný úrokový swap považujme swap se stejným nominálem, frekvencí plateb a fixní popř. pohyblivé úrokové sazby na shodnou měnu.}. Dopad kreditního rizika na ocenění úrokových swapů je výrazně nižší než v případě měnových swapů. Důvod je ten, že očekávaná expozice na úrokové swapy začíná na nule, s v průběhu jejich životnosti vrcholí a postupně konverguje zpět k nule. V případě měnových swapů však očekávaná expozice narůstá po celou dobu životnosti. Důvodem vyšší expozice v případě měnových swapů je především finální směna nominálů v době splatnosti.
\begin{center}
	\begin{pspicture}(0,0)(10,7)
		\rput(5.0,0.5){Očekávaná expozice na dvojici vzájemně se kompenzujících}
		\rput(5.0,0.0){úrokových a měnových swapů}

		\psline[arrows=->](0.5,1.0)(9.5,1.0)
		\psline[arrows=->](0.5,1.0)(0.5,7.0)

		\pscurve[linewidth=0.5mm](0.5,1.0)(4.0,4.5)(9.0,6.0)
		\pscurve[linewidth=0.5mm](0.5,1.0)(4.75,2.0)(9.0,1.0)

		\rput(9.2,1.4){\small{čas}}
		\rput(2.2,6.8){\small{očekávaná expozice}}

		\rput(8.5,5.5){\tiny{měnový swap}}
		\rput(8.2,1.8){\tiny{úrokový swap}}
	\end{pspicture}
\end{center}

\section{Konvertibilní dluhopisy}

Majitel konvertibilního dluhopisu je zároveň majitelem opce, která mu umožňuje vyměnit dluhopis za akcie emitenta. Počet akcií, které lze získat za jeden dluhopis nazýváme konverzním poměrem. Konverzní poměr je nejčastěji konstatní, někdy může být funkcí času. Konvertibilní dluhopisy jsou také většinou svolatelné. To znamená, že jejich emitent je majitelem opce, která mu umožňuje dluhopis odkoupit za předem stanovenou cenu. Majitel dluhopisu má pak v případě jeho svolání právo provést výměnu dluhopisu za akcie. Svolání dluhopisu tak plní funkci vynucené konverze.

Kreditní riziko hraje v ocenění konvertibilního dluhopisu významnou roli. Jestliže bychom ignorovali kreditní riziko, získali bychom nízkou cenu konverzní opce, protože cash-flow generované dluhopisem by bylo za předpokladu, že nedojde k jeho konverzi, nadhodnocené. Důvodem je, že toto cash-flow by bylo diskontováno bezrizikovou úrokovou mírou.

\subsection{Ocenění konveritibilního dluhopisu}
Nejčastěji používaný model pro ocenění konvertibilního dluhopisu vychází z modelování ceny akcie emitenta konvertibilního dluhopisu. Cena akcie se stejně jako v případě opcí modeluje pomocí binomického stromu. Délka binomického stromu se shoduje s životností konvertibilního dluhopisu. Pro konečné uzly binomického stromu je nejprve určena hodnota konvertibilního dluhopisu s ohledem na konverzní opci, kterou vlastní majitel dluhopisu. Následně se v binomickém stromu pokračuje směrem k prvnímu uzlu. V těch uzlech binomického stromu, se kterými je spojeno právo konverze, se testuje zda-li uplatnění konverze optimální. Dále se také testuje, zda-li je z pohledu emitenta konvertibilního dluhopisu vhodné tento dluhopis svolat. V případě, že emitent provede svolání dluhopisu, je třeba znovuotestovat optimálnost případné konverze. Tento postup je ekvivalentní stanovení hodnoty konvertibilního dluhopisu v příslušném uzlu jako
\begin{equation*}
max[min(Q_1, Q_2), Q_3]
\end{equation*}
kde $Q_1$ je hodnota získaná zpětným průchodem jednotlivých uzlů binomického stromu za předpokladu, že dluhopis nebyl konvertován ani svolán, $Q_2$ je svolací cena dluhopisu a $Q_3$ je hodnota dluhopisu v případě provedené konverze.

Jednou z komplikací výše uvedeného postupu je volba vhodné diskotní úrokové sazby. V případě, že by konverzní opce nebyla uplatněna, měla by být použita úroková míra, která by zohledňovala kreditní riziko emitenta. Naopak v případě, že by konverzní opce byla uplatněna, měla by být použita bezriziková úroková míra, protože konvertibilní dluhopis by byl ve skutečnosti akcií. V praxi však dopředu nevíme, zda-li bude konverzní opce uplatněna či nikoliv. Hodnota dluhopisu je tak v každém uzlu binomického stromu rozdělena na dvě části a to (a) cena za předpokladu, že dluhopis bude konvertován na akcii a (b) cenu za předpokladu, že konverzní opce nebude uplatněna.\\

\noindent \textbf{Příklad:} Uvažujme devíti měsíční diskontní dluhopis s nominální hodnotou 100 USD emitovaný společností XYZ. Předpokládejme že tento dluhopis můžeme vyměnit za dvě akcie firmy XYZ v libovolný okamžik v průběhu devítiměsíčního horizontu. Dále předpokládejme, že tento dluhopis může být svolán v libovolný okamžik za cenu 115 USD. Počáteční cena akcie je 50 USD, její roční volatilita 30\% a z akcie není ve sledovaném období vyplácena dividenda. Bezriziková úroková sazba je konstatní na úrovni 10\% p.a. Úroková míra zohledňující kreditní riziko emitenta XYZ je 15\% p.a. Níže uvedený obrázek představuje binomický strom, který může být použit pro ocenění konvertibilního dluhopisu.
\begin{center}
	\begin{pspicture}(0,0)(12,12)
		\rput(6.0,0.0){Binomický strom pro ocenění konvertibilního dluhopisu}

		\psline[arrows=->](1,4.0)(4,5.0)
		\psline[arrows=->](1,4.0)(4,3.0)

		\psline[arrows=->](4,5.0)(7,6.0)
		\psline[arrows=->](4,5.0)(7,4.0)
		\psline[arrows=->](4,3.0)(7,4.0)
		\psline[arrows=->](4,3.0)(7,2.0)

		\psline[arrows=->](7,6.0)(10,7.0)
		\psline[arrows=->](7,6.0)(10,5.0)
		\psline[arrows=->](7,4.0)(10,5.0)
		\psline[arrows=->](7,4.0)(10,3.0)
		\psline[arrows=->](7,2.0)(10,3.0)
		\psline[arrows=->](7,2.0)(10,1.0)

		\rput(0.2,4.0){\psframebox*{\tiny{\begin{tabular}{r} 50.00 \\  76.55 \\ 28.40 \\ 104.95 \end{tabular}}}}
		\rput(4.0,5.8){\psframebox*{\tiny{\begin{tabular}{r} 58.09 \\ 116.18 \\  0.00 \\ 116.18 \end{tabular}}}}
		\rput(4.0,3.8){\psframebox*{\tiny{\begin{tabular}{r} 43.04 \\  33.03 \\ 65.05 \\  98.08 \end{tabular}}}}
		\rput(7.0,6.8){\psframebox*{\tiny{\begin{tabular}{r} 67.49 \\ 134.98 \\  0.00 \\ 134.98 \end{tabular}}}}
		\rput(7.0,4.8){\psframebox*{\tiny{\begin{tabular}{r} 50.00 \\  61.95 \\ 43.66 \\ 105.61 \end{tabular}}}}
		\rput(7.0,2.8){\psframebox*{\tiny{\begin{tabular}{r} 37.04 \\   0.00 \\ 96.32 \\ 96.32 \end{tabular}}}}
		\rput(10.7,7.0){\psframebox*{\tiny{\begin{tabular}{r} 78.42 \\ 156.84 \\  0.00 \\ 156.84 \end{tabular}}}}
		\rput(10.7,5.0){\psframebox*{\tiny{\begin{tabular}{r} 58.09 \\ 116.18 \\  0.00 \\ 116.18 \end{tabular}}}}
		\rput(10.7,3.0){\psframebox*{\tiny{\begin{tabular}{r} 43.04 \\   0.00 \\ 100.00 \\ 100.00 \end{tabular}}}}
		\rput(10.7,1.0){\psframebox*{\tiny{\begin{tabular}{r} 31.88 \\   0.00 \\ 100.00 \\ 100.00 \end{tabular}}}}

		\rput(1.0,3.7){\small{A}}
		\rput(4.0,4.7){\small{B}}
		\rput(4.0,2.7){\small{C}}
		\rput(7.0,5.7){\small{D}}
		\rput(7.0,3.7){\small{E}}
		\rput(7.0,1.7){\small{F}}
		\rput(10.0,6.7){\small{G}}
		\rput(10.0,4.7){\small{H}}
		\rput(10.0,2.7){\small{I}}
		\rput(10.0,0.7){\small{J}}
	\end{pspicture}
\end{center}
U každého uzlu jsou uvedena čtyři čísla. První z těchto čísel je cenou akcie, druhé číslo representuje hodnotu dluhopisu v případě jeho konverze za akcii (tzv. akciová složka konvertibilního dluhopisu) a třetí číslo představuje hodnotu v dluhopisu v případě, že není uplatněna konverzní opce (tzv. dluhová složka konvertibilního dluhopisu). Poslední čtvrtě číslo je výslednou hodnotou konvertibilního dluhopisu. Parametry výše uvedeného stromu jsou $u=1.1618, d = 0.8607, a=1.0253$ a $p=0.5467$.

V konečných uzlech je hodnota konvertibilního dluhopisu rovna $max[100, 2S_T]$. V uzlu G je hodnota akcie 78.48 USD. Konverzní opce tedy bude uplatněna a hodnota konvertibilního dluhopisu je rovna jeho akciové složce 156.84 USD\footnote{$2 \cdot 78.48 = 156.84$ USD}. V uzlu I je hodnota akcie 43.04 USD. Konverzní opce tedy nebude uplatněna a hodnota akciové složky konvertibilního dluhopisu je nulová. Hodnota konvertibilního dluhopisu je tak zcela dána jeho dluhovou složkou a je rovna 100.00 USD. S tím, jak postupujeme binomickým stromem směrem k jeho vrcholu, je třeba v každém uzlu provést kontrolu, zda-li je z pohledu majitele optimální uplatnit konverzní opci a z pohledu emitenta provést svolání dluhopisu.

V uzlu D je hodnota akciové složky konvertibilního dluhopisu rovna
\begin{equation*}
(0.5467 \cdot 156.84 + 0.4533 \cdot 116.18)e^{-0.10 \cdot 0.25} = 134.98
\end{equation*}
Vzhledem k tomu, že by racionální investor v uzlu D vždy uplatnil konverzní opci, je hodnota dluhové složky konvertibilního dluhopisu nulová. Svolání dluhopisu nebo uplatnění konverzní opce nemění v uzlu D jeho hodnotu, protože se již v podstatě jedná o akcii\footnote{Jestliže se emitent rozhodne v uzlu D konvertibilní dluhopis svolat, uplatní jeho vlastník konverzní opci. Jestliže emitent dluhopis nesvolá, bude konverzní opce uplatněna v uzlu G nebo H.}.

Naopak v uzlu F nebude konverzní opce uplatněna a konverzní dluhopis je tak možné považovat za standardní dluhopis. Hodnota akciové složky je tedy nulová a hodnota dluhové složky je rovna
\begin{equation*}
(0.5467 \cdot 100.00 + 0.4533 \cdot 100.00)e^{-0.15 \cdot 0.25} = 96.32
\end{equation*}

V případě uzlu E je situace o něco složitější. Hodnota akciové složky je totiž rovna
\begin{equation*}
(0.5467 \cdot 116.18 + 0.4533 \cdot 0.00)e^{-0.10 \cdot 0.25} = 61.95
\end{equation*}
a hodnota dluhové složky je rovna
\begin{equation*}
(0.5467 \cdot 0.00 + 0.4533 \cdot 100)e^{-0.15 \cdot 0.25} = 43.66
\end{equation*}
Celková hodnota konvertibilního dluhopisu je tedy rovna $61.95 + 43.66 = 105.61$ USD. Je zřejmé, že v uzlu E by dluhopis neměl být svolán ani konvertován na akcie.

V uzlu B je ma akciová složka konvertibilního dluhopisu hodnotu
\begin{equation*}
(0.5467 \cdot 134.98 + 0.4533 \cdot 61.95)e^{-0.10 \cdot 0.25} = 99.36
\end{equation*}
a dluhová složka hodnotu
\begin{equation*}
(0.5467 \cdot 0 + 0.4533 \cdot 43.66)e^{-0.15 \cdot 0.25} = 19.06
\end{equation*}
Celková hodnota konvertibilního dluhopisu je tak $99.36 + 19.06 = 118.42$ USD. V tomto uzlu je z pohledu emitenta racionální svolat dluhopis, protože to má za následek konverzi dluhopisu a pokles jeho hodnoty na $2 \cdot 58.09 = 116.18$ USD.

Jestliže budeme výše naznačenou logikou postupovat ke špičce binomického stromu, získáme v uzlu A hodnotu konvertibilního dluhopisu rovnu 104.95 USD. Jesliže by dluhopis nebyl svolán, jeho hodnota je $100e^{-0.15 \cdot 0.75} = 89.36$ USD. Hodnota konverzní opce je tedy $104.95 - 89.35 = 15.59$ USD.

V případě, že jsou z akcie ve sledovaném období vypláceny dividendy nebo konvertibilní dluhopis přináší svému majiteli kupóny, je třeba toto cash-flow zohlednit ve výpočtech. Nejprve v každém uzlu předpokládáme, že nebude uplatněna konverzní opce. Do dluhové složky započteme současnou hodnotu všech kupónů vyplácených v následujících uzlech. Po té v průběhu testováni, zda-li je optimální uplatnit konverzi dluhopisu, je třeba zohlednit současnou hodnotu dividend vyplácených v následujících uzlech.

\chapter{Opce na reálná aktiva}

\section{Oceňování budoucího cash-flow současnou hodnotou}

Nejčastější způsob oceňování reálných aktiv je pomocí metod, které vycházejí ze současné hodnoty očekávaného cash-flow. Nejznámější oceňovací metodou, která spadá do této skupiny, je NPV (net present value). Nejslabším místem metody NPV je odhad očekávaného cash-flow, který může být v případě některých aktiv, jako např. výrobní podnik, značně komplikovaný.

V rámci metody NPV je nejprve určeno očekávané cash-flow. To je následně diskontováno úrokovou sazbou odrážející rizikovost daného aktiva. Diskontní úrokovou sazbu je možné získat pomocí modelu oceňování kapitálových aktiv (CAPM model). Postup určení úrokové sazby je následující
\begin{itemize}
\item Vybereme skupinu společností, jejichž předmět podnikání odpovídá uvažovanému aktivu.
\item Vypočteme parametry beta pro dané společnosti. Tyto parametry zprůměrujeme a průměr použijeme jako odhad pro parametr beta uvažovaného aktiva.
\item Riziková sazba aktiva je rovna bezrizikové sazbě navýšené o součin parametru beta a rozdílu mezi výnosem akciového indexu a bezrizikové sazby.
\end{itemize}
\begin{equation*}
r_A = r_f + \beta_A(i_{index} - r_f)
\end{equation*}

Další komplikací NPV metody jsou vnořené opce, které řada aktiv obsahuje. Ty například umožňují investorovi daný projekt předčasně ukončit nebo naopak rozšířit. Vnořené opce s sebou kromě problémů s predikcí očekávaného cash-flow přináší také problém při stanovení rizikové úrokové sazby pro základní projekt (tj. projekt bez vnořených opcí). Společnosti, které byly použity pro odhad parametru beta, mají své vlastní vnořené opce, které se ve výpočtu promítly, a nemusí být proto z tohoto pohledu srovnatelné s uvažovaným aktivem.

\section{Rozšíření rizikově neutrálního ocenění}

\subsection{Jedna podkladová proměnná}

Uvažujme aktivum, jehož cena $f$ je funkcí proměnné $\theta$ a času $t$. Předpokládejme, že proměnná $\theta$ sleduje proces
\begin{equation*}
\frac{d \theta}{\theta} = m dt + s dz
\end{equation*}
kde $dz$ je Wienerův proces. Parametr $m$ je očekávanou mírou růstu a $s$ volatilitou proměnné $\theta$. Cena aktiva $f$ pak sleduje proces
\begin{equation*}
d f = \mu f dt + \sigma f dz
\end{equation*}
Dle kapitoly 18 platí
\begin{equation*}
\lambda = \frac{\mu - r}{\sigma}
\end{equation*}
kde $r$ je bezriziková úroková míra a $\lambda$ je tržní cenou rizika $\theta$. Protože $f$ je funkcí $\theta$ a $t$, je možné použít It\^o lemu k vyjádření $\mu$ a $\sigma$ pomocí parametrů $m$ a $s$.
\begin{equation*}
\mu f = m \theta \frac{\partial f}{\partial \theta} + \frac{\partial f}{\partial t} + \frac{1}{2}s^2 \theta^2 \frac{\partial^2 f}{\partial \theta^2}
\end{equation*}
\begin{equation*}
\sigma f = s \theta \frac{\partial f}{\partial \theta}
\end{equation*}
Substitucí do první ze dvou uvedených rovnic získáváme diferenciální rovnici
\begin{equation}
rf = \frac{\partial f}{\partial t} + \theta \frac{\partial f}{\partial \theta}(m-\lambda s) + \frac{1}{2}s^2\theta^2\frac{\partial^2 f}{\partial \theta^2}
\end{equation}
Tato rovnice je velmi podobná Black-Scholově diferenciální rovnici (10.7). Jedná se totiž o totožnou rovnici s tím rozdílem, že $\theta$ nahrazuje $S$, tj. cenu akcie s nulovým dividendovým výnosem. Protože $\theta$ je dle této diferenciální rovnice cenou aktiva a protože platí $m - r = \lambda s$, lze upravit druhý člen pravé strany rovnice (23.1) do tvaru
\begin{equation*}
r\theta \frac{\partial f}{\partial \theta}
\end{equation*}
Rovnice (23.1) po této substituci odpovídá diferenciální rovnici klasického Black-Scholes modelu, kdy cena derivátu závisí na ceně podkladového aktiva $S$ s dividendovým výnosem $q$. Cena podkladového aktiva $S$ je v tomto případě představována parametrem $\theta$ a pro výnos plynoucí z $\theta$ platí $q = r - m -\lambda s$. Očekávaná míra růstu parametru $\theta$ je tedy rovna $r - q = m - \lambda s$. Očekávané cash-flow generované derivátem diskontujeme bezrizikovou sazbou $r$. Obecně platí, že libovolné aktivum, které závisí na $\theta$, můžeme ocenit tak, že snížíme očekávanou míru růstu $\theta$ o $\lambda s$ a budeme předpokládat rizikově neutrální svět.

\subsection{Vícero podkladových proměnných}

Uvažujme $n$ proměnných $\theta_1$, $\theta_2$, ..., $\theta_3$, které sledují stochastický proces
\begin{equation*}
\frac{d \theta_i}{\theta_i} = m_i dt + s_i d z_i
\end{equation*}
pro $i = 1, 2, ..., n$, kde $dz_i$ představuje Wienerův proces. Parametry $m_i$ a $s_i$ představují očekávanou míru růstu a volatilitu a mohou být funkcí $\theta_i$ popř. času $t$. Korelaci mezi $\theta_i$ a $\theta_j$ označme jako $\rho_{ij}$. Proces, který sleduje cena aktiva $f$ závislého na $\theta_i$, má formu
\begin{equation*}
\frac{df}{f}=\mu dt + \sum^n_{i=1}\sigma_i dz_i
\end{equation*}
Ve výše uvedené rovnici představuje $\mu$ očekávanou míru růstu aktiva a $\sigma_i dz_i$ je rizikovou složkou výnosu z titulu parametru $\theta_i$. V kapitole 18 jsme prokázali
\begin{equation}
\mu - r = \sum^n_{i=1}\lambda_i \sigma_i
\end{equation}
kde $\lambda_i$ je tržní cenou rizika $\theta_i$. S využitím It\^o lemy lze $\mu$ a $\sigma_i$ vyjádřit pomocí $m_i$ a $s_i$. Rovnice (23.2) pak přejde do tvaru
\begin{equation*}
rf = \frac{\partial f}{\partial t} + \sum_i \theta_i \frac{\partial f}{\partial \theta_i}(m_i - \lambda_i s_i)+ \frac{1}{2}\sum_{i,k}\rho_{ik}s_i s_k \theta_i \theta_k \frac{\partial^2 f}{\partial \theta_i \partial \theta_k}
\end{equation*}
Libovolné aktivum tedy může být oceněňo, jako kdyby svět byl rizikově neutrální, za předpokladu, že očekávaná míra růstu každého podkladového parametru je $m - \lambda_i s_i$. Dalším nezbytným předpokladem jsou konstantní volatility $s_i$ a korelační koeficienty $\rho_{ij}$.

\subsection{Odhad tržní ceny rizika}

Oceňování v rizikově neutrálním světě obchází nutnost stanovení rizikových diskontních faktorů. Nicméně je zapotřebí určit tržní cenu rizika pro jednotlivé parametry $\theta_i$. Jestliže jsou pro daný parametr k dispozici historická data, je možné tržní cenu rizika odvodit pomocí modelu CAPM. Pro zjednodušení předpokládejme, že cena aktiva je výhradně závislá na parametru $\theta$ a definujme
\begin{center}
\begin{tabular}{l l}
$\mu$ & očekávaná míra růstu aktiva\\
$\sigma$ & volatilita očekávané míry růstu aktiva\\
$\lambda$ & tržní cena rizika parametru $\theta$\\
$\rho$ & korelace mezi procentní změnou ceny aktiva a akciového indexu\\
$\mu_m$ & očekávaná míra růstu akciového indexu\\
$\sigma_m$ & volatilita očekávané míry růstu akciového indexu\\
$r$ & krátkodobá bezriziková úroková sazba\\
\end{tabular}
\end{center}
Ze spojitého modelu CAPM získáváme
\begin{equation*}
\mu - r = \frac{\rho \sigma}{\sigma_m}(\mu_m - r)
\end{equation*}
Dále v souladu s kapitolou 18 platí
\begin{equation*}
\mu - r = \lambda \sigma
\end{equation*}
Z toho vyplývá
\begin{equation*}
\lambda = \frac{\rho}{\sigma_m}(\mu_m - r)
\end{equation*}
Tuto rovnici je možné použít pro odhad tržní ceny rizika parametru $\theta$. V případě, že nejsou k dispozici historická data, je možné použít data pro podobné parametry.

Dalším problematickým místem je hodnota korelace $\rho$, která musí být často stanovena odhadem.

\section{Postup při oceňování nového projektu}

Klasické metody oceňování jako např. P/E pro dané odvětví násobené aktuálními tržbami nemusí být v případě nových projektů aplikovatelné\footnote{Jedním z důvodů je také to, že nové projekty mívají často zpočátku záporné cash-flow.}. Vhodnější je projekt ocenit pomocí odhadu budoucích tržeb a cash-flow v rámci několika scénářů možného vývoje (projekce prodejů, nákladů, marží atd.). Možný vývoj klíčových parametrů by měl být popsán pomocí stochastických procesů. Hodnota projektu je určena jako současná hodnota očekávaného cash-flow diskontovaná bezrizikovou sazbou.\\

V roce 2000 Schwartz a Moon aplikovali tento postup na ocenění tehdy začínající firmy Amazon. Předpokládali, že tržby $R$ jsou dány stochastickým procesem
\begin{equation*}
\frac{dR}{R}=\mu dt + \sigma(t)dz_1
\end{equation*}
kde
\begin{equation*}
d \mu = \kappa(\tilde{\mu} - \mu)dt + \eta(t)dz_2
\end{equation*}
Očekávaná míra tržeb je tedy $\mu$, která sama sleduje stochastický proces s dlouhodobým průměrem $\tilde{\mu}$ a kovergencí $\kappa$ k tomuto dlouhodobému průměru. Volatilita tržeb $\sigma(t)$ v rámci modelu klesala z původní hodnoty 10\% za čtvrtletí na 5\% za čtvrtletí. Směrodatná odchylka růstu tržeb $\eta(t)$ je také deterministická a klesá z 3\% za čtvrtletí na nulu. Počáteční úroveň tržeb byla 356 miliónů USD za čtvrtletí a počáteční míra růstu tržeb byla 11\% za čtvrtletí. Hodnoty parametrů $\kappa$ a $\tilde{\mu}$ byly nastaveny na 7\% a 1.5\% za čtvrtletí. Uvažované Wienerovy procesy $dz_1$ a $dz_2$ nejsou korelované. Schwartz a Moon dále předpokládali, že náklady na prodané zboží představuje 75\% tržeb, ostatní variabilní náklady představují 19\% z tržeb a fixní náklady jsou ve výši 75 miliónů USD za čtvrtletí. Počáteční daňově uznatelná ztráta z titulu založení podniku představovala 559 miliónů USD a míra zdanění byla uvažována na úrovni 35\%. Tržní cena rizika $\lambda_R$ parametru $R$ byla odhadnuta s pomocí historických dat. Tržní cena rizika parametru $\mu$ byla odhadnuta na nulu. Rizikově neutrální proces pro $R$ je tedy
\begin{equation*}
\frac{d R}{R} = (\mu - \lambda_R \sigma)dt + \sigma dz_1
\end{equation*}
zatímco rizikově neutrální proces sledovaný $\mu$ zůstal beze změny.

Časový horizont analýzy byl stanoven na 25 roků a konečná hodnota projektu byla odhadnuta na desetinásobek operačního zisku před zdaněním. Dále bylo uvažováno, že počáteční hotovost v rámci projektu je 906 miliónů USD a že projekt zbankrotuje v případě, že kumulativní ztráta přesáhne tuto částku. Jednotlivé budoucí scénáře byly modelovány pomocí metody Monte Carlo.

\section{Ceny komodit}

V řadě případů je možné pro odhad rizikově neutrálního procesu, který sledují ceny komodit, použít ceny futures. Komodity jsou tedy podobné dosud uvažovaným investičním aktivům v tom, že je možné obejít přímý výpočet tržní ceny rizika. V rizikově neutrálním světě je očekávaná budoucí cena komodit rovna ceně futures této komodity. Jestliže budeme předpokládat, že míra růstu ceny komodity je funkcí času a že volatilita je konstantní, je rizikově neutrální proces ceny komodity $S$ dán vztahem
\begin{equation}
\frac{dS}{S} = \mu(t)dt + \sigma dz
\end{equation}
a cena futures vztahem
\begin{equation*}
F(t) = \hat{E}[S(t)] = S(0)e^{\int^t_0\mu(\tau)d\tau}
\end{equation*}
kde $\hat{E}$ značí očekávanou cenu komodity v rizikově neutrálním světě. Dalšími úpravami získáváme
\begin{equation*}
\ln F(t) = \ln S(0) + \int^t_0 \mu(\tau) d \tau
\end{equation*}
\begin{equation*}
\mu(t) = \frac{\partial}{\partial t}[\ln F(t)]
\end{equation*}

\subsection{Modelování ceny komodit pomocí trinomického stromu}

Rovnice (23.3) může být považována za zjednodušenou. V praxi totiž ceny komodit zpravidla oscilují kolem dlouhodobého průměru. Jako vhodnější by se pak jevila rovnice
\begin{equation}
d \ln S = [\theta(t) - a \ln S]dt + \sigma dz
\end{equation}
Rovnice (23.4) je tak analogií k modelům krátkodobých úrokových sazeb. Pro modelování tohoto procesu je možné použít trinomický strom a dlouhodobý průměr $\theta(t)$ je možné odhadnout ze vztahu $F(t) = \hat{E}[S(t)]$.\\

Ilustrujme tento postup pomocí trinomického stromu pro ropu. Předpokládejme, že spotová cena ropy je 20 USD za barel a jednoroční, dvouroční a tříleté futures mají cenu 22 USD, 23 USD a 24 USD. V rovnici (23.4) použijme parametry $a = 0.1$, $\sigma = 0.2$. Nejprve definujme pomocnou proměnnou $X$, jejíž počáteční hodnota je nula a která sleduje proces
\begin{equation*}
dX = -adt + \sigma dz
\end{equation*}
S využitím postupu popsaného v kapitole 20.1.6 je možné zkonstruovat trinomický strom.
\begin{center}
	\begin{pspicture}(0,0)(7,7)
		\rput(3.5,0){Trinomický strom proměnné $X$}

          \rput(0.2,3.5){\tiny{A}}
		\rput(0.2,3.3){\tiny{0.0000}}

		\psline[arrows=->](0.5, 3.0)(2.5,4.0)
		\psline[arrows=->](0.5, 3.0)(2.5,3.0)
		\psline[arrows=->](0.5, 3.0)(2.5,2.0)
		
          \rput(2.5,4.5){\tiny{B}}
		\rput(2.5,4.3){\tiny{0.3464}}
		\rput(2.5,3.5){\tiny{C}}
		\rput(2.5,3.3){\tiny{0.0000}}
		\rput(2.5,2.5){\tiny{D}}
		\rput(2.5,2.3){\tiny{-0.3464}}
		
          \psline[arrows=->](2.5, 4.0)(4.5,5.0)
          \psline[arrows=->](2.5, 4.0)(4.5,4.0)
          \psline[arrows=->](2.5, 4.0)(4.5,3.0)
          \psline[arrows=->](2.5, 3.0)(4.5,4.0)
          \psline[arrows=->](2.5, 3.0)(4.5,3.0)
          \psline[arrows=->](2.5, 3.0)(4.5,2.0)
          \psline[arrows=->](2.5, 2.0)(4.5,3.0)
          \psline[arrows=->](2.5, 2.0)(4.5,2.0)
          \psline[arrows=->](2.5, 2.0)(4.5,1.0)
          
          \rput(4.5,5.5){\tiny{E}}
		\rput(4.5,5.3){\tiny{0.6928}}
          \rput(4.5,4.5){\tiny{F}}
		\rput(4.5,4.3){\tiny{0.3464}}
		\rput(4.5,3.5){\tiny{G}}
		\rput(4.5,3.3){\tiny{0.0000}}
		\rput(4.5,2.5){\tiny{H}}
		\rput(4.5,2.3){\tiny{-0.3464}}
		\rput(4.5,1.5){\tiny{I}}
		\rput(4.5,1.3){\tiny{-0.6928}}

          \psline[arrows=->](4.5, 5.0)(6.5,5.0)
          \psline[arrows=->](4.5, 5.0)(6.5,4.0)
          \psline[arrows=->](4.5, 4.0)(6.5,5.0)
          \psline[arrows=->](4.5, 4.0)(6.5,4.0)
          \psline[arrows=->](4.5, 4.0)(6.5,3.0)
          \psline[arrows=->](4.5, 3.0)(6.5,4.0)
          \psline[arrows=->](4.5, 3.0)(6.5,3.0)
          \psline[arrows=->](4.5, 3.0)(6.5,2.0)
          \psline[arrows=->](4.5, 2.0)(6.5,3.0)
          \psline[arrows=->](4.5, 2.0)(6.5,2.0)
          \psline[arrows=->](4.5, 2.0)(6.5,1.0)
          \psline[arrows=->](4.5, 1.0)(6.5,2.0)
          \psline[arrows=->](4.5, 1.0)(6.5,1.0)
          
          \rput(6.5,5.5){\tiny{J}}
		\rput(6.5,5.3){\tiny{0.6928}}
          \rput(6.5,4.5){\tiny{K}}
		\rput(6.5,4.3){\tiny{0.3464}}
		\rput(6.5,3.5){\tiny{L}}
		\rput(6.5,3.3){\tiny{0.0000}}
		\rput(6.5,2.5){\tiny{M}}
		\rput(6.5,2.3){\tiny{-0.3464}}
		\rput(6.5,1.5){\tiny{N}}
		\rput(6.5,1.3){\tiny{-0.6928}}

	\end{pspicture}
\end{center}
Proměnná $S$ sleduje stejný proces jako $X$ navýšený o trend, který je funkcí času. Trinomický strom náhodné veličiny $S$ je tak možné získat posunutím trinomického stromu náhodné veličiny $X$. Počáteční cena ropy je 20 USD. Výchozí bod pro modelování náhodné veličiny $S$ tak bude $\ln 20$. Hodnoty náhodné veličiny $X$ na konci prvního roku jsou 0.3464, 0.0000, -0.3464. Uvažujme, že posunutí náhodné veličiny $S$ proti náhodné veličině $X$ na konci prvního roku je $\alpha_1$. Odpovídající hodnoty $\ln S$ jsou tak $3.464 + \alpha_1$, $\alpha_1$ a $-3.464 + \alpha_1$. Modelované hodnoty $S$ jsou tak rovny $e^{3.464 + \alpha_1}$, $e^{\alpha_1}$ a $e^{-3.464 + \alpha_1}$. Zároveň požadujeme, aby se očekávaná hodnota $S$ rovnala ceně futures. Musí tedy platit
\begin{equation*}
0.1667 e^{3.364 + \alpha_1} + 0.6666 e^{\alpha_1} + 0.1667 e^{-3.364 + \alpha_1} = 22
\end{equation*}
Řešením této rovnice je $\alpha_1 = 3.071$. Modelované hodnoty $S$ na konci roku jsou tedy 30.49, 21.56 a 15.25. Stejným způsobem lze vypočítat parametry posunu $\alpha_2$ a $\alpha_3$ pro druhý a třetí rok.
\begin{center}
	\begin{pspicture}(0,0)(7,7)
		\rput(3.5,0){Trinomický strom proměnné $S$}

          \rput(0.2,3.5){\tiny{A}}
		\rput(0.2,3.3){\tiny{20.00}}

		\psline[arrows=->](0.5, 3.0)(2.5,4.0)
		\psline[arrows=->](0.5, 3.0)(2.5,3.0)
		\psline[arrows=->](0.5, 3.0)(2.5,2.0)
		
          \rput(2.5,4.5){\tiny{B}}
		\rput(2.5,4.3){\tiny{30.49}}
		\rput(2.5,3.5){\tiny{C}}
		\rput(2.5,3.3){\tiny{21.56}}
		\rput(2.5,2.5){\tiny{D}}
		\rput(2.5,2.3){\tiny{15.25}}
		
          \psline[arrows=->](2.5, 4.0)(4.5,5.0)
          \psline[arrows=->](2.5, 4.0)(4.5,4.0)
          \psline[arrows=->](2.5, 4.0)(4.5,3.0)
          \psline[arrows=->](2.5, 3.0)(4.5,4.0)
          \psline[arrows=->](2.5, 3.0)(4.5,3.0)
          \psline[arrows=->](2.5, 3.0)(4.5,2.0)
          \psline[arrows=->](2.5, 2.0)(4.5,3.0)
          \psline[arrows=->](2.5, 2.0)(4.5,2.0)
          \psline[arrows=->](2.5, 2.0)(4.5,1.0)
          
          \rput(4.5,5.5){\tiny{E}}
		\rput(4.5,5.3){\tiny{44.35}}
          \rput(4.5,4.5){\tiny{F}}
		\rput(4.5,4.3){\tiny{31.37}}
		\rput(4.5,3.5){\tiny{G}}
		\rput(4.5,3.3){\tiny{22.18}}
		\rput(4.5,2.5){\tiny{H}}
		\rput(4.5,2.3){\tiny{15.69}}
		\rput(4.5,1.5){\tiny{I}}
		\rput(4.5,1.3){\tiny{11.10}}

          \psline[arrows=->](4.5, 5.0)(6.5,5.0)
          \psline[arrows=->](4.5, 5.0)(6.5,4.0)
          \psline[arrows=->](4.5, 4.0)(6.5,5.0)
          \psline[arrows=->](4.5, 4.0)(6.5,4.0)
          \psline[arrows=->](4.5, 4.0)(6.5,3.0)
          \psline[arrows=->](4.5, 3.0)(6.5,4.0)
          \psline[arrows=->](4.5, 3.0)(6.5,3.0)
          \psline[arrows=->](4.5, 3.0)(6.5,2.0)
          \psline[arrows=->](4.5, 2.0)(6.5,3.0)
          \psline[arrows=->](4.5, 2.0)(6.5,2.0)
          \psline[arrows=->](4.5, 2.0)(6.5,1.0)
          \psline[arrows=->](4.5, 1.0)(6.5,2.0)
          \psline[arrows=->](4.5, 1.0)(6.5,1.0)
          
          \rput(6.5,5.5){\tiny{J}}
		\rput(6.5,5.3){\tiny{45.68}}
          \rput(6.5,4.5){\tiny{K}}
		\rput(6.5,4.3){\tiny{32.30}}
		\rput(6.5,3.5){\tiny{L}}
		\rput(6.5,3.3){\tiny{22.85}}
		\rput(6.5,2.5){\tiny{M}}
		\rput(6.5,2.3){\tiny{16.16}}
		\rput(6.5,1.5){\tiny{N}}
		\rput(6.5,1.3){\tiny{11.43}}

	\end{pspicture}
\end{center}

\section{Ocenění opcí v investičním projektu}

Řada investičních projektů obsahuje vnořenou opci, která může mít významný vliv na jeho výsledné ocenění. Opce mají často charakter rozšíření popř. ukončení projektu.\\

Uvažujme firmu, která se rozhodla investovat 15 miliónů USD do těžby 6 miliónů barelů ropy. Uvažovaný projekt trvá tři roky a během každého roku vytěží 2 milióny barelů ropy. Fixní náklady projektu jsou 6 miliónů USD ročně a variabilní náklady jsou 17 USD na vytěžený barel. Předpokládejme, že bezriziková úroková sazba je 10\% p.a. po celou dobu životnosti projektu. Dále předpokládejme, že spotová cena ropy je 20 USD za barel a ceny futures po následující tři roky jsou 22 USD, 23 USD a 24 USD za barel. Vývoj ceny ropy je dán (23.4) s parametry $a = 0.1$ a $\sigma = 0.2$. Pro modelování vývoje ceny ropy v rizikově neutrálním světě je tedy možné použít výše odvozený trinomický strom.

\subsection{Projekt bez vnořené opce}

Nejprve předpokládejme, že projekt nemá vnořenou opci. Očekávaná cena ropy v rizikově neutrálním světě odpovídá cenám futures, tj. 22 USD, 23 USD resp 24 USD v prvním, druhém resp. třetím roce životnosti projektu. Očekávaná výplata generovaná projektem tedy může být na základě údajů o nákladech a trinomického stromu vývoje cen ropy stanovena na 4.0, 6.0 a 8.0 miliónů USD v prvním, druhém a třetím roce životnosti projektu. Hodnota projektu je tedy
\begin{equation*}
-15.0 + 4.0e^{-0.1 \cdot 1} + 6.0e^{-0.1 \cdot 2} + 8.0e^{-0.1 \cdot 3} = -0.54
\end{equation*}
Uvažovaný projekt by tedy neměl být realizován.

Následující trinomický strom představuje hodnotu projektu v jednotlivých uzlech. Tento trinomický strom by zkonstruován na základě údajů o nákladech projektu a trinomického stromu ceny ropy $S$.
\begin{center}
	\begin{pspicture}(0,0)(7,7)
		\rput(3.5,0){Projekt bez vnořené opce}

          \rput(0.2,3.5){\tiny{A}}
		\rput(0.2,3.3){\tiny{14.46}}

		\psline[arrows=->](0.5, 3.0)(2.5,4.0)
		\psline[arrows=->](0.5, 3.0)(2.5,3.0)
		\psline[arrows=->](0.5, 3.0)(2.5,2.0)
		
          \rput(2.5,4.5){\tiny{B}}
		\rput(2.5,4.3){\tiny{38.32}}
		\rput(2.5,3.5){\tiny{C}}
		\rput(2.5,3.3){\tiny{10.60}}
		\rput(2.5,2.5){\tiny{D}}
		\rput(2.5,2.3){\tiny{-9.65}}
		
          \psline[arrows=->](2.5, 4.0)(4.5,5.0)
          \psline[arrows=->](2.5, 4.0)(4.5,4.0)
          \psline[arrows=->](2.5, 4.0)(4.5,3.0)
          \psline[arrows=->](2.5, 3.0)(4.5,4.0)
          \psline[arrows=->](2.5, 3.0)(4.5,3.0)
          \psline[arrows=->](2.5, 3.0)(4.5,2.0)
          \psline[arrows=->](2.5, 2.0)(4.5,3.0)
          \psline[arrows=->](2.5, 2.0)(4.5,2.0)
          \psline[arrows=->](2.5, 2.0)(4.5,1.0)
          
          \rput(4.5,5.5){\tiny{E}}
		\rput(4.5,5.3){\tiny{42.24}}
          \rput(4.5,4.5){\tiny{F}}
		\rput(4.5,4.3){\tiny{21.42}}
		\rput(4.5,3.5){\tiny{G}}
		\rput(4.5,3.3){\tiny{5.99}}
		\rput(4.5,2.5){\tiny{H}}
		\rput(4.5,2.3){\tiny{-5.31}}
		\rput(4.5,1.5){\tiny{I}}
		\rput(4.5,1.3){\tiny{-13.49}}

          \psline[arrows=->](4.5, 5.0)(6.5,5.0)
          \psline[arrows=->](4.5, 5.0)(6.5,4.0)
          \psline[arrows=->](4.5, 4.0)(6.5,5.0)
          \psline[arrows=->](4.5, 4.0)(6.5,4.0)
          \psline[arrows=->](4.5, 4.0)(6.5,3.0)
          \psline[arrows=->](4.5, 3.0)(6.5,4.0)
          \psline[arrows=->](4.5, 3.0)(6.5,3.0)
          \psline[arrows=->](4.5, 3.0)(6.5,2.0)
          \psline[arrows=->](4.5, 2.0)(6.5,3.0)
          \psline[arrows=->](4.5, 2.0)(6.5,2.0)
          \psline[arrows=->](4.5, 2.0)(6.5,1.0)
          \psline[arrows=->](4.5, 1.0)(6.5,2.0)
          \psline[arrows=->](4.5, 1.0)(6.5,1.0)
          
          \rput(6.5,5.5){\tiny{J}}
          \rput(6.5,5.3){\tiny{0.00}}
          \rput(6.5,4.5){\tiny{K}}
          \rput(6.5,4.3){\tiny{0.00}}
          \rput(6.5,3.5){\tiny{L}}
          \rput(6.5,3.3){\tiny{0.00}}
          \rput(6.5,2.5){\tiny{M}}
          \rput(6.5,2.3){\tiny{0.00}}
          \rput(6.5,1.5){\tiny{N}}
          \rput(6.5,1.3){\tiny{0.00}}

	\end{pspicture}
\end{center}
Níže uvedená tabulka představuje pravděpodobnosti přesunu mezi jednotlivými uzly. V trinomickém stromě je možné se z každého uzlu přesunout do tří sousedících vyšších uzlů.
\begin{center}
\begin{tabular}{l l l l l l l l l l}
\hline
\textbf{Uzel} & \textbf{A} & \textbf{B} & \textbf{C} & \textbf{D} & \textbf{E} & \textbf{F} & \textbf{G} & \textbf{H} & \textbf{I} \\
\hline
$p_u$ & 0.1667 & 0.1217 & 0.1667 & 0.2217 & 0.8867 & 0.1217 & 0.1667 & 0.2217 & 0.0867\\
$p_m$ & 0.6666 & 0.6566 & 0.6666 & 0.6566 & 0.0266 & 0.6566 & 0.6666 & 0.6566 & 0.0266\\
$p_d$ & 0.1667 & 0.2217 & 0.1667 & 0.1217 & 0.0867 & 0.2217 & 0.1667 & 0.1217 & 0.8867\\
\end{tabular}
\end{center}
Uvažujme uzel H. V tomto uzlu je pravděpodobnost 0.2217, že cena ropy bude na konci třetího roku 22.85 USD za barel, což představuje zisk $2 \cdot 22.85 - 2 \cdot 17 - 6 = 5.70$ miliónů USD. Obdobně pravděpodobnost, že cena se posune na 16.16 USD za barel, je 0.6566 a pravděpodobnost, že cena bude na úrovni 11.43 USD je 0.1217. Při těchto cenách skončí projekt ve ztrátě -7.68 resp. -17.14 miliónů USD. Hodnota projektu v uzlu H je tedy
\begin{equation*}
[0.2217 \cdot 5.70 + 0.6566 \cdot (-7.68) + 0.1217 \cdot (-17.14)]e^{-0.1 \cdot 1} = -5.31
\end{equation*}
Hodnotu projektu je nutné analogicky počítat od koncových uzlů až k vrcholu trinomického stromu. Ve výchozím uzlu A je hodnota projektu 14.46 miliónů USD. Vezmeme-li v potaz prvnotní investici 15 miliónů USD, je výsledná hodnota projektu -0.54 miliónů USD.

\subsection{Projekt s vnořenou opcí}

\subsubsection{Ukončení projektu}

Předpokládejme, že máme možnost ukončit projekt s nulovými náklady v libovolný čas. Jedná se tedy o americkou opci s nulovou realizační cenou. Cena této opce je namodelována v níže uvedeném trinomickém stromě. Trinomický strom opce lze snadno odvodit od trinomického stromu projektu bez vnořené opce. Hodnota opce v jednotlivých uzlech je
\begin{equation*}
v = -MIN(0,P)
\end{equation*}
kde $P$ je hodnota projektu bez vnořené opce v odpovídajícím uzlu.
\begin{center}
	\begin{pspicture}(0,0)(7,7)
		\rput(3.5,0){Vnořená americká prodejní opce}

          \rput(0.2,3.5){\tiny{A}}
		\rput(0.2,3.3){\tiny{1.94}}

		\psline[arrows=->](0.5, 3.0)(2.5,4.0)
		\psline[arrows=->](0.5, 3.0)(2.5,3.0)
		\psline[arrows=->](0.5, 3.0)(2.5,2.0)
		
          \rput(2.5,4.5){\tiny{B}}
		\rput(2.5,4.3){\tiny{0.00}}
		\rput(2.5,3.5){\tiny{C}}
		\rput(2.5,3.3){\tiny{0.80}}
		\rput(2.5,2.5){\tiny{D}}
		\rput(2.5,2.3){\tiny{9.65}}
		
          \psline[arrows=->](2.5, 4.0)(4.5,5.0)
          \psline[arrows=->](2.5, 4.0)(4.5,4.0)
          \psline[arrows=->](2.5, 4.0)(4.5,3.0)
          \psline[arrows=->](2.5, 3.0)(4.5,4.0)
          \psline[arrows=->](2.5, 3.0)(4.5,3.0)
          \psline[arrows=->](2.5, 3.0)(4.5,2.0)
          \psline[arrows=->](2.5, 2.0)(4.5,3.0)
          \psline[arrows=->](2.5, 2.0)(4.5,2.0)
          \psline[arrows=->](2.5, 2.0)(4.5,1.0)
          
          \rput(4.5,5.5){\tiny{E}}
		\rput(4.5,5.3){\tiny{0.00}}
          \rput(4.5,4.5){\tiny{F}}
		\rput(4.5,4.3){\tiny{0.00}}
		\rput(4.5,3.5){\tiny{G}}
		\rput(4.5,3.3){\tiny{0.00}}
		\rput(4.5,2.5){\tiny{H}}
		\rput(4.5,2.3){\tiny{5.31}}
		\rput(4.5,1.5){\tiny{I}}
		\rput(4.5,1.3){\tiny{13.49}}

          \psline[arrows=->](4.5, 5.0)(6.5,5.0)
          \psline[arrows=->](4.5, 5.0)(6.5,4.0)
          \psline[arrows=->](4.5, 4.0)(6.5,5.0)
          \psline[arrows=->](4.5, 4.0)(6.5,4.0)
          \psline[arrows=->](4.5, 4.0)(6.5,3.0)
          \psline[arrows=->](4.5, 3.0)(6.5,4.0)
          \psline[arrows=->](4.5, 3.0)(6.5,3.0)
          \psline[arrows=->](4.5, 3.0)(6.5,2.0)
          \psline[arrows=->](4.5, 2.0)(6.5,3.0)
          \psline[arrows=->](4.5, 2.0)(6.5,2.0)
          \psline[arrows=->](4.5, 2.0)(6.5,1.0)
          \psline[arrows=->](4.5, 1.0)(6.5,2.0)
          \psline[arrows=->](4.5, 1.0)(6.5,1.0)
          
          \rput(6.5,5.5){\tiny{J}}
          \rput(6.5,5.3){\tiny{0.00}}
          \rput(6.5,4.5){\tiny{K}}
          \rput(6.5,4.3){\tiny{0.00}}
          \rput(6.5,3.5){\tiny{L}}
          \rput(6.5,3.3){\tiny{0.00}}
          \rput(6.5,2.5){\tiny{M}}
          \rput(6.5,2.3){\tiny{0.00}}
          \rput(6.5,1.5){\tiny{N}}
          \rput(6.5,1.3){\tiny{0.00}}

	\end{pspicture}
\end{center}
Pomocí trinomického stromu projektu bez vnořené opce a trinomického stromu pro vnořenou opci je možné ocenit projekt s uvažovanou vnořenou opcí. Jestliže se podíváme na trinomický strom projektu bez vnořené opce, je zřejmé že by vnořená opce s ohledem na hodnotu projektu neměla být uplatněna v uzlech E, F a G a naopak měla být uplatněna v uzlech H a I. V bodech H a I je hodnota 5.31 a 13.49 miliónů USD. V uzlu D je hodnota opce, pokud nebude uplatněna, rovna
\begin{equation*}
(0.1217 \cdot 13.49 + 0.6566 \cdot 5.31 + 0.2217 \cdot 0)e^{-0.1 \cdot 1} = 4.64
\end{equation*}
Hodnota uplatnění vnořené opce v uzlu D je 9.65\footnote{Uplatnění opce umožní vyhnout se očekávané ztrátě ve výši -9.65 miliónů USD.}, a proto by mělo dojít k uplatnění opce. Podobně hodnota opce v uzlu C je rovna
\begin{equation*}
(0.1667 \cdot 0 + 0.6666 \cdot 0 + 0.1667 \cdot 5.31)e^{-0.1 \cdot 1} = 0.80
\end{equation*}
a v uzlu A rovna
\begin{equation*}
(0.1667 \cdot 0 + 0.6666 \cdot 0.80 + 0.1667 \cdot 9.65)e^{-0.1 \cdot 1} = 1.94
\end{equation*}
Hodnota vnořené opce na začátku projektu je tedy rovna 1.94 miliónů USD. Hodnota projektu je tedy $1.94 - 0.54 = 1.40$ miliónů USD.

\subsubsection{Rozšíření projektu}

Předpokládejme, že máme možnost zvýšit těžbu o 20\%, tj. o 0.40 miliónů barelů. Variabilní náklady zůstanou na úrovni 17 USD za barel ropy a fixní náklady vzrostou o 20\%, tj. o 2 milióny USD.
\begin{center}
	\begin{pspicture}(0,0)(7,7)
		\rput(3.5,0){Vnořená americká prodejní opce}

          \rput(0.2,3.5){\tiny{A}}
		\rput(0.2,3.3){\tiny{1.06}}

		\psline[arrows=->](0.5, 3.0)(2.5,4.0)
		\psline[arrows=->](0.5, 3.0)(2.5,3.0)
		\psline[arrows=->](0.5, 3.0)(2.5,2.0)
		
          \rput(2.5,4.5){\tiny{B}}
		\rput(2.5,4.3){\tiny{5.66}}
		\rput(2.5,3.5){\tiny{C}}
		\rput(2.5,3.3){\tiny{0.34}}
		\rput(2.5,2.5){\tiny{D}}
		\rput(2.5,2.3){\tiny{0.00}}
		
          \psline[arrows=->](2.5, 4.0)(4.5,5.0)
          \psline[arrows=->](2.5, 4.0)(4.5,4.0)
          \psline[arrows=->](2.5, 4.0)(4.5,3.0)
          \psline[arrows=->](2.5, 3.0)(4.5,4.0)
          \psline[arrows=->](2.5, 3.0)(4.5,3.0)
          \psline[arrows=->](2.5, 3.0)(4.5,2.0)
          \psline[arrows=->](2.5, 2.0)(4.5,3.0)
          \psline[arrows=->](2.5, 2.0)(4.5,2.0)
          \psline[arrows=->](2.5, 2.0)(4.5,1.0)
          
          \rput(4.5,5.5){\tiny{E}}
		\rput(4.5,5.3){\tiny{6.45}}
          \rput(4.5,4.5){\tiny{F}}
		\rput(4.5,4.3){\tiny{2.28}}
		\rput(4.5,3.5){\tiny{G}}
		\rput(4.5,3.3){\tiny{0.00}}
		\rput(4.5,2.5){\tiny{H}}
		\rput(4.5,2.3){\tiny{0.00}}
		\rput(4.5,1.5){\tiny{I}}
		\rput(4.5,1.3){\tiny{0.00}}

          \psline[arrows=->](4.5, 5.0)(6.5,5.0)
          \psline[arrows=->](4.5, 5.0)(6.5,4.0)
          \psline[arrows=->](4.5, 4.0)(6.5,5.0)
          \psline[arrows=->](4.5, 4.0)(6.5,4.0)
          \psline[arrows=->](4.5, 4.0)(6.5,3.0)
          \psline[arrows=->](4.5, 3.0)(6.5,4.0)
          \psline[arrows=->](4.5, 3.0)(6.5,3.0)
          \psline[arrows=->](4.5, 3.0)(6.5,2.0)
          \psline[arrows=->](4.5, 2.0)(6.5,3.0)
          \psline[arrows=->](4.5, 2.0)(6.5,2.0)
          \psline[arrows=->](4.5, 2.0)(6.5,1.0)
          \psline[arrows=->](4.5, 1.0)(6.5,2.0)
          \psline[arrows=->](4.5, 1.0)(6.5,1.0)
          
          \rput(6.5,5.5){\tiny{J}}
          \rput(6.5,5.3){\tiny{0.00}}
          \rput(6.5,4.5){\tiny{K}}
          \rput(6.5,4.3){\tiny{0.00}}
          \rput(6.5,3.5){\tiny{L}}
          \rput(6.5,3.3){\tiny{0.00}}
          \rput(6.5,2.5){\tiny{M}}
          \rput(6.5,2.3){\tiny{0.00}}
          \rput(6.5,1.5){\tiny{N}}
          \rput(6.5,1.3){\tiny{0.00}}

	\end{pspicture}
\end{center}
V uzlu E by měla být opce uplatněna. Výplata z jejího uplatnění je $0.2 \cdot 42.24 - 2 = 6.45$. Podobně uplatnění opce v uzlu F generuje výplatu $0.2 \cdot 21.42 - 2 = 2.28$. V uzlech G, H a I by opce uplatněna být neměla. U uzlu B generuje okamžité uplatnění opce vyšší platbu než její případné uplatnění v budoucnu, a proto je hodnota opce v tomto uzlu dána
\begin{equation*}
0.2 \cdot 38.32 - 2 = 5.66
\end{equation*}
Naopak v uzlu C je hodnota opce v případě jejího možného pozdějšího uplatnění
\begin{equation*}
(0.1667 \cdot 2.28 + 0.6666 \cdot 0 + 0.1667 \cdot 0.00)e^{-0.1 \cdot 1} = 0.34
\end{equation*}
zatímco hodnota v případě jejího okamžitého uplatnění je pouze
\begin{equation*}
0.2 \cdot 10.80 - 2 = 0.16
\end{equation*}
V uzlu C tedy není optimální uplatnit opci a její hodnota je rovna 0.34 miliónů USD. V bodě A je hodnota opce v případě jejího neuplatnění rovna
\begin{equation*}
(0.1667 \cdot 5.66 + 0.6666 \cdot 0.34 + 0.1667 \cdot 0.00)e^{-0.1 \cdot 1} = 1.06
\end{equation*}
zatímco v případě jejího neuplatnění pouze $0.2 \cdot 14.46 - 2 = 0.89$. Okamžité uplatnění opce není tedy v bodě A optimální a celková hodnota projektu s touto vnořenou opcí je tedy $1.06 - 0.54 = 0.52$ miliónů USD.

\subsection{Vícero vnořených opcí}

V případě, že má projekt vícero vnořených opcí, nejsou tyto opce zpravidla nezávislé. Hodnota portofilia opcí A a B tak zpravidla neodpovídá součtu hodnot opce A a B. Jako příklad uvažujme situaci, kdy společnost má možnost realizovat projekt s vnořenou opcí na jeho ukončení popř. rozšíření. Je logické, že projekt nemůže být rozšířen, jestliže byl před tím ukončen. Vzájemné iterace vnořených opcí lze vyřešit tak, že v každém uzlu trinomického stromu budeme definovat čtyři možné stavy
\begin{itemize}
\item projekt nebyl rozšířen ani ukončen
\item projekt byl rozšířen a nebyl ukončen
\item projekt nebyl rozšířen a byl ukončen
\item projekt byl rozšířen a ukončen
\end{itemize}
S tím, jak postupujeme od koncových uzlů trinomického stromu k jeho vrcholu, je třeba vypočíst hodnotu opcí pro všechny čtyři varianty. Z těchto dílčích výsledků je pak možné "poskládat" výsledný trinomický strom.

Složitost výpočtu roste exponenciálně s počtem vnořených opcí. V těchto případech je vhodnější ocenit projekt pomocí metody Monte Carlo.

\end{document}
