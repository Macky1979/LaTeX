\chapter{Teorie extrémní hodnoty}

Pravděpodobnostní rozdělení ztráty kreditního porfolia je typicky značně sešikmené, což má za následek tzv. efekt ``tlustých'' chvostů. Pro odhad vysokých kvantilů tohoto pravděpodobnostního rozdělení je tak třeba zkombinovat simulaci Monte-Carlo s analytickým ``fitování'' chvostů. Tím se dostáváme k teorie extrémní hodnoty (extreme value theory - EVT). Ta se, jak napovídá její název, zaměřuje právě na extrémní ztráty a jejich pravděpodobnosti.

Z pohledu teorie extrémní hodnoty lze rozlišovat tři typy ztrát a to
\begin{itemize}
\item očekávanou ztrátu,
\item neočekávanou ztrátu a
\item extrémní ztrátu.
\end{itemize}
Teorie extrémní hodnoty se pak snaží o nalezení hranic mezi těmito typy ztrát.

\subsection{Základy teorie extrémní hodnoty}

Uvažujme posloupnost nezávislých náhodných veličin $\{X_1, X_2, ..., X_n\}$ z identického pravděpodonostního rozdělení $F$, které však neznáme. Předpokládejme, že tyto náhodné veličiny představují ztráty. Pak lze extrémní ztráta definovat jako největší ztrátu v rámci pozorování
\begin{equation}
M_n = max[X_1, X_2, ..., X_n]
\end{equation}
popř. pomocí prahové hodnoty $u_{\alpha}$
\begin{equation}
P[X > u_{\alpha}] = 1 - F(u_\alpha) = \alpha
\end{equation}
kde $\alpha$ představuje požadovaný kvantil. Práh $u_{\alpha}$ tak představuje minimální úroveň ekonomického kapitálu, kterým musí banka disponovat, aby přežila extrémní ztrátu.

V případě, že je stanoveno $u_{\alpha}$, pak nás zpravidla také zajímá podmíněná pravděpodobnost
\begin{equation*}
F_{u_{\alpha}} \equiv P[X - u_{\alpha} \le x | X > u_{\alpha}]
\end{equation*}
Tuto pravděpodobnost lze přeformulovat do tvaru
\begin{multline}
F_{u_{\alpha}}(x) \equiv P[X - u_{\alpha} \le x | X > u_{\alpha}]\\
= \frac{P[X - u_{\alpha} \le x, X > u_{\alpha}]}{P[X > u_{\alpha}]}\\
= \frac{F_{u_{\alpha}}(x + u_{\alpha}) - F_{u_{\alpha}}(u_{\alpha})}{1 - F_{u_{\alpha}}(u_{\alpha})}, x \ge 0
\end{multline}
V případě, že máme dostatečné množství dat, lze podmíněnou pravděpodobnost $F_{u_{\alpha}}$ vypočíst přímo. V opačném případě je třeba vhodným způsobem tuto pravděpodobnost aproximovat.

\subsection{Obecné Pareto rozdělení}

Obecné Pareto rozdělení je rodina pravděpodobnostních rozdělení s třemi stupni volnosti a pravděpodobností funkcí
\begin{equation}
G_{\xi, \mu, \psi}(x) = e^{-\left(1 + \xi \frac{x - \mu}{\psi}\right)_{+}^{-\frac{1}{\xi}}}, ~~~ \xi \ne 0, \psi > 0
\end{equation}
kde $\psi$ je škálovací parametr, $\mu$ je lokační parametr a $\xi$ je parametr, který ovlivňuje tvar rozdělení. Pro $\xi = 0$ přejde rozdělení do limitního tvaru
\begin{equation}
G_{0, \mu, \psi}(x) = e^{-e^{-\frac{x - \mu}{\psi}}}
\end{equation}
V praxi se pak často namísto (8.4) používá zjednodušený tvar
\begin{equation}
G_{\xi, \mu, \psi}(x) = 1 - \left(1 + \xi \frac{x - \mu}{\psi}\right)_{+}^{-\frac{1}{\xi}}, ~~~ \xi \ne 0
\end{equation}
Obě rovnice mají stejnou asymptotickou formu a pro praktické účely je tedy lhostejné, kterou z nich použijeme.

Podle hodnoty parametru $\xi$ lze obecné Pareto rozdělení dále klasifikovat na
\begin{itemize}
\item $\xi = 0$ - Gumbel popř. dvojité exponenciální rozdělení (zahrnuje normální, exponenciální, gamma a lognormální rozdělení)
\item $\xi > 0$ - Frechet rozdělení s tlustým chvostem na pravo (zahrnuje Pareto, Burr, log-gamma, Cauchy a studentovo rozdělení)
\item $\xi < 0$ - Weibull rozdělení s tlustým chvostem na levo (zahrnuje beta a uniformní rozdělení)
\end{itemize}

OBRÁZEK

\subsection{Kritéria konvergence}

\begin{theorem}[Fisher-Tippett, 1928]
Uvažujme i.i.d posloupnost náhodných veličin $\{X_1, X_2, ..., X_n\}$ z neznámé distribuční funkce $F$, pro kterou známe neúplnou empirickou pravděpodobnostní funkci $F^{emp}$. Předpokládejme, že existuje posloupnost reálných čísel $a_n$ a $b_n$ taková, že posloupnost normalizovaného maxima $(M_n - a_n) / b_n$ konverguje k pravdnedegerativnímu pravděpodobnostnímu rozdělení
\begin{equation}
P\left[\frac{M_n - a_n}{b_n} \le x \right] = F^{emp}(a_nx + b_n) \rightarrow G(x) ~~~ pro ~~~ n \rightarrow \infty
\end{equation}
V tomto případě patří rozdělení $G$ do rodiny obecného Pareto rozdělení $G_{\xi, \nu, \psi}(x)$. 
\end{theorem}
Jestliže je splněna podmínka (8.7), pak pravděpodobnostní rozdělení $F$ v maximální doméně aktrakce $G_{\xi, \mu, \psi}$, což zapisujeme jako
\begin{equation}
F \in MDA(G_{\xi, \mu, \psi})
\end{equation}
Jestliže jsme tedy schopni na přijatelně velkém vzorku pozorování prokázat limitní chování, pak limitní rozdělení musí být obecené Pareto rozdělení. Pro účely řízení rizik se pak nejčastěji používá Frechet rozdělení.
\begin{theorem}[Gnedenko, 1943]
Pro Frechet rozdělení platí
\begin{equation}
F \in MDA(G_{\xi, \mu, \psi}) \Leftrightarrow 1 - F(x) ~ x^{-1/\xi}L(x)
\end{equation}
pro nějakou pomalu se měnící funkci $L(x)$.
\end{theorem}
Tato věta říká, že pokud chvost rozdělení $F(x)$ ``vymírá'' jako mocninná funkce, pak se nachází v doméně atrakce Frechet rozdělení.

\subsection{Prahové hodnoty}

Následující věta garantuje, že distribuční funkce přesahu přes určitou prahovou hodnotu konverguje k limitě obecného Pareto rozdělení.

\begin{theorem}[Picklands, 1975; Balkema-de Haan, 1974]
Nechť je $x_0$ konečný či nekonečný bod pravé strany pravděpodobnostního rozdělení, tj.
\begin{equation}
x_0 = sup\{x \in R:F(x) < 1\} \le \infty
\end{equation}
Nechť je pravděpodobnostní funkce přesahu přes prahovou hodnotu $u_{\alpha}$ dána rovnicí
\begin{equation}
F_{u_{\alpha}}(x) \equiv P[X - u_{\alpha} \le x | X > u_{\alpha}] ~~~ pro 0 \le x < x_0 - u_{\alpha}
\end{equation}
Pak $F \in MDA(G)$ právě tehdy a jen tehdy jestliže je $G$ obecné Pareto rozdělení s tím, jak se prahová hodnota blíží koncovému bodu pravé strany pravděpodobnostního rozdělení. To znamená, že existuje kladná měřitelná funkce $\Psi(u_{\alpha})$ taková, že
\begin{equation}
\lim_{u_{\alpha} \rightarrow x_0} \sup_{0 \le x < x_0 - u_{\alpha}}|F_{u_{\alpha}}(x) - G_{\xi, \mu, \psi(u_\alpha)}(x)| = 0
\end{equation}
tehdy a právě tehdy jestliže $F \in MDA(G)$.
\end{theorem}
Pro dostatečně vysoké prahové hodnoty $u_{\alpha}$ lze tedy pravděpodobnostní funkci přesahu $F$ aproximovat pomocí obecného Pareto rozdělení $G_{\xi, \mu, \psi}$. Věta nám tak dává teoretický základ toho, že je-li zvolena dostatečně vysoká prahová hodnota, sledují data za touto prahovou hodnotou obecné Pareto rozdělení.

\subsection{Funkce střední hodnoty přesahu}

Stále ještě nebyla zodpovězena klíčová otázka, totiž jakým způsobem zvolit prahovou hodnotu. V předchozím textu jsme prahovou hodnotu odvodili od ratingu, který se odjíví od určitého cílovaného kvantilu. Existují však i jiné metody. Řada z nich je vizuálního charakteru.
\begin{itemize}
\item QQ grafy - Konkávní popř. konvexní odchylka od trendové přímky indikuje těžké popř. lehké chvosty v empirickém pravděpodobnostním rozdělení.
\item Hill funkce odhadu - Tato funkce není nic jiného než inverzí průměru logaritmu podílů pořadových statistik empirických dat. Tato funkce odhadu se používá pro aproximaci $\xi^{-1}$ v obecném Pareto rozdělení.
\item funkce střední hodnoty přesahu - Tato funkce je definována jako
\begin{equation}
\hat{e}(u) = \frac{\sum_{i = 1}^n (X_i - u)_{+}}{\sum_{i = 1}^n 1_{\{x_i > u\}}}
\end{equation}
Funkce střední hodnoty přesahu $\hat{e}(u)$ je empirickým odhadem střední hodnoty náhodné veličiny
\begin{equation}
e(u) \equiv E[X - u | X > u]
\end{equation}
Pro obecné Pareto rozdělení lze střední hodnotu jednoduše vypočíst jako
\begin{equation}
e(u) = \frac{\psi + \xi u}{1 - \xi}
\end{equation}
kde $\psi + \xi u > 0$. Všimněme si, že pro obecné Pareto rozdělení je $e(u)$ je lineární funkcí prahové hodnoty $u$. To znamená, že pokud graf $\hat{e}(u)$ přibližně sleduje přímku (nebo alespoň má kladný gradient) nad určitou hodnotu $u$, pak to indikuje, že odpovídající hodnota přesahu sleduje obecné Pareto rozdělení s kladným parametrem $\xi$. Kladný gradient je jasnou známkou těžkého chvostu. 
\end{itemize}
