\chapter{Kreditní ztráta v potrfoliu}

Kreditní riziko nelze, narozdíl od tržního rizika zajistit. Jedniným nástrojem na redukci kreditního rizika tak zůstává diverzifikace.

\section{Očekávaná vs neočekávaná ztráta}

Připomeňme, že v případě jednoho úvěrového aktiva lze očekávanou ztrátu vyjádřit jako
\begin{equation*}
EL = AE \times LGD \times EDF
\end{equation*}
a neočekávanou ztrátu jako
\begin{equation*}
UL = AE \times \sqrt{EDF \times \sigma_{LGD}^2 + LGD^2 \times \sigma_{EDF}^2}
\end{equation*}
Kdybychom do stejného grafu vykreslili očekávanou a neočekávanou ztrátu, zjistili bychom, že s klesající kvalitou úvěru (tj. s rostoucím $EDF$)
\begin{itemize}
\item očekávaná ztráta roste lineárně a
\item neočekávaná ztráta roste nelineárně a v porovnání s očekávanou ztrátou mnohem rychleji
\end{itemize}

OBRÁZEK

\subsection{Očekávaná ztráta portfolia}

Z pohledu očekávané ztráty portfolia je lhostejné, zda-li dvě úvěrová aktiva náleží stejnému či dvěma rozdílným dlužníkům. Očekávaná ztráta portfolia je prostým součetem dílčích očekávaných ztrát\footnote{V následujícím textu jsou $EL$ a $UL$ vztaženy k $AE$, tj. $EL^* = \frac{EL}{AE}$ a $UL^* = \frac{UL}{AE}$.}.
\begin{equation*}
EL_P = \sum_i EL_i = \sum_i \left(AE_i \times LGD_i \times EDF_i \right)
\end{equation*}

\subsection{Neočekávaná ztráta portfolia}

V případě neočekávané ztráty je výpočet poněkud komplikovanější. Jestliže $\rho_{ij}$ je korelace defaultu mezi úvěrovými aktivy $i$ a $j$, pak lze tuto ztrátu vyjádřit jako
\begin{equation}
UL_P = \left[\rho_{ij} UL_i UL_j \right]^{\frac{1}{2}}
\end{equation}
kde
\begin{equation*}
UL_i = AE_i \times \sqrt{EDF_i \times \sigma_{LGD(i)}^2 + LGD_i^2 \times \sigma_{EDF(i)}^2}
\end{equation*}
Z důvodu diverzifikace pak platí
\begin{equation*}
UL_P << \sum_i UL_i
\end{equation*}

\subsection{Riziková kontribuce}

Riziková kontribuce $i$-tého aktiva k neočekávané ztrátě v rámci portfolia je definována jako
\begin{equation*}
RC_i \equiv UL_i \frac{\partial UL_P}{\partial UL_i}
\end{equation*}
Jestliže aplikujeme derivaci na (4.1), získáme
\begin{equation*}
RC_i = \frac{UL_i \sum_j UL_j \rho_{ij}}{UL_p}
\end{equation*}

\subsection{Nediversifikovatelné riziko}

Riziková kontribuce je mírou nediverzifikovatelného rizika úvěrového aktiva. Tato kontribuce totiž vyjadřuje výši kreditního rizika, kterého se nelze zbavit umístěním úvěrového aktiva do portfolia. Lze totiž dokázat, že
\begin{equation*}
UL_P = \sum_i RC_i
\end{equation*}

\begin{example}
Uvažujme dvě úvěrová aktiva s následujícími charakteristikami. Nechť je první úvěrové aktivum definováno parametry $COM_1 = 10~000~000~USD, OS_1 = 5~000~000~USD, UGD_1 = 65\%, EDF_1 = 0.15\%, LGD_1 = 50\%, \sigma_{LGD_1} = 25\%$ a druhé úvěrové aktivum parametry $COM_2 = 2~000~000~USD, OS_2 = 1~500~000~USD, UGD_2 = 48\%, EDF_2 = 4.85\%, LGD_2 = 35\%, \sigma_{LGD_2} = 24\%$. Předpokládejme, že korelace defaultu mezi oběma úvěrovými aktivy je $\rho = 3\%$.

Vypočtěme expozici v případě defaultu
\begin{align*}
AE_1 = OS_1 + (COM_1 - OS_1) \times UGD_1 = 8~250~000~USD\\
AE_2 = OS_2 + (COM_2 - OS_2) \times UGD_2 = 1~740~000~USD
\end{align*}
směrodatnou odchylku defaultu
\begin{align*}
\sigma_{EDF_1} = \sqrt{EDF_1 \times (1 - EDF_1)} = 3.87\%\\
\sigma_{EDF_2} = \sqrt{EDF_2 \times (1 - EDF_2)} = 21.48\%
\end{align*}
dílčí očekávanou ztrátu
\begin{align*}
EL_1 = AE_1 \times EDF_1 \times LGD_1 = 6~188~USD\\
EL_2 = AE_2 \times EDF_2 \times LGD_2 = 29~537~USD
\end{align*}
a dílčí neočekávanou neočekávanou ztrátu
\begin{align*}
UL_1 = AE_1 \times \sqrt{EDF_1 \times \sigma_{LGD_1}^2 + LGD_1^2 \times \sigma_{EDF_1}^2} = 178~511~USD\\
UL_2 = AE_2 \times \sqrt{EDF_2 \times \sigma_{LGD_2}^2 + LGD_2^2 \times \sigma_{EDF_2}^2} = 159~916~USD
\end{align*}
Očekávaná ztráta portfolia je tedy rovna
\begin{equation*}
EL_P = EL_1 + EL_2 = 35~724~USD
\end{equation*}
a neočekávaná ztráta pak
\begin{equation*}
UL_P = \sqrt{UL_1^2 + UL_2^2 + 2 \times \rho \times UL_1 \times UL_2} = 243~212~USD
\end{equation*}
což je méně než prostý součet 338~427~USD. Rizikové kontribuce jsou pak
\begin{align*}
RC_1 = \frac{UL_1 \times (UL_1 + UL_2 \times \rho)}{UL_P} = 134~543~USD\\
RC_2 = \frac{UL_2 \times (UL_2 + UL_1 \times \rho)}{UL_P} = 108~669~USD
\end{align*}
\end{example}

\section{Příloha A - Odvození rizikové kontribuce}

Derivací (3.1) lze snadno dokázat
\begin{multline*}
RC_k = UL_k \frac{\partial UL_P}{\partial UL_k} = UL_k \times \frac{1}{2 UL_P}\left[2UL_k + 2\sum_{i \neq k}UL_i \rho_{ik}\right]\\
= \frac{UL_k}{UL_p}\left[UL_k + \sum_i UL_i \rho_{ik} - UL_k \rho_{kk} \right] = \frac{UL_k}{UL_p}\left[UL_k(1 - \rho_{kk}) + \sum_i UL_i \rho_{ik} \right]\\
= \frac{UL_k}{UL_P}\left[\sum_i UL_i \rho_{ik} \right]
\end{multline*}

V praxi může být počet párových korelací $\rho_{ik}$ neúnosný, a proto je vhodné jejich počet zredukovat přiřazením jednotlivých úvěrových aktiv odvětvím. Korelace pak budou definovány nikoliv na úrovni úvěrových aktiv ale na úrovni těchto odvětví. Uvažujme dvě odvětví $\alpha$ a $\beta$. Výše uvedená rovnice pak přejde do tvaru
\begin{multline*}
RC_k = \frac{UL_k}{UL_P}\left[UL_{k \in \alpha}(1 - \rho_{\alpha \alpha}) + \sum_{\beta} \sum_{j \in \beta}UL_j \rho_{\alpha \beta} \right]\\
= \frac{UL_k}{UL_P}\left[\sum_{\beta} \sum_{j \in \beta} UL_j \rho_{\alpha \beta} \right]
\end{multline*}
