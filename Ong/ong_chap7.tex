\chapter{Simulace Monte-Carlo}

Postup při simulaci kreditních ztrát metodou Monte-Carlo je následující.
\begin{enumerate}
\item Prvním krokem je odhad pravděpodobností defaultu $PD$, expozice v okamžiku defaultu $AE$ a podmíněné ztráty $LGD$ pro uvažované portfolio.
\item Následně odhadneme korelace aktiv mezi jednotlivými dlužníky.
\item Vygenerujeme set nezávislých náhodných čísel ze standardizovaného normálního rozdělení. Tyto náhodné veličiny zkorelujeme pomocí korelační matice odhadnuté v předchozím kroku (např. pomocí Cholesky dekompozice).
\item Pro každého dlužníka vypočteme bod defaultu na základě jeho pravděpodobnosti defaultu.
\item Porovnáme bod defaultu s nasimulovanými náhodnými veličinami a rozhodneme, zda-li došlo k defaultu.
\item Jestliže došlo k defaultu, simulujeme podmíněnou ztrátu jako náhodnou veličinu $lgd$. K tomu je zapotřebí znalost nejen střední hodnoty podmíněné ztráty $LGD$ ale také jejího rozptylu $\sigma_{LGD}$.
\item Následně u dlužníků v defaultu vypočteme  ztrátu jako $AE \times lgd$.
\item Posledním krokem je výpočet celkové ztráty, která je součtem dílčích ztrát přes všechny dlužníky. Opakováním kroků 3 až 7 získáme požadovaný počet simulací.
\end{enumerate}
Typický výsledek simulace zachycuje následující obrázek.

OBRÁZEK

\section{Matematika za simulací Monte-Carlo}

V následujícím textu se zaměříme na body 3 až 7.

\subsection{Generování náhodných veličin}

Pro každého z $N$ dlužníků v rámci potrfolia vygenerujeme náhodné číslo ze standardizovaného normálního rozdělení.
\begin{equation}
\epsilon =
\begin{bmatrix}
\epsilon_1\\
\epsilon_2\\
\hdots
\epsilon_N
\end{bmatrix}
\end{equation}
kde $\epsilon_i ~ N(0, 1)$ pro všechna $i$. Abychom získali vektor vzájemně korelovaných náhodných veličin, je třeba provést transformaci
\begin{equation*}
\epsilon'_u = \sum_{p = 1}^N m_{up}\epsilon_p
\end{equation*}
což lze maticově vyjádřit jako
\begin{equation}
\epsilon' = M \epsilon
\end{equation}
Úkolem je tedy nalézt vhodnou transformační matici $M$, s jejíž pomocí jsme schopni převést vektor $\epsilon$ na vektor $\epsilon'$, který bude náhodným výběrem z vícerozměrného normálního rozdělení s požadovanou kovarianční strukturou. Hledaná transformační matice $M$ musí splňovat podmínku
\begin{equation}
\Sigma = M^T M
\end{equation}
kde $\Sigma$ je korelační matice. Tuto podmínku lze odvodit následovně.
\begin{multline}
\Sigma = D[\epsilon'] = E[(M\epsilon)^2] - E[M\epsilon]^2\\
= E[M^T \epsilon \epsilon^T M] - E[M \epsilon]^2\\
= M^TE[\epsilon \epsilon^T] M - M^T E[\epsilon]^2 M\\
= M^T I M = M^TM
\end{multline}

\subsubsection{Cholesky dekompozice}

Cholesky dekompozice rozloží positivně definitní matici $\Sigma$ na
\begin{equation}
\Sigma = A^T A
\end{equation}
kde $A$ je horní trojúhelníková matice
\begin{equation*}
A =
\begin{bmatrix}
a_{11} & a_{12} & a_{13} & \dots & a_{1n}\\
0 & a_{22} & a_{23} & \dots & a_{2n}\\
0 & 0 & a_{33} & \dots & a_{3n}\\
\dots & \dots & \dots & \dots & \dots\\
0 & 0 & 0 & \dots & a_{nn}
\end{bmatrix}
\end{equation*}
Lze dokázat, že
\begin{equation*}
a_{ii} = \left(s_{ii} - \sum_{k = 1}^{i - 1}a_{ik}\right) ^ {\frac{1}{2}}
\end{equation*}
a
\begin{equation*}
a_{ij} = \frac{1}{a_{ii}}\left(s_{ij} - \sum_{k = 1}^{i - 1} a_{ik} a_{jk}\right) ^ {\frac{1}{2}}
\end{equation*}
kde $s_{ij}$ je prvkem korelační matice $\Sigma$.

Hlavní nevýhodou Cholesky dekompozice je skutečnost, že neumí rozložit korelační matici, pokud je signulární nebo není positivně definitní.

\subsubsection{Eigenvalue dekompozice}

Eigenvalue rovnice je
\begin{equation*}
\Sigma U = U \Lambda
\end{equation*}
kde $\Sigma$ je daná matice, $\Lambda$ je diagonální matice, jejíž prvky tvoří eigenvalues matice $\Sigma$. Matici $U$ je třeba určit. Úpravou výše uvedené rovnice získáme
\begin{equation*}
\Sigma = U \Sigma U ^ {-1} = U \Lambda^{\frac{1}{2}} \Lambda^{\frac{1}{2}} U^{-1} = Q^{T}Q
\end{equation*}
Tento přístup vyžaduje ortogonální matici $U$, tj. $U^T = U^{-1}$. Korelované náhodné veličiny $epsilon'$ vypočteme pomocí
\begin{equation*}
\epsilon' = Q \epsilon = \Lambda^{\frac{1}{2}} U^{-1} \epsilon
\end{equation*}

\subsubsection{SVD dekompozice}

Pro korelační matice, které jsou singulární, lze použít SVD dekompozici. Jádrem dekompozice je rovnice
\begin{equation*}
\Sigma = V D V^T
\end{equation*}
kde $V$ je ortogonální matice, tj. $V^T = V^{-1}$ a $D$ je diagonální matice se signulárními hodnotami matice $\Sigma$ na diagonále. Korelované náhodné veličiny je tak možné získat pomocí vztahu
\begin{equation*}
\epsilon' = D^{\frac{1}{2}}V^{-1}\epsilon
\end{equation*}

\subsection{Výpočet bodu defaultu}

Předpokládejme, že pro každého dlužníka máme k dispozici $EDF$. Bod defaultu vypočteme jako $DP \equiv N^{-1}(EDF)$. Dalším krokem je porovnání korelovaných náhodných veličin $\epsilon'$ s bodem defaultu pro $i$-tého dlužníka.
\begin{align*}
\epsilon' < DP_i ~~~ \textit{default dlužníka}\\
\epsilon' \ge DP_i ~~~ \textit{dlužník není v defaultu}
\end{align*}

\subsection{Podmíněná ztráta}

Jestliže je daný dlužník v defaultu, je třeba určit podmíněnou ztrátu. Ačkoliv bychom správně na podmíněnou ztrátu měli pohlížet jako na náhodnou veličinu, v praxi se často používá deterministická hodnota odpovídající střední hodnotě $LGD$. Pokud bychom chtěli na podmíněnou ztrátu pohlížet jako na náhodnou veličinu, pak se nejčastěji používá beta rozdělení. Pro simulaci z beta rozdělení je za potřebí nejen znalost střední hodnoty $LGD$ ale také směrodatné odchylky $\sigma_{LGD}$.

\subsection{Výpočet ztráty}

Celkovou ztrátu v rámci jedné simulace lze snadno vypočíst jako
\begin{equation*}
L = \sum_{\textit{dlužníci v defaultu}} AE_i \times lgd_i
\end{equation*}

\subsection{Simulované pravděpodobnostní rozdělení ztráty}

Simulované pravděpodobnostní rozdělení ztráty je možné získat opakováním výše uvedených kroků. Počet simulací zavisí na velikosti portfolia a požadovaném kvantilu, nicméně obecně se uvádí, že je třeba 10~000 až 1~000~000 simulací, abychom dosáhli přijatelné stability řešení.
