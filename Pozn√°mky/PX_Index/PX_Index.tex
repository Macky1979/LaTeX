\documentclass[a4paper]{book}
\usepackage[czech]{babel}
\usepackage[utf8]{inputenc}
\usepackage{pstricks}
\usepackage{amsmath}

\setlength{\unitlength}{1.0mm}

\begin{document}

\chapter{PX index}

\section{Výpočet}

PX index je hlavním idexem Pražské burzy. Výpočet indexu je dán vzorcem
\begin{equation*}
PX = K(t)\frac{M(t)}{M(0)} \cdot 1000
\end{equation*}
kde $K(t)$ je tzv. chaining faktor\footnote{Chaining faktor slouží k přepočtu báze PX indexu, aby nedocházelo k cenovým skokům v důsledku její pouhé změny. Pomocí chaining faktoru je tedy zajištěna "cenová kontinuita" indexu. Tento faktor se průběžně mění a je k dispozici na stránkách www.pse.cz.}, $M(t)$ představuje aktuální tržní kapitalizaci indexu, $M(0)$ je tržní kapitalizací indexu v roce 1994\footnote{Hodnota parametru $M(0)$ je tudíž neměnná a rovna 379,786,853,620 CZK}.

Bloomberg používá namísto chaing faktoru tzv. divisor. Vztah mezi chaining faktorem $K(t)$ a divisorem $D(t)$ je dán rovnicí
\begin{equation*}
D(t) = \frac{M(0)}{K(t) \cdot 10^9}
\end{equation*}

\section{Databáze}
Hodnoty potřebné pro výpočet PX indexu jsou k dispozici v tabulce RM01..LIM\_IndexWeight. Do této tabulky jsou případné změny dohrány automaticky spouštěným skriptem\footnote{Konkrétně se jedná o skript PXIndexWeights.vbs.} z počítače WA1000079\footnote{Jedná se o počítač s Bloombergem.}. Velice často se stává, že Bloomberg zapomene provést aktualizaci údajů po té, co byla změněna báze. V tomto případě je třeba poslat požadavek na Bloomberg Help, kde tento problém obratem ruky vyřeší.

\end{document}
