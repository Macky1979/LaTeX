\documentclass[a4paper]{book}
\usepackage[czech]{babel}
\usepackage[utf8]{inputenc}
\usepackage{pstricks}
\usepackage{amsmath}

\setlength{\unitlength}{1.0mm}

\begin{document}

\chapter{FX Volatilita}

\textbf{Poznámka:} Ve finanční terminologii označuje pojem "volatilita" směrodatnou odchylku (tj. druhou odmocninu rozptylu).

\section{Výpočet}

Uvažujme měnové páry TRY-USD a EUR-USD. Jestliže pro tyto dva měnové páry známe směrodatnou odchylku a korelaci, je možné dopočítat směrodatnou odchylku pro měnový pár TRY-EUR.

Definujme měnové kurzy $X_t$, $Y_t$ a $Z_t$ pro časový okamžik $t$ jako
\begin{equation*}
X_t = \frac{TRY_t}{USD_t} 
\end{equation*}
\begin{equation*}
Y_t = \frac{EUR_t}{USD_t}
\end{equation*}
\begin{equation*}
Z_t = \frac{TRY_t}{EUR_t}
\end{equation*}
Je zřejmé že, za předpokladu neexistence arbitráže, platí pro měnový kurz $Z_t$
\begin{equation*}
Z_t = \frac{X_t}{Y_t}
\end{equation*}
resp.
\begin{equation*}
\ln Z_t = \ln X_t - \ln Y_t
\end{equation*}
Nechť je relativní změna $X$ v časovém intervalu $t-1$ až $t$ dána vztahem
\begin{equation*}
X^{\%}_{t, t-1} = \frac{X_t - X_{t-1}}{X_{t-1}} \approx \ln \frac{X_t}{X_{t-1}} = \ln X_t - \ln X_{t-1}
\end{equation*}
Relativní změnu pro měnové kurzy $Y$ a $Z$ definujme analogicky.

Z logiky věci vyplývá, že $Z^{\%}_{t, t-1}$ lze vypočítat dle
\begin{equation*}
Z^{\%}_{t, t-1} \approx \ln Z_t - \ln Z_{t-1} = (\ln X_t - \ln Y_t) - (\ln X_{t-1} -  \ln Y_{t-1}) =
\end{equation*}
\begin{equation*}
= (\ln X_t - \ln X_{t-1}) - (\ln Y_t - \ln Y_{t-1}) = X^{\%}_{t, t-1} - Y^{\%}_{t, t-1} 
\end{equation*}
S ohledem na definici rozptylu složené náhodné veličiny $A - B$
\begin{equation*}
D[A - B] = D[A] + D[B] - 2cov(A,B)
\end{equation*}
tedy pro rozptyl měnového kurzu $Z$ platí
\begin{equation*}
D[Z^{\%}_{\Delta t}] = D[X^{\%}_{\Delta t} - Y^{\%}_{\Delta t}] = D[X^{\%}_{\Delta t}] + D[Y^{\%}_{\Delta t}] - 2 cov(X^{\%}_{\Delta t}, Y^{\%}_{\Delta t})
\end{equation*}
kde $\Delta t$ představuje časový interval délky $t, t-1$. Výše uvedenou rovnici lze dále upravit do tvaru
\begin{equation*}
D[Z^{\%}_{\Delta t}] = D[X^{\%}_{\Delta t}] + D[Y^{\%}_{\Delta t}] - 2 \rho(X^{\%}_{\Delta t}, Y^{\%}_{\Delta t}) \sqrt{D[X^{\%}_{\Delta t}]D[Y^{\%}_{\Delta t}]}
\end{equation*}

\subsection{Přepočet volatility na roční bázi}

Takto vypočtený rozptyl je vztažen k časovému intervalu $t, t-1$. Je tedy třeba ho přepočítat na roční bázi. Klasicky je časový interval $t, t-1$ roven jednomu pracovnímu dni. Jestliže budeme předpokládat, že rok má 252 pracovních dní a že jednotlivé denní změny kurzu jsou vzájemně nezávislé \footnote{To znamená, že kovariance mezidenních změn je rovna nule.}, platí pro roční rozptyl
\begin{equation*}
D[Z^{\%}_Y] = 252 \cdot D[Z^{\%}_D]
\end{equation*}
kde $D[Z^{\%}_D]$ je denním rozptylem relativní změny měnového kurzu $Z$.

\section{Reuters}

Na Reuters jsou velice často k dispozici volatity pouze ve vztahu k USD. Jestliže je zapotřebí volatilita např. vůči EUR, je třeba ji vypočítat křížově dle výše uvedeného postupu. Jediné, co je třeba dopočítat "ručně" z historických dat, je korelace $\rho(X^{\%}_{\Delta t}, Y^{\%}_{\Delta t})$ mezi měnovými páry.

\end{document}
